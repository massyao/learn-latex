\documentclass[10pt]{article}
\usepackage{makeidx}
\usepackage{multirow}
\usepackage{multicol}
\usepackage[dvipsnames,svgnames,table]{xcolor}
\usepackage{graphicx}
\usepackage{epstopdf}
\usepackage{ulem}
\usepackage{hyperref}
\usepackage{amsmath}
\usepackage{amssymb}
\author{Bruce Bueno de Mesquita, Alastair Smith}
\title{The Dictator's Handbook}
\usepackage[paperwidth=595pt,paperheight=841pt,top=72pt,right=90pt,bottom=72pt,left=90pt]{geometry}

\makeatletter
	\newenvironment{indentation}[3]%
	{\par\setlength{\parindent}{#3}
	\setlength{\leftmargin}{#1}       \setlength{\rightmargin}{#2}%
	\advance\linewidth -\leftmargin       \advance\linewidth -\rightmargin%
	\advance\@totalleftmargin\leftmargin  \@setpar{{\@@par}}%
	\parshape 1\@totalleftmargin \linewidth\ignorespaces}{\par}%
\makeatother 

% new LaTeX commands


\begin{document}

{\large Acknowledgments}

{\large Notes}

{\large Index Copyright Page}

{\large To our dictators, who have treated us so well }

{\large         -----------Arlene and Fiona}

{\large What is important here is cash. [A] leader needs money, gold and
diamonds to run his hundred castles, feed his thousand women, buy cars for the
millions of boot-lickers under his heels, reinforce the loyal military forces and
still have enough change left to deposit into his numbered Swiss accounts.}

{\large         -----------MOBUTU SESE SEKOOF ZAIRE, PROBABLY APOCRYPHAL}
{\large           }
{\large Men at some time are masters of their fates. The fault, dear Brutus, is
not in our stars, but in ourselves, that we are underlings.}

{\large         -----------WILLIAM SHAKESPEARE, Julius Caesar (I, II, 140-141)}
\pagebreak{}


\section{Introduction}

{\large Rules to Rule By}

{\large What remarkable puzzles politics provides. Every day's headlines shock
and surprise us. Daily we hear of frauds, chicanery, and double-dealing by
corporate executives, new lies, thefts, cruelties and even murders perpetrated by
government leaders. We cannot help but wonder what flaws of culture, religion,
upbringing, or historical circumstance explain the rise of these malevolent
despots, greedy Wall Street bankers, and unctuous oil barons. Is it true, as
Shakespeare's Cassius said, that the fault lies not in the stars but in
ourselves? Or, more particularly, in those who lead us?}

{\large Most of us are content to believe that. And yet the truth is far
different.}

{\large Too often we accept the accounts of historians, journalists, pundits,
and poets without probing beneath the surface to discover deeper truths that
point neither to the stars nor to ourselves. The world of politics is dictated by
rules. Short is the term of any ruler foolish enough to govern without submitting
to these rules to rule by.}

{\large Journalists, authors, and academics have endeavored to explain politics
through storytelling. They'll explore why this or that leader seized power, or
how the population of a far-flung country came to revolt against their
government, or why a specific policy enacted last year has reversed the fortunes
of millions of lives. And in the explanations of these cases, a journalist or
historian can usually tell us what happened, and to whom, and maybe even why. But
beneath the particulars of the many political stories and histories we read are a
few questions that seem to emerge time after time, some profound, some seemingly
minor, but all nagging and enduring in the back of our minds: How do tyrants hold
on to power for so long? For that matter, why is the tenure of successful
democratic leaders so brief?}

{\large How can countries with such misguided and corrupt economic policies
survive for so long? Why are countries that are prone to natural disasters so
often unprepared when they happen? And how can lands rich with natural resources
at the same time support populations stricken withpoverty?}

{\large Equally, we may well wonder: Why are Wall Street executives so
politically tone-deaf that they dole out billions in bonuses while plunging the
global economy into recession? Why is the leadership of a corporation, on whose
shoulders so much responsibility rests, decided by so few people?}

{\large Why are failed CEOs retained and paid handsomely even as their company's
shareholders lose their shirts?}

{\large In one form or another, these questions of political behavior pop up
again and again. Each explanation, each story, treats the errant leader and his
or her faulty decision making as a one-off, one-of-a-kind situation. But there is
nothing unique about political behavior.}

{\large These stories of the horrible things politicians or business executives
do are appealing in their own perverse way because they free us to believe we
would behave differently if given the opportunity. They liberate us to cast blame
on the flawed person who somehow, inexplicably, had the authority to make
monumental---and monumentally bad---decisions. We are confident that we would
never act like Libya's Muammar Qaddafi who bombed his own people to keep himself
in power. We look at the huge losses suffered under Kenneth Lay's leadership by
Enron's employees, retirees, and shareholders and think we aren't like Kenneth
Lay. We look at each case and conclude they are different, uncharacteristic
anomalies.}

{\large Yet they are held together by the logic of politics, the rules ruling
rulers.}

{\large The pundits of politics and the nabobs of news have left us ignorant of
these rules. They are content to blame the doers of evil without inquiring why
the worlds of politics and business seem to succor miscreants or to turn good
people into scoundrels. That's why we are still asking the same old questions.
We're still surprised by the prevalence of drought-induced food shortages in
Africa, 3,500 years after the pharaohs worked out how to store grain. We're still
shocked by the devastation of earthquakes and tsunamis in places like Haiti,
Iran, Myanmar, and Sri Lanka, and by the seemingly lesser intensity of such
natural disasters in North America and Europe. We're still troubled by the
friendly handshakes and winks exchanged between democratic leaders and the
tyrants that they somehow justify empowering.}

{\large In this book, we're going to provide a way to make sense of the
miserable behavior that characterizes many---maybe most---leaders,whether in
government or business. Our aim is to explain both good and bad conduct without
resorting to ad hominem claims. At its heart, this will entail untangling the
reasoning and reasons behind how we are governed and how we organize.}

{\large The picture we paint will not be pretty. It will not strengthen hope for
humankind's benevolence and altruism. But we believe it will be the truth and it
will point the way to a brighter future. After all, even if politics is nothing
more than a game that leaders play, if only we learn the rules, it becomes a game
we can win.}

{\large To improve the world, however, all of us must first suspend faith in
conventional wisdom. Let logic and evidence be the guide and our eyes will be
opened to the reasons why politics works the way it does. Knowing how and why
things are as they are is a first, crucial step toward learning how to make them
better.Bell's Bottomless Blues In politics, as in life, we all have desires and
contend with obstacles that keep us from getting what we want. A government's
rules and laws, for example, limit what we can do. Those in power differ from the
rest of us: they can design rules to their advantage and make it easier for them
to get what they want. Understanding what people want and how they get it can go
a long way to clarifying why those in power often do bad things. In fact, bad
behavior is more often than not good politics. This dictum holds up whether one
governs a tiny town, a mom-and-pop business, a megacorporation, or a global
empire.}

{\large Let's start with a tale of a small town's team of seemingly greedy,
grasping, avaricious louts so that we can appreciate how the world looks from a
leader's perspective. And yet it's vital that we remember that this is a story
about politics, not personality. Whether we're discussing a cabal of corrupt
reprobates or not, what really matters is that these are people who value power
and recognize how to get it and keep it. Soon enough we will come to appreciate
that this small tale of miserable conduct recurs at every level of politics and
corporate governance, and that there is nothing out of the ordinary in the
extraordinary story of Bell, California.}

{\large Robert Rizzo is a former city manager of the small town of Bell
(population about 36,600). Bell, a suburb of Los Angeles, is a poor, mostly
Hispanic and Latino town. Per capita income may be as low as \$10,000 or as high
as \$25,000---estimates vary---but either way it is way below both the California
and national average. More than a quarter of the town's hard-working people live
below the poverty line. Life is not easy in Bell.}

{\large Still, it is a community that takes pride in its accomplishments, its
families, and its prospects. Despite its many challenges, Bell consistently
outperforms other California communities in keeping violent crime and property
crime below average. A cursory glance at Bell's official website suggests a
thriving, happy community brimming over with summer classes, library events,
water play, and fun-filled family trips. And Bell seems to be aconcerned
community too. The town offers, for instance, Housing and Urban Development (HUD)
grants to pay for repairs to single-family homes provided certain basic residency
and income requirements are met.1 Robert Rizzo, in his job for seventeen years,
must surely look back on his time as city manager with pride. In 2010, Bell's
then-mayor, Oscar Hernandez (later jailed on corruption charges), said the town
had been on the verge of bankruptcy in 1993 when Rizzo (also ultimately charged
with corruption) was hired. For fifteen consecutive years of Rizzo's leadership,
up until he stepped down in 2010, the city's budget had been balanced.}

{\large Hernandez credits Rizzo with making the town solvent and helping to keep
it that way.2 That, of course, is no mean feat. Surely he and the town leaders
with whom he worked were deserving of praise and tangible rewards for their good
service to the people of Bell.}

{\large Behind the idyllic fa\c{c}ade, however, lies a story that embodies how
politics really works. You see, Robert Rizzo, hired at \$72,000 a year in 1993,
and in his job for seventeen years before being forced to step down in the summer
of 2010, at the end of his tenure was earning a staggering \$787,000 per year.}

{\large Let's put that in perspective. If his salary had just kept up with
inflation, he would have made \$108,000 in 2010. He made seven times more! During
long years of low inflation, his salary went up at an annual, compounded rate of
more than 15 percent, almost exactly the return promised by Bernie Madoff, the
master Ponzi schemer, to his hapless investors.}

{\large How does Rizzo's city manager's pay compare to other responsible
government jobs? The president of the United States is paid \$400,000.3 The
governor of California's salary is just over \$200,000. The mayor of Los Angeles,
just a hop, skip, and a jump from Bell, is paid only a bit over \$200,000. To be
sure, Robert Rizzo was not even close to the highest paid public employee in
California. That distinction, as in most states, went to the coach of a
university football team---UC Berkeley's coach earned about \$1,850,000 in 2010,
but then he probably brought in a lot more revenue than Mr. Rizzo.4 Robert Rizzo
was indeed credited with doing a good job for Bell, but was it really that good?
It seems that he was the highest paid city manager in the entire United States
(or at least until we discover another Bell).The natural thought is that somehow
Robert Rizzo must have been stealing money, dipping into the proverbial cookie
jar, taking funds that were not rightfully and legally his or, at least, doing
something or other that was immoral and illegal. The California attorney general
(and Democratic candidate for governor) at the time of the Bell scandal in the
summer of 2010, Jerry Brown, promised an investigation to find out if any laws
had been violated. The implicit message in his action was clear enough: No one
would pay a small town city manager nearly \$800,000 a year. The truth, however,
is quite a bit more complicated.}

{\large The actual story is one of clever (and reprehensible) political
maneuvering implicitly sanctioned by Bell's voters and the city council members
who represent them, supplemented only by a touch of larceny.}

{\large Cities comparable to Bell pay their council members an average of
\$4,800 a year. But four of Bell's five council members received close to
\$100,000 a year through the simple mechanism of being paid not only their
(minimal) base council salaries but also nearly \$8,000 per month to sit on city
agency boards. Only poor councilman Lorenzo Velez failed to reap such rewards.
Velez apparently received only \$8,076 a year as a council member, approximately
equal to what his fellow council members were getting each month. How can we
possibly explain these disparities, let alone the outrageous salaries and
pensions provided not only to Mr. Rizzo, but also to the assistant city manager
and Bell's chief of police (all subsequently jailed on corruption charges)?}

{\large The answers lie in a clever manipulation of election timing. The city's
leaders ensured that they depended on very few voters to hold power and to set
their compensation. To see how a poor community could so handsomely reward its
town leaders we must start with the 2005 special election to convert Bell from a
``general city'' to a ``charter city.'' What, you may well ask between yawns, is
the difference between a general city and a charter city? The answer is day and
night: decisions are made in the open daylight in general cities and often in
secret, behind closed doors in charter cities. While a general city's governing
system is dictated by state or federal law, a charter city's governance is
defined by---well, as you would expect---its own charter.}

{\large The California legislature decided in 2005 to limit salaries for city
council members in general cities. No sooner did the state legislaturemove to
impose limits than creative politicians in Bell---some allege Robert Rizzo led
the way---found a way to insulate themselves from the ``whims'' of those sent to
California's state capitol, Sacramento. A special election was called, supported
by all five council members, to turn Bell into a charter city. The selling point
of the change to charter city was to give Bell greater autonomy from decisions by
distant state officials. Local authorities know best what is right for their
community, more so than distant politicians who are not in touch with local
circumstances. Or, at least, so the leaders of Bell, California, argued.}

{\large Special elections on technical questions---to be a charter city or to
remain a general city---are less than captivating to the general voter. Of
course, if the decision had been made in the context of a major national or even
statewide election, the proposition would likely have been scrutinized by many
prospective voters, but as it happens---surely by political design ---the special
election, associated with no other ballot decisions, attracted fewer than 400
voters (336 in favor, 54 opposed) in a town of 36,000 people. And so the charter
passed, placing within the control of a handful of people the right to allocate
city revenues and form the city budget, and to do so behind closed doors. As best
as one can tell, the charter changed nothing else of consequence concerning
Bell's governance. It just provided a means to give vast discretion over taxing
and spending decisions to a tiny group of people who were, as it happens, making
choices about their own compensation.}

{\large Lest one think the council members were stupid as well as venal, it is
worth noting how clever they were in disguising what they had done.}

{\large Should anyone care to ask a city council member's part-time salary, any
councilman or councilwoman could say openly and honestly that they were each paid
just a few hundred dollars a month, a pittance for their services.}

{\large As we have already seen, the bulk of their pay---the part denied to
Lorenzo Velez---was for participation on city agency boards. That, as it turns
out, may ultimately have been their Achilles' heel.}

{\large As of this writing all of the principal players in Bell's scandal have
been jailed, but not for their lavish salaries. As reprehensible as these may
have been, it seems they were perfectly legal. No, they were jailed for receiving
payments for meetings that allegedly never took place. It seems they collected a
lot of money while overlooking their obligation to actually attendcommittee
meetings. This is to say that the well-paid managers of Bell may end up falling
victim to what one might describe as a legal technicality.}

{\large Outrageous salaries were okay, but getting paid for attending meetings
while being absent from them was not. We cannot help but wonder how many
government officials are held to that standard. How many senators and
representatives, for instance, draw their full salaries while skipping meetings
of the Senate or House so that they can raise campaign funds, give speeches, or
go on boondoggles?}

{\large You may well wonder how a little town like Bell could balance its budget
---one of Mr. Rizzo's significant accomplishments---while paying such high
salaries. (Indeed, we anticipate a high probability that once Bell's governance
is cleaned up, its spending will involve indebtedness rather than a balanced
budget.) Remember, the town's leaders got to choose not only how to spend money
but also how much tax to levy. And did they ever tax their constituents. Here's
what the Los Angeles Times reported about property taxes in Bell: Bell's rate is
1.55\%---nearly half again as much as those in such affluent enclaves as Beverly
Hills and Palos Verdes Estates and Manhattan Beach, and significantly higher than
just about everywhere else in Los Angeles County, according to records provided
by the county Auditor-Controller's Office at the Times request. That means that
the owner of a home in Bell with an assessed value of \$400,000 would pay about
\$6,200 in annual property taxes. The owner of the same home in Malibu, whose
rate is 1.10\%, would pay just \$4,400.5 In plain and simple terms, Bell's
property tax was about 50 percent higher than nearby communities. With such high
taxes, the city manager and council certainly could pay big salaries and balance
the budget, all the while enriching themselves and their key cronies.}

{\large Now that we have Bell's story let's look at the subtext. In the city,
council members are elected, although their election was not contested for many
years before 2007. That means that council members are beholden to the voters, or
at least the voters whose support was needed to win office.}

{\large Before 2007 that was hardly anyone since elections were not contested.}

{\large Since 2007, as it turns out, even with contested elections, it still
took veryfew votes to win a council seat. For instance, Bell had about 9,400
registered voters in 2009, of which only 2,285---that is, 24.3 percent--- turned
out to vote. Each voter could cast a ballot for two candidates for city council
out of the six candidates seeking that office. The two winners, Luis Artiga and
Teresa Jacobo, received 1,201 and 1,332 votes respectively, out of 2,285 votes
that were cast, but they didn't need that many votes to win. Speaking generously,
election was achieved with supportive votes from only about 13 percent of the
registered electorate. We say ``speaking generously'' because to get elected to
the city council in 2009 all that was necessary was to have one more vote than
the third largest vote-getter among the candidates. Remember, two were to be
elected. The number three candidate had just 472 votes. So, 473 votes---about 5
percent of the registered voters, just over 1 percent of the city population, and
only about one fifth of those who actually turned out to vote---is all that was
needed to win election. Whatever the reason for the vote being divided among so
many candidates, it is evident that election could be achieved with support from
only a tiny percentage of Bell's adult population. This goes a long way to
explaining the city government's taxing and spending policies.}

{\large One thing we can be sure of: those on the city council could not have
been eager for competing candidates (or even fellow council member Velez) to get
wind of the truth about their compensation package. City manager Rizzo had to
maintain the council's confidence to keep his job and they needed his support to
keep theirs. He could have exposed how deeply they were dipping into the public's
hard-earned money, which would have sent them packing (as it now has). It is in
this need for mutual loyalty that we see the seeds of Bell's practices and of
politics in general. Rizzo served at the pleasure of the mayor and city council.
They, in turn, served at the pleasure of a tiny group of Bell's citizens, the
essential supporters among Bell's considerably larger prospective electorate.
Without the council's support, Rizzo would be, as he now is, out on his
ear---albeit with a fabulous pension estimated at \$650,000 per year. How best to
keep their loyalty? That was easy: promote the means to transfer great private
rewards in the form of lavish compensation packages to council members.6 Of
course, if all were being done in the open, or if Bell remained a general city
subject to control over compensation from Sacramento, Rizzocould not have
provided the means to ensure that he would scratch the city council members'
backs and they his. When a leader's hold on power---his or her political
survival---depends on a small coalition of backers (remember the small percentage
of voters needed to actually win a seat on the city council), then providing
private rewards is the path to long tenure in office: Mr. Rizzo kept his job for
seventeen years. Furthermore, when that small coalition is drawn from a
relatively large pool---just five council members, elected under a city charter
ratified by only 354 voters out of a registered voter population (in 2009) of
9,395---then not only are private rewards to the small coalition an efficient way
to govern, but so much budgetary and taxing discretion is created that the folks
at the top have ample opportunity for handsome compensation, an opportunity that
the city's top leadership did not fail to exploit.}

{\large Bell presents a number of lessons to teach us about the rules to rule
by.}

{\large First, politics is about getting and keeping political power. It is not
about the general welfare of ``We, the people.'' Second, political survival is
best assured by depending on few people to attain and retain office. That means
dictators, dependent on a few cronies, are in a far better position to stay in
office for decades, often dying in their sleep, than are democrats.}

{\large Third, when the small group of cronies knows that there is a large pool
of people waiting on the sidelines, hoping to replace them in the queue for
gorging at the public trough, then the top leadership has great discretion over
how revenue is spent and how much to tax. All that tax revenue and discretion
opens the door to kleptocracy from many leaders, and publicspirited programs from
a very few. And it means enhanced tenure in power. Fourth, dependence on a small
coalition liberates leaders to tax at high rates, just as was true in Bell.
Taxing at high rates has a propensity to foment the threat of popular uprisings,
just as happened in Bell. Of course, in Bell it was easy for the people to rise
up and end Rizzo's rule because they have essential freedoms: the rights to free
speech and assembly. We shall see that how the structure of government and the
economy works explains variation in how many of these rights people have. This in
turn accounts for whether the people take to the streets and whether they can
succeed in orchestrating change, as we recently saw in some parts of the Middle
East, or remain oppressed, as we saw in others.}

{\large We will see that Bell's story offers a nearly perfect script for how
togovern when the hold on office depends on very few people, especially when they
are selected from among many. The politicians of Bell intuitively understood the
rules of politics. Leaders who follow these rules faithfully truly can stay on
top without ever having to do ``the right thing'' for their subjects. The people
governing Bell clung to power for a very long time before probes from outside
uncovered their means of holding on to office.}

{\large As we will see, what works for those at the top usually works against
those at the bottom, hence our shock and surprise at headlines of the misdoings
of so many in high positions. The way places like Bell are governed (and that is
the way most places and most businesses are governed) assure the Bell Bottom
Blues.}

{\large One important lesson we will learn is that where politics are concerned,
ideology, nationality, and culture don't matter all that much. The sooner we
learn not to think or utter sentences such as ``the United States should do . .}

{\large . ''or ``the American people want . . . '' or ``China's government ought
to do . .}

{\large . ,'' the better we will understand government, business, and all other
forms of organization. When addressing politics, we must accustom ourselves to
think and speak about the actions and interests of specific, named leaders rather
than thinking and talking about fuzzy ideas like the national interest, the
common good, and the general welfare. Once we think about what helps leaders come
to and stay in power, we will also begin to see how to fix politics. Politics,
like all of life, is about individuals, each motivated to do what is good for
them, not what is good for others. And that surely is the story of Robert Rizzo
of Bell, California.Great Thinker Confusion As Robert Rizzo's story highlights,
politics is not terribly complicated. But by the same measure, history's most
revered political philosophers haven't explained it very well. The fact is,
people like Niccol\`{o} Machiavelli, Thomas Hobbes, James Madison, and
Charles-Louis de Secondat (that is, Montesquieu), not to forget Plato and
Aristotle, thought about government mostly in the narrow context of their times.}

{\large Hobbes sought the best form of government. His search, however, was
blinded by his experience of the English civil war, the rise of Cromwell, and his
fear of rule by the masses. Fearing the masses, Hobbes saw monarchy as the
natural path to order and good governance. Believing in the necessary benevolence
of an absolute leader, the Leviathan, he also concluded that, ``no king can be
rich, nor glorious, nor secure, whose subjects are either poor, or contemptible,
or too weak through want, or dissension, to maintain a war against their
enemies.''7 Taking a bit of liberty with Hobbes's more nuanced philosophy, we
must wonder how Robert Rizzo, by Hobbesian lights, could grow so rich when his
subjects, the citizens of Bell, were so demonstrably poor.}

{\large Machiavelli, an unemployed politician/civil servant who hoped to become
a hired hand of the Medici family---that is, perhaps the Robert Rizzo of his
day---wrote The Prince to demonstrate his value as an adviser. It seems the
Medicis were not overly impressed---he didn't land the job. He had, we believe, a
better grasp than Hobbes on how politics can create self-aggrandizing practices
such as were experienced in Bell half a millennium later. Writing in The
Discourses, Machiavelli observes that anyone seeking to establish a government of
liberty and equality will fail, ``unless he withdraws from that general equality
a number of the boldest and most ambitious spirits, and makes gentlemen of them,
not merely in name but in fact, by giving them castles and possessions, as well
as money and subjects; so that surrounded by these he may be able to maintain his
power, and that by his support they may satisfy their ambition.. . .''8 Robert
Rizzo might have done well to study Machiavelli as the best source of his defense
against public opprobrium. He maintained his power for long years by satisfying
the ambition for wealth and position of those loyal to him on Bell's city
council, and they really were the only people whose support he had to have.}

{\large James Madison, a revolutionary trying to bring his brand of politics
into power, was, like Hobbes, looking revolution in the face. Unlike Hobbes,
however, Madison actually liked what he saw. In Federalist 10, Madison
contemplated the problem that was to bedevil the citizens of Bell a quarter of a
millennium later, ``whether small or extensive Republics are most favorable to
the election of proper guardians of the public weal: and it is clearly decided in
favor of the latter.''9 His conclusion, not easily reached as he was fearful
about tyranny of the majority, is close to what we argue is correct although, as
always, the devil is in the details and Madison, we believe, fell a bit short on
the details of good governance. In describing a republic as large or small, he
failed to distinguish between how many had a say in choosing leaders and how many
were essential to keeping a leader in place. The two, as we will see, can be
radically different.}

{\large Madison's view was at odds with that of Montesquieu, who maintained
that, ``In a large republic the public good is sacrificed to a thousand views; it
is subordinate to exceptions; and depends on accidents. In a small one, the
interest of the public is easier perceived, better understood, and more within
the reach of every citizen; abuses have a lesser extent, and of course are less
protected.''10 Not so in Bell---and in Bell we trust.}

{\large For Montesquieu, the Enlightenment, the new Cartesian thinking, and the
emerging constitutional monarchy of Britain all combined to stimulate his
insightful ideas of political checks and balances. Through these checks and
balances he hoped to prevent exactly the corruption of public welfare that the
charter city election in Bell foisted on its citizens.}

{\large Of course, the option of forming a charter city was motivated, in
theory, exactly by a quest for checks and balances on the authority of
California's state legislature. But the electoral public in the charter city
special election was a meager 390 souls, and even in Bell's contested elections
before the scandal, fewer than a quarter of registered voters, themselves only a
quarter of the city's population, bothered to vote. That's not enough toprevent
the very corruption Montesquieu hoped to avoid.}

{\large Now there is no doubt that Montesquieu, Madison, Hobbes, and Machiavelli
were very clever and insightful thinkers (and surely brighter than us). However,
they got an awful lot of politics wrong simply because they were coping with
momentary circumstances. They were looking at but a small sample of data, the
goings-on surrounding them, and bits and pieces of ancient history. They also
lacked modern tools of analysis (which we, luckily, have at our disposal).
Consequently, they leapt to partially right, but often deeply wrong, conclusions.
In all fairness to these past luminaries, their shortcomings often have to do
with the fact that, besides being bound by their then-present contexts, these
thinkers were also caught up in ``the big questions''---what the highest nature
of man ought to be, or what the ``right'' state of government really is, or what
``justice'' truly means in political terms. This shortsightedness extends not
only to history's legends in political thought, but also to contemporary thinkers
like J\"{u}rgen Habermas, Michel Foucault, and John Rawls---thinkers who someday
may be viewed in the same light.}

{\large The big questions of how the world ought to be are indeed important.}

{\large But they are not our focus. Questions of philosophical values and
metaphorical abstractions---these simply don't apply to the view of politics that
we'll present in the pages ahead. We do not start with a desire to say what we
think ought to be. It is hard to imagine that anyone, including ourselves, cares
much about what we think ought to be. Neither do we exhort others to be better
than they are. Not that we do not hope to find ways to improve the world
according to our lights. But then, we believe that the world can only be improved
if first we understand how it works and why.}

{\large Working out what makes people do what they do in the realm of politics
is fundamental to working out how to make it in their interest to do better
things.}

{\large The modern vernacular of politics and international relations, from
balances of power and hegemony to partisanship and national interest, is the
stuff of high school civics and nightly news punditry. It has little to do with
real politics. And so, you may be delighted---or disappointed---to hear that this
particular book of politics is not concerned with any of this. Our account of
politics is primarily about what is, and why what is, is. In this book, we hope
to explain the most fundamental and puzzling questionsabout politics, and in the
process give all of us a better way to think about why the worlds of rulers and
subjects, of authorities and rights, of war and peace, and, in no small way, of
life and death all work in the ways that they do. And maybe, just maybe, from
time to time we will see paths to betterment.}

{\large The origins of the ideas developed here came years ago during heated
lunchtime discussions between one of the authors of this book---Bruce Bueno de
Mesquita---and a coauthor of many earlier works, Randolph M.}

{\large Siverson (now Professor Emeritus at the University of California,
Davis).}

{\large While munching on burritos, Randy Siverson and Bueno de Mesquita
discussed a rather basic question: What are the consequences for leaders and
their regimes when a war is lost?}

{\large Oddly, that question had not been much addressed in the copious research
on international affairs, and yet surely any leader would want to know before
getting involved in a risky business like war what was going to happen to him
after it was over. This question hadn't been asked because the standard ideas
about war and peace were rooted in notions about states, the international
system, and balances of power and polarity, and not in leader interests. From the
conventional view of international relations, the question just didn't make
sense. Even the term ``international relations'' presumes that the subject is
about nations rather than being about what Barack Obama or Raul Castro or any
other named leader wants. We so easily speak of United States grand strategy or
China's human rights policy or Russian ambitions to restore Russia to great power
status, and yet, from our point of view, such statements make little sense.}

{\large States don't have interests. People do. Amidst all the debate about
national interest, what did President Obama fret about in formulating his Afghan
policy? If he did not announce a timetable for withdrawal from Afghanistan he
would lose support from his Democratic---not his national, but his
Democratic---electoral base. President Kennedy similarly fretted that if he took
no action in what became the Cuban missile crisis, he would be impeached and the
Democrats would pay a heavy price in the 1962 midterm election.11 National
interest might have been on each of their minds, but their personal political
welfare was front and center.}

{\large The prime mover of interests in any state (or corporation for that
matter)is the person at the top---the leader. So we started from this single
point: the self-interested calculations and actions of rulers are the driving
force of all politics.}

{\large The calculations and actions that a leader makes and takes constitute
how she governs. And what, for a leader, is the ``best'' way to govern? The
answer to how best to govern: however is necessary first to come to power, then
to stay in power, and to control as much national (or corporate) revenue as
possible all along the way.}

{\large Why do leaders do what they do? To come to power, to stay in power and,
to the extent that they can, to keep control over money.}

{\large Building on their lunchtime question about leaders and war, Randy and
Bruce wrote a couple of academic journal articles in which they looked at
international relations as just ordinary politics in which leaders, above all
else, want to survive in power. These articles caught on quickly.}

{\large Researchers saw that this was a different way to think about their
subject, one tied to real people making real decisions---in their own interest---
rather than metaphors like states, nations, and systems. (It seems obvious now,
but among the dominant realist school of international relations this is still
heresy.) But Siverson and Bueno de Mesquita also saw that the theory could be
stretched across a bigger canvas. Every type of politics could be addressed from
the point of view of leaders trying to survive.}

{\large The idea that the canvas was that big was scary. It meant trying to
recast everything (or nearly everything) we knew or thought we knew about
politics in a single theoretical whole. It was a humbling moment, and Bueno de
Mesquita and Siverson felt in need of help. Enter James D. Morrow---now a
professor at the University of Michigan but back then a Senior Research Fellow at
Stanford's Hoover Institution, where Bueno de Mesquita was also based---and
Alastair Smith. And so a foursome was born (sometimes affectionately known as
BdM2S2). Together we wrote a thick, dense, technical tome called The Logic of
Political Survival, as well as a long list of journal articles, that remain the
foundation for this translation of our ideas into an account that we hope anyone
can follow, argue with, and maybe even come to accept.12 Today the theory behind
this body of research has inspired many spin-off studies by us and by other
researchers, theoretical expansions and elaborations by us and by others, and
some lively debate ---and no shortage of controversy as well.Using this
foundation, we look at politics, the choices of public policies, and even
decisions about war and peace as lying outside of conventional thinking about
culture and history. It also means that we put ideas of civic virtue and
psychopathology aside as central to understanding what leaders do and why they do
it. Instead, we look at politicians as self-interested louts, just the sort of
people you wouldn't want to have over for dinner, but without whom you might not
have dinner at all.}

{\large The structure of the book is simple. After outlining the essentials of
ruling in Chapter 1, each subsequent chapter will probe a specific feature of
politics. We'll assess why taxes are higher in many poor countries than in rich
countries; or why leaders can spend a fortune on the military and yet have a weak
and almost useless army when it comes to the national defense. Together, the
chapters will detail how the political logic of political survival---the rules to
rule by---connects dots of political consequence across the widest canvas
imaginable, deepening our understanding of the dynamics of all rulers and their
populations. It is because of this capacity to ``connect the dots'' that many of
our students have called our list of rules to rule by the ``Theory of
Everything.'' We are content to codify it simply as ``The Dictator's Handbook.''
We fully admit that our view of politics requires us to step outside of
wellentrenched habits of mind, out of conventional labels and vague generalities,
and into a more precise world of self-interested thinking. We seek a simpler and,
we hope, more compelling way to think about government. Our perspective,
disheartening though it may be to some, offers a way to address other facets of
life than just government. It easily describes businesses, charities, families,
and just about any other organization. (We're sure many readers will be comforted
to have confirmation that their companies really are run like tyrannical
regimes.) All of this may be sacrilege to some, but we believe that, in the end,
it's the best way to understand the political world---and the only way that we
can begin to assess how to use the rules to rule by to rule for the better. If we
are going to play the game of politics, and we all must from time to time, then
we ought to learn how to win the game. We hope and believe that is just what we
all can take away from this book: how to win the game of politics and perhaps
even improve the world a bit as we do so.}
\pagebreak{}


\section{Chapter 1 - The Rules of Politics}

{\large THE LOGIC OF POLITICS IS NOT COMPLEX. IN FACT, it is surprisingly easy
to grasp most of what goes on in the political world as long as we are ready to
adjust our thinking ever so modestly. To understand politics properly, we must
modify one assumption in particular: we must stop thinking that leaders can lead
unilaterally.}

{\large No leader is monolithic. If we are to make any sense of how power works,
we must stop thinking that North Korea's Kim Jong Il can do whatever he wants. We
must stop believing that Adolf Hitler or Joseph Stalin or Genghis Khan or anyone
else is in sole control of their respective nation. We must give up the notion
that Enron's Kenneth Lay or British Petroleum's (BP) Tony Hayward knew about
everything that was going on in their companies, or that they could have made all
the big decisions. All of these notions are flat out wrong because no emperor, no
king, no sheikh, no tyrant, no chief executive officer (CEO), no family head, no
leader whatsoever can govern alone.}

{\large Consider France's Louis XIV (1638--1715). Known as the Sun King, Louis
reigned as monarch for over seventy years, presiding over the expansion of France
and the creation of the modern political state. Under Louis, France became the
dominant power in Continental Europe and a major competitor in the colonization
of the Americas. He and his inner circle invented a code of law that helped shape
the Napoleonic code and that forms the basis of French law to this day. He
modernized the military, forming a professional standing army that became a role
model for the rest of Europe and, indeed, the world. He was certainly one of the
preeminent rulers of his or any time. But he didn't do it alone.}

{\large The etymology of monarchy may be ``rule by one,'' but such rule does
not, has not, and cannot exist. Louis is thought famously (and probably falsely)
to have proclaimed, L'etat, c'est moi: the state, it is me. Thisdeclaration is
often used to describe political life for supposedly absolute monarchs like
Louis, likewise for tyrannical dictators. The declaration of absolutism, however,
is never true. No leader, no matter how august or how revered, no matter how
cruel or vindictive, ever stands alone. Indeed, Louis XIV, ostensibly an absolute
monarch, is a wonderful example of just how false this idea of monolithic
leadership is.}

{\large After the death of his father, Louis XIII (1601--1643), Louis rose to
the throne when he was but four years old. During the early years actual power
resided in the hands of a regent---his mother. Her inner circle helped themselves
to France's wealth, stripping the cupboard bare. By the time Louis assumed actual
control over the government in 1661, at the age of twenty-three, the state over
which he reigned was nearly bankrupt.}

{\large While most of us think of a state's bankruptcy as a financial crisis,
looking through the prism of political survival makes evident that it really
amounts to a political crisis. When debt exceeds the ability to pay, the problem
for a leader is not so much that good public works must be cut back, but rather
that the incumbent doesn't have the resources necessary to purchase political
loyalty from key backers. Bad economic times in a democracy mean too little money
to fund pork-barrel projects that buy political popularity. For kleptocrats it
means passing up vast sums of money, and maybe even watching their secret bank
accounts dwindle along with the loyalty of their underpaid henchmen.}

{\large The prospect of bankruptcy put Louis's hold on power at risk because the
old-guard aristocrats, including the generals and officers of the army, saw their
sources of money and privilege drying up. Circumstances were ripe to prompt these
politically crucial but fickle friends to seek someone better able to ensure
their wealth and prestige. Faced with such a risk, Louis needed to make changes,
or else risk losing his monarchy.}

{\large Louis's specific circumstances called for altering the group of people
who had the possibility of becoming members of his inner circle---that is, the
group whose support guaranteed his continued dignity as king. He moved quickly to
expand the opportunities (and for a few, the actual power) of new aristocrats,
called the noblesse de robe. Together with his chancellor, Michel Le Tellier, he
acted to create a professional, relatively meretricious army. In a radical
departure from the practice observed by just about all of his neighboring
monarchs, Louis opened the doors toofficer ranks---even at the highest
levels---to make room for many more than the traditional old-guard military
aristocrats, the noblesse d'\'{e}p\'{e}e. In so doing, Louis was converting his
army into a more accessible, politically and militarily competitive
organization.}

{\large Meanwhile, Louis had to do something about the old aristocracy. He was
deeply aware of their earlier disloyalty as instigators and backers of the
antimonarchy Fronde (a mix of revolution and civil war) at the time of his
regency. To neutralize the old aristocracy's potential threat, he attached
them---literally---to his court, compelling them to be physically present in
Versailles much of the time. This meant that their prospects of income from the
crown depended on how well favored they were by the king. That, of course,
depended on how well they served him.}

{\large By elevating so many newcomers, Louis had created a new class of people
who were beholden to him. In the process, he was centralizing his own authority
more fully and enhancing his ability to enforce his views at the cost of many of
the court's old aristocrats. Thus he erected a system of ``absolute'' control
whose success depended on the loyalty of the military, the new aristocrats, and
on tying the hands of the old aristocrats so that their welfare translated
directly into his welfare.}

{\large The French populace in general did not figure much into Louis's
calculations of who needed to be paid off---they did not represent an imminent
threat to him. Even so, it's clear that his absolutism was not absolute at all.
He needed supporters and he understood how to maintain their loyalty. They would
be loyal to him only so long as being so was more profitable for them than
supporting someone else.}

{\large Louis's strategy was to replace the ``winning coalition'' of essential
supporters that he inherited with people he could more readily count on. In place
of the old guard he brought up and into the inner circle members of the noblesse
de robe and even, in the bureaucracy and especially in the military, some
commoners. By expanding the pool of people who could be in the inner circle, he
made political survival for those already in that role more competitive. Those
who were privileged to be in his winning coalition knew that under the enlarged
pool of candidates for such positions, any one of them could easily be replaced
if they did not prove sufficiently trustworthy and loyal to the king. That, in
turn, meant they could lose their opportunity for wealth, power, and privilege.
Few were foolish enough totake such a risk.}

{\large Like all leaders, Louis forged a symbiotic relationship with his inner
circle. He could not hope to thrive in power without their help, and they could
not hope to reap the benefits of their positions without remaining loyal to him.
Loyal they were. Louis XIV survived in office for seventy-two years until he died
quietly of old age in 1715.}

{\large Louis XIV's experience exemplifies the most fundamental fact of
political life. No one rules alone; no one has absolute authority. All that
varies is how many backs have to be scratched and how big the supply of backs
available for scratching.}

\subsection{Three Political Dimensions}

{\large For leaders, the political landscape can be broken down into three
groups of people: the nominal selectorate, the real selectorate, and the winning
coalition.}

{\large The nominal selectorate includes every person who has at least some
legal say in choosing their leader. In the United States it is everyone eligible
to vote, meaning all citizens aged eighteen and over. Of course, as every citizen
of the United States must realize, the right to vote is important, but at the end
of the day no individual voter has a lot of say over who leads the country.
Members of the nominal selectorate in a universalfranchise democracy have a toe
in the political door, but not much more. In that way, the nominal selectorate in
the United States or Britain or France doesn't have much more power than its
counterparts, the ``voters,'' in the old Soviet Union. There, too, all adult
citizens had the right to vote, although their choice was generally to say Yes or
No to the candidates chosen by the Communist Party rather than to pick among
candidates. Still, every adult citizen of the Soviet Union, where voting was
mandatory, was a member of the nominal selectorate. The second stratum of
politics consists of the real selectorate. This is the group that actually
chooses the leader. In today's China (as in the old Soviet Union), it consists of
all voting members of the Communist Party; in Saudi Arabia's monarchy it is the
senior members of the royal family; in Great Britain, the voters backing members
of parliament from the majority party. The most important of these groups is the
third, the subset of the real selectorate that makes up a winning coalition.
These are the people whose support is essential if a leader is to survive in
office. In the USSR the winning coalition consisted of a small group of people
inside the Communist Party who chose candidates and who controlled policy. Their
support was essential to keep the commissars and general secretary in power.
These were the folks with the power to overthrow their boss---and he knew it. In
the United States thewinning coalition is vastly larger. It consists of the
minimal number of voters who give the edge to one presidential candidate (or, at
the legislative level in each state or district, to a member of the House or
Senate) over another. For Louis XIV, the winning coalition was a handful of
members of the court, military officers, and senior civil servants without whom a
rival could have replaced the king.}

{\large Fundamentally, the nominal selectorate is the pool of potential support
for a leader; the real selectorate includes those whose support is truly
influential; and the winning coalition extends only to those essential supporters
without whom the leader would be finished. A simple way to think of these groups
is: interchangeables, influentials, and essentials.}

{\large In the United States, the voters are the nominal selectorate
---interchangeables . As for the real selectorate---influentials---the electors
of the electoral college really choose the president (just like the party
faithful picked their general secretary back in the USSR), but the electors
nowadays are normatively bound to vote the way their state's voters voted, so
they don't really have much independent clout in practice.}

{\large In the United States, the nominal selectorate and real selectorate are
therefore pretty closely aligned. This is why, even though you're only one among
many voters, interchangeable with others, you still feel like your vote is
influential---that it counts and is counted. The winning coalition
---essentials---in the United States is the smallest bunch of voters, properly
distributed among the states, whose support for a candidate translates into a
presidential win in the electoral college. And while the winning coalition
(essentials) is a pretty big fraction of the nominal selectorate
(interchangeables), it doesn't have to be even close to a majority of the US
population. In fact, given the federal structure of American elections, it's
possible to control the executive and legislative branches of government with as
little as about one fifth of the vote, if the votes are really efficiently
placed. (Abraham Lincoln was a master at just such voter efficiency.) It is worth
observing that the United States has one of the world's biggest winning
coalitions both in absolute numbers and as a proportion of the electorate. But it
is not the biggest. Britain's parliamentary structure requires the prime minister
to have the support of a little over 25 percent of the electorate in two-party
elections to parliament. That is, the prime minister generally needs at least
half the members of parliament to be fromher party and for each of them to win
half the vote (plus one) in each twoparty parliamentary race: half of half of the
voters, or one quarter in total.}

{\large France's runoff system is even more demanding. Election requires that a
candidate win a majority in the final, two-candidate runoff.}

{\large Looking elsewhere we see that there can be a vast range in the size of
the nominal selectorate, the real selectorate, and the winning coalition.}

{\large Some places, like North Korea, have a mass nominal selectorate in which
everyone gets to vote---it's a joke, of course---a tiny real selectorate who
actually pick their leader, and a winning coalition that surely is no more than
maybe a couple of hundred people (if that) and without whom even North Korea's
first leader, Kim Il Sung, could have been reduced to ashes.}

{\large Other nations, like Saudi Arabia, have a tiny nominal and real
selectorate, made up of the royal family and a few crucial merchants and
religious leaders. The Saudi winning coalition is perhaps even smaller than North
Korea's.}

{\large How does Bell, California, measure up? We saw that in 2009, the
interchangeables in Bell consisted of 9,395 registered voters; the influentials,
the 2,235 who actually voted; and the essentials, not more than the 473 voters
whose support was essential to win a seat on the city council. Bell definitely
looks better than North Korea or Saudi Arabia--- we'd hope so. It looks
alarmingly close, however, to the setup of a regime with mostly phony elections,
such as prerevolutionary Egypt, Venezuela, Cambodia, and maybe Russia. Most
publicly traded corporations have this structure as well. They have millions of
shareholders who are the interchangeables. They have big institutional
shareholders and some others who are the influentials. And the essentials are
pretty much those who get to pick actual board members and senior management.
Bell doesn't look much like Madison's or Montesquieu's idealization of democracy
and neither do corporations, regardless of how many shareholders cast proxy
ballots.}

{\large Think about the company you work for. Who is your leader? Who are the
essentials whose support he or she must have? What individuals, though not
essential to your CEO's power, are nonetheless influential in the governance of
the company? And then, of course, who is there every day at the office---working
hard (or not), just hoping for the breakthrough or the break that will catapult
them into a bigger role?These three groups provide the foundation of all that's
to come in the rest of this book, and, more importantly, the foundation behind
the working of politics in all organizations, big and small. Variations in the
sizes of these three groups give politics a three-dimensional structure that
clarifies the complexity of political life. By working out how these dimensions
intersect---that is, each organization's mix in the size of its interchangeable,
influential, and essential groups---we can come to grips with the puzzles of
politics. Differences in the size of these groups across states, businesses, and
any other organization, as you will see, decide almost everything that happens in
politics---what leaders can do, what they can and can't get away with, to whom
they answer, and the relative qualities of life that everyone under them enjoys
(or, too often, doesn't enjoy).}

\subsection{Virtues of 3 - D Politics}

{\large You may find it hard to believe that just these three dimensions govern
all of the varied systems of leadership in the world. After all, our experience
tends to confirm that on one end of the political spectrum we have autocrats and
tyrants---horrible, selfish thugs who occasionally stray into psychopathology. On
the other end, we have democrats---elected representatives, presidents, and prime
ministers who are the benevolent guardians of freedom. Leaders from these two
worlds, we assure ourselves, must be worlds apart! It's a convenient fiction, but
a fiction nonetheless. Governments do not differ in kind. They differ along the
dimensions of their selectorates and winning coalitions. These dimensions limit
or liberate what leaders can and should do to keep their jobs. How limited or
liberated a leader is depends on how selectorates and winning coalitions
interact.}

{\large No question, it is tough to break the habit of talking about democracies
and dictatorships as if either of these terms is sufficient to convey the
differences across regimes, even though no two ``democracies'' are alike and
neither are any two ``dictatorships.'' In fact, it is so hard to break that habit
that we will continue to use these terms much of the time throughout this
book---but it is important to emphasize that the term ``dictatorship'' really
means a government based on a particularly small number of essentials drawn from
a very large group of interchangeables and, usually, a relatively small batch of
influentials. On the other hand, if we talk about democracy, we really mean a
government founded on a very large number of essentials and a very large number
of interchangeables, with the influential group being almost as big as the
interchangeable group. When we mention monarchy or military junta, we have in
mind that the number of interchangeables, influentials, and essentials is small.}

{\large The beauty of talking about organizations in terms of essentials,
influentials, and interchangeables is that these categories permit us torefrain
from arbitrarily drawing a line between forms of governance, pronouncing one
``democratic'' and another ``autocratic,'' or one a large republic and another
small, or any of the other mostly one-dimensional views of politics expressed by
some of history's leading political philosophers.}

{\large The truth is, no two governments or organizations are exactly alike. No
two democracies are alike. Indeed, they can be radically different one from the
other and still qualify perfectly well as democracies. The more significant and
observable differences in the behavior of governments and organizations are
dependent on the absolute and relative size of the interchangeable, influential,
and essential groups. The seemingly subtle differences between, say, France's
government and Britain's, or Canada's and the United States's are not
inconsequential. However, the variations in their policies are the product of the
incentives leaders face as they contend with their particular mix of
interchangeable, influential, and essential groups.}

{\large There is incredible variety among political systems, mainly because
people are amazingly inventive in manipulating politics to work to their
advantage. Leaders make rules to give all citizens the vote---creating lots of
new interchangeables---but then impose electoral boundaries, stacking the deck of
essential voters to ensure that their preferred candidates win.}

{\large Democratic elites may decide to require a plurality to win a particular
race, giving themselves a way to impose what a majority may otherwise reject.}

{\large Or they might favor having runoff elections to create a majority, even
though it may end up being a majority of the interchangables' second-place
choices. Alternately democratic leaders might represent political views in
proportion to how many votes each view got, forging governments out of coalitions
of minorities. Each of these and countless other rules easily can fall within our
belief in democracy, yet each can---and does---produce radically different
results.}

{\large We must remember that labels like democracy or dictatorship are a
convenience---but only a convenience.}

\subsection{Change the Size of Dimensions and Change the World}

{\large 1 Changing the relative size of interchangeables, influentials, and
essentials can make a real difference in basic political outcomes. As an example,
we can look to the seemingly prosaic election of members of San Francisco's board
of supervisors.}

{\large San Francisco used to elect its board of supervisors in citywide
elections. That meant that the selectorate consisted of the city's voters, and
the essentials were the minimum number needed to elect a member to the board. In
1977 the method changed, and at-large, citywide elections were replaced by
district voting. Under the old rules, members of the board of supervisors were
elected by and represented the whole city as if it were one large constituency.
Under the new rules, they were elected by and represented their district; that
is, their neighborhood, so each supervisor was chosen by a much smaller
constituency. The policy and candidate preferences of San Francisco residents as
a whole were little different between 1975 and 1977---nevertheless in 1975 a
candidate named Harvey Milk failed in his bid to be elected to the board, but
went on to be elected in 1977 (and tragically assassinated not long after). As
Time magazine reported later, Harvey Milk was ``the first openly gay man elected
to any substantial political office in the history of the planet.''2 What changed
in Harvey Milk's favor between 1975 and 1977 was simple enough. In 1975, he
needed broad-based support among San Francisco's influentials to get elected. He
got 52,996 votes. This meant he finished seventh in the election of supervisors,
with the top five being elected. Milk did not have enough support, and so he
lost. In 1977 he only needed support within the neighborhood from which he ran,
the Castro, a dominantly gay area. He was, as he well knew, popular within his
district.}

{\large He received 5,925 votes, giving him a plurality of support with
29.42percent of the vote in district 5, which placed him first in the 5th
Supervisory District contest and so he was elected.}

{\large Strange as it may seem, the same ideas and subtle differences that held
true in San Francisco can be applied to illiberal governments like Zimbabwe,
China, and Cuba, and even to the more ambiguous sorts of governments like
current-day Russia or Venezuela or Singapore. Each is easily and uniquely placed
on the three organizational dimensions: interchangeables, influentials, and
essentials.}

{\large Once we learn to think along these three dimensions, we can begin to
unravel some of politics' most enduring puzzles. Our starting point is the
realization that any leader worth her salt wants as much power as she can get,
and to keep it for as long as possible. Managing the interchangeables,
influentials, and essentials to that end is the act, art, and science of
governing.}

\subsection{Rules Ruling Rulers}

{\large Money, it is said, is the root of all evil. That can be true, but in
some cases, money can serve as the root of all that is good about governance. It
depends on what leaders do with the money they generate. They may use it to
benefit everyone, as is largely true for expenditures directed toward protecting
the personal well-being of all citizens and their property. Much public policy
can be thought of as an effort to invest in the welfare of the people. But
government revenue can also be spent on buying the loyalty of a few key cronies
at the expense of general welfare. It can also be used to promote corruption,
black marketeering, and a host of even less pleasant policies.}

{\large The first step in understanding how politics really works is to ask what
kinds of policies leaders spend money on. Do they spend it on public goods that
benefit everyone? Or do they spend mostly on private goods that benefit only a
few? The answer, for any savvy politician, depends on how many people the leader
needs to keep loyal---that is, the number of essentials in the coalition.}

{\large In a democracy, or any other system where a leader's critical coalition
is excessively large, it becomes too costly to buy loyalty through private
rewards. The money has to be spread too thinly. So more democratic types of
governments, dependent as they are on large coalitions, tend to emphasize
spending to create effective public policies that improve general welfare pretty
much as suggested by James Madison.}

{\large By contrast, dictators, monarchs, military junta leaders, and most CEOs
all rely on a smaller set of essentials. As intimated by Machiavelli, it is more
efficient for them to govern by spending a chunk of revenue to buy the loyalty of
their coalition through private benefits, even though these benefits come at the
expense of the larger taxpaying public or millions of small shareholders. Thus
small coalitions encourage stable, corrupt, privategoods-oriented regimes. The
choice between enhancing social welfare orenriching a privileged few is not a
question of how benevolent a leader is.}

{\large Honorable motives might seem important, but they are overwhelmed by the
need to keep supporters happy, and the means of keeping them happy depends on how
many need rewarding.}

\subsection{Taxing}

{\large To keep backers happy a leader needs money. Anyone aspiring to rule must
first ask how much can he extract from his constituents---whether they are
citizens of a nation or shareholders in a corporation. This extraction can take
many forms---personal income taxes, property taxes, duties on imports, licenses,
and government fees---but we will refer to it generically as taxation to keep the
discussion from wandering too far afield. As we've already seen, those who rule
based on a large coalition cannot efficiently sustain themselves in power by
focusing on private benefits. Their bloc of essential supporters is too large for
that. Since they must sustain themselves by emphasizing public goods more than
private rewards, they must also keep tax rates low, relatively speaking. People
prefer to keep their money for themselves, except when that money can be pooled
to provide something they value that they cannot afford to buy on their own.}

{\large For example, we all want to be sure that a reliable fire department will
put out a fire that threatens our home. We could conceivably hire a personal
firefighter to protect our house alone. However, not only is that expensive, we
would also have to worry about whether our neighbor's house is itself well enough
protected that it won't catch fire and threaten our home. Furthermore, our
neighbor, realizing that we won't want his house to burn if in doing so it
threatens ours, may attempt to free ride on the fact that we hired a personal
firefighter who will have to step in to protect the neighbor's house as well. In
no time we are in the position of paying for neighborhoodwide fire protection
single-handedly, a very costly proposition. The easiest way to get neighbors to
share the burden of fire protection is to let government leaders take the
responsibility for fire protection. To provide such protection we happily pay
taxes.}

{\large Though we may willingly pay taxes for programs that provide tangible
benefits to us, for instance protection from fire, felons, and foreign foes, we
would not be so willing to see our tax money used to pay a tremendoussalary to
our president or prime minister---or, in the case of Bell, California, to our
local government officials. As a result, heads of governments reliant on a large
coalition tend not to be among the world's best paid executives.}

{\large Because the acceptable uses of taxation in a regime that depends on a
large coalition are few---just those expenditures thought to buy more welfare
than people can buy on their own---taxes tend to be low when coalitions are
large. But when the coalition of essential backers is small and private goods are
an efficient way to stay in power, then the well-being of the broader population
falls by the wayside, contrary to the view expressed by Hobbes. In this setting
leaders want to tax heavily, redistributing wealth by taking as much as they can
from the poor interchangeables and the disenfranchised, giving that wealth in
turn to the members of the winning coalition, making them fat, rich, and loyal.
For example, a married couple in the United States pays no income tax on the
first \$17,000 they earn. At that same income, a Chinese couple's marginal tax
rate is 45 percent. That is well above the highest personal income tax rate in
the United States and so no one, no matter how high their income, pays that much
to the US federal government. And then there are small coalition regimes like
Bell, California. Chief Administrative Officer Rizzo's small number of supporters
did not complain about the excessively high level of property taxes. They had to
pay these taxes, but then so did thousands of others. And unlike others they
received the rewards financed by those same taxes. The private gains the few
crucial cronies got from their city government more than repaid the high taxes
everyone had to pay.}

{\large Obviously, self-interest plays a large role in these equations. We must
wonder, therefore, why incumbents don't take all the revenue they've raised and
sock it away in their personal bank accounts. This question is especially
pertinent for corporate executives. Once investors have entrusted money in the
hands of a CEO or chairman of the board, what can the investors do to assure
themselves that the money will be invested wisely to produce benefits for them?
Investors want increased value. They want share prices to rise, their portion of
ownership to go up, and dividend payments to be large and predictably regular. To
be sure, focusing on selfinterest tells us that rulers and business leaders, and
in fact, all of us, would love to take other people's money and keep it for
ourselves. This means that the next step in explaining the calculus of politics
is to figure out howmuch a leader can keep and how much must be spent on the
coalition and on the public if the incumbent is to stay in power.}

\subsection{Shuffling the Essential Deck}

{\large Staying in power, as we now know, requires the support of others. This
support is only forthcoming if a leader provides his essentials with more
benefits than they might expect to receive under alternative leadership or
government. When essential followers expect to be better off under the wing of
some political challenger, they desert.}

{\large Incumbents have a tough job. They need to offer their supporters more
than any rival can. While this can be difficult, the logic of politics tells us
that incumbents have a huge advantage over rivals, especially when office holders
rely on relatively few people and when the pool of replacements for coalition
members is large. Lenin designed precisely such a political system in Russia
after the revolution. This explains why, from the October 1917 Revolution through
to Gorbachev's reforms in the late 1980s, only one Soviet leader, Nikita
Khrushchev, was successfully deposed in a coup.}

{\large All the other Soviet leaders died of old age or infirmity. Khrushchev
failed to deliver what he promised to his cronies. It is the successful, reliable
implementation of political promises to those who count that provides the basis
for any incumbent's advantage.}

{\large The story of survival is not much different, although the particulars
are very different, in political settings that rely on many essential backers. As
even a casual observer of election campaigns knows, there is a big discrepancy
between what politicians promise when making a bid for power and what they
actually deliver once there. Once in power, a new leader might well discard those
who helped her get to the top, replacing them with others whom she deems more
loyal.}

{\large Not only that, but essential supporters can't just compare what the
challenger and incumbent offer today. The incumbent might pay less now, for
instance, but the pay is expected to continue for those kept on or brought into
the new incumbent's inner circle. True, the challenger may offer more today, but
his promises of future rewards may be nothing morethan political promises without
any real substance behind them. Essentials must compare the benefits expected to
come their way in the future because that future flow adds up in time to bigger
rewards. Placing a supporter in his coalition after a new leader is ensconced as
the new incumbent is a good indicator that he will continue to rely on and reward
that supporter, exactly because the new incumbent has made a concerted effort to
sort out those most likely to remain loyal from those opportunists who might
bring the leader down in the future. The challenger might make such a promise to
keep backers on if she reaches the heights of power, but it is a political
promise that might very well not be honored in the long run.}

{\large Lest there be doubt that those who share the risks of coming to power
often are then thrown aside---or worse---let us reflect on the all-too-typical
case of the backers of Fidel Castro's revolution in Cuba. Of the twenty-one
ministers appointed by Castro in January 1959, immediately after the success of
his revolution, twelve had resigned or had been ousted by the end of the year.
Four more were removed in 1960 as Castro further consolidated his hold on power.
These people, once among Fidel's closest, most intimate backers, ultimately faced
the two big exes of politics. For the luckier among them, divorce from Castro
came in the form of exile. For others, it meant execution. This includes even
Castro's most famous fellow revolutionary, Che Guevara.}

{\large Che may have been second in power only to Fidel himself. Indeed, that
was likely his greatest fault. Castro forced Che out of Cuba in 1965 partly
because of Che's popularity, which made him a potential rival for authority.}

{\large Castro sent Che on a mission to Bolivia, but towards the end of March
1967 Castro simply cut off Guevara's support, leaving him stranded.}

{\large Captain Gary Prado Salmon, the Bolivian officer who captured Che,
confirmed that Guevara told him that the decision to come to Bolivia was not his
own, it was Castro's. One of Fidel's biographers remarked, In a very real sense
Che followed in the shadows of Frank Pais, Camilo Cienfuegos, Huber Matos, and
Humberto Sori Marin [all close backers of Castro during the revolution]. Like
them, he was viewed by Castro as a `competitor' for power and like them, he had
to be moved aside `in one manner or another.' Che Guevara was killed in
Boliviabut at least he escaped the ignominy of execution by his revolutionary
ally, Fidel Castro. Humberto Sori Marin was not so `fortunate.' Marin, the
commander of Castro's rebel army, was accused of conspiring against the
revolution. In April 1961, like so many other erstwhile backers of Fidel Castro,
he too was executed.3 Political transitions are filled with examples of
supporters who help a leader to power only to be replaced. This is true whether
we look at national or local governments, corporations, organized crime families,
or, for that matter, any other organization. Each member of a winning coalition,
knowing that many are standing on the sidelines to replace them, will be careful
not to give the incumbent reasons to look for replacements.}

{\large This was the relationship Louis XIV managed so well. If a small bloc of
backers is needed and it can be drawn from a large pool of potential supporters
(as in the small coalition needed in places like Zimbabwe, North Korea, or
Afghanistan), then the incumbent doesn't need to spend a huge proportion of the
regime's revenue to buy the coalition's loyalty. On the other hand, more must be
spent to keep the coalition loyal if there are relatively few people who could
replace its members. That is true in two circumstances: when the coalition and
selectorate are both small (as in a monarchy or military junta), or the coalition
and selectorate are both large (as in a democracy). In these circumstances, the
incumbent's ability to replace coalition members is pretty constrained.
Essentials can thereby drive up the price for keeping them loyal. The upshot is
that there is less revenue available to be spent at the incumbent's discretion
because more has to be spent to keep the coalition loyal, fending off credible
counteroffers by political foes.}

{\large When the ratio of essentials to interchangeables is small (as in
riggedelection autocracies and most publicly traded corporations), coalition
loyalty is purchased cheaply and incumbents have massive discretion.}

{\large They can choose to spend the money they control on themselves or on pet
public projects. Kleptocrats, of course, sock the money away in secret bank
accounts or in offshore investments to serve as a rainy-day fund in the event
that they are overthrown. A few civic-minded autocrats slip a little into secret
accounts, preferring to fend off the threat of revolt by using their
discretionary funds (the leftover tax revenue not spent on buying
coalitionloyalty) to invest in public works. Those public works may prove
successful, as was true for Lee Kwan Yew's efforts in Singapore and Deng
Xiaoping's in China. They may also prove to be dismal failures, as was true for
Kwame Nkrumah's civic-minded industrial program in Ghana or Mao Zedong's Great
Leap Forward, which turned out to be a great leap backwards for China.}

{\large We have seen how the desire to survive in office shapes some key revenue
generation decisions, key allocation decisions, and the pot of money at the
incumbent's discretion. Whether the tax rate is high or low, whether money is
spent more on public or private rewards, and how much is spent in whatever way
the incumbent wants dictates political success within the confines of the
governance structure the leader inherits or creates. And our notion of governing
for political survival tells us that there are five basic rules leaders can use
to succeed in any system: Rule 1: Keep your winning coalition as small as
possible. A small coalition allows a leader to rely on very few people to stay in
power. Fewer essentials equals more control and contributes to more discretion
over expenditures.}

{\large Bravo for Kim Jong Il of North Korea. He is a contemporary master at
ensuring dependence on a small coalition.}

{\large Rule 2: Keep your nominal selectorate as large as possible. Maintain a
large selectorate of interchangeables and you can easily replace any
troublemakers in your coalition, influentials and essentials alike. After all, a
large selectorate permits a big supply of substitute supporters to put the
essentials on notice that they should be loyal and well behaved or else face
being replaced.}

{\large Bravo to Vladimir Ilyich Lenin for introducing universal adult suffrage
in Russia's old rigged election system. Lenin mastered the art of creating a vast
supply of interchangeables.}

{\large Rule 3: Control the flow of revenue. It's always better for a ruler to
determine who eats than it is to have a larger pie from which the people can feed
themselves. The most effective cash flow for leaders is one that makes lots of
people poor and redistributes money to keep select people---their
supporters---wealthy.}

{\large Bravo to Pakistan's president Asif Ali Zardari, estimated to be worth up
to \$4 billion even as he governs a country near the world's bottom in per capita
income.}

{\large Rule 4: Pay your key supporters just enough to keep them loyal.}

{\large Remember, your backers would rather be you than be dependent on you.}

{\large Your big advantage over them is that you know where the money is and
they don't. Give your coalition just enough so that they don't shop around for
someone to replace you and not a penny more.}

{\large Bravo to Zimbabwe's Robert Mugabe who, whenever facing a threat of a
military coup, manages finally to pay his army, keeping their loyalty against all
odds.}

{\large Rule 5: Don't take money out of your supporter's pockets to make the
people's lives better. The flip side of rule 4 is not to be too cheap toward your
coalition of supporters. If you're good to the people at the expense of your
coalition, it won't be long until your ``friends'' will be gunning for you.}

{\large Effective policy for the masses doesn't necessarily produce loyalty
among essentials, and it's darn expensive to boot. Hungry people are not likely
to have the energy to overthrow you, so don't worry about them. Disappointed
coalition members, in contrast, can defect, leaving you in deep trouble.}

{\large Bravo to Senior General Than Shwe of Myanmar, who made sure following
the 2008 Nargis cyclone that food relief was controlled and sold on the black
market by his military supporters rather than letting aid go to the people---at
least 138,000 and maybe as many as 500,000 of whom died in the disaster.}

\subsection{Do the Rules Work in Democracies?}

{\large At this point, you may be saying, Hold on! If an elected leader followed
these rules she'd be out of the job in no time flat. You're right---almost.}

{\large As we'll see throughout the chapters to follow, a democratic leader does
indeed have a tougher time maintaining her position while looting her country and
siphoning off funds. She's constrained by the laws of the land, which also
determine---through election procedures---the size of the coalition that she
needs in order to come to power. The coalition has to be relatively large and she
has to be responsive to it, so she does have a problem with Rule 1. But that
doesn't mean she doesn't try to follow Rule 1 as closely as she can (and all of
the other rules too).}

{\large Why, for example, does Congress gerrymander districts? Precisely because
of Rule 1: Keep the coalition as small as possible.}

{\large Why do some political parties favor immigration? Rule 2: Expand the set
of interchangeables.}

{\large Why are there so many battles over the tax code? Rule 3: Take control of
the sources of revenue.}

{\large Why do Democrats spend so much of that tax money on welfare and social
programs? Or why on earth do we have earmarks? Rule 4: Reward your essentials at
all costs.}

{\large Why do Republicans wish the top tax rate were lower, and have so many
problems with the idea of national health care? Rule 5: Don't rob your supporters
to give to your opposition.}

{\large Just like autocrats and tyrants, leaders of democratic nations follow
these rules because they, like every other leader, want to get power and keep it.
Even democrats almost never step down unless they're forced to.5 The problem for
democrats is that they face different constraints and have to be a little more
creative than their autocratic counterparts. And they succeed less often. Even
though they generally provide a much higher standard of living for their citizens
than do tyrants, democrats generallyhave shorter terms in office.}

{\large Political distinctions are truly continuous across the intersection of
the three dimensions that govern how organizations work. Some ``kings'' in
history have actually been elected. Some ``democrats'' rule their nations with
the authority of a despot. In other words, the distinction between autocrats and
democrats isn't cut and dried.}

{\large Having laid the foundation for our new theory of politics and having
revealed the five rules of leadership, we'll turn to the big questions at the
heart of the book, often using the terms autocrats and democrats throughout, to
show how the games of leadership change as you slide from one extreme to the
other on the spectrum of small and large coalitions. But just remember, there's
always a little mix of both worlds regardless of the country or organization in
question. The lessons from both extremes apply ---whether you're talking about
Saddam Hussein or George Washington.}

{\large After all, the old saw still holds true---politicians are all the same.}
\pagebreak{}


\section{Chapter 2 - Coming to Power}

{\large FOR CENTURIES, ``JOHN DOE'' HAS SERVED AS THE placeholder name assigned
to unidentified nobodies. And while his first name may have been Samuel, not
John, in every other respect Liberia's Sergeant Doe was just such a nobody until
April 12, 1980. Born in a remote part of Liberia's interior and virtually
illiterate, he, like hundreds of thousands of others in his predicament, moved
out of the West African jungle in search of work. He headed to the capital city,
Monrovia, where he found that the army held great opportunities even for men,
like him, who had no skills.}

{\large One of these opportunities presented itself when Doe found himself in
President William Tolbert's bedroom on April 12. As the president slept, he
seized the day, bayoneted the president, threw his entrails to the dogs, and
declared himself Liberia's new president.1 Thus did he rise from obscurity to
claim the highest office in his land.}

{\large Together with sixteen other noncommissioned officers, Doe had scaled the
fence at the Executive Mansion, hoping to confront the president and find out why
they had not been paid. Seeing the opportunity before him, he ended the dominance
of Tolbert's True Whig Party, a political regime created by slaves repatriated
from America in 1847. He immediately rounded up thirteen cabinet ministers, who
were then publically executed on the beach in front of cheering crowds. Many more
deaths would follow.}

{\large Doe then headed the People's Redemption Council that suspended the
constitution and banned all political activity.}

{\large Doe had no idea what a president was supposed to do and even less idea
of how to govern a country. What he did know was how to seize power and keep it:
remove the previous ruler; find the money; form a small coalition; and pay them
just enough to keep them loyal. In short order, he proceeded to replace virtually
everyone who had been in the government or the army with members of his own small
Krahn tribe, which made uponly about 4 percent of the population. He increased
the pay of army privates from \$85 to \$250 per month. He purged everyone he did
not trust.}

{\large Following secret trials, he had no fewer than fifty of his original
collaborators executed.}

{\large Doe funded his government, as his predecessors had, with revenues from
Firestone, which leased large tracts of land for rubber; from the Liberian Iron
Mining Company, which exported iron ore; and by registering more than 2,500
ocean-going ships without requiring safety inspections.}

{\large Further, he received direct financial backing from the United States
government. The United States gave Doe's government \$500 million over ten years.
In exchange the United States received basing rights and made Liberia a center
for US intelligence and propaganda. It is believed that Doe and his cronies
personally amassed \$300 million.}

{\large As for Doe's policies, they couldn't be called successful. Indeed he
produced virtually no policies at all. He was lazy, and spent his days hanging
out with the wives of his presidential guards. The economy collapsed, foreign
debt soared, and criminal enterprises became virtually the only successful
businesses in Liberia. Monrovian banks became money-laundering operations. Little
wonder that the people of Liberia ended up hating Doe. And yet, provided he knew
where the money was and who needed paying off, he managed to survive in power.}

{\large Damn the idea of good governance and don't elevate the concerns of the
people over your own and those of your supporters: That's a good mantra for
would-be dictators. In such a way any John Doe---even a Samuel Doe---can seize
power, and even keep it.}

\subsection{Paths to Power with Few Essentials}

{\large To come to power a challenger need only do three things. First, he must
remove the incumbent. Second, he needs to seize the apparatus of government.
Third, he needs to form a coalition of supporters sufficient to sustain him as
the new incumbent. Each of these actions involves its own unique challenges. The
relative ease with which they can be accomplished differs between democracies and
autocracies.}

{\large There are three ways to remove an incumbent leader. The first, and
easiest, is for the leader to die. If that convenience does not offer itself, a
challenger can make an offer to the essential members of the incumbent's
coalition that is sufficiently attractive that they defect to the challenger's
cause. Third, the current political system can be overwhelmed from the outside,
whether by military defeat by a foreign power, or through revolution and
rebellion, in which the masses rise up, depose the current leader, and destroy
existing institutions.}

{\large While rebellion requires skill and coordination, its success ultimately
depends heavily upon coalition loyalty, or more precisely, the absence of loyalty
to the old regime. Hosni Mubarak's defeat by a mass uprising in Egypt is a case
in point. The most critical factor behind Mubarak's defeat in February 2011 was
the decision by Egypt's top generals to allow demonstrators to take to the
streets without fear of military suppression.}

{\large And why was that the case? As explained in a talk given on May 5, 2010,
based on the logic set out here, cuts in US foreign aid to Egypt combined with
serious economic constraints that produced high unemployment, meant that
Mubarak's coalition was likely to be underpaid and the people were likely to
believe the risks and costs of rebellion were smaller than normal.2 That is, the
general rule of thumb for rebellion is that revolutions occur when those who
preserve the current system are sufficiently dissatisfied with their rewards that
they are willing to look for someone new to take care of them. On the other hand,
revolts are defeated throughsuppression of the people---always an unpleasant
task---so coalition members need to receive enough benefits from their leader
that they are willing to do horribly distasteful things to ensure that the
existing system is maintained. If they do not get enough goodies under the
current system, then they will not stop the people from rising up against the
regime.}

\subsection{Speed Is Essential}

{\large Once the old leader is gone, it is essential to seize the instruments of
power, such as the treasury, as quickly as possible. This is particularly
important in small coalition systems. Anyone who waits will be a loser in the
competition for power.}

{\large Speed is of the essence. The coalition size in most political systems is
much smaller than a majority of the selectorate. Furthermore, even though we tend
to think that if one leader has enough votes or supporters, then the other
potential candidate must be short, this is wrong. There can simultaneously be
many different groups trying to organize to overthrow a regime and each might
have sufficient numbers of lukewarm or doubledealing supporters who could aid
them in securing power---or just as easily aid someone else, if the price is
right. This is why it is absolutely essential to seize the reins of power quickly
to make sure that your group gets to control the instruments of the state, and
not someone else's.}

{\large Samuel Doe ruled because his group had the guns. He did not need half
the nation to support him. He needed just enough confederates so that he could
control the army and suppress the rest of the population. There were many other
coalitions that could have formed, but Doe grabbed hold of power first and
suppressed the rest. This is the essence of coming to power.}

{\large Consider a room filled with 100 people. Anyone could take complete
control if only she had five supporters with automatic weapons pointed at the
rest. She would remain in power so long as the five gunmen continue to back her.
But there need be nothing special about her or about the gunmen beyond the fact
that they grabbed the guns first. Had someone else secured the guns and given
them to five supporters of their own, then it would be someone else telling
everyone what to do.}

{\large Waiting is risky business. There is no prize for coming in second.}

\subsection{Pay to Play}

{\large Paying supporters, not good governance or representing the general will,
is the essence of ruling. Buying loyalty is particularly difficult when a leader
first comes to power. When deciding whether to support a new leader, prudent
backers must not only think about how much their leader gives them today. They
must also ponder what they can expect to receive in the future.}

{\large The supporting cast in any upstart's transitional coalition must
recognize that they might not be kept on for long. After Doe took over the
Liberian government, he greatly increased army salaries. This made it immediately
attractive for his fellow army buddies to back him. But they were mindful that
they might not be rewarded forever. Don't forget that fifty of his initial
backers ended up executed.}

{\large Allaying supporters' fears of being abandoned is a key element of coming
to power. Of course, supporters are not so na\"{\i}ve that they will be convinced
by political promises that their position in the coalition is secure.}

{\large But such political promises are much better than tipping your hand as to
your true plans. Once word gets out that supporters are going to be replaced,
they will turn on their patron. For instance, Ronald Reagan won the pro-choice
vote in the 1980 US presidential election over the pro-life incumbent, Jimmy
Carter. When Reagan's true abortion stance became apparent, the pro-choice voters
abandoned him in droves. Walter Mondale won the pro-choice vote in the 1984
presidential election despite Reagan's reelection in a landslide.}

{\large Leaders understand the conditions that can cost them their heads. That
is why they do their level best to pay essential cronies enough that these
partners really want to stay loyal. This makes it tough for someone new to come
to power. But sometimes circumstances conspire to open the door to a new ruler.}

\subsection{Mortality: The Best Opportunity for Power}

{\large Most unavoidably, and therefore first, on the list of risks of being
deposed is the simple, inescapable fact of mortality. Dead leaders cannot deliver
rewards to their coalition. Dying leaders face almost as grave a problem. If
essential backers know their leader is dying, then they also know that they need
someone new to assure the flow of revenue into their pockets. That's a good
reason to keep terminal illnesses secret since a terminal ailment is bound to
provoke an uprising, either within the ranks of the essential coalition or among
outsiders who see an opportunity to step in and take control of the palace.}

{\large Ayatollah Ruhollah Khomeini in Iran and Corazon Aquino in the
Philippines both chose the right time to seize power. Take the case of Ayatollah
Khomeini. He was one of the most senior Shia clerics in Iran and a vehement
opponent of Shah Mohammad Reza Pahlavi's secular regime.}

{\large During early 1960 he spoke out against the regime, and organized
protests. His activities resulted in his being repeatedly arrested. In 1964, he
went into exile, first to Turkey, then Iraq, and eventually to France, continuing
to preach his opposition to the shah wherever he was. Tapes of his speeches were
popular throughout Iran.}

{\large In 1977, with the death of the shah's rival, Ali Shariati, Khomeini
became the most influential opposition leader. Although he urged others to oppose
the shah, he refused to return to Iran until the shah was gone. Except for a
privileged few, almost everyone in Iran hungered for change. The shah's regime
and those associated with it were widely disliked. Seeing that there was a chance
for real change, people threw their support behind the one clearly viable
alternative: Khomeini. After the shah fled the country, an estimated 6 million
people turned out to cheer Khomeini's return. Judging from what he did next, they
may have cheered too soon.}

{\large Immediately after his return, Khomeini challenged the interim
government, which was headed by the shah's former prime minister. Muchof the army
defected and joined Khomeini, and when he ordered a jihad against soldiers
remaining loyal to the old regime resistance collapsed.}

{\large Then he ordered a referendum to be held in which the people would choose
between the old monarchy of the shah or an Islamic republic. With the endorsement
of 98 percent for the latter, he rewrote the constitution basing it on rule by
clerics. After some dubious electoral practices this constitution was approved
and he became the Supreme Leader with a Council of Guardians to veto non-Islamic
laws and candidates. The many secular and moderate religious groups who had taken
to the streets on his behalf, providing the critical support needed for his rise
to power, found they were left out, excluded from running the new regime.}

{\large Khomeini became leader because he provided a focal point for opposition
to the shah's regime, and because the army did not stop the people from rising up
against the monarchy. Once the shah was gone, Khomeini quickly asserted that it
was he, not an interim government or a council representing all interests, who
was in charge. Although the masses brought down the old regime in hopes of
obtaining a more democratic government, Khomeini ensured that real power was
retained by a small group of clerics. The parliament, while popularly elected,
could only contain politicians who would support and be supported by the Council
of Guardians.}

{\large There is nothing special or unique about Khomeini's success. That
millions wanted the shah's regime overturned is unsurprising. The shah ran a
brutal, oppressive government under which thousands disappeared.}

{\large Imprisonment, torture, and death were commonplace. But that was equally
true fourteen years earlier when Khomeini went into exile and the shah's
government seemed invulnerable. The key to Khomeini's success at the end of the
1970s was that the army refused to stop the unhappy millions from taking to the
streets. They had not allowed such protests before. What had changed? The army
was no longer willing to fight to preserve the regime because they knew that the
shah was dying. The NewYork Times3 published accounts of the farce of a sick
leader desperate to hide the progression of his cancer. A dead shah couldn't
guarantee rewards.}

{\large Neither could his successor. The incumbency advantage unraveled. Faced
with the unpleasant task of suppressing the people with only a modest prospect of
continuing to enjoy the lavish rewards of coalition membership,the army sat on
its hands, smoothing the way for revolution.}

{\large The story of the rise of democracy in the Philippines is not much
different. Benigno Aquino Jr. was an outstanding man. At the age of eighteen he
was awarded the Philippine Legion of Honor for his journalism during the Korean
War. He then negotiated the surrender of a rebel group.}

{\large He was mayor of Concepcion by age twenty-two, governor of Tarlac
Province at twenty-nine, and a senator by thirty-four. In a dangerous move, he
became an outspoken critic of President Ferdinand Marcos. In 1983, Benigno
returned from exile in the United States. On the flight back to Manila he warned
journalists that it might all be over in minutes. And it was.}

{\large He was immediately taken from the plane and assassinated on the tarmac.}

{\large He should have followed Khomeini's example and bided his time.}

{\large His wife, Corazon, did not have his political skills or experience, but
she had one critical advantage: She was alive! In late 1985 Ferdinand Marcos
announced snap elections a year earlier than scheduled. Corazon Aquino stepped in
as her late husband's surrogate and ran as the main opposition candidate. There
was widespread fraud at the elections on February 7, 1986, so it was of little
surprise when just over a week later the electoral commission declared Marcos the
winner. But Marcos's supporters swiftly deserted him. President Ronald Reagan
expressed concern about the electoral result. Cardinal Jamie Sin, leader of the
influential Philippines' Catholic Church, spoke out. At Corazon Aquino's urging,
the people protested. Key members of the army and other leading political figures
resigned from the government and joined the demonstrations. Without the army to
stop them, hundreds of thousands of people joined the protest, resulting in still
more military leaders defecting.}

{\large In an attempt to avoid bloodshed, Marcos and his family sought sanctuary
in the United States. They left the Philippines and settled in Hawaii but, as
insiders and many others knew, Marcos would not live long.}

{\large That, in fact, had been his problem all along. He was dying of lupus and
all his key backers knew it. He could not deliver goodies from beyond the grave
so his supporters sought to ingratiate themselves with someone who might benefit
them. Corazon Aquino had no experience in government. Yet she succeeded where her
more accomplished husband had failed. She challenged Marcos at a time when his
supporters knew his time was coming to an end. They were looking for a new
partner to defend inexchange for their rewards. Corazon Aquino was inaugurated as
president and voted Time magazine's Woman of the Year for 1986.}

{\large These are not isolated examples. Laurent Kabila, once maligned by Che
Guevara as lacking ``revolutionary seriousness'' and being ``too addicted to
alcohol and women,'' took on the mighty Mobuto Sese Seko of Zaire and won.4
Kabila lacked much in talent, but his timing was excellent. Mobuto was dying of
prostate cancer and everybody knew it. His military simply refused to fight back
as Kabila's insurgents captured more and more territory. Mobutu's erstwhile
backers knew that their own future would be brighter by abandoning their dying
patron, a sentiment captured in the clich\'{e}, ``the King is dead, long live the
King!'' Health concerns for North Korea's Kim Jong Il and Cuba's Fidel Castro
have engendered similarly intense political speculation. Both have attempted to
stave off defection by their essential coalition members by nominating heirs. Kim
Jong Il promoted his youngest son, Kim Jong Un, to a variety of posts, including
the rank of four-star general, even though his son has no military experience.
Fidel Castro likewise promoted his brother, Raul, to president when Fidel's
survival was in doubt following major surgery. By designating heirs who might
keep the existing winning coalition largely intact, these leaders sought to
prevent the incumbency advantage from disappearing as their ability to deliver on
political promises was brought into jeopardy.}

{\large Impending death often induces political death. The sad truth is that if
you want to come to power in an autocracy you are better off stealing medical
records than you are devising fixes for your nation's ills.}

\subsection{Inheritance and the Problem of Relatives}

{\large We don't mean to say that healthy leaders don't face hazards of their
own.}

{\large If an incumbent runs out of money he cannot continue to pay his
supporters.}

{\large Why might he run out of money? Because he has taxed so heavily and
stolen so much that the masses choose siestas over labor, stymieing the future
flow of revenue into the government's treasury. Worse, the masses could choose
revolution over siestas, emboldened by the realization that things will only get
worse if they do not act now to overthrow their masters.}

{\large Mismanagement of coalition dynamics and the incentives of revolutionary
entrepreneurs can create changes in institutions that topple the incumbent regime
and bring new leaders to power.}

{\large Normally one of the most difficult tasks a challenger faces is removing
the incumbent. But this is instantly achieved when a leader dies or, as in the
case of William Tolbert, is murdered. Once an incumbent is dead, there is still
the issue of fending off competitors for the dead leader's job.}

{\large Ambitious challengers still need to grab control of the state apparatus,
reward supporters, and eliminate rivals. To resolve this issue, the Ottomans, who
ruled what is today's Turkey from 1299 until 1923, eventually instituted the law
of fratricide.5 When the sultan died, the succession depended upon who could
capture control of the state and reward his coalition. In practice this meant
grabbing the treasury and paying off the army. Succession became a battle of
survival of the fittest to see which son would become the next sultan. Each of
the sultan's sons governed a province of his own. When the sultan died, the sons
raced back to the capital, Constantinople, in an attempt to seize the treasury
and pay the army for its loyalty. The result could often be civil war, as rival
brothers each used their provincial forces to achieve sole, total control of the
state. The sultan could have already shown favor to one son over others simply by
giving him a province to govern that was closer to the capital, thereby favoring
that son even fromthe grave.}

{\large Ottoman succession could be bloody. Unsuccessful brothers were typically
killed. Mehmet II (1429--1481) institutionalized this practice with the
fratricide law, under which all unsuccessful male heirs were strangled with a
silk cord. A century later, Mehmet III allegedly killed nineteen brothers, two
sons, and fifteen slaves who were pregnant by his own father, thereby eliminating
all present and future potential rivals. By the middle of the seventeenth century
this practice was replaced by the kinder, gentler practice of locking all male
relatives in the Fourth Court of the Topkapi Palace---quite literally the
original Golden Cage. With relatives like this, it is perhaps no wonder why
Shakespeare's Hamlet or Robert Graves's Claudius chose to feign madness.}

{\large The general dilemma of succession is hardly unique to the Ottomans.}

{\large England's King Richard the Lionheart died in 1199. Since Richard had no
direct heirs, at least three people had a strong claim to the English throne
following his death. Richard's father was the previous king, Henry II, meaning
that succession could be claimed by Henry II's wife, Eleanor of Aquitaine, then
nearing eighty years old; by Henry's eldest surviving son, John; or by Henry's
eldest---but deceased---son Geoffrey's eldest surviving male child (himself but
eleven years old), Arthur.}

{\large Eleanor was too pragmatic to put herself at risk for the crown,
especially given her advanced years. She understood the likely consequences for
her if she pushed her claim. Being the loving mother and grandmother that she
surely must have been, she stepped aside, leaving John and Arthur to fight it
out. Or, more precisely, she looked at who was likely to win and threw her
support in that direction, allowing herself to change directions as the winds of
fortune switched from time to time.}

{\large Would-be autocrats must be prepared to kill all comers---even members of
the immediate family. The Ottomans formalized this while the English merely
relied on the tradition of doing in their rivals. Murder seems to be a favored
solution under the extreme conditions of fear and insecurity that accompany
monarchic and autocratic successions. What did John do?}

{\large Even after assuming the crown he continued to fear Arthur's quest for
power, a quest that grew more intense as the boy aged into his teenage years.
Finally, in 1203, John had Arthur taken prisoner and murdered.}

{\large Some rumors suggested that he killed his nephew personally. With
Arthurout of the way, no one stood as a further threat to John's crown---not
until the nobles rose up against him, promulgating Magna Carta, twelve years
later.}

{\large Inheritance holds a number of advantages for leaders and their
supporters alike. Paying off the right people is the essence of good
government---and princes are well equipped to continue to reward supporters. They
know where the money is and who to pay off. Even so, why should the court be so
keen to go along with inheritance? After all, if the prince takes the top job,
then the other courtiers cannot be king (or dictator or president) in his place.
Supporting inheritance inevitably means giving up the chance to become king
yourself. Yet, that is just one side of the calculation. With so many people who
would like to be king, the chance of landing the top job is tiny. In reality,
supporters of the late king are often best off to elevate his son and hope that
he then dances with the one that brought him to the ball.}

{\large New leaders need supporters to stay in power, and with inheritance those
supporters are all already in place. The prince knows who they are and how to pay
them. Of course, as we saw with France's Louis XIV, the prince might radically
alter the coalition. But supporters of the old king correctly believe in the old
adage, Like father like son. It's not a bad gamble for them. Essential supporters
have a much greater chance of retaining their privileged position when power
passes within a family, from father to son, from king to prince, than when power
passes to an outsider. If you are a prince and you want to be king, then you
should do nothing to dissuade your father's supporters of their chances of being
important to you too. They will curry favor with you. You should let them. You
will need them to secure a smooth transition. If you want them gone (and you may
not), then banish them from court later. But the first time they need to know
your true feelings for them is when you banish them from court, well after your
investiture and not a minute before.}

{\large Naturally, if you're a young prince who hopes to be king, you'll have to
make sure to outlive your ``supporters'' first. History has shown that regents
are notoriously bad caregivers. Provided a regent is prepared to kill his charge,
being entrusted with the care of the would-be future king is a great way to
become king. England's King Richard III provides an example.}

{\large When Edward IV died in 1483, the crown fell to his twelve-year-old
son,Edward V. Richard III, King Edward IV's brother, was appointed Lord Protector
of the Commonwealth and charged with looking after the prince's interests. He was
supposed to manage the crown for a few years and then hand it over. Like many
leaders, however, Richard didn't relish the idea of giving up power.}

{\large As the trusted executor of his brother, King Edward IV's, wishes,
Richard was able to manipulate events to his own benefit. First he had
twelve-year-old Edward and his younger brother taken to the Tower of London.
Richard then had Parliament declare both princes illegitimate by questioning the
legitimacy of their parents' marriage. The princes were never seen again. Richard
may not have been much of an executor but he seems to have had no trouble with
execution. (It is believed that two skeletons found under a staircase in 1674
belonged to the two young boys.) Even in systems that rely on inheritance, the
door can nevertheless be opened for a designated successor who is not a blood
relative. Leaders often nominate their successor and sometimes choose from
outside of their immediate relations, perhaps because they understand the dire
risks to family if they turn to one member and not another. For instance, the
first Roman emperor, Augustus, formally adopted his successor, Tiberius. Mob
bosses often do the same. Carlo Gambino nominated ``Big'' Paul Castellano to
succeed him as head of his New York mafia family. In each case, the designated
successor was seen as someone likely to continue the programs and projects of the
prior leader. Therefore, there wasn't much rush to replace the old leader. The
new, designated successors might even enhance the old boss's reputation.}

{\large For sick and decrepit leaders, nominating a new heir can help them live
out the rest of their life in power. Provided the essentials in the coalition
believe the heir will retain sufficient continuity in the coalition's makeup,
inheritance makes it very difficult for outsiders to offer essential coalition
members more than they expect from the father-son succession.}

\subsection{Papal Bull - ying for Power}

{\large Some of the greatest stories and movies of all time portray how the
outcome of whole nations, peoples, and faiths come down to the actions of a
single individual. Whether it is Luke Skywalker wrestling with father issues or
Frodo disposing of a ring, massed battles have only secondary importance compared
to an individual's triumph. It makes for great fiction certainly, but such events
happen in fact too.}

{\large For Christianity's first several hundred years, the Bishop of Rome---the
pope---was a relatively minor figure even within the Christian community.}

{\large Bishops were the arbiters of Christian practice and belief, but not
until Damasus I, pope from 366 to 384, was the Bishop of Rome truly elevated
above all other Roman Catholic bishops, becoming the head of the western Roman
Catholic Church.6 Eventually sainted for his extraordinary accomplishments,
Damasus's actions were a case study in the manipulation of essentials,
influentials, and interchangeables.}

{\large By the late 300s, the east had a seemingly insurmountable advantage in
the long struggle between the eastern and western branches of Christianity. The
apostles and, of course, Jesus himself, all came from the east. The holy places
were in Jerusalem and Galilee and the nearby cities of today's Israel, Palestine,
Jordan, and Syria. With such incontestable credentials, how could Christianity be
seen first and foremost as anything other than an eastern religion? Damasus had
the insight to find an answer.}

{\large True, the apostles came from the east, but Peter and Paul were martyred
in Rome and it was in Rome that they were buried. Thus he could argue that Rome
was privileged by being the scene of apostolic missions intended to spread the
word and by the profound example of martyrdom carried out of the east and to
Rome.}

{\large Damasus made the compelling case that only the See of Peter in Rome
could be the heart of Christianity because, as Jesus reportedly said (Matthew
16:17--20), ``I tell you that you are Peter, and on this rock I willbuild my
church, and the gates of Hell will not overcome it. I will give you the keys of
the kingdom of heaven; whatever you bind on earth will be bound in heaven, and
whatever you loose on earth will be loosed in heaven.'' Rome, then, must have a
superior claim compared to the eastern Sees. On the surface, this may seem an
explicitly religious argument---but powerful though it is, it obscures the
coalition-building strategies that actually made Damasus pope and made the Roman
Church the new locus of power.}

{\large Nowadays a new pope is elected by the College of Cardinals on the death
of the pope. In Damasus's day, the method was different. The
interchangeables---the selectorate---consisted of all of the Christians in the
Roman diocese. The influentials included at least the local clergy and other
bishops from the province. Defining the winning coalition---the essentials ---is
where the tale of Damasus's success must begin.}

{\large Damasus had a rival for election as pope, Ursinus. Ursinus was popular
with the lay Christians and with much of the clergy. Damasus, in contrast,
enjoyed the support of the aristocracy. Both men had worked closely with the
previous pope, Liberius. When Liberius was exiled to Berea by Emperor Constantius
II in 354, Damasus, like Ursinus, followed him into exile. Unlike Ursinus,
however, Damasus wasted no time returning to Rome, abandoning Liberius, and
throwing his support behind the antipope Felix II who was favored by the emperor.
This most assuredly helped cement Damasus's popularity with the controlling
classes while alienating the lay Christian community and clergy.}

{\large With Liberius dead, parallel papal elections were held, resulting in
both Damasus and Ursinus claiming election. Ursinus was chosen by the faithful
plebian worshippers and Damasus by the powerful. Riots ensued, leading to a
bloody massacre in which 137 people were slaughtered in the basilica of
Sicininus, a popular Roman church. The city's prefects---the secular leaders of
Rome---stepped in and restored order by establishing Damasus as the one and only
pope. They dealt with the threat Ursinus represented by exiling him to Gaul. So
it was that Ursinus's larger coalition of lay worshipers was defeated by the
smaller, but much more powerful, support coalition behind Damasus.}

{\large Damasus did not come by his upper-crust backing by accident. We have
already seen that he had supported Felix II over Liberius. He assiduously pursued
support from the upper classes of Romans, many ofthem pagans, before (and during)
his papacy, thereby ensuring their loyalty to him in return for his loyal pursuit
of policies that benefited them.}

{\large Damasus, for instance, made a habit of cultivating the upperclass women
of Rome. His detractors, noting his close associations with Rome's leading
ladies, accused him of adultery (and murder). He was exonerated thanks to direct
intervention by the emperor himself. His promoters, in contrast, note that he
converted many aristocratic pagan women to Christianity and they, in turn,
brought their husbands into the fold, thereby expanding the selectorate and
perhaps the influentials in Rome's Christian community. That, of course, was good
for the growth of the Church, but it also was good for Damasus's ability to
secure and hold power. He relied on a small coalition---unlike Ursinus---and he
worked on drawing that coalition from an enlarged set of influentials and
interchangeables.}

{\large Being a sophisticated strategist, he also worked to further expand the
set of interchangeables by reaching out to the Christian masses of Rome.}

{\large This could only help him shore up his political power and his
discretionary authority over Church funds, discretionary authority he later used
to build important public works and to employ (Saint) Jerome to write the
Vulgate, the first accessible Latin translation of the Bible, which further
solidified the pope in Rome's ability to dictate the meaning of the gospels.}

{\large How did Damasus expand his appeal to the masses---the
interchangeables---many of whom had opposed his papacy? It seems that many of the
recently converted lay people of the declining Roman Empire missed their many
pagan Roman gods. Damasus recognized that these same people seemed happy to
substitute the many Christian martyrs for those gods. Damasus focused his energy
on discovering the burial places of martyrs and erecting great marble monuments.
Some of his monuments and inscriptions to martyrs can still be seen in Rome to
this day.}

{\large Damasus's efforts bore fruit. He won over and expanded the Christian
laity, gained support among the upper classes, and even captured the support of
the emperor himself, who endorsed Damasus's view of the preeminence of the See of
Rome. On February 28, 380, Emperor Theodosius declared that everyone must abide
by the Christian principles as declared by ``the Apostle Peter to the Romans, and
now followed by Bishop Damasus and Peter of Alexandria.''7 Damasus understood
what to do to come to power and how to retain it.Indeed, after his ignominious
road to election as pope, he did good works from the perspective of the Roman
Catholic Church and achieved sainthood for himself. The door to his coming to
power was opened by the errors of Liberius, his predecessor, who alienated the
emperor instead of cultivating him as an ally. Damasus did not make that error.
He built a small winning coalition drawn from an expanded set of influentials and
interchangeables, thereby ensuring loyal, long-lasting support for himself and
his papacy. And, in the process, his battle for power shifted Christianity away
from its Eastern origins and set it on the path to becoming a Western faith.}

{\large Leaders, like Liberius, who fail to do the right thing, provide
opportunities for someone new to come to power. But remember, what constitutes
doing the right thing must be understood from the perspective of a potential
supporter; it may have nothing to do with what is best for a community or nation.
Anyone who thinks leaders do what they ought to do ---that is, do what is best
for their nation of subjects---ought to become an academic rather than enter
political life. In politics, coming to power is never about doing the right
thing. It is always about doing what is expedient.}

\subsection{Seizing Power from the Bankrupt}

{\large As it turns out, one thing that is always expedient is remaining
solvent. If a ruler has run out of money with which to pay his supporters, it
becomes far easier for someone else to make coalition members an attractive
offer.}

{\large Financial crises are an opportune time to strike.}

{\large The Russian Revolution is often portrayed through the prism of Marxist
ideology and class warfare. The reality might be much simpler. Kerensky's
revolutionaries were able to storm the Winter Palace in February 1917 because the
army did not stop them. And the army did not bother to stop them because the czar
did not pay them enough. The czar could not pay them enough because he foolishly
cut the income from one of his major sources of revenue, the vodka tax, at the
same time that he fought World War I.}

{\large Czar Nicholas confused what might seem like good public policy with bad
political decision making. He had the silly idea that a sober army would prove
more effective than an army that was falling-over drunk.}

{\large Nicholas, it seems, thought that a ban on vodka would improve the
performance of Russia's troops in World War I. He missed the obvious downsides,
however. Vodka was vastly popular with the general populace and, most assuredly,
with the troops. So popular and widely consumed was vodka that its sale provided
about a third of the government's revenue.}

{\large With vodka banned, his revenue diminished sharply. His expenses, in
contrast, kept on rising due to the costs of the war.}

{\large Soon Nicholas was no longer able to buy loyalty. As a result, his army
refused to stop strikers and protesters. Alexander Kerensky formed Russia's
short-lived democratic government after toppling the czar's regime. But he
couldn't hang on to power for long. His mistake was operating a democratic
government, which necessitated a large coalition, and implementing an unpopular
policy---continuing the czar's war---thereby alienating his coalition right from
the start. Lenin and the Bolsheviks madeno such mistakes.}

{\large The czar fell once there was no one to stop the revolution. Louis XVI
suffered much the same fate in the French Revolution. Successful leaders must
learn the lesson of these examples and put raising revenue and paying supporters
above all else. Consider Robert Mugabe's success in staying on as Zimbabwe's
president. The economy has collapsed in Zimbabwe thanks to Mugabe's terrible
policies. Starvation is common and epidemics of cholera regularly sweep the
country. Mugabe ``succeeds'' because he understands that it does not matter what
happens to the people provided that he makes sure to pay the army. And despite
regular media speculation, so far he has always managed to do so and to keep
himself in office well into his eighties. He has reduced a once thriving
agricultural exporting nation into one that depends on foreign aid. Mugabe is
certainly horrible for what he's done to the people he rules, but he is a master
of the rules to rule by. Where policy matters most, when it comes to paying off
cronies, he has delivered. That is why no one has deposed him.}

\subsection{Silence Is GoldenSilence Is Golden}

{\large We all grew up hearing the lesson that silence is golden. As it turns
out, violating that basic principle is yet another path by which incumbents can
succumb to their political rivals.}

{\large The incumbent's advantage in offering rewards disappears as soon as
coalition members come to suspect their long run access to personal benefits will
end. An incumbent's failure to reassure his coalition that he will continue to
take care of them provides competitors with a golden opportunity to seize power.
Houari Boumedi\`{e}ne was able to seize the Algerian presidency from Ahmed Ben
Bella in 1965 after Ben Bella foolishly opened his mouth. Silence would have
served him better.}

{\large Ben Bella achieved fame both on the soccer pitch and as a war hero. He
joined the French army in 1936 and, while posted to Marseille, played for their
professional soccer team. He was awarded the Croix de Guerre and the M\'{e}daille
militaire for his gallantry during World War II. After the war he joined the
struggle to liberate Algeria from France. He became a popular figure in the
independence movement and was elected president of Algeria in 1963. But despite
his many talents, he made a serious mistake.}

{\large On June 12, 1966, he announced that there would be a politburo meeting a
week later and that the purpose of the meeting was to discuss three major issues:
(1) Changes in the cabinet; (2) Changes in the army command; (3) The liquidation
of the military opposition. He then left Algiers for Oran.}

{\large This announcement was tantamount to telling his essential supporters
that he was getting rid of some of them. Since he did not say who was to go, he
created a common interest among the whole group in getting rid of Ahmed Ben
Bella.}

{\large Ben Bella's foolish announcement was just the opening that Houari
Boumedi\`{e}ne needed. No one was certain who would be replaced, but given Ben
Bella's sweeping statement, clearly many would be. In this unforced error, Ben
Bella threw away his incumbency advantage and leftBoumedi\`{e}ne a week to
organize a plot of his own. Ben Bella returned to Algiers the day before the
scheduled meeting and he was awakened at gunpoint by his friend, Colonel Tahar
Zbiri. Boumedi\`{e}ne grasped his opportunity and the essential supporters
defected.8 Silence, as Ben Bella learned far too late, truly is golden. There is
never a point in showing your hand before you have to; that is just a way to
ensure giving the game away.}

\subsection{Institutional Change}

{\large There is a common adage that politicians don't change the rules that
brought them to power. This is false. They are ever ready and eager to reduce
coalition size. What politicians seek to avoid are any institutional changes that
increase the number of people to whom they are beholden.}

{\large Yet much as they try to avoid them, circumstances do arise when
institutions must become more inclusive. This can make autocrats vulnerable
because the coalition they have established and the rewards they provide are then
no longer sufficient to maintain power.}

{\large Under the old Soviet system, Boris Yeltsin had no chance of rising to
power.9 His first effort at becoming a major player relied on a proposal every
bit as foolish as the decision of Czar Nicholas to ban vodka sales.}

{\large He sought to end Communist Party members' access to special stores,
privileged access to the best universities, and other benefits not shared by the
working people of the Soviet Union. Sure, that was popular with the masses but
the masses didn't have much say in choosing who ran the Soviet Union---Party
members did. Mikhail Gorbachev, seeing that Yeltsin was a loose cannon, sent him
packing. After this setback, Yeltsin only survived by being resilient and
inventive in the face of a changed environment.}

{\large By the late 1980s the Soviet economy had stagnated. This left the
recently promoted Soviet leader, Gorbachev, with a serious dilemma.}

{\large Unless he could somehow resuscitate the economy, he was liable to run
out of money. As we have seen, this situation can get leaders into serious
trouble. In order to get the economy moving so that there would continue to be
enough money, Gorbachev needed to loosen control over the people, freeing their
suppressed entrepreneurial potential.}

{\large Economic liberalization wasn't a simple matter for the Soviets. It
entailed giving Soviet citizens many more personal and political freedoms. On the
up side, this allowed the people to communicate, coordinate, and interact,which
can be good for economic growth. On the downside, allowing people to communicate,
coordinate, and interact facilitates mass political protest. Gorbachev was no
fool and presumably he knew liberalization could get him in trouble.
Unfortunately for him, he was between a rock and a hard place. Without a stronger
economy his Soviet Union could not hope to compete with the United States and
maintain its superpower status. And more importantly he could not pay party
members the rewards they were used to. To get a stronger economy he had to put
his political control at risk, both from the masses who wanted a speedier path to
prosperity and from within his coalition by those who feared losing their
privileges.}

{\large Gorbachev rolled the dice and ultimately lost.}

{\large First Gorbachev faced a coup from within his own coalition. In 1991,
harder line antireform party members, fearful of losing their special privileges
(a loss openly advocated by Boris Yeltsin), deposed Gorbachev and took control of
the government. But then Boris Yeltsin, standing atop a tank in Red Square,
ensured that the Soviet military would not fire on protestors who wanted reform.
The mass movement, with Boris Yeltsin at its head, overthrew the coup that wanted
to return to the Soviet Union's more repressive policies of the past. The mass
movement returned Gorbachev ever so briefly to power, leaving him with a much
diminished rump Soviet Union, and paving the way for the dissolution of the
Soviet empire just a few months later.}

{\large Yeltsin, having gotten over his privileges fiasco, understood that he
could not forge a winning coalition out of the inner circles of the Communist
Party, but he could win over the apparatchiks by promoting greater budgetary
autonomy for the Russian Republic within the Soviet structure.}

{\large They could become richer and more powerful in Russia than they had been
in the Soviet Union. In this way, Yeltsin picked off essential members of
Gorbachev's coalition and made himself a winner. Yeltsin was, as it turned out,
much better at working out how to come to power than he was at governing well,
but that is a tale for another time.}

\subsection{Coming to Power in Democracy}

{\large Most of the examples we have discussed so far have involved
autocracies.}

{\large Although generally much less violent, leader transitions in democracies
operate via the same mechanisms. Just as in an autocracy, a democratic challenger
needs to ensure the deposition of the incumbent, seize command of the instruments
of state, and sufficiently reward a coalition of supporters so that they back her
as the new democratic incumbent. Yet achieving these goals is quite different in
democracies.}

{\large In some respects, it is an easier task. In a democracy it is less
difficult, for instance, to detach supporters from the dominant coalition because
democrats need such a large number of supporters. Leaders rely heavily on public
goods to reward their backers, but precisely because so many of the rewards are
public goods that benefit everyone, those in the coalition are not much better
off than those outside the coalition. Furthermore, since personal rewards are
relatively modest once the essential bloc is so large, loyalty is further
diluted. The risk of exclusion from the next leader's coalition remains
relatively small---after all, the next leader will need a lot of backers
too---further weakening the incumbency advantage.}

{\large Challengers succeed when they offer better rewards than the government.
Given that there are so many who need rewarding, this means coming up with better
or at least more popular public policies.}

{\large Unfortunately, because it is easy to erode the support of the
incumbent's coalition, it remains difficult for the challenger to pay off her own
supporters.}

{\large When democratic leaders come to power they need to seize control of
government, but there is not the frenzied rush that we observe in autocracies. In
the United States, for instance, leaders elected in the November election are not
sworn in until the following January. This lag gives incoming presidents time to
prepare, nominating their cabinet and appointing people to positions that need to
be filled. Originally the delay(which lasted until March) was required because
leaders needed months to travel to the Capitol from the state that elected them.
Contenders to become a new dictator or monarchs never extend the courtesy of
waiting for their more distant kin to travel great distances to compete with
them.}

{\large Democrats lack urgency when assuming power because the democratic rules
that determine that the incumbent has been defeated simultaneously create a
coalition of supporters.}

\subsection{Democratic Inheritance}

{\large Democrats, because they rely on a large coalition, cannot lavish great
wealth on their supporters personally. They simply do not have enough money to go
around. Instead democrats need to find effective public policies that their
supporters like and reward their loyalty that way. But this is not to say there
are no private goods in democratic politics. There are.}

{\large And this explains why dynastic rule is common even in democracies. It
may be surprising to learn, for instance, that a careful study finds that 31.2
percent of American female legislators (and 8.4 percent of men) had a close
relative precede them in their political role.10 Nearly 20 percent of American
presidents were close relatives of each other. That's a lot more than chance and
fair competition suggest.}

{\large Dynastic rule is commonplace in democracies for exactly the same reasons
that it is popular among autocrats and monarchs. Who better to protect the wealth
and prestige of the family than family members? Elected officials get to dole out
money and enjoy power and money in return. They are as eager to see their progeny
enjoy the same benefits---and protect their own legacy---as Emperor Augustus or
Carlo Gambino. And so it is that the Tafts of Ohio have held high office
generation after generation.}

{\large Ohio's governor from 1999--2007, Bob Taft, has an illustrious pedigree.}

{\large His father and his grandfather were both US senators, his great
grandfather was president of the United States, and his great great grandfather
was attorney general and secretary of war. The Kennedys, the Rockefellers, the
Roosevelts, the Bush family, and many other American families also have long and
distinguished political histories.}

{\large Of course, dynastic rule is more common outside of democracy. Even if
you don't have the good fortune to be born into a political dynasty, you can come
to power in a democracy if you have good, or at least popular, ideas.}

{\large Good ideas that help the people are rarely the path to power in a
dictatorship.}

\subsection{Democracy Is an Arms Race for Good Ideas}

{\large Competition in democracies is cerebral, not physical. Killing foes works
for dictators, but it is a pretty surefire path to political oblivion in a
democracy.}

{\large That's a good thing, from a moral standpoint, of course. But from a
democrat's point of view, the corollary is that even good public policy does not
buy much loyalty.}

{\large Everyone consumes policy benefits whether they support the incumbent or
not. If a leader cleans up the environment or solves global warming then everyone
is a winner, although of course the extent to which individuals value these
things will vary. But past deeds don't buy loyalty. When a rival appears with a
cheaper way to fix the environment, or the rival finds policy fixes for other
problems that people care about more, then the rival can seize power through the
ballot box. Autocratic politics is a battle for private rewards. Democratic
politics is a battle for good policy ideas. If you reward your cronies at the
expense of the broader public, as you would in a dictatorship, then you will be
out on your ear so long as you rely on a massive coalition of essential backers.}

{\large Winston Churchill is certainly a candidate for Britain's greatest
statesman. He is deservedly famous for his wonderful oratory. Yet patriotic
rhetoric alone was not enough to defeat Hitler's Nazi Germany in World War II.
Churchill did not just deliver rhetoric; he delivered policy results too.}

{\large He convinced President Franklin D. Roosevelt to implement the LendLease
program that enabled a virtually bankrupt Britain to keep fighting.}

{\large He converted the British economy to an efficient wartime footing and
found ways to pressure the Axis powers on multiple fronts. He was fondly admired
and praised by the vast majority of Britons at the end of the war.}

{\large Yet Clement Atlee's Labour Party decisively defeated Churchill's
Conservative Party at elections held in July 1945. Technically speaking, World
War II, a war that Winston Churchill, as much as any single individual, might be
credited with having won, wasn't even over yet. Andalready the people of Britain
were ready to toss Winston out.}

{\large Churchill famously stated in November 1942, following Britain's victory
at El Alamein, that, ``I have not become the King's First Minister in order to
preside over the liquidation of the British Empire.'' British voters ensured he
did not have to. Churchill offered the policies of continued austerity to make
Britain great again. After six hard years of war, rationing, and sacrifice, these
policies had little appeal. Atlee chose to promote the National Health Service
and the creation of a welfare state over reestablishing international dominance.
He won the battle for good ideas.}

{\large Few would deny Churchill did a magnificent job and he was much loved.}

{\large But it was Atlee who won.}

\subsection{Coalition Dynamics}

{\large That democrats need so many supporters makes them vulnerable. If you can
find an issue over which the incumbent's supporters disagree, then it will soon
be your turn to lead. Divide and conquer is a terrific principle for coming to
power in a democracy---and one of the greatest practitioners of this strategy was
Abraham Lincoln, who propelled himself to the US presidency by splitting the
support for the Democratic Party in 1860.}

{\large During the Illinois senate race in 1858, Abraham Lincoln forced Stephen
Douglas to declare his position on slavery just one year after the Supreme
Court's Dredd Scott decision made clear that Congress did not have the right to
ban slavery in federal territories. Douglas was cornered. If he said that slavery
could be excluded, he would win the election in Illinois but he would shake the
foundations of his party; if he said that it couldn't, he would lose the election
and thereby diminish his chances of being the Democrat's presidential nominee in
1860. Douglas declared that the people could exclude slavery and won the race, of
course, but his response on slavery came at the expense of dividing the
Democratic Party two years later in the 1860 presidential election, clearing the
way for Lincoln's coalition to elect him president.}

{\large Lincoln, more than any other winner of the presidency, foresaw that he
would not be popular among a vast segment of voters in the presidential election.
He understood that his best chance, maybe even his only chance for election in
1860, lay in dividing and conquering. Had Douglas answered Lincoln's question
with a pro-slavery response (that is, in support of the Dredd Scott Decision as
the law of the land), he almost certainly would have lost the senate race to
Lincoln. That might have kept the Democrats united in 1860, but it would have
boosted Lincoln's prospects as the senate incumbent with a popular following. By
answering as he did, Douglas guaranteed that his own party would divide over his
presidential bid. With competitors Breckinridge and Bell contesting
thepresidency, Douglas lost his opportunity to win the southern vote, dooming
him---and his Democratic rivals---to defeat, even though Lincoln's vote total was
slim. Lincoln beat the divided Democrats with less than 40 percent of the popular
vote and almost no votes in the South. Similarly, Bill Clinton, with just 43
percent of the vote beat the incumbent President George H. W. Bush (who won 38
percent of the popular vote) in 1992, in no small measure thanks to the run by H.
Ross Perot (who got 19 percent of the vote).11 Lincoln understood that he needed
to keep the coalition as small as possible---even in an inherently large
coalition system.}

{\large Lincoln did not lose sight of this important principal as he sought
reelection in 1864. Seeing that his prospects were not great, he maneuvered to
expand the set of interchangeables and influentials so that he could forge a
winning coalition out of those who previously had no say at all. How did he do
this? He introduced absentee ballots so that soldiers could vote, with an
especially important impact in New York. It is widely believed that the vote of
soldiers carried the state for Lincoln in his 1864 race against General George B.
McClellan. Lincoln was a master at using the rules of politics to his advantage,
winning while being unpopular with a large swath of the American people.}

{\large In democracies, politics is an arms race of ideas. Just as the democrat
has to be responsive to the people when governing, when seeking office it helps
to propose policies that the voters like and it pays to want to do more (as
opposed to less)---even if the economic consequences are damaging down the road
(when you're no longer in office). Satisfy the coalition in the short run. When
democratic politicians lament ``mortgaging our children's future,'' they're
really regretting that it was not them who came up with the popular policy that
voters actually want. Sure, voters might feel guilty about the latest \$1
trillion program, but see if they actually vote to reject it. With parents like
that, what children need enemies?}

\subsection{A Last Word on Coming to Power: The Ultimate Fate of Sergeant Doe}

{\large Our account of coming to power began with the story of Liberia's
Sergeant Doe. His end provides a useful cautionary tale for those seeking power.}

{\large Coming to power and staying in power, as the rest of this book makes
clear, are very different things.}

{\large Sergeant Doe knew where Liberia's money was. And so long as he knew
where it was and used it to keep the army faithful he was able to survive
numerous attempts to overthrow him. The trouble is that you only have to lose
once, and that question---Where's the money?---ended up being the last thing that
Sergeant Doe ever heard.}

{\large With the end of the cold war, the United States no longer needed Doe's
assistance, and in 1989 the US government cut off his future aid. Rivals Charles
Taylor and Prince Johnson, backed by the governments of Burkina Faso and C\^{o}te
d'Ivoire, saw their opportunity and launched an insurgency.}

{\large Doe sent soldiers to counter them, but rather than act as a professional
army ought to, his soldiers proceeded to rape, pillage, and kill, not exactly
endearing themselves to the very people whose support might have saved Doe.}

{\large Civilians flocked to join the revolt. Showing his characteristic lack of
statesmanship or judgment, Doe decided to take a car and personally go off in
search of recently arrived Nigerian peacekeepers. Following a gun battle that
killed all of Doe's entourage, Prince Johnson captured the president and
videotaped his subsequent interrogation. The interrogators repeated the same
questions over and over again before Johnson turned to cutting off Doe's ear and
eating it: ``Where is the money? What is the bank account number?'' Doe didn't
answer. Maybe, knowing he was going to die regardless, he figured at least by
keeping silent his family could enjoy the fruits of his labor, living out their
lives in a comfortable exile.}

{\large Doe was incompetent at running a country. He drove an already poornation
into even deeper poverty and civil war. But he knew the essence of coming to
power. Although dressed up in many forms, successful challengers follow basic
principles. They offer greater expected rewards to the essential supporters of
the current leader than those essentials currently receive. Unfortunately for the
challenger, the incumbent has a significant advantage because the members of the
established winning coalition can be confident that their leader will keep on
lining their pockets or providing the public policies they want. But if the
incumbent is known to be dying, takes too much for himself, chooses the wrong
policies, or is seen to have only weak loyalty from his critical backers, then
the door swings wide open for a challenger to step in and depose the incumbent.}

{\large To achieve power means recognizing the moment of opportunity, moving
fast, and moving decisively to seize the day. And, for good measure, coming to
power also means seizing any opponents, figuratively in democracies, and
physically in dictatorships. Coming to power is not for the faint of heart.}

{\large Politics, however, does not end with becoming a leader. Even as you take
up the reins of power and enjoy its rewards, others are gunning for you. They
want the same job that you so desperately sought! Politics is a risky business.
As we will see, successful leaders manage these risks by locking in a loyal
coalition. Those who fail at this first task open the door for someone else to
overthrow them.}
\pagebreak{}


\section{Chapter 3 - Staying in Power}

{\large AT LONG LAST, THE ASPIRANT TO HIGH OFFICE HAS triumphed.}

{\large Whether through inheritance, coup, election, revolt, murder, or mayhem,
he has seized power. Now he faces a new challenge: hanging on to it.}

{\large As Sergeant Doe's brutal career has taught us, rising to a high position
often requires skills altogether different from those needed to maintain control.
And even the rules for surviving in power do not always resemble the skills
necessary for ruling well. The novelist Italo Calvino has clearly and succinctly
described the tribulations of those who have risen to power: ``The throne, once
you have been crowned, is where you had best remain seated, without moving, day
and night. All your previous life has been only a waiting to become king; now you
are king; you have only to reign. And what is reigning if not this long wait?
Waiting for the moment when you will be deposed, when you will have to take leave
of the throne, the scepter, the crown, and your head.''1 What, then, must a newly
minted leader do to keep his (or her) head? A good starting place is to shore up
the coalition of supporters. This may seem like a simple enough task. After all,
as we've seen, the heights of power are unattainable without the backing of a
coalition strong enough to beat back rivals. However, a wise leader does not
count too much on those who helped her gain power. Remember the fate of many of
Fidel Castro's closest allies. After toppling the previous leader, it's only a
matter of time until they realize that they can do the same again.}

{\large A prudent new incumbent will act swiftly to get some of them out of the
way and bring in others whose interests more strongly assure their future
loyalty. Only after sacking, shuffling, and shrinking their particular set of
essentials can a leader's future tenure be assured.}

{\large Nor is this only true of dictators. To see this urge to build a modified
coalition at work in the seemingly less ferocious world of business, let'stake a
look at Carly Fiorina's rise and fall as CEO at Hewlett-Packard.}

\subsection{Governance in Pursuit of Heads}

{\large CEOs, just like national leaders, are susceptible to removal. Being
vulnerable to a coup, they need to modify the corporate coalition (usually the
board of directors and senior management), bringing in loyalists and getting rid
of potential troublemakers. Usually, they have a large potential pool of people
to draw from and prior experience to help guide their choices. But, also like
national leaders, they face resistance from some members of their inherited
coalition and that may be hard to overcome.}

{\large Most publicly traded corporations have millions of interchangeables
(their shareholders), a considerably smaller set of influentials (big individual
shareholders and institutional shareholders), and a small group of essentials,
often not more than ten to fifteen people. In a group of this size, even
seemingly minor variations in the number of coalition members can have profound
consequences for how a company is run. As we will see, this was particularly true
for Hewlett-Packard (HP), because, as in all companies, small shifts in coalition
numbers can lead to large percentage changes in the expected mix of corporate
rewards.}

{\large In the case of HP, the CEO's winning coalition made up a relatively
large fraction of the real selectorate because ownership is heavily concentrated
in a few hands. That is, we might count corporate coalition size in terms of the
number of its members or in terms of the number of shares owned by them. In HP's
case, the essential bloc and the influential bloc are a tiny part of the total
selectorate because the families of the company's founders, William Hewlett and
David Packard, retain significant ownership just as was true of the Ford Motor
Company, Hallmark Cards, and quite a few other businesses for many years.}

{\large Involvement in a corporation can yield benefits, just like any other
form of government. These benefits can take the form of rewards given to everyone
or private payments directed just to the essentials. In a corporate setting,
private benefits typically come as personal compensation in theform of salary,
perks, and stock options. Rewards to everyone---what economists call ``public
goods''---take the shape of dividends (an equal amount per share) and increased
stock value. When the winning coalition is sufficiently large that private
rewards are an inefficient mechanism for the CEO to buy the loyalty of
essentials, public goods tend to be the benefit of choice. Usually, coalition
members are eager to receive private benefits. However, dividends and growth in
share value are preferred over private rewards by very large shareholders who
happen also to be in the winning coalition---this would make them the biggest
recipients of the rewards that go to all shareholders. That was precisely the
situation in HP, where the Hewlett and Packard families owned a substantial
percentage of the company.}

{\large Who makes up the essentials in a corporation? The coalition typically
includes no more than a few people in senior management and the members of the
board of directors. These directors are drawn from a mix of senior management in
the company, large institutional shareholders, handpicked friends and relatives
of the CEO (generally described as civic leaders, no doubt), and the CEO herself.
In the parlance of economists who study corporations, the makeup of these boards
boils down to insiders (employees), grey members (friends, relatives), and
outsiders.}

{\large One part of any corporate board's duties is to appoint, retain, or
remove CEOs. Generally CEOs keep their job for a long time and that certainly was
true of HP's first CEO, founder David Packard. He was replaced in 1992 by an
insider, Lewis Platt, who had worked for the firm since the 1960s. Platt retired
in 1999 and was replaced by outsider Carly Fiorina.}

{\large The HP board has repeatedly deposed CEOs since then.}

{\large It should be obvious that any board members involved in deposing the
former CEO have the potential to be a problem for a new CEO. Having once been a
coup maker, there is little reason to doubt that they stand ready to start
trouble once again if they think the circumstances warrant it.}

{\large And what could those circumstances be but application of one or more of
the rules of governance we set out earlier, especially if that application harms
their interests.}

{\large Research into CEO longevity teaches us, not surprisingly, that time in
office lengthens as one maintains close personal ties to members of theboard.
Just as sons and daughters may make attractive inheritors of the mantle of power
in a dictatorship, friends, relatives, and fellow employees can generate the
expectation of more loyal supporters after power is achieved. This logic probably
contributed to Lewis Platt's elevation to CEO of HP. Putting more outsiders on a
board translates on average into better returns for shareholders, a benefit to
everyone. At the same time, it also translates into greater risk for the CEO.2
Since the CEO's interest is rarely the same as the shareholder's interest, CEOs
prefer to avoid outsider board members if they can.}

{\large Corporate problems, especially those serious enough to oust a sitting
CEO, can serve to galvanize attention and enhance oversight by the board, making
existing coalitions less reliable. Furthermore it is likely that the new,
replacement CEO will face real impediments to his efforts to create and shape a
board of directors in the wake of an older CEO's deposition.}

{\large After all, the old board members did not get rid of the prior incumbent
with the idea that they would also make it easy for the successor CEO to get rid
of them. Nevertheless, any new CEO worth her salt will try to do just that.}

{\large The long-lasting CEOs are the ones who succeed.}

{\large Carly Fiorina became Hewlett-Packard's CEO in 1999. After six turbulent
years she was deposed from that position and as chairwoman of the board in early
2005. Prior to being removed she was the target of an unsuccessful proxy fight
mounted by Walter Hewlett and David Woodley Packard, sons of HP's founders. The
board, in keeping with the power of inherited insider influence, also included
Susan Orr, founder David Packard's daughter. All were individuals with big
financial stakes in HP.}

{\large Furthermore, as big shareholders Hewlett, Packard, and Orr were more
concerned about HP's overall performance than about any private benefits they got
from being on the board. Good news for shareholders---potentially bad news for
Fiorina.}

{\large The board that selected Fiorina as CEO consisted of fourteen members. As
we've seen, three were relatives of HP's founders; three more were current or
retired HP employees.3 Fiorina's initial board, in other words, had a substantial
group of insider and grey members who were not of her choosing and who had big
stakes in the corporation's stock value. It is not hard to see that Carly Fiorina
needed to make changes to build a leaner board with stronger attachments to her.
It wouldnot be easy---while the previous board selected her, they were not her
handpicked loyalists.}

{\large She achieved results, nonetheless. A year after Fiorina's ascension,
HP's proxy statement to its shareholders in 2000 listed only eleven board
members, 20 percent fewer than the group that selected her. Three, including
David Woodley Packard, were gone. As Fiorina became more entrenched in her
position, the board continued to shrink---the 2001 statement listed only ten
board members, a reduction of nearly 30 percent from the board she originally
inherited. Seemingly growing more secure in her control, Fiorina launched an
effort to merge Compaq with HP, an effort with both beneficial prospects and
serious risks for her continued rule.}

{\large Naturally, Fiorina presented the merger as a boon for HP and its
shareholders. As Fiorina explained on February 4, 2002, It is a rare opportunity
when a technology company can advance its market position substantially and
reduce its cost structure substantially at the same time. And this is possible
because Compaq and HP are in the same businesses, pursuing the same strategies,
in the same markets, with complementary capabilities. So, yes, we thought about a
go-slow approach. But, we concluded, after two-and-a-half years of careful
deliberation and preparation, that standing still had enormous risks.... Standing
still means choosing the path of retreat, not leadership.4 There is no reason to
doubt the sincerity of Fiorina's expectations for the Compaq merger. But it is
instructive to examine a major indicator of how Fiorina's appointment and how her
views meshed with broader market sentiments. The day before the announcement of
Carly Fiorina's appointment as HP's new CEO, HP's shares traded at \$53.43. The
market's reaction to her appointment can reasonably be described as uncertain.
The price of HP shares was flat immediately following the announcement and then
began a decline, falling to under \$39 by midOctober 1999, about three months
later. Of course, markets are forward looking and so investors were watching and
learning, modifying their expectations as Fiorina took charge. The news and
modified expectations must have been good for a while because by early April
2000, HP's shareshad risen markedly to about \$78. But good feelings and good
circumstances were not to prevail for long. After April 7 the share price went
into a tailspin, bottoming out in September 2002 at around \$12 a share, and
significantly underperforming the major stock market indexes.}

{\large By the time Fiorina resigned in February 2005, HP's share price had only
rebounded to about \$20.}

{\large With respect to the Compaq merger, the market was similarly pessimistic.
The plan to merge with Compaq was announced on September 3, 2001. The shares rose
on the news, with a peak in December of that year of about \$23, though still
well below the value the day before Carly Fiorina became CEO. Over the period
from July 1999 (the announcement of Fiorina's appointment) to the end of December
2001, the adjusted Dow Jones index fell 9.4 percent while HP's adjusted share
price fell 47 percent.}

{\large From the perspective of any big investor in HP, including the Hewlett
and Packard families, Fiorina must have looked like a disaster. Their company was
doing worse than the general stock market; their fortunes were being hammered.
She was a CEO in trouble. Nevertheless, the upward tick in the share value
indicated a renewed, if temporary, boost of optimism at the announced intention
to merge with Compaq. But markets don't like infighting, and when Walter Hewlett
and David Woodley Packard declared their opposition to the merger, the gains were
reversed. Soon the price collapsed even further, halving as it became apparent
that there was to be a proxy fight in which Hewlett and Packard sought to muster
support from enough shareholders to defeat the board's proposed slate at the
corporation's annual meeting. No doubt Fiorina realized that she was going to be
in for a tough time, perhaps even before her public announcement of the intended
and eventually successfully completed merger. It also seems likely that she would
already have known Hewlett and Packard's views. We can only conclude that this
was an intentional gamble on a major policy shift, one that could---and
did---adversely affect the wealth of HP's large shareholders (such as present or
former board members Hewlett and Packard).}

{\large Looking at the Compaq-HP merger politically, we can see several critical
themes emerging. Fiorina was already in some trouble because of declining share
value. She had successfully diminished the board's sizeand shuffled its
membership, both wise choices for a CEO seeking longevity in office. Yet despite
these actions, she still faced significant opposition from the inner circle of
essentials and influentials. She had not yet secured the board's loyalty. The
Compaq merger might have made good business sense and could therefore have been
good for the stock price, thus softening internal opposition to her. Or else,
seeing the merger as a fait accompli, her opponents might have given up their
fight. That didn't happen. And the disgruntled board members, heavily invested as
they were in HP's stock value, could not be mollified with private rewards.}

{\large However, what in retrospect may seem like a political nonstarter at the
time held great political advantages. What, for instance, had to be an
implication for board composition of Fiorina's multibillion dollar merger with
Compaq? Once the deal was sealed Fiorina would have to bring some Compaq leaders
onto the postmerger HP board. This could be done either by expanding the existing
board to accommodate Compaq influentials or by pruning the existing board to make
room for the new, Compaq representatives drawn from Compaq's selectorate. Fiorina
apparently saw that the merger would provide an opportunity to reconstitute the
board, providing an undeniable opportunity to weaken the board faction that
opposed her. That seems to be exactly what she tried to do.}

{\large Of course, her rivals would not sit idly by and be purged. Unless such a
purge can be accomplished in the dark, presented as a fait accompli to the old
group of influentials, the risk of failure is real. As it happens, Securities and
Exchange Commission (SEC) regulations require disclosures, which make turning a
board purge into a fait accompli extremely difficult when the opportunity to
purge the board depends on a prospective merger.}

{\large There are two potential responses to a rebellion such as the one Fiorina
faced over HP's weak share price and the Compaq merger. A CEO can either purge
essentials and boost the private benefits to remaining coalition members, or
expand the coalition and increase rewards to the general selectorate of
interchangeables (that is, shareholders). Having survived the proxy fight in
2002, Fiorina faced an eleven-member board that included five new members carried
over from Compaq as part of the merger. HP's board had shifted materially with
only six previous HP boardmembers on it. Since Fiorina had been the mover and
shaker behind the Compaq deal it is reasonable to believe that she assumed the
new members would be likely to work with her as opposed to lining up with board
members who had supported Walter Hewlett's fight against the merger. Walter
Hewlett and Robert P. Wayman, meanwhile, left the board.}

{\large By this time Fiorina had expanded the total board size by only one, from
ten to eleven, while overseeing the departure of several old board members at the
same time to make room for five Compaq representatives. Surely she had reason to
believe she now enjoyed the support of a majority of the new board.}

{\large Perhaps in an effort to shore up the support of remaining old hands on
the board, or perhaps coincidentally, there also was a notable shift in board
compensation. Just before Fiorina became HP's head, board members earned
compensation (that is, private benefits) that ranged from \$105,700 to \$110,700.
With Fiorina in office and the board diminished in size, this amount dropped
slightly to \$100,000--\$105,000 and remained there in the years 2000--2003. But
in 2004, according to HP's 2005 proxy statement, board members received \$200,000
to \$220,000. During the same period, dividends remained steady at \$0.32 a share
annually and HP's shares significantly underperformed the main stock market
indexes.}

{\large Clearly something was up: HP's stock price performance was poor;
dividends were steady; and directors' pay doubled.}

{\large Fiorina's board shuffling and their improved compensation seem aimed at
getting the right loyalists in place to help her survive. Although the Compaq
merger resulted in the board growing from ten to eleven, what is most noteworthy
is that this net growth of one member was achieved while adding five new members
(one of whom stepped down at the end of the year). So, the old members
constituted only about half the board, shifting the potential balance of power
toward Fiorina. Presumably that is just what she hoped, although it is not how
things turned out.}

{\large Expanding the board was not, and generally is not, the optimal response
to a threat from within. To her credit, in terms of political logic, she
significantly expanded the size of the interchangeables by adding Compaq's
shareholders to HP's list of shareholders. This normally helps to induce
strengthened loyalty, but declining share value could not have been good for new
HP board members, who had been heavily invested inCompaq since their economic
well-being was now tied to HP's share performance. Nor could Fiorina mollify HP's
large shareholders on the board with better board compensation, since their
welfare depended on producing the ``public good'' of greater returns to
shareholders. Those grey board members who owned lots of shares made the
seemingly small board of eleven actually pretty large in terms of shares they
could vote.}

{\large Under enormous pressure, Carly Fiorina stepped down. She was replaced by
Patricia Dunn as chairwoman, with HP's Chief Financial Officer (CFO), Robert
Wayman, emerging again as a significant HP player. He was made interim CEO.
Wayman, unable or uninterested in translating his interim position into a
full-time job, stepped down a month later while continuing in his role as a
member of the board and an HP employee. Mark Hurd in turn replaced him as CEO.}

{\large In the immediate aftermath of Fiorina's ouster the board separated two
key positions, CEO and chairperson, presumably in a good Montesquieulike effort
to promote the separation of powers and protect themselves against future adverse
choices by the CEO. If that was their intention, they certainly failed. Following
Hurd's ascent to the position of CEO he successfully brought the two posts back
under one person's control: his own.}

{\large Within a year of Fiorina's ouster, all the leading coup makers who acted
against her were gone. Mark Hurd had risen to the top and, as suggested by the
quote from Italo Calvino, he had to watch day and night to keep his head. Four
years later, despite stellar HP performance, Hurd was, in turn, forced out amidst
a personal scandal. This is the essential lesson of politics: in the end ruling
is the objective, not ruling well.}

\subsection{The Perils of Meritocracy}

{\large One lesson to be learned from Mark Hurd's ultimate removal at HP is that
doing a good job is not enough to ensure political survival. That is true whether
one is running a business, a charity, or a national government.}

{\large How much a leader's performance influences remaining in office is a
highly subjective matter. It might seem obvious that it is important to have
people in the coalition of key backers who are competent at performing the duties
associated with implementing the leader's policies. But autocracy isn't about
good governance. It's about what's good for the leader, not what's good for the
people. In fact, having competent ministers, or competent corporate board
members, can be a dangerous mistake. Competent people, after all, are potential
(and potentially competent) rivals.}

{\large The three most important characteristics of a coalition are: (1)
Loyalty; (2) Loyalty; (3) Loyalty. Successful leaders surround themselves with
trusted friends and family, and rid themselves of any ambitious supporters.}

{\large Carly Fiorina had a hard time achieving that objective and as a result
she failed to last long. Fidel Castro, by contrast, was a master (of course, he
had fewer impediments to overcome in what he could do than did Fiorina) and he
lasted in power for nearly half a century.}

{\large The implications of this aspect of political logic are profound,
particularly in small coalition governments. Saddam Hussein in Iraq, like Idi
Amin in Uganda and so many other eventual national leaders, started as a street
thug. Autocrats don't need West Point graduates to protect them. Once in power,
people like Amin and Hussein wisely surround themselves with trusted members of
their own tribe or clan, installing them in the most important positions---those
involving force and money---and killing anyone that may turn out to be a rival.}

{\large Saddam Hussein came to power after compelling his predecessor (and
cousin) Ahmed Hassan al-Bakr to resign in 1979.5 Before that, however, he had
carefully laid the groundwork for his control over Iraq. In 1972, forinstance, he
spearheaded the nationalization of international oil interests in Iraq. Oil, of
course, was and is where the money is in Iraq, so he had fulfilled the essential
ingredient to come to power: he knew where the money was. Once in power, he
ruthlessly pruned his support base.}

{\large Just six days after President al-Bakr ``resigned,'' Saddam Hussein
convened a national assembly of the ruling Ba'ath Party's leaders (the
Revolutionary Command Council). The assembly was videotaped at Saddam Hussein's
insistence. During the session, Muhyi Abdel-Hussein, secretary of the
Revolutionary Command Council, read out a confession that he plotted against
Saddam Hussein, and then sixty-eight more ``enemies of the state'' were named as
coconspirators. Each, one at a time, was removed from the assembly. Twenty-two
were sentenced to death by firing squad and summarily executed by members of the
Ba'ath Party, each branch of which was required to send a delegate with a rifle
to participate in the executions. Hundreds more were executed within the next few
days. Saddam Hussein's biographer asked Saddam about the decision to eliminate
these people, most of whom had risen in the ranks of the Ba'ath Party with
Saddam's support. He reports, ``The answer was that as long as there is a
revolution, there will be a counter-revolution.''6 As we said before, those who
can bring a leader to power can also bring the leader down. It is best to shrink
the ranks of those who represent a threat and keep those who are most trusted to
be loyal.}

{\large How competent were the approximately 450 Ba'ath leaders who were
executed as part of Saddam's consolidation of power? It is difficult to say from
this remove, but we do know that among their ranks were professors, military
officers, lawyers, judges, business leaders, journalists, religious leaders, and
many other well-educated and accomplished men. For good measure, Hussein also
threw in leaders of competing political parties who, after all, might have
conspired to replace him.}

{\large Survivors included people like Saddam's cousin, ``Chemical Ali,'' Ali
Hassan al-Majid. Chemical Ali most notably demonstrated his loyalty in 1988 when,
under orders from Saddam, he launched a successful campaign to commit genocide
against Iraq's restive Kurds. Long before that al-Majid had established his
commitment to Saddam Hussein. In the infamous videotape mentioned earlier,
al-Majid is seen speaking to Saddam, saying, ``What you have done in the past was
good. What you willdo in the future is good. But there's one small point. You
have been too gentle, too merciful.''7 Unlike many who were executed following
the July 22, 1979, party assembly, al-Majid, previously a motorcycle
courier/delivery boy, had little formal education. Although he held the posts of
defense minister, interior minister, and head of Iraq's intelligence service, it
seems his main area of competence was murder.}

{\large Saddam Hussein's pattern of appointments is quite typical. His
successor, Prime Minister Nouri al-Maliki, purged the security services of all
Sunnis and replaced them with Shia supporters, albeit with a gentler hand than
his predecessor.8 These replacements did not have the experience and training of
the existing security personnel. Both leaders knew that it is better to have
loyal incompetents than competent rivals.}

{\large Sometimes, of course, having competent advisers is unavoidable.}

{\large Byzantine, Mughal, Chinese, Caliphate, and other emperors devised a
creative solution that guaranteed that these advisers didn't become rivals: They
all relied on eunuchs at various times. In the Byzantine Empire in the ninth and
tenth centuries, the three most senior posts below emperor were held almost
exclusively by eunuchs. The most senior position of Grand Administrator had
evolved from the position of Prefect of the Sacred Bedchamber and included the
duties of posting eunuch guards and watching over the sleeping emperor. Michael
III made an exception and gave this position to his favorite, Basil, rather than
a eunuch. This decision cost him his life. When Basil perceived that Michael was
starting to favor another courtier, he murdered the emperor and seized the
throne.9 Even in modern times the principle of choosing close advisers who cannot
rise to the top spot remains good advice. It is surely no coincidence that Saddam
Hussein as president of Islamic Iraq had a Christian, Tariq Aziz, as his number
two.}

\subsection{Keep Essentials Off-Balance}

{\large  What we can begin to appreciate is that no matter how well a tyrant
builds his coalitions, it is important to keep the coalition itself off-balance.}

{\large Familiarity breeds contempt. As noted, the best way to stay in power is
to keep the coalition small and, crucially, to make sure that everyone in it
knows that there are plenty of replacements for them. This is why you will often
read about regular elections in tyrannical states. Everyone knows that these
elections don't count, and yet people go along with them. Rigged elections are
not about picking leaders. They are not about gaining legitimacy. How can an
election be legitimate when its outcome is known before the vote even occurs?
Rigged elections are a warning to powerful politicians that they are expendable
if they deviate from the leader's desired path.}

{\large Vladimir Ilyich Lenin was the first to really exploit the idea of
substitute coalition members. In a one-party state, he nonetheless perfected a
rigged election, universal adult suffrage system. Any action he took---say,
sending so-and-so to Siberia---was the will of the people, and any of the people
in the replacement pool had a chance, albeit a slight one, of being called up to
serve as an influential or maybe even an essential somewhere down the line.
Everybody in the Soviet selectorate could, with a very small probability, grow up
to be general secretary of the Communist Party, just like the petty criminal
Joseph Stalin and the uneducated Nikita Khrushchev. Those already in the inner
circle knew they had to stay in line to keep their day jobs. Bravo, Lenin.}

{\large Although Lenin perfected the system and probably came up with it on his
own, the always fascinating country of Liberia experimented earlier on with the
same phenomenon. Prior to Samuel Doe's takeover, Liberia had been ruled by the
True Whig Party. The country originated when a number of American liberal
organizations, appalled by the evils of slavery, paid to repatriate former slaves
to West Africa. Despite the nation's philanthropicorigins, the most important
lesson the former slaves took from their experiences appears to be that slavery
and forced labor worked much better for the masters than the slaves. These former
slaves instituted universal adult suffrage in 1904, but with a property
qualification that effectively excluded indigenous Africans from becoming
insiders, making the selectorate large but the influential group relatively
small. Thus, they established a system run for a small group of insiders despite
the appearance of a universal franchise. This structure provided for strong
loyalty to the incumbent that ensured the opportunity to suppress any opposition
that might arise to their forced labor policies, a system whose policies differed
from Soviet ones but whose security in office was the same.10 Virtually every
publicly traded company in the world has adopted the Leninist rigged-election
system and for much the same reasons. It, along with a packed board, is one of
the major factors ensuring that poorly performing CEOs hardly ever get fired.
Carly Fiorina had the misfortune of heading a company that might have looked like
a rigged election autocracy but up close and personal remained more akin to a
monarchy.}

{\large Although there were millions of shareholders who in theory could shape
HP policy, so many shares were concentrated in a few hands that HP had more of
the characteristics of a small coalition drawn from a small group of influentials
within a mostly small, concentrated group of interchangeables; that is, members
of the Hewlett and Packard families.}

{\large The essence of keeping coalition members off-balance is to make sure
that their loyalty is paid for and that they know they will be ousted if their
reliability is in doubt. The USSR's Mikhail Gorbachev, thought to be a good guy
in western political circles, certainly understood the necessity of rewarding
loyalty and shucking off all those whose faithfulness was questionable. He
replaced much of the politburo within his first two years in office, picking and
choosing from the Communist Party (the real selectorate) those most loyal to him.
It turns out, though, that Gorbachev was much less ruthless than contemporaries
of the autocratic class. He forced adversaries, like Boris Yeltsin, out of the
politburo to be sure. But, as Yeltsin surely realized, he would have been killed
under Stalin. Equally, he and many others must have known that it was much better
to cross swords with Gorbachev, an intellectual reformer, than with
suchcontemporaries as Mobutu Sese Seko of Zaire or even Deng Xiaoping of China.
Deng, after all, used ruthless force to end the prodemocracy uprising at
Tiananmen Square in 1989. Gorbachev, as we will see, did not hesitate to use
force outside of Russia, but he also did not go around killing his political
rivals. His reward was a short time in power first because he left himself
vulnerable to a coup by hard-line communists and then because he allowed Yeltsin
to resurrect himself politically, defeat the coup, and make himself into
Gorbachev's replacement.}

{\large The execution of opponents is a longstanding practice among most
autocrats. We should not fail to appreciate the moral significance of Gorbachev's
restraint. Adolf Hitler, Mao Zedong, Fidel Castro, Samuel Doe, and so many others
showed no such restraint. They had their erstwhile backers murdered once they
worked out who was most likely to be loyal and who was not. We see a nicer
version of such behavior as a routine part of corporate changes when there's a
new CEO. Although the CEO is supposed to answer to the board, it is commonplace
for boards to be reconstituted after a new CEO comes to power; the tail
apparently wags the dog.}

{\large Being purged from the initial coalition is often fatal. Hitler became
chancellor of Germany on January 30, 1933. During his rise to power he relied
heavily on the Sturmabteilung, a paramilitary force also known by the
abbreviation, SA, or by a description of their uniforms, the Brownshirts.}

{\large Hitler perceived the SA's leader, Ernst Rohm, as a threat. He built up
an alternative paramilitary, the Schutzstaffel, or SS, and then, on what became
known as the night of the long knives, he ordered the assassination of at least
eighty-five and possibly many hundreds of people between June 30 and July 2,
1934. Thousands more were imprisoned. Despite Rohm's long term and essential
backing (Rohm had been with Hitler during his failed 1923 Munich Beer Hall
Putsch), Hitler showed no sentimentality. He replaced him with men like SS leader
Heinrich Himmler, whom he deemed more loyal.}

{\large Robert Mugabe is likewise a master at keeping his coalition
off-balance.}

{\large He was elected president of Zimbabwe in 1980 following a negotiated
settlement to a long civil war. The struggle against the white-only rule of the
previous Rhodesian regime was led by two factions that crystallized into
political parties behind their respective leaders: Robert Mugabe's ZANU(Zimbabwe
African National Union) and Joshua Nkomo's ZAPU (Zimbabwe African People's
Union). Initially, Mugabe preached reconciliation: If yesterday I fought you as
an enemy, today you have become a friend and ally with the same national
interest, loyalty, rights and duties as myself. If yesterday you hated me, you
cannot avoid the love that binds you to me and me to you. . . . Draw a line under
the past.... The wrongs of the past must now stand forgiven and forgotten. If
ever we look to the past, let us do so for the lesson the past has taught us,
namely that oppression and racism are inequalities that must never find scope in
our political and social system. It could never be a correct justification that
because the whites oppressed us yesterday when they had power, the blacks must
oppress them today because they have power. An evil remains an evil whether
practiced by white against black or black against white.11 A na\"{\i}ve observer
might have thought that Mugabe planned to bring ZAPU elites into his winning
coalition. That might have made sense at the outset, but once ZANU's power was
consolidated there would be no reason to keep ZAPU loyalists around. And once
Mugabe's power was consolidated, he'd have no need to keep some of his old
friends from ZANU around either.}

{\large Mugabe also reached out to many in the white community, and particularly
former leaders and administrators, to help him run the country.}

{\large Many whites who had feared the transition, began to refer to him as
``Good Old Bob.'' Mugabe needed their support. He could not run the country
without them and he needed to know where the money was. In this he was greatly
assisted by the international community. They pledged \$900 million during his
first year. However, once he was ensconced in power, Mugabe's attitude changed.}

{\large In 1981 he called for a one party state and began arresting whites,
saying ``we will kill those snakes among us, we will smash them completely.''
Mugabe was even harsher towards his former comrades in arms. He forced Nkomo out
of the cabinet and sent a North Korean trained paramilitary group, the Fifth
Brigade, to terrorize Matabeleland, Nkomo'sregional stronghold. As one ZANU
minister put it, ``Nkomo and his guerillas are germs in the country's wounds and
they will have to be cleaned up with iodine. The patient will scream a bit.'' The
operation was called Gukurahundi---a Shona word that means, Wind that blows away
the chaff before the spring rains. Many veterans from the fight against white
rule resisted. In retaliation Matabeleland was effectively sealed off and 400,000
people faced starvation. As one of Mugabe's henchmen, a brigade officer, stated,
``First you will eat your chickens, then your goats, then your cattle, then your
donkeys. Then you will eat your children and finally you will eat the
dissidents.''12 Mugabe needed the assistance of ZAPU fighters to defeat white
only rule. He needed the assistance of white farmers and administrators and the
international community to find the money to solidify his control over the state.
Only when he was entrenched in power did ``Good Old Bob'' show his true colors.}

\subsection{Democrats Aren't Angels}

{\large  As we all know, the victor writes history. Leaders should therefore
never refrain from cheating if they can get away with it. Democrats may have to
put up with real and meaningful elections in order to stay in power, but it
shouldn't be shocking to see that whenever they can, they'll happily take a page
out of Lenin's book. There's no election better than a rigged one, so long as
you're the one rigging it.}

{\large The list of tried and trusted means of cheating is long. Just as quickly
as electoral rules are created to outlaw corrupt practices, politicians find
other means. For instance, leaders can restrict who is eligible and registered to
vote and who is not. In Malaysia, under a system known as Operation IC,
immigration is controlled so as to create demographics favorable to the incumbent
party. New York City's infamous Democratic Party machine, Tammany Hall, acquired
its Irish flavor by meeting and recruiting immigrants as they left the boat,
promising citizenship and jobs for their vote.}

{\large When leaders can't restrict who is eligible to vote or else are unable
to buy enough votes, they can use intimidation and violence to restrict access to
polling places. North Indian states, such as Bihar and Uttar Pradesh, experience
``booth capture,'' where party supporters capture the polling place and cast
every eligible voter's vote for their party.}

{\large Cheating does not stop once ballots are cast, of course. Leaders never
hesitate to miscount or destroy ballots. Coming to office and staying in office
are the most important things in politics. And candidates who aren't willing to
cheat are typically beaten by those who are. Since democracies typically work out
myriad ways to make cheating difficult, politicians in power in democracies have
innovated any number of perfectly legal means to ensure their electoral victories
and their continued rule.}

{\large One counterintuitive strategy is for leaders to encourage additional
competitors. This is why some states have so many political parties, eventhough
only one really wins. The conventional wisdom about America's two-party system
tells us that fringe parties allow for a more vibrant and responsive government.
But even in multiparty states, there are always leading parties---you have to ask
yourself whether the leading parties would allow the fringe parties to exist if
they weren't somehow serving their interests.}

{\large Tanzania's parliament and presidency are perennially controlled by the
Chama Cha Mapinduzi party (CCM), even though as many as seventeen parties
routinely compete in Tanzania's free and fair elections. The CCM government
actually provided campaign financing, as we would expect, in an opaque way, to
small parties until quite recently, thereby encouraging them to compete and
divide the opposition vote. This makes it easier for the relatively centrist CCM
to win. Although the CCM wins a large percentage of the vote, all it needs to win
is one more vote than the second largest party in half the parliamentary
constituencies. That turns out to mean the CCM needs much less than 10 percent in
most districts. The number of supporters a party needs affects the kinds of
policies it pursues.}

{\large In those constituencies in Tanzania where an opposition party generates
lots of votes, the CCM needs to appeal to many voters and therefore generally
provides better health care, education, and services. In constituencies where the
CCM needs fewer votes, cash transfers, such as vouchers for subsidized
fertilizer, are more common.13 Multiparty democracy provides a similar means for
one or two parties to dominate governments in democracies from Botswana to Japan
and Israel. There is more to representing the people than just allowing them to
vote, even when the vote is done honestly.}

{\large Designated seats for underrepresented minorities is another means by
which leaders reduce the number of people upon whom they are dependent. Such
policies are advertised as empowering minorities, whether they are women, or
members of a particular caste or religion. In reality they empower leaders. That
a candidate is elected by a small subset of the population reduces the number of
essentials required to retain power. At a very basic level, electoral victory in
a two-party parliamentary system requires the support of half the people in half
the districts; that is, in principle, 25 percent of the voters. Suppose 10
percent of the seats were reserved for election by one specific group that
happensto be geographically concentrated (such as gay voters in the Castro in our
earlier account of Harvey Milk's election in San Francisco). To retain half the
seats in parliament, the incumbent party need only retain 40 percent of the
regular single member district seats, which is readily done with just over 22
percent of the vote. So by focusing on districts in which the privileged minority
is prevalent, a party can reduce the number of votes it requires by 12 percent.}

{\large Delegated positions also make it easier to form a small coalition.}

{\large Consider Tanzania's Parliament, the Bunge. There are 232 directly
elected seats, seventy-five seats reserved for women who are nominated by the
parties in relation to the number of seats they capture in the election, and five
seats nominated by the Zanzibar Assembly. (Zanzibar is a beautiful island off the
mainland that united with mainland Tanganyika in 1964 to form Tanzania.) In
addition, the president gets to nominate ten cabinet appointees and an attorney
general to serve in parliament. This gives a total of 323 seats, of which the
president needs 162 to control the Bunge. Given that he appoints eleven, and that
the CCM is regionally based in Zanzibar, he already controls sixteen seats. If
the CCM wins 111 elected seats, then he controls parliament. That is, 111
directly elected seats, 16 appointed seats, and 35 of the appointed women's seats
(75 seats x 111/232), which totals 162. The CCM needs substantially less than
half the directly elected seats. And as we have already seen, by funding many
opposition parties the CCM can win many seats with less than a 10 percent vote
share. In practice the president controls nearly all the women's appointments and
he tends to appoint women who lack an independent base of support. Indeed, few
women win direct election to Tanzania's parliament.}

{\large While Tanzania has free and fair elections, the reality is that the
incumbent CCM party can sustain itself in office with as little as 5 percent of
the vote. Of course, in most districts they get much more support because
politicians find inventive ways to incentivize voters. One of these ways is the
creation of voting blocs.14}

\subsection{Bloc Voting}

{\large Bloc voting is a feature common in many fledgling democracies. It was
also the norm under party machines in large US cities. For instance, under the
influence of Tammany Hall, whole neighborhoods in New York City would turn up to
vote Democratic. Many of India's electoral districts have followed a pattern
similar to the old Tammany Hall. That is, a small group of local notables or
village patrons can deliver their community's vote and extract great rewards for
themselves in return.}

{\large During Bueno de Mesquita's time doing field work in India in 1969-- 1970
he observed firsthand how the quest for power coupled with the influence of power
blocs undermined any notion of the pursuit of political principles other than the
principles, win, and get paid off.}

{\large Senior people in villages and towns, and indeed, up and down the levels
of governance in India's states, would pledge to a particular party the support
of those they led. In return, they would receive benefits and privileges. By and
large, all the ``clients'' of these ``patrons'' followed their patron's lead and
voted for the designated party. What is most fascinating is that the affiliations
between voters and parties need not have had any ideological rhyme or reason. In
Uttar Pradesh, India's most populace state, for instance, the free-market,
anticommunist Swatantra Party, the socially conservative and anticommunist Jana
Sangh Party, and the Communist Party of India formed a coalition government with
each other following India's 1967 election. This was true despite the Swatantra
Party's leadership's description of the Communist Party of India as ``public
enemy number 1.'' What did these parties have in common? Only their desire to
band together and beat the Congress Party so as to enjoy the benefits of power.
This sort of odd bedfellows coalition-building strategy was long rampant
throughout India.15 Perhaps the most egregious case of bald opportunism occurred
in the state of Bihar. There ideologically disparate parties formed a
government,relying heavily on currying favor with the Raja of Ramgarh. The raja,
owner of much of the mining interests in Bihar, switched parties every few
months, bringing coalition governments down---and up---with him. Each time he
switched, he garnered greater private goods for himself and his backers,
including the dismissal of criminal charges against him. As the newspaper, The
Patriot, reported on June 26, 1968, following one of the raja's frequent
defections to an alternative coalition, leading to the formation of a new
government, ``The Raja who had been able to get his terms from Mr. Mahamaya
Prasad [the former head of the Bihar government] assumed that he could demand
from Mr. Paswan [the new head of the Bihar government] a higher price. This
amounted to Deputy Chief Ministership and the Mines portfolio for himself and
withdrawal of the innumerable cases filed against him and members of his family
by the Bihar government.'' 16 The raja understood that he could manipulate his
bloc of backers to make and break governments and, in doing so, he could enrich
himself a lot and help his followers a little bit in turn. That, indeed, is the
lesson of bloc voting whether based on personal ties in Bihar, trade union
membership among American teachers, tribal clans in Iraq, linguistic divisions in
Belgium, or religion in Northern Ireland. Bloc leaders gain a lot, their members
gain less, and the rest of society pays the price.}

{\large Bloc voting takes seemingly democratic institutions and makes them
appear like publicly traded companies. Every voter or share has a nominal right
to vote, but effectively all the power lies with a few key actors who can control
the votes of large numbers of shares or deliver many votes from their village.
Bloc voting makes nominally democratic systems with large coalitions function as
if they are autocratic by making the number of influentials---that is, people
whose choices actually matter---much smaller than the nominal selectorate of the
rest of the voters. Since this is such an important aspect of winning elections
we are obliged to explore how politicians do it.}

{\large The traditional approach has been to treat emerging democracies as
patronage systems in which politicians deliver small bribes to individual voters.
The New York Times, for instance, reported on September 17, 2010, in an article
with the headline ``Afghan Votes Come Cheap, and Often in Bulk,'' that the
typical price paid for an Afghan voter's support was about \$5 or \$6. But the
article also noted that widespread vote fraudprobably made vote buying
unnecessary in any event.}

{\large The explanation for fraudulent electoral outcomes based on vote buying
in exchange for patronage is simple, but it is also incomplete. First, parties
don't bribe enough people, and second, once in the voting booth, voters can
renege. Historically parties used to issue their own ballots. For instance, your
party might print a ballot on pink paper. In such a way, party representatives
could check that those who took bribes voted with pink ballots. Although we could
fill a whole book with the tricks parties use to monitor vote choices, the
reality is that today votes are likely to be anonymous, at least in real
democracies.}

{\large Bribing voters works far better at the bloc level. Suppose there are
just three villages, and suppose a party, call it party A, negotiates with senior
community figures in the villages and makes the following offer: if party A wins
it will build a new hospital (or road, or pick up the trash, send police patrols,
plow the snow, and so on) in the most supportive of the three villages. Once a
village elder declares for party A, voters in that village can do little better
than support party A, even if they don't like it. The reality is that there are
so many voters that the chance that any individual's vote matters is
inconsequential. Yet, voters are much more influential about where the hospital
gets built or whose streets get swept than they are about who wins the election.
To see why, consider the case where two or three of the village elders declare in
favor of party A and most voters in these villages go along with them.}

{\large Consider the incentives of an individual voter. Since at least two of
three villages have declared for party A, an alternative party is unlikely to win
so an individual's vote has little influence on the electoral outcome. Voting for
party B is a waste of time. Yet the voter could influence where the hospital is
built by turning out to vote for A. If everyone else supports A, but she does
not, then her village gives one less vote for A than another village and so loses
out on the hospital. If she votes for A, then her village has a shot at getting
the hospital. In the extreme case, where absolutely everyone votes for party A,
our voter would give up a one third chance of getting the hospital in her village
if she did not vote for party A. Voters have little incentive but to go along
with their village elders.}

{\large By rewarding supportive groups over others, individual voters are
motivated to follow the choice of their group leader, be that a village elder,a
ward organizer, a church leader, or a union boss. The real decisions are made by
the group leaders who deliver blocs of votes. They are the true influentials. It
is therefore unsurprising that it is common for the rewards to flow through them,
so that they can take their cut, rather than go directly to the people. Milton
Rakove describes the process of handing out rewards to different ethnic groups
under Mayor Richard Daley's party machine in Chicago in the early 1970s: ``The
machine co-opts those emerging leaders in the black and Spanishspeaking
communities who are willing to cooperate; reallocates perquisites and
prerogatives to the blacks and the Spanish speaking, taking them from ethnic
groups such as the Jews and Germans, who do not support the machine as loyally as
their fathers did. . .}

{\large .''17 Of course, leaders can use sticks as well as carrots. Lee Kuan Yew
ruled Singapore from 1959 until 1990, making him, we believe, the longest serving
prime minister anywhere. His party, the People's Action Party (PAP), dominated
elections and that dominance was reinforced by the allocation of public housing,
upon which most people in Singapore rely.}

{\large Neighborhoods that fail to deliver PAP votes come election time found
the provision and maintenance of housing cut off.18 In Zimbabwe, Robert Mugabe
went one step further. In an operation called Murambatsvina (Operation Drive Out
the Rubbish), he used bulldozers to demolish the houses and markets in
neighborhoods that failed to support him in the 2005 election.}

{\large Ownership of a public company works in the same way as bloc voting.}

{\large We could hold our shares in our own name and vote at stockholder
meetings. However, except for a very wealthy few of us, our votes are
inconsequential and turning up is burdensome. Thus we hold stock via mutual funds
and pensions (there are tax and management reasons to do so too, but then think
about who has the incentive to lobby for these regulations). These institutional
investors, like village elders, are influential enough that CEOs court their
support. But it is much cheaper to buy the loyalty of the institutional investor
by private goods, such as fees for board membership, than it is to reward all the
little investors he represents with great stock performance.}

{\large So what can a politician do when elections are fair and the risk of
electoral defeat is rising? When an incumbent is at risk of electoral defeat,he
can always mitigate that risk by redrawing the boundaries of the constituency to
exclude opposition voters. That is to say, the district can be gerrymandered,
although this opportunity only comes once in a while so it may come too late to
save an unpopular incumbent. The practice of gerrymandering has made it such that
the odds of being voted out of a US congressional seat are not that different
from the odds of defeat faced by members of the Supreme Soviet under the Soviet
Union's one-party communist regime. And, while gerrymandering virtually ensures
reelection, it also makes the voters in a congressional district happy. After
all, the gerrymander means that they get the candidate favored by a majority in
the district. If gerrymandering isn't an option, then other rule changes can be
instituted, such as prohibiting rallies---in the name, of course, of public
safety.}

{\large Have a look at the map of Maryland's 3rd Congressional district in
Figure 3.1. Need any more be said about why, in many districts, one party always
wins?}

\subsection{Leader Survival}

{\large Building a small coalition is key to survival. The smaller the number of
people to whom a leader is beholden the easier it is for her to persist in
office. Autocrats and democrats alike try to cull supporters. It remains very
difficult to measure the size of coalitions precisely. However, if we arrange
political systems into broad groups of autocracy and democracy, then we can
compare the survival of different political leaders.}

{\large FIGURE 3.1 Maryland's 3rd Congressional District Figure 3.2 looks at the
risk for democrats and autocrats of being replaced given different lengths of
time that they have already been in office. On average, for instance, democrats
who make it through the first six months in office have about a 43 percent chance
of being out by the end of their second year; autocrats only have about a 29
percent chance of being ousted in the same amount of time.19 Making it to ten
years, democrats are three times more likely to be replaced than their
autocratic, small-coalition counterparts.These simple comparisons, however, miss
an interesting and important detail. Although autocrats survive longer, they find
surviving the initial period in office particularly difficult. During their first
half year they are nearly twice as likely to be deposed as their democratic
counterparts.}

{\large However, if they survive those first turbulent months, then they have a
much better chance of staying in power than democrats. Those early months are
difficult because they have not yet worked out where the money is, making them
unreliable sources of wealth for their coalition, and they have yet to work out
whose support they really need and who they can dump from their transitional
coalition. But once autocrats have reshaped and purged their supporters, survival
becomes easier. Democrats, in contrast, are constantly engaged in a battle for
the best policy ideas to keep their large constituencies happy. As a result,
although democrats survive the early months in office more easily (they get a
honeymoon), the perpetual quest for good policy takes a toll, such that only 4
percent of democrats survive in office for ten or more years. Nearly three times
as many autocrats manage to accomplish this feat, 11 percent.}

{\large FIGURE 3.2 The Risk of Ouster by Type of GovernmentStaying in power
right after having come to power is tough, but a successful leader will seize
power, then reshuffle the coalition that brought him there to redouble his
strength. A smart leader sacks some early backers, replacing them with more
reliable and cheaper supporters. But no matter how much he packs the coalition
with his friends and supporters, they will not remain loyal unless he rewards
them. And as we will see in the next chapter, rewards don't come cheaply.}
\pagebreak{}


\section{Chapter 4 - Steal from the Poor, Give to the Rich}

{\large WHETHER YOU'RE TAKING CHARGE OF THE OTTOMAN Empire, a corporation, or
Liberia, controlling the flow of funds is essential to buying support. However
once you've emptied the state's or the corporate coffers by buying off both your
essential supporters and their replacements, if necessary, you must reckon with
the entirely new challenge of refilling the treasury. If a leader cannot find a
reliable source of income, then it is only a matter of time until someone else
will offer his supporters greater rewards than he can.}

{\large Money is essential for anyone who wants to run any organization.}

{\large Without their share of the state's rewards, hardly anyone will stick
with an incumbent for long. Liberia's Prince Johnson knew this when he tortured
Samuel Doe, demanding the number of the bank accounts where the state's treasure
had been hidden. Without getting his question answered, Johnson would not be able
to secure power for himself. In fact, neither he nor rival insurgent Charles
Taylor could secure state revenue enough to buy control of Liberia's government
immediately after Doe was overthrown.}

{\large The upshot: Samuel Doe died under Prince Johnson's torture without
answering the question, and Liberia degenerated into civil war. Each faction was
able to extract enough resources to buy support in a small region, but no one
could control the state as a whole.}

{\large The succession process in the Ottoman Empire is another illustration of
the same point. Upon the death of their father, the Ottoman princes rushed from
their provinces to secure the treasury, buy the loyalty of the army and have all
their potential rivals (also known as brothers) strangled. Whoever first secured
control over the money was likely to win. If no one son triumphed, cleanly
wresting the treasury out of his siblings' control, then no one could summon up
the necessary revenues to pay his backers. The common result, as in Liberia, was
civil war.``Knowing where the money is'' is particularly important in
autocracies--- and particularly difficult. Such systems are shrouded in secrecy.}

{\large Supporters must be paid but there are no accurate accounts detailing
stocks and flows of wealth. Of course, this lack of transparency is by design.1
Thus does chaotic bookkeeping become a kind of insurance policy: it becomes
vastly more difficult for a rival to promise to pay supporters if he cannot match
existing bribes, or, for that matter, put his hands on the money. Indeed, secrecy
not only provides insurance against rivals, it also keeps supporters in the dark
about what other supporters are getting. Anyone who has tried to read the annual
reports of publicly traded firms will quickly realize that this is a practice
induced by dependence on a small winning coalition. In the corporate setting
opacity occurs despite having to satisfy strict regulations and accounting
standards. Secrecy ensures that everyone gets the deal they can negotiate, not
knowing how much it might cost to replace them. Thus every supporter's price is
kept as low as possible, and woe to any supporter who is discovered trying to
coordinate with his fellow coalition members to raise their price.}

{\large As we saw at the end of the last chapter, it is very difficult for
autocrats to survive their early months in office. Good governance is a luxury
they cannot afford at a time when they must scramble to find revenues. Little
surprise, then, that we so often see looting, confiscations, extraction, and fire
sales during political transitions, or conversely, and perhaps ironically,
temporary liberal reforms by would-be dictators who are mindful that it is easier
for a public goods--producing democrat than an autocrat to survive the first
months in office. Thus it is that in the immediate aftermath of a leadership
transition we see a few new leaders acting as if they care about the people and
many new leaders seizing the people's wealth and property. Such confiscations of
property might well damage long-run revenue, but if a leader does not find money
in the short term then the long run is someone else's problem.}

{\large Democrats are generally fortunate enough to know where most of the money
is. When David Cameron became prime minister of Britain or Barack Obama became
president of the United States, neither needed to torture their predecessor to
find the money. Because democracies have well-organized and relatively
transparent treasuries, their flow of funds is left undisturbed by leader
turnover. There are two reasons for thistransparency. First, as we are about to
explore, democratic leaders best promote their survival through policies of open
government. Second, a larger proportion of revenue in democracies than in
autocracies tends to be from the taxation of people at work. Such taxes need to
be levied in a clear and transparent way, because just as surely as leaders need
money, their constituents want to avoid taxes.}

\subsection{Taxation}

{\large We all hate taxes and are impressively inventive in looking for ways to
avoid them. Leaders, however, are rather fond of taxes---as long as they don't
have to pay them. Being a dictator can be a terrific job, but it also can be
terribly stressful, especially if money is in short supply. Taxes are one of the
great antidotes to stress for heads of governments. Taxes, after all, generate
much-needed revenue, which can then reward supporters. As a general principle
leaders always want to increase taxes. That gives them more resources with which
to reward their backers and, not to be forgotten, themselves. Nevertheless, they
will find it difficult to raise taxes with impunity.}

{\large Leaders face three constraints on how much money they can skim from
their subjects. First, taxes diminish how hard people work. Second, some of the
tax burden inevitably will fall upon the essential backers of the leader.}

{\large (In general, the first constraint limits taxes in autocracies and the
second constraint sets the boundary on taxes in democracies.) The third
consideration is that tax collection requires both expertise and resources.}

{\large The costs associated with collecting taxes limit what leaders can
extract and shapes the choice of taxation methods.}

{\large The first and most common complaint about taxes is that they discourage
hard work, enterprise, and investment. This is true. People are unlikely to work
as hard to put money in government coffers as they do to put money in their own
pocket. Economists often like to express taxation and economic activity in terms
of pies---when taxes are low, they say, the people work hard to enlarge the pie,
but the government only gets a thin slice of the pie. As the government increases
taxes, its share of the pie increases but people begin to do less work so the
overall size of the pie shrinks. If the government sets tax rates to be extremely
low or extremely high, its take will approach zero. In the first case it gets
very little of a large pie; in the latter case there is hardly any pie because
hardly anyone works.Somewhere between these extremes there is an ideal tax rate
that produces the most revenue the state can get from taxation. What that ideal
rate is depends on the precise size of the winning coalition. That, in fact, is
one of many reasons that it is more helpful to talk about organizations in terms
of how many essentials they depend on than to talk about imprecise notions such
as autocracy or democracy. The general rule is that the larger the group of
essentials, the lower the tax rate. Having said that, we return to the less
precise vocabulary of autocracy and democracy, but always mindful that we really
mean smaller or bigger coalitions.}

{\large Autocrats aim for the rate that maximizes revenue. They want as much
money as possible for themselves and their cronies. In contrast, good governance
dictates that taxes should only be taken to pay for things that the market is
poor at providing, such as national defense and large infrastructure projects.
Taking relatively little in taxes therefore encourages the people to lead more
productive lives, creating a bigger pie. Democrats are closer to this good
governance ideal than autocrats, but they too overtax. The centerpiece of
Reaganomics, the economic plan of US president Ronald Reagan (1981--1989), was
that US taxes were actually higher than this revenue maximizing level. By
reducing taxes, he argued, people would do so much extra work that government
revenue would actually go up. That is, a smaller share of a bigger pie would be
larger than the bigger share of a smaller pie. Such a win-win policy proved
popular, which is why similar appeals are again in vogue. Of course, it did not
quite work out this way in fact.}

{\large To a certain extent, Reagan was right: lower taxes encouraged people to
work and so the pie grew. However, crucially, in democracies it is the
coalition's willingness to bear taxes that is the true constraint on the tax
level. Since taxes had not been so high as to squash entrepreneurial zeal in the
first place, there wasn't much appreciable change as a result of Reagan's tax
cuts. The pie grew a little, but not by so much that revenues went up.}

{\large Today, the Tea Party wing of the Republican Party seeks to reenact
taxcutting policies similar to Reagan's. Like him, they argue that tax cuts will
grow the economy. The lesson from the Tea Party movement's electoral success in
2010 is that people don't like paying taxes. Politician who raise or even
maintain current taxes are politically vulnerable, but then so too arepoliticians
who fail to deliver the policies their coalition wants. Herein lies the rub. It
may well be that cutting taxes, while increasing the size of the economic pie,
fails to make it big enough to generate both more wealth and more effective
government policies. The question is and always must be the degree to which the
private sector's efficient but unequal distribution of wealth trumps government's
more equitable, less efficient, but popular economic programs.}

{\large Ruling is about staying in power, not about good governance. To this
end, leaders buy support by rewarding their essential backers relative to others.
Taxation plays a dual role in generating this kind of loyalty. First it provides
leaders with the resources to enrich their most essential supporters. Second, it
reduces the welfare of those outside of the coalition.}

{\large Taxation, especially in small-coalition settings, redistributes from
those outside the coalition (the poor) to those inside the coalition (the rich).
Small coalition systems amply demonstrate this principle, for these are places
where people are rich precisely because they are in the winning coalition, and
others are poor because they are not. Phillip Chiyangwa, a prot\'{e}g\'{e} of
Robert Mugabe in Zimbabwe, has stated it bluntly, ``I am rich because I belong to
Zanu-PF [Mugabe's ruling party].''2 When the coalition changes, so does who is
rich and who is poor.}

{\large Nor is Zimbabwe an isolated case. Robert Bates, a professor of
government at Harvard University, described the link between wealth and political
backing in Kenya: I recall working in western Kenya shortly after Daniel Arap Moi
succeeded Jomo Kenyatta as President of Kenya. With the shift in power, the
political fortunes of elite politicians had changed. As I drove through the
highlands, I encountered boldly lettered signs posted on the gateways of farms
announcing the auction of cattle, farm machinery, and buildings and lands. Once
they were no longer in favor, politicians found their loans cancelled or called
in, their subsidies withdrawn, or their lines of business, which had once been
sheltered by the state, exposed to competition. Some whom I had once seen in the
hotels of Nairobi, looking sleek and satisfied, I now encountered in rural bars,
looking lean and apprehensive, as they contemplated the magnitude of their
reversal.3Needless to say, people want to be sleek and satisfied and not lean and
apprehensive. That is why they remain loyal. A heavy tax burden emphasizes the
differences between being rich and poor---in or out of the coalition. At the same
time, the resulting revenues fund spoils for the lucky few, leaving little for
everyone else. Further, the misery such heavy taxes inflict on the general
population makes participation in the coalition even more valuable. Fearing
exclusion and poverty under an alternative leadership, supporters are all the
more fiercely loyal. They will do anything to keep what they have and keep on
collecting goodies. Gerard Padr\'{o} i Miguel of the London School of Economics
has shown that the leaders of numerous African nations tax ``too'' highly (that
is, beyond the maximum revenue point) and then turn around and provide subsidies
to chosen groups. This may be economic madness, but it is also political genius.4
Democrats tax heavily too and for the same reason as autocrats: they provide
subsidies to groups that favor them at the polls at the expense of those who
oppose them. We will see, for instance, that Democrats and Republicans each use
taxation when they can to redistribute wealth from their opponents to their
supporters. So democratic governments also have an appetite for taxation but they
cannot indulge that appetite to the extent autocrats can. Since their numbers are
small, an autocrat can easily compensate his essential backers for the tax burden
that falls on them.}

{\large This option is not available to a democrat because his number of
supporters is so large. Tax rates are therefore limited by the need to make
coalition members better off than they can expect to be under alternative
leadership. On the campaign trail, US president George H. W. Bush told the
American people, ``Read my lips, no new taxes.'' Yet, budget shortfalls left him
scrambling for revenue. The result was more taxes. In the wake of the First Gulf
War, just eighteen months earlier, Bush had approval ratings of over 90 percent.
But a declining economy and his broken promise on taxes led to his ouster in the
1992 election. While all leaders want to generate revenue with which to reward
supporters, democratic incumbents are constrained to keep taxes relatively low. A
democrat taxes above the good governance minimum, but he does not raise taxes to
the autocrat's revenue maximization point.}

{\large The relationship between regime type and taxation can be seen in
therecent history of Mexico. Mexico's first free election came in 1994, and the
incumbent party, the Partido Revolucionario Institucional (PRI), lost nationally
for the first time in 2000. As can be seen in Figure 4.1, onset of competitive
elections (and of democratization) marks the start of the decline in government
revenue as a percentage of gross domestic product (GDP). As the size of the
winning coalition enlarged, Mexico's tax rates followed suit by declining, just
as they should when politicians need to curry favor with many instead of a few.
For instance, the highest marginal tax rate in Mexico in 1979, with the PRI
firmly in control, was 55 percent. As the PRI's one-party rule declined, so did
tax rates. By 2000, marking the first truly free, competitive presidential
election, Mexico's highest tax rate was 40 percent.5 Members of any autocrat's
small coalition also dislike paying taxes, but they readily endorse high taxes
when those taxes are used to funnel great wealth back to them. This was just the
case in Bell, California. City Manager Robert Rizzo raised property taxes. The
city council could have stopped such increases, but had they done so, the city
could not have afforded their bloated consultancy fees. Suppose for simplicity
that Rizzo's coalition was composed of 1 percent of Bell's 36,000 residents. For
every dollar increase in tax per person, Rizzo would have had up to \$100 of
services and payments he could transfer to each coalition member. Had his
coalition been composed of half of Bell's residents, each dollar increase in tax
would provide only \$2 per coalition member for transfers and services. It is
easy to see why the coalition would sooner endorse higher taxes in the former
setup than the latter.}

{\large FIGURE 4.1 Mexico's Tax Take and DemocratizationNevertheless, to most of
us who live in democracies, the idea that our taxes are actually less than in
other systems might sound frankly absurd. If you live in New York City, as we do,
you pay federal, state, and local taxes (as well as social security, Medicare,
and sales tax). If you earn a reasonably good income, then income taxes suck up
about 40 percent of your earnings. By the time you factor in sales, property, and
other taxes, a reasonably wealthy New Yorker will have paid more than half her
income in tax---hardly a low figure. European democracies, with their extensive
social safety nets and universal health care, can tax at even higher rates. In
contrast, some autocracies don't even have income taxes. But the comparison of
average tax rates is misleading.}

{\large At the income levels taxed in much of the world's poor autocracies, the
tax rate in Europe and the United States is zero. We have to compare taxes at
given income levels, not across the board, since most income tax systems are
designed to be progressive, taxing higher incomes at higher rates than lower
incomes. By looking at how much tax has to be paid at a given income level across
countries we get close to comparing apples to apples and oranges to oranges. In
the United States, for example, a couple with one child and an income under about
\$32,400 pays no income tax. If their income were, say, \$20,000 they would
receive \$1,000 from thefederal government to help support their child. In China,
a family with an income of \$32,400 is expected to pay about \$6,725 in income
tax.6 Further, even when nominal rates are low, autocracies have high implicit
taxes---if you have something valuable then it simply gets taken.7 It's worth
remembering that the wealthiest man in China and the wealthiest man in Russia are
both currently in prison.}

{\large In 2004, Mikhail Khodorkovsky was the wealthiest man in Russia and the
sixteenth wealthiest man in the world. He made his money building up Yukos, an
oil company founded in the privatization wave in Russia in 1993.}

{\large Yukos was the largest nonstate oil company in the world and accounted
for about 20 percent of Russian oil production. Khodorkovsky, who had initially
been close to the government, spoke out about Putin's autocratic rule of Russia
and he funded several opposition political parties. In 2003, he was arrested on
fraud charges and subsequently convicted. The Russian government accused Yukos of
tax evasion. According to Yukos, the tax take claimed by Russia from Yukos was
substantially higher than that levied on other oil companies and, in some years,
exceeded gross revenue. These enormous tax burdens forced Yukos into bankruptcy.
With the end of his first eight-year sentence, Khodorkovsky, apparently still
seen as a liability by the Russian government, was recently given a second
sentence for embezzlement and money laundering.}

{\large His Chinese counterpart, Huang Guangyu, also known as Wong Kwong Ku,
fared little better. Starting with nothing but \$500 and a street cart, Guangyu
created Gome, the largest electrical retailer in China. He was repeatedly ranked
as China's richest individual---until he was sentenced to fourteen years in
prison for bribery. It is likely that he was guilty since bribery is commonplace
in Chinese business dealings. It is also likely that he and others who have been
prosecuted for corruption in China were ``chosen for political reasons.''8 In
autocracies, it is unwise to be rich unless it is the government that made you
rich. And if this is the case, it is important to be loyal beyond all else. As we
noted, it is quite possible that Guangyu and Khodorkovsky were both guilty of
fraud and bribery. That is the nature of business in their respective countries.
Even so, many others were surely guilty of the same crimes and yet walk free
today. What singled them out was that they did not support the government and
they had enormous wealth. White farmersin Zimbabwe suffer a similar fate. Robert
Mugabe's government seizes their land. The cover for these seizures is land
redistribution to poor blacks who were dispossessed under colonial and white
minority rule. The reality is much different. The land invariably ends up in the
hands of cronies, none of whom are farmers. When the new owners invariably allow
the land to fall into disuse, the farmers lose their investments, farm workers
are evicted from their houses, and Zimbabwe, once a huge agricultural exporter,
becomes hungrier. But on the other side of the ledger, Robert Mugabe is still in
power.}

{\large Democrats are less inclined to rewrite the rules and seize wealth.}

{\large Tempting though extra revenue is, it comes at the cost of lost
productivity to the masses. In Shakespeare's Merchant of Venice, the heroine,
Portia, disguises herself as a judge and adjudicates at the trial between
Antonio, who pledged his person as security on a loan, and Shylock, who demands
his pound of Antonio's flesh when Antonio does not pay in time.9 Bassanio,
Antonio's friend (and Portia's husband), offers to pay many times the debt due
and when Shylock refuses, he appeals to the mercy of the court: ``And I beseech
you, Wrest once the law to your authority: To do a great right, do a little
wrong. . . .'' But Portia recognizes the sanctity of the law: ``It must not be;
there is no power in Venice Can alter a decree established: 'Twill be recorded
for a precedent, And many an error by the same example, Will rush into the state:
it cannot be.''' The many messages of the Merchant of Venice are complicated and
controversial, but one message, epitomized by the passage just quoted (but not
the reversal in the enforcement of contracts just a bit later in the play)
reminds us that rule of law is essential to successful commerce. As one
examination of the demands of commerce as seen in the Merchant of Venice makes
clear both for Venice and in general, ``Contract does not require friendship, but
it does require a degree of trust that the market is well-regulated or that the
institutions of contract enforcement are appropriately strong.''}

\subsection{Tax Collectors}

{\large Democrats need resources so they can reward their coalition, but they
can't take too much or they risk alienating those very same supporters.}

{\large Similar concerns shape how taxes are collected. Leaders want to collect
taxes in a ``fair'' or at least transparent way. Few US citizens would regard the
Internal Revenue Service (IRS) as a transparent tax authority, but it is at least
governed by rules (albeit an awful lot of them) and enforced by an independent
judiciary. As for all the rules and exceptions that make the US tax code so
complicated, these inevitably result from politicians doing what politicians
inevitably do: rewarding their supporters at the expense of everybody else. This
is why sheaves of pages in the tax code are dedicated to farmers---a crucial
coalition for some politicians, who need to receive their rewards if their
senators and representatives are to remain in power.}

{\large Autocrats can be less transparent. As we saw in the case of the
unfortunate Messrs Khodorkovsky and Guangyu, when the opportunity arises
autocrats will grab whatever they can. Yet even as they work without the
constraint of being bound by people's feelings, autocrats face real issues in the
realm of collecting taxes. High taxes will inevitably drive people to hide their
work and profits. This makes monitoring their income difficult. Furthermore, the
large bureaucracy required to run a comprehensive tax system, such as the one in
the United States, can be prohibitively expensive. To put this in context, the
US's Internal Revenue Service spends about \$38 per person, or about 0.5 percent
of the IRS take, on collecting an average of \$7,614 in tax per person.11 This is
fine in a nation with per capita GDP of \$46,000, but in nations with incomes of
only \$1,000 per year, such a cost of collecting taxes would be about 23 percent
of the revenue. Further, setting up a large bureaucracy makes an autocrat
beholden to those who run it. The first rule of office holding is to minimize the
number of people whose support you need. To avoidbecoming a slave of their own
tax collectors, autocrats often use indirect taxation instead. With indirect
taxes, the cost of the tax is passed on to someone other than the person actually
paying it. For instance, sellers pay sales taxes to municipal governments but
sellers pass the cost on to buyers, making sales taxes indirect.}

{\large Agricultural marketing boards are a common indirect means of taxing poor
farmers in autocracies. In principal such organizations are designed to fulfill a
similar function as the European Union's Common Agricultural Policy (CAP). The
CAP guarantees farmers minimum prices for their goods---thus it provides a
benefit to the farmers. In many democracies, as in the United States, rural areas
are overrepresented electorally. Given the desire to rule with as few supporters
as possible, it should be of little surprise that democrats often include farm
groups in their coalition and reward them accordingly.12 In contrast, farmers are
rarely key supporters in autocracies. Farm marketing boards are set up to
exploit, rather than help them. Consider Ghana's Cocoa Marketing Board (CMB).
Cocoa is Ghana's major agricultural export. The CMB fixes a price for cocoa---an
implicit tax---and insists that farmers sell all their cocoa to the board at that
price, an indirect tax. The board then resells the cocoa on world markets at a
higher price and pockets the difference: ``The first rung in the long ladder of
leeches that feed on the sweat of the cocoa farmers is the Cocoa Marketing
Board.''13 These rents have been a major source of government revenue in Ghana.}

{\large Taxing the poor to pay the rich has plenty of bad economic consequences,
but these tend to be ``in the long run''---that is, on another leader's watch.
For instance, in Ghana, heavily taxing famers had the longer term consequence of
reducing crops. Ghanaian farmers simply stopped planting and caring for cocoa
trees. By the 1980s cocoa production had collapsed and farmers tried to smuggle
what little they did grow to neighboring C\^{o}te d'Ivoire. Case after case
proves the point: when taxes are too high, then people either stop working or
they find ways to avoid the formal economy.}

\subsection{Privatized Tax Collection}

{\large When even indirect taxation proves to be too much trouble, autocrats
sometimes turn to outsiders for help extracting funds from their people. For
autocrats and for their tax collectors this has a virtue and a liability. People
hired to extract money for the government, keeping a portion of what they collect
for themselves, have a strong incentive to take in lots of tax revenue.}

{\large That's good for them and good for the leader who receives the
substantial remainder not kept by prudent tax collectors. But people hired to
extract money can also use the power of that money to become a threat to a
leader, and that, of course, is dangerous for them and for the incumbent.}

{\large The Caliphate was the Muslim empire created by military conquest
following the death of the Prophet Muhammad in 632. It ruled much of the Middle
East, North Africa, and parts of Europe until 1258. In the tradition of the
Romans before them, the caliphs avoided the technical difficulties of tax
collection by outsourcing the task altogether. A tax farmer would pay the
treasury for the right to collect taxes from a particular territory.}

{\large Obviously, once they had paid for the privilege, tax farmers extracted
everything they could. They were notoriously brutal and always looking for
ingenious ways to take more. For instance, they would demand payment in silver
coins rather than crops, and then collude with merchants to fix prices.}

{\large Those who could not pay were punished or even killed.}

{\large Naturally, the people resisted. Tax farmers contended with a persistent
problem of people fleeing the land rather than paying their property taxes.}

{\large To prevent this, tax farmers set up patrols to check identities.
Non-Muslims were often tattooed or forced to wear ``dog tags'' with their name
and address to prevent them from fleeing.14 Initially some of the tax collected
by the tax farmers was only applicable to non-Muslims. This proved to be a very
successful, if not wholly intended, means of encouraging religious conversion. It
seems that many non-Muslims, realizing that they could reduce the tax collectors'
reach by becoming Muslim, put their religiousbeliefs aside and converted. As long
as these conversions did not assume massive proportions, the tax farmers made
themselves incredibly rich at the expense of the average citizen. When conversion
became commonplace, tax farmers adjusted, no longer excluding Muslims from some
of the taxes they levied. And from the perspective of the Caliph, they ensured
reliable revenue. That they terrorized the people was of no political importance:
impoverished and persecuted farmers were not part of the winning coalition.}

{\large Autocrats can avoid the technical difficulties of gathering and
redistributing wealth by authorizing their supporters to reward themselves
directly. For many leaders, corruption is not something bad that needs to be
eliminated. Rather it is an essential political tool. Leaders implicitly or
sometimes even explicitly condone corruption. Effectively they license the right
to extract bribes from the citizens. This avoids the administrative headache of
organizing taxation and transferring the funds to supporters.}

{\large Saddam Hussein's sons were notorious for smuggling during the 1990s when
Iraq was subject to sanctions. They made a fortune from the sanctions that were
supposed to harm the regime.}

\subsection{Extraction}

{\large ``Oil is the Devil's excrement,'' at least according to Juan Pablo Perez
Alfonzo, a Venezuelan who founded the Organization of Petroleum Exporting
Countries (OPEC), the cartel of oil-producing nations. ``Ten years from now,
twenty years from now, you will see: oil will bring us ruin.'' And he was right.}

{\large As many leaders have learned, the problem with raising revenue through
taxation is that it requires people to work. Tax too aggressively or fail to
provide an environment conducive to economic activity and people simply don't
produce. Actually extracting revenue from the land itself provides a convenient
alternative, cutting the people out of the equation altogether.}

{\large Take oil, for example. It flows out of the ground whether it is taxed at
0 percent or 100 percent. Labor represents but a small part of the value of oil
extraction. This makes it a leader's dream and the people's nightmare. In a
phenomenon often called the resource curse, nations with readily extractable
natural resources systematically underperform nations without such resources.15
Resource-rich nations have worse economic growth, are more prone to civil wars,
and become more autocratic than their resource-poor counterparts.}

{\large Nigeria, the most populous nation in Africa, achieved independence from
Britain in 1960. At the time of independence it was a poor nation, but
expectations were high. These expectations grew with the discovery of oil.}

{\large Nigeria is believed to have the world's tenth largest reserves. With the
rise in oil prices during the oil crises in the early and late 1970s, Nigeria
found itself awash with funds. And yet, by the early 1980s the country was
swamped by debt and poverty. From 1970 to 2000, Nigeria had accumulated \$350
billion in oil revenue.16 It has not helped the people.}

{\large Over the same years, average annual income per capita went from
US\$1,113 in 1970 to US\$1,084 in 2000, making Nigeria one of the poorest nations
in the world, in spite of its vast oil wealth. Poverty has risentoo. One dollar
per day is a common standard used for assessing poverty: in 1970, 36 percent of
Nigerians lived on less; by 2000 this figure had jumped to nearly 70 percent. The
situation can hardly be said to have improved since then. Even with today's
inflated dollars, a majority of Nigerians earn less than a dollar a day and per
capita income has continued to fall. Adjusted for inflation, income is below what
it was when Nigeria became independent.}

{\large Nigeria is not exceptional. Figure 4.2 shows exactly that. The
horizontal axis shows natural resource exports as a percentage of GDP in 1970.
The vertical axis shows the average level of economic growth between 1970 and
1990. The trend is clear. Nations flush with oil, copper, gold, diamond, or other
minerals grow more slowly.}

{\large Nevertheless, natural resources are wonderful for leaders. Unlike
getting their subjects to work, leaders don't have to encourage natural resources
to work. Admittedly the minerals need to be extracted, but by and large autocrats
can achieve this without the participation of the local population.}

{\large In Nigeria, for instance, the oil is concentrated in the Niger Delta
region.}

{\large Foreign firms with foreign workers do most of the extraction. Few
Nigerians participate. The oil companies run security firms, effectively small
private armies, to keep the locals from obstructing the business or complaining
about the environmental degradation that results. BP and other foreign firms are
free to act with impunity, provided they deliver royalty checks to the
government. This is not so much a failing of these companies as the way business
must be conducted in countries whose leaders rely on a few cronies to back them
up. A company that acts responsibly will necessarily have less money to deliver
to the government and that will be enough for them to be replaced by another
company that is willing to be more ``cooperative.'' FIGURE 4.2 Growth and Natural
Resource Abundance, 1970--1990One interesting manifestation of the differences
between wealth and poverty in resource-rich lands is the cost of living for
expatriates living in these countries. While it is tempting to think that cities
like Oslo, Tokyo, or London would top the list as the most expensive places, they
don't. Instead it is Luanda, the capital of the southwestern African state of
Angola. It can cost upwards of \$10,000 per month for housing in a reasonable
neighborhood, and even then water and electricity are intermittent. What makes
this so shocking is the surrounding poverty. According to the United Nations
Development Program, 68 percent of Angola's population lives below the poverty
line, more than a quarter of children die before their fifth birthday, and male
life expectancy is below forty-five years. The most recent year for which income
inequality data are available is 2000. These data suggest that the poorest 20
percent of the population have only 2 percent of the wealth. Angola is ranked 143
out of 182 nations in terms of overall human development. Prices in Angola, as in
many other West African states, are fueled by oil.}

{\large The resource curse enables autocrats to massively reward their
supporters and accumulate enormous wealth. This drives prices to the
stratospheric heights seen in Luanda, where wealthy expatriates and
luckycoalition members can have foie gras flown in from France every day. Yet to
make sure the people cannot coordinate, rebel, and take control of the state,
leaders endeavor to keep those outside the coalition poor, ignorant, and
unorganized. It is ironic that while oil revenues provide the resources to fix
societal problems, it creates political incentives to make them far worse.}

{\large This effect is much less pernicious in democracies. The trouble is that
once a state profits from mineral wealth, it is unlikely to democratize. The
easiest way to incentivize the leader to liberalize policy is to force him to
rely on tax revenue to generate funds. Once this happens, the incumbent can no
longer suppress the population because the people won't work if he does.}

{\large The upshot is that the resource curse can be lifted. If aid
organizations want to help the peoples of oil-rich nations, then the logic of our
survivalbased argument suggests they would achieve more by spending their
donations lobbying the governments in the developed world to increase the tax on
petroleum than by providing assistance overseas. By raising the price of oil and
gas, such taxes would reduce worldwide demand for oil.}

{\large This in turn would reduce oil revenues and make leaders more reliant on
taxation.}

{\large Effective taxation requires that the people are motivated to work, but
people cannot produce as effectively if they are forbidden such freedoms as
freedom to assemble with their fellow workers and free speech---with which to
think about, among other things, how to make the workplace perform more
effectively, and how to make government regulations less of a burden on the
workers.}

\subsection{Borrowing}

{\large Borrowing is a wonderful thing for leaders. They get to spend the money
to make their supporters happy today, and, if they are sensible, set some aside
for themselves. Unless they are fortunate enough to survive in office for a
really long time, repaying today's loan will be another leader's problem.
Autocratic leaders borrow as much as they can, and democratic leaders are
enthusiastic borrowers as well.}

{\large We are all at least a little bit impatient. It's in our nature to buy
things today when better financial acumen might suggest saving our money.}

{\large Politics makes financial decision making even more suspect. To
understand the logic and see why politicians are profligate borrowers, suppose
everyone in a country earns \$100 per year and is expected to do so in the future
too. The more we spend today, the more we must pay in interest and debt repayment
tomorrow. Suppose, for instance, that to spend an extra \$100 today we have to
give up \$10 per year as interest payments in the future. It is reasonable to see
that people could differ on whether this is a good idea or not, but politics
certainly makes it more attractive. To simplify the issue vastly, suppose leaders
simply divide the money they borrow among the members of their coalition. This
encourages leaders to borrow more. If a leader has a coalition of half the people
and he borrows an amount equivalent to \$100 per person, then everyone has to
give up \$10 in each future year (as taxes to pay the interest). However, since
the coalition is only half the population, each coalition member's immediate
benefit from the borrowing is \$200. While to some this might still not seem like
an attractive deal, it is certainly better than incurring the same debt
obligation for \$100. Governments of all flavors are more profligate spenders and
borrowers than the citizens they rule. And that profligacy is greatly multiplied
when we look at small coalition regimes.}

{\large As the size of the coalition shrinks, the benefits that the coalition
gains from indebtedness go up. If, for instance, the coalition includes one
personper hundred then, in exchange for the debt obligation, each coalition
member receives \$10,000 today instead of the mere \$200 in the 50 percent
coalition example. This is surely a deal that most of us would jump at. As the
coalition size becomes smaller, the incentive to borrow increases.}

{\large Of course, borrowing more today means higher indebtedness and a smaller
ability to borrow tomorrow. But such arguments are rarely persuasive to a leader.
If he takes a financially reasonable position by refusing to incur debt, then he
has less to spend on rewards. No such problem will arise for a challenger who
offers to take on such debt in exchange for support from members of the current
incumbent's coalition.}

{\large This makes the current leader vulnerable. Incurring debt today is
attractive because, after all, the debt will be inherited by the next
administration. That way, it also ties the hands of any future challenger.}

{\large A leader should borrow as much as the coalition will endorse and markets
will provide. There is surely a challenger out there who will borrow this much
and, in doing so, use the money to grab power away from the incumbent. So not
borrowing jeopardizes a leader's hold on power. Heavy borrowing is a feature of
small coalition settings. It is not the result, as some economists argue, of
ignorance of basic economics by third-world leaders.}

{\large In an autocracy, the small size of the coalition means that leaders are
virtually always willing to take on more debt. The only effective limit on how
much autocrats borrow is how much people are willing to lend them. Earlier we saw
the paradoxical result that as Nigeria's oil revenues grew so did its debt. It
wasn't that the oil itself encouraged borrowing---autocrats always want to borrow
more. Rather revenues from oil meant that Nigeria could service a larger debt and
so people were more willing to lend.}

{\large Although the large coalition size in a democracy places some
restrictions on the level of borrowing, democratic leaders are still inclined to
be financially irresponsible. Remember, while the debt is paid by all, the
benefits disproportionately flow to coalition members. Over the last ten years
the economies of many Western nations boomed. This would have been a perfect time
to reduce debt. Yet in many cases this did not happen.}

{\large In 1990, US debt was \$2.41 trillion, which was equivalent to 42 percent
of GDP. By 2000 this debt had grown in nominal terms to \$3.41 trillion,although
in relative terms this was a decline to 35.1 percent of GDP.}

{\large However, as the economy prospered during the 2000s, debt continued to
slowly accumulate instead of shrink. In 2007, before the financial crisis, US
debt stood at \$5.04 trillion or 36.9 percent of GDP. A bigger economy means a
greater ability to service debt and a capacity to borrow more.}

{\large We may be inclined to explain the expansion of the debt by citing the
party politics of the leader in charge. However, ideology offers a poor account
of these trends. The major accumulations of US debt in the postwar period both
began under Republican administrations: Ronald Reagan (1981--1989) and George W.
Bush (2001--2009). This debt grew at a staggering pace during the 2007--2010
recession as the United States underwrote troubled banks and embarked on
Keynesian policies of fiscal stimulus. By the third quarter of 2010, debt was
\$9.13 trillion or 62 percent of GDP. British debt follows a similar pattern. In
2002 debt stood at 29 percent of GDP, but by 2007 it was 37 percent and this has
exploded in the wake of the 2008 financial crisis to 71 percent of GDP.}

{\large From a Keynesian perspective, many governments are taking the perverse
steps of trying to cut spending during a recession instead of stimulating demand.
This does not reflect a desire by politicians to borrow less. Rather debt crises
in Iceland, Greece, and Ireland have led many investors to doubt the ability of
nations to repay. This has pushed up the cost of borrowing and made it much
harder to secure new loans. It is supply, not demand, that has shrunk.}

{\large Markets limit how much a nation can borrow. If individuals borrow too
much and either cannot or will not repay it, then banks and other creditors can
seize assets to recover the debt. With sovereign lending to countries, however,
creditors cannot repossess property. On a few occasions creditors have tried. For
instance, France invaded Mexico in 1862 in an attempt to get Mexico to repay
loans. France also invaded the Ruhr, an industrial area of Germany, in 1923 to
collect reparation payments due from World War I that Germany had not paid. Both
attempts failed. In practice, the only leverage lenders have over nations is to
cut them off from future credit. Nevertheless, this has a profound effect, as the
ability to engage in borrowing in financial markets is valuable. For this reason
nations generally pay their debt.}

{\large However, once the value of access to credit is worth less than the cost
ofservicing the debt then leaders should default. If they don't then surely a
challenger will come along who will offer to do so. This was one of the appeals
of Adolf Hitler to the German people in the 1930s. Germany faced a huge debt, in
part to pay reparations from World War I. Hitler defaulted on this debt. It was a
popular policy with the German people since the cost of servicing the debt was so
high.}

{\large As debt approaches the balance point where the value of access to credit
equals the cost of debt service, lenders refuse to increase the overall size of
debt. At this point, if leaders want to borrow more, then they need to increase
revenues such that they could service this additional debt. As in the Nigerian
case, the discovery of exploitable natural resources provides one means to
increase debt service and hence more borrowing. However, without such
discoveries, the only way to increase borrowing is to increase tax revenue. For
autocratic leaders this means liberalizing their policies to encourage people to
work harder because they already tax at a high (implicit) rate. Only when facing
financial problems are leaders willing to even consider undertaking such
politically risky liberalization. They don't do it frequently or happily. They
liberalize, opening the door to a more democratic, representative and accountable
government only when they have no other path to save themselves from being
deposed today.}

\subsection{Debt Forgiveness}

{\large Debt forgiveness is a popular policy, but one that is generally
misguided.}

{\large Those in favor of forgiving the debt of highly indebted poor countries
argue that the debt burden falls on the poor people of the nation who did not
benefit in a consequential way from the borrowed funds. This is certainly true.
As we have explained, the benefits go to the leader and the coalition while the
debt obligation falls on everyone. But people who argue for debt forgiveness
construct their arguments in terms of how they think the world should operate,
rather than how it actually works.}

{\large In the late 1980s, as many poor nations struggled to repay debts,
creditors coordinated to reschedule and forgive debt. The French Ministry of the
Economy, Finance, and Industry became an important center for negotiations,
helping ensure that creditors shared similar losses. These meetings became known
as the Paris Club. In 1996 the International Monetary Fund (IMF) and World Bank
launched the Heavily Indebted Poor Country (HIPC) initiative. Instead of the
previous case-by-case approach, this program provided systematic help to poor
nations with writing down their debts. However, nations could only receive debt
relief when they passed or made substantial progress towards meeting explicit
criteria concerned with poverty alleviation and budget reform. The program
received a huge boost under the Millennium Goals program. From 2006 onwards many
HIPCs saw very large reductions in their debt. We will have to wait to see the
consequences of these programs. However, it is useful to look back at some of the
largest debt-relief efforts prior to 2000. It is particularly illustrative to
observe how important the nature of governance is. Even though creditors
carefully chose those nations that they thought would behave sensibly, in the
wake of debt relief many nations started increasing debt again.}

{\large As a percentage of debt, the largest debt reliefs prior to 2000 were
given to Ethiopia in 1999 (42 percent of debt), Yemen in 1997 (34percent),
Belarus in 1996 (33 percent), Angola in 1996 (33 percent), Nicaragua in 1996 (30
percent) and Mozambique in 1990 (27 percent). 17 With the exceptions of Angola
and Nicaragua, each of these nations promptly started reaccumulating debt. For
instance, after a series of small debt reductions, in 1999, with the forgiveness
of \$4.4 billion, Ethiopia had its debt reduced to \$5.7 billion. But by 2003
this debt had risen to \$6.9 billion. Despite the forgiveness of \$589 million of
debt in 1996, Belarus's debt has steadily risen from \$1.8 billion in 1995 to
over \$4.1 billion in 2005. Even though debt-reduction programs vet candidates,
these examples suggest that in many cases for-giveness without institutional
reform simply allows leaders to start borrowing again.}

{\large FIGURE 4.3 Mozambique's Debt and Forgiveness Democrats also like to
borrow but they are not as profligate as autocrats. They prefer lower levels of
indebtedness than autocrats.}

{\large Democratization promotes successful debt reduction, as the history of
Mozambique and Nicaragua illustrate. In 1990, 27 percent of Mozambique's debt was
forgiven. The result was the further accumulation of debt. By the time Mozambique
democratized in 1994, debt was over \$8 billion. However, this debt has gradually
been reduced, as seen in Figure 4.3, aided in part by further forgiveness.The
HIPC program has received much criticism for its slow pace in reducing the debt
of poor nations. Our criticism of HIPC is the opposite.}

{\large These programs are actually too eager to forgive debt. Debt forgiveness
simply allows autocratic leaders to start borrowing more money. As we'll see a
little later, financial crises are one of the important reasons leaders are
compelled to democratize. Debt reduction, however, relieves financial pressure
and enables autocrats to stay in office without reform, continuing to make the
lives of their subjects miserable.}

{\large It is no wonder that autocrats love debt relief. But how does debt
relief look to those who would like to improve governance and who want to start,
as we do, from understanding how leaders behave rather than engaging in wishful
thinking? We depart for the moment from looking at governance through a leader's
eyes and instead turn our attention to how to change an autocrat's vision. That
is, we turn for the moment to thinking about how we can use the logic of
dictatorial rule to give autocrats the right incentives to change their
government for the better. We wonder, can we create a desire by at least some
autocrats to govern for the people as the best way to ensure their own political
survival?}

{\large Debt reduction might work in democracies. Since such nations want to
reduce excessive debt anyway, debt reduction clearly helps speed the process. But
as can be seen in Figure 4.3, Mozambique was already tackling its debt problem
prior to large scale forgiveness in 2001 and in 2006. Therefore, we advocate a
conservative approach of little or no debt relief as a way to improve the quality
of governance and the quality of life of people currently living under wretched,
oppressive regimes. We know that debt relief allows autocrats to entrench
themselves in office. Debt forgiveness with the promise of subsequent
democratization never works.}

{\large An autocrat might be sincere in his willingness to have meaningful
elections in return for funds. Yet once the financial crisis is over and the
leader can borrow to pay off the coalition, any promised election will be a sham.
For democrats, debt relief, while helpful, is unnecessary. By eliminating debt
relief for autocrats we can help precipitate the sorts of rebellions seen in the
Middle East in 2011, rebellions that, as discussed later, may very well open the
door to better governments in the future.}

{\large Taxation, resource extraction, and borrowing are the foremost ways
ofacquiring funds for enriching a coalition. Discussions that portray taxation
differently are either window dressing to make the process seem more palatable or
are making arguments based on how people would like the world to work. Leaders
tax because they need to spend on their coalition.}

{\large Successful leaders raise as much revenue as they can. The limits of
taxation are: (1) the willingness of people to work as they are taxed; (2) what
the coalition is willing to bear; and (3) the cost of collecting taxes.}

{\large Having filled government coffers, leaders spend resources in three ways.
First they provide public goods. That is, policies that benefit all.}

{\large Second, they deliver private rewards to their coalition members. This
mix of private and public benefits differs across political systems, and it's
worth noting that any resources left over after paying off the coalition are
discretionary. Leaders therefore have a third choice to make about spending
money. They could spend discretionary money promoting their pet projects.
Alternatively, and all too commonly, as we shall see, they can hide them in a
rainy-day fund.}
\pagebreak{}


\section{Chapter 5 - Getting and Spending}

{\large AT LAST, A NEW RULER HAS SHAKEN UP THE COALITION that first brought him
to power and he has the right supporters in place. Money is coming in thanks to
the taxes being levied. Now comes the real task of governing: allocating money to
keep the coalition happy---but not too happy---and providing just enough to keep
the interchangeables from rising up in revolt. As we have seen in North Africa
and the Middle East in the past few years, and as we saw in Eastern Europe in the
1980s and 1990s, this can be an awkward tightrope to traverse for any leader. The
last few decades encourage us by showing that in time many autocrats fall off
that tightrope. It is really hard to strike just the right balance between
benefits for one's coalition and for the mass of interchangeables.}

{\large Any new incumbent who wants to be around for a long time needs to
fine-tune the art of spending money. Of course, he can err on the side of
generosity to the coalition or to the people---but only with any money that is
left for his own discretionary use after taking care of the coalition's needs.}

{\large He had better not err on the side of shortchanging anyone who could
mount a coup or a revolution. Shortchange the wrong people and any leader's fate
will confirm our abuse of William Wordsworth's famous line: ``Getting and
spending, we lay waste our powers.'' Thus we turn to the essential question of
all democracies: how to allocate resources aimed at providing policies of benefit
to everyone in a society. These public goods come in a variety of different forms
depending on the tastes of those in a position to demand such policies. Those in
such a position, of course, are the incumbent's essential backers. Different
groups of essentials will have different baskets of public goods--oriented
policies in mind. Some will want to spend more on a social welfare safety net;
others on education; still others on benefits for the elderly or for the young;
benefits to the arts and so forth. Although all of these are of interestand will
be touched upon, however briefly, we are especially interested in core public
benefits like education, health care, and such freedoms as a free press, free
speech, and freedom of assembly.}

{\large Although security against foreign invasion certainly is a central public
good, we leave consideration of foreign threats to a later chapter and focus here
on domestic policy choices. For now, let's have a look at how public goods can
help society as a whole and how they help entrenched leaders.}

\subsection{Effective Policy Need Not Be Civic Minded}

{\large To balance between spending policies that benefit the masses and those
that favor the essentials, leaders would do well to reflect on those parts of
Thomas Hobbes's philosophy of government, which we touched on briefly in the
introduction. He had a lot right, but Hobbes's ideas about government weren't
infallible. While he realized that anyone who enriched society would avoid a
revolution such as the one he lived through in England, he failed to distinguish
between what it takes to keep the people at bay and what it takes to keep
essential backers from betraying their leader---whether that leader is Hobbes's
Leviathan, Plato's Philosopher King, Rousseau's General Will, or Madison's
factionridden representatives of the people. Hobbes was sure his Leviathan had to
be a benign ruler.}

{\large That, so Hobbes seems to have thought, was the way to prevent a
revolution such as he experienced. Without a ruler who enriched his people,
Hobbes feared that for many, life would be, in his well-coined phrase, solitary,
nasty, poor, brutish, and short.}

{\large Hobbes was only half right. It is true, as Hobbes's believed, that
happy, well-cared-for people are unlikely to revolt. China's prolonged economic
growth seems to have verified that belief (at least for now). Keep them fat and
happy and the masses are unlikely to rise up against you. It seems equally true,
however, that sick, starving, ignorant people are also unlikely to revolt. All
seems quiet among North Korea's masses, who deify their Dear Leader as the sole
source of whatever meager, life-sustaining resources they have. Who makes
revolution? It is the great in-between; those who are neither immiserated nor
coddled. The former are too weak and cowered to revolt. The latter are content
and have no reason to revolt.}

{\large Truly it is the great in-between who are a threat to the stability of a
regime and its leaders. Therefore, a prudent leader balances resources between
keeping the coalition content and the people just fit enough to produce the
wealth needed to enrich the essentials and the incumbent. We should notbe
surprised that those countries whose governments rely on few essential
backers---that is, those that are least democratic---are the very places where
Hobbes's state of nature is most likely to be an apt description of life for the
masses. They are also, as we saw earlier, the places where leaders have the best
prospect of staying in control for years and years.}

{\large Leaders who depend on a large coalition have to work hard to make sure
that their citizens' lives are not solitary, nasty, poor, brutish, and short.}

{\large That doesn't mean democratic rulers have to be civic minded, nor would
they need to harbor warm and cuddly feelings for their citizens. All they need is
to ensure that there are ample public benefits to provide a high quality of life.
They just need to follow the rules by which successful leaders rule, adapting
them to the difficult circumstances that any democrat faces: being stuck with
dependence on an unruly crowd of essentials to keep them in power.}

{\large Just as democrats have no need of being civic visionaries, dictators
likewise aren't bound to make life miserable. It just happens, more often than
not, to work out better for them to do so. There are exceptions, but those
exceptions tend simply to reaffirm the importance of obeying the rules of
politics. As we have noted, it is okay for a leader to spend her pot of
discretionary money on trying to bring the good life to her citizenry. By
definition, discretionary money is money that is not required to keep the
coalition loyal; the coalition has already been paid off before any resources
become discretionary. Singapore, for example, has managed through benevolent
dictatorship to produce a high quality of material life for its citizens, albeit
without many of the freedoms that others hold dear. Maybe Lee Kwan Yew,
Singapore's long-time benefactor, was the embodiment of Hobbes's Leviathan. But
benevolent dictators like Singapore's are hard to find.}

{\large The most reliable means to a good life for ordinary people remains the
presence of institutional incentives in the form of dependence on a big coalition
that compels power-seeking politicians to govern for the people.}

{\large Democracy, especially with little or no organized bloc voting, aligns
incentives such that politicians can best serve their own self-interest,
especially their interest in staying in office, by promoting the welfare of a
large proportion of the people. That, we believe, is why most democracies are
prosperous, stable, and secure places to live.Perhaps you doubt that the path to
a good life is ensured by the presence of a big coalition. You wouldn't be alone.
Many distinguished economists, even quite a few with the Nobel Prize under their
belts, are convinced that the best way to promote democracy is by promoting
prosperity. That is why whenever they see an economic crisis looming on the
horizon, such as a government so indebted that it is on the verge of default and
bankruptcy, they call for debt forgiveness, new loans, lots of foreign aid, and
other economic fixes. They resist the cry of people like us who demand improved
governance before any bailout money is offered up to rescue a troubled autocratic
economy. They are convinced that wealth--- and not politics---is the better route
to escape Hobbes's state of nature.}

{\large Simply reviewing the history of economic bailouts that we began in the
previous chapter makes clear that the very forgiveness without political change
so eagerly embraced for third-world financial crises is rarely sought when the
crisis arises in a society that relies on a large coalition.}

\subsection{Bailouts and Coalition Size}

{\large The politics of economic bailouts can be quite different in small and
large coalition regimes. Bailouts come in many forms: shifts in domestic taxing
and spending; loans, whether from banks at home or abroad; debt forgiveness; or
foreign aid. Any bailout is accompanied by demands for economic reform, whether
the money comes from the IMF, the German Central Bank, or the taxpayers. A big
difference between large- and smallcoalition bailout recipients is that the
former almost always institute reforms and the latter only infrequently do.}

{\large Just like debt forgiveness, a bailout in the face of economic stress for
autocrats is a way to solve an impending political crisis. When their economy
becomes too feeble to provide sufficient money to buy political loyalty,
autocrats face being overthrown either by a rival or a revolution.}

{\large This, in a nutshell, is the story of the politico-economic crises faced
by places like Tunisia and Egypt in 2011. A bailout, whether generated from
within or through outside loans or aid, can buy off opposition and thwart the
threat to the leader's hold on power. Therefore, during an economic crisis
autocrats shop around for bailout money from others to save themselves in the
name of relieving their country's financial woes.}

{\large For large coalition leaders bailouts are a curse, or at least a
necessary evil. A poorly performing economy is likely to be understood by voters
as a policy failure by the leadership, resulting in their being thrown out by the
voters at the first opportunity. That was very much a part of the story of the
defeat of the Republican Party in 2008 and, when the economy did not turn around
fast enough to satisfy voters, the defeat of the Democrats in the House of
Representatives in 2010. So the need for an economic bailout strongly signals the
voters to find new leaders with new policy ideas.}

{\large Foreign aid rarely comes to the rescue of democrats, for reasons we
explain later. Therefore, financial crises and the need for a bailout are just
about always bad news for democrats.The rich countries of the world faced a
severe economic crisis in 2008 and 2009. Both the Bush and Obama administrations
sought to stem the worst of the crisis by providing massive financial bailouts to
save the banking industries and other large businesses, restore liquidity in the
market, and bring the US economy back onto a path toward sustained growth. Much
the same was done in Europe. These bailouts were accompanied by regulatory
change. The Dodd-Frank Wall Street Reform and Consumer Protection Act, signed by
President Obama in 2010, is a case in point. Faced with a severe recession, the
Congress passed the largest regulatory reform since the presidency of Franklin
Roosevelt.}

{\large In contrast, in small-coalition regimes, bailouts all too often are the
means to preserve business as usual. Economic bailouts in autocracies rarely
precipitate a serious review of economic or business policies. They are almost
never accompanied by regulatory reform. Consequently, economic crises happen more
often than in democracies and, so long as rich nations feel an urge to provide
loans, debt forgiveness, or aid, rarely result in betterment for the society
although they do result in security for lousy leaders.}

\subsection{Is Democracy a Luxury?}

{\large Are dictators and economists right that economic solvency needs to come
ahead of political change? Is becoming materially rich the precursor for the
luxury of democracy? We think not. There are plenty of well-to-do places that
nevertheless suffer under oppressive governments that keep ordinary peoples'
lives as solitary, nasty, poor, brutish, and short as their neighbors in poorer
nations. Take a look at just about any nondemocratic oil-rich or diamond-studded
regime.}

{\large Yes, the world has produced wise, well-intentioned leaders even among
those who depend on few essentials, but it neither produces a lot of them nor
does it ensure that they have good ideas about how to make life better for
others. Indeed, a common refrain among small-coalition rulers is that the very
freedoms, like free speech, free press, and especially freedom of assembly, that
promote welfare-improving government policies are luxuries to be doled out only
after prosperity is achieved and not before.}

{\large This seems to be the self-serving claim of leaders who keep their people
poor and oppressed. The People's Republic of China is the poster boy for this
view. When Deng Xiaoping introduced economic liberalization to China in the
1980s, experts in wealthy Western countries contended that now China's economy
would grow and the growth would lead to rapid democratization. Today, more than
thirty years into sustained rapid growth we still await these anticipated
political reforms. Growth does not guarantee political improvement but neither
does it preclude it. The Republic of China (aka Taiwan) and the Republic of Korea
(aka South Korea) are models of building prosperity ahead of democracy. Needless
to say, the People's Republic of China certainly is not fond of promoting either
of those countries' experiences.}

{\large Many economists arrive at the same inference as dictators, though from
an entirely different perspective. For many economists, the contention that
nations work on becoming rich before becoming free follows from how theythink
about politics. They treat politics as just so much friction, to be written off
instead of dealt with.}

{\large No doubt it is good to be rich, and many of the world's rich countries
are democratic. But dependence on a large coalition of essentials is a powerful
explanation of quality of life even when wealth is absent, just as it seems to be
a harbinger of future wealth. Economic growth and success, in contrast, does not
seem to be an assurance of improved governance and, indeed, may hinder it.1 This
is a question worth exploring in greater depth, though for now our subject is how
variations in the size of the group of essentials and interchangeables determines
how resources are allocated between public and private rewards so as to pay just
the right amount to one's coalition while also paying just enough to keep the
citizenry from making trouble.}

\subsection{Public Goods Not for the Public's Good}

{\large From a leader's point of view, the most important function of the people
is to pay taxes. All regimes need money. As a result, certain basic public goods
must be made available even by the meanest autocrat, unless he has access to
significant revenue from sources, like oil or foreign aid, that are not based on
taxing workers. Public benefits like essential infrastructure, education, and
health care, need to be readily available to ensure that labor is productive
enough to pay taxes to line the pockets of rulers and their essential supporters.
These policies are not instituted for the betterment of the masses, even though,
of course, some members of the masses, especially workers, benefit from them.}

{\large Education, as a means for getting ahead in life, is a big deal for any
country's citizenry. Indeed, a popular refrain among many liberalminded thinkers
is to extol the quality of education in otherwise oppressive states like Castro's
Cuba or even Kim Jong-Il's North Korea. And they have a good point. Both Cuba and
North Korea have impressive primary education. For instance a 1997 UNESCO study
finds that Cuban third and fourth graders far outperform their counterparts in
other Latin American countries, As for North Korea, it has a 100 percent literacy
rate. In contrast, only 81 percent of democratic India's people can read and
write.2 But these facts can be misleading, or even downright wrong. That basic
education is mandatory and extensive in such places often is used to argue that
autocracy isn't so bad.}

{\large Rarely do any of us stop to probe beneath these observations to find out
why dictators pay to have well-educated third graders---but do not carry that
quality of education forward to higher learning. The logic behind political
survival teaches us to be suspicious. We cannot help but believe that these
public goods are not intended to uplift and assist the people unfortunate enough
to live in such places. The rules of politics, as we know, instruct leaders to do
no more for the people than is absolutely essential toprevent rebellion. Leaders
who spend on public welfare at the expense of their essentials are courting
disaster.}

{\large These leaders, whether dictators or democrats, are all grappling with
the same question: How much education is the right amount? For those who rely on
few essential backers the answer is straightforward. Educational opportunity
should not be so extensive as to equip ordinary folks, the interchangeables, to
question government authority. A na\"{\i}ve person might look at any number of
awful regimes and yet come to the conclusion that, because they provide such
public benefits as nationalized health care or sound primary education, they're
actually better to their people than many democratic states are to theirs. This
is nonsense, of course---in the vast majority of cases autocrats are simply
keeping the peasants healthy enough to work and educated enough to do their jobs.
Either way, literate or not, they're still peasants and they're going to stay
that way.}

{\large A far better measure of leaders' interest in education is the
distribution of top universities. With the sole exceptions of China and
Singapore, no nondemocratic country has even one university rated among the
world's top 200. Despite its size, and not counting universities in Hong Kong,
which were established under British rule before Hong Kong's return to China in
1997, the best-ranked Chinese university is only in 47th place despite China's
opportunity to draw top minds from its vast population. The highest ranking
Russian university, with Russia's long history of dictatorship, is 210th. By
contrast, countries with relatively few people but with dependence on many
essential backers, like Israel, Finland, Norway, the Netherlands, Belgium, and
Canada, have several universities ranked among the top 200.3 That this uneven
distribution of top-notch universities favors large-coalition locales is no
accident.}

{\large Highly educated people are a potential threat to autocrats, and so
autocrats make sure to limit educational opportunity. Autocrats want workers to
have basic labor skills like literacy, and they want their own children---their
most likely successors---to be truly well educated, and so send them off to
schools in places like Switzerland, where Kim Jong Un, Kim Jong Il's youngest son
and designated successor, was educated.}

{\large Dictators also like to have their children educated in leading
universities in the United States, and especially at Oxford University in the
United Kingdom. In fact, one might almost conclude that Oxford is a
breedingground for authoritarians. It certainly is the alma mater of many,
including Zimbabwe's Robert Mugabe, the Bhutto family of Pakistan, kings of
Jordan, Bhutan, Malaysia, and even little Tonga. England's big coalition system
opens the door pretty broadly to give access to higher education.}

{\large When the leadership relies on few essentials, higher education is for
the children of the powerful; when the bloc of essentials is big, it is for the
betterment of everyone. One of the features of the old Soviet regime that Boris
Yeltsin balked at, for instance, was exactly the privileged access that children
of Communist Party leaders had to the best universities regardless of their
ability. Kids of loyal families were helped to get ahead.}

{\large The capable children of potentially dissident families were kept down by
being excluded from broad access to the best schools.4 One thing that both
dictatorships and democracies have in common is the special advantage insiders
seem to have when it comes to the top universities. Even places with lots of
essentials seem to work that way, although the privileged access is granted by
the universities and not imposed on them by the government. The president of the
United States doesn't get to tell Harvard who to admit. A closer look at the
system demonstrates the reasoning---Harvard and many other prestigious
universities favor ``legacies,'' that is, the children of alumni, because such
students are likely to bring the university more donations from its wealthy
graduates. What may seem like a case of privileged access in an otherwise open,
large-coalition system, actually reflects the internal dynamics of universities
themselves.}

{\large We shouldn't fail to notice that universities in their own right
constitute small-coalition political systems with a pretty big batch of
interchangeables. No surprise, then, that they behave like autocracies, favoring
the rich and connected at the expense of those who lack political clout. If you
doubt it, have a look one day at how many administrators university presidents
like to hire compared to faculty. It seems you can never have too many
supporting-cast administrators whose jobs depend on keeping the person at the top
happy. Faculty, on the other hand, don't depend on keeping their ``bosses''
happy; they depend on keeping their colleagues happy long enough to get tenure
and then they are pretty free to do whatever they want---why do you think we can
write this passage! There is a delicate balance to be struck, to be sure, and so
successful universityleaders are especially skilled at doling out private rewards
to anyone who could be a threat. People who raise money for the university
frequently get a percentage of what they bring in to incentivize them. Faculty
who are cooperative are likely to more readily be granted sabbaticals, get
research funds, pick the classes (usually small) that they teach, and so forth.
So we shouldn't be surprised by distortions in merit in universities; they really
are small-coalition regimes.}

{\large We might hope for a rosier picture when it comes to secondary education
in places that need few essentials. Why wouldn't all political leaders favor open
access to secondary school, where students learn higher levels of math, science,
language, literature, history, and social thought? That's easy to answer. These
are dangerous public goods that should be doled out carefully. It just isn't
necessary to have lots of people around with skills that are not absolutely
required to produce revenue for the autocrat's regime. Why, for instance, would
any autocrat eager to stay in power want to open the secondary schools to people
who are not likely to contribute to the coalition's wealth and security? Math and
science are great subjects for study in China; sociology and political science
are the subjects of democracies.}

\subsection{Who Doesn't Love a Cute Baby?}

{\large The incentives to provide good health care are not so different from the
incentives to provide basic education. Keeping the labor force humming is the
primary concern for leaders of small-coalition countries---everything, and
everyone, else is inessential. There is no point in spending lots of money on the
health of people who are not in the labor force and who won't be in the labor
force for a long time. One of the more depressing ways in which this can be seen
is in the relation between the performance of health care systems for infants and
the size of a government's winning coalition.}

{\large It seems that a lot of dictators and their essential backers don't love
babies. This is true whether we think of seeming monsters like Saddam Hussein or
the oft-praised Fidel Castro for his efforts to foster high quality health care
in Cuba.}

{\large Saddam Hussein built lavish palaces while his people suffered under the
impact of economic sanctions. The UN provided baby formula intended to offset the
impact of this hardship on little children---however, Hussein allowed his cronies
to steal it. The formula found its way to markets throughout the Middle East,
yielding profits for Hussein, even as shortages in Iraq resulted in a doubling of
the infant mortality rate. There is no doubt that Hussein was a miserable human
being---it may be that his record is no better or worse than any other heartless
brute. Alternately, it could be that dependence on a small group of cronies made
him act as if he were a brute. Perhaps in a different place and circumstance he
might have gone around kissing babies to garner political support. Perhaps
Chemical Ali was right that Saddam was too merciful.5 Indeed, it so happens that
even in many autocracies with reportedly good health-care systems, infant
mortality is high. This may be because helping little children does not
particularly help leaders survive in power. Not that they don't like a cute baby
as much as the next guy, but they recognize that helping babies doesn't help
them.Cuba has the lowest infant mortality rate in Latin America. It's a
commendable accomplishment. However, the real question is whether we should
attribute this to Fidel Castro's beneficence in building a quality health-care
system, or whether he just inherited a good healthcare system from his
predecessor, Fulgencio Batista.}

{\large Batista originally rose in prominence as a participant in Cuba's 1933
coup. Although a succession of other figures became president of Cuba, Batista,
as army chief of staff, remained a key figure behind the ``throne'' throughout
this transitional period. He defeated former president (under the coup regime)
Ram\'{o}n Grau San Mart\'{\i}n in the free, democratic election of 1940. Batista
then served as Cuba's democratically elected president from 1940 until his term
ended in 1944. It is noteworthy that during this period Batista enjoyed the
support of Cuba's communist party because of his strongly pro-labor and
pro--labor union policies. Indeed, during his period as a democratic leader under
the rules of Cuba's 1940 constitution, Batista proved to be an effective social
reformer who also helped promote successful economic policies.6 In short, he
governed just the way we would expect a large-coalition leader to govern.}

{\large After his term expired, Batista moved to the United States. His
preferred successor for the presidency lost to Grau in the 1944 Cuban
presidential election. Although still in the United States in 1948, Batista was
elected to the Cuban senate. He returned home to serve, and then ran again for
the presidency in 1952, but was a distant third in the polls to Robert Agramonte,
the front-runner, and Carlos Hevia. Seeing that he had no chance to be elected,
and possessed of the backing of the United States government, pro-American
Batista launched a coup before the election took place. Backed by the army,
Batista now assumed the presidency as a small-coalition dictator rather than as a
democratically elected largecoalition leader. Surely this is yet another example
of power taking precedence over political principles.}

{\large The Cuban economy depended largely on agriculture, especially growing
sugarcane, a highly labor intensive activity. As a result, both Batista's and
Castro's regimes, bereft of natural-resource wealth, had to rely on workers to
generate revenue. Each did have the benefit of supplementing that revenue with a
significant amount of foreign aid, from the United States in Batista's case and
from the Soviet Union in Castro's.Nevertheless, to stay afloat, both needed to
maintain a healthy and reasonably educated workforce. Therefore, both Batista's
and Castro's Cuba needed good health care as well as good basic education.}

{\large We should not expect vast differences in the public goods provided by
Batista after he became a dictator and those provided by Castro. To be sure,
their ideological songs were radically different, but they both depended on a
small clique to keep them in power. In both cases, the military was essential and
so were loyal bureaucrats. So, putting aside the window dressing each used to
justify their form of rule, stripping out the ideology, they were running similar
regimes. The main difference in our terms was that Batista had a small coalition
and a small selectorate once he overthrew the constitution. Castro ran a
rigged-election regime, so he, like Batista, had a small coalition, but unlike
Batista his nominal selectorate was pretty big. Of course, the real Cuban
selectorate under Castro, the influentials, was probably no bigger than Batista's
real selectorate. So we should anticipate that both of them were good at
producing good health care and good primary education. And the facts bear out
these expectations.}

{\large Although a headline fact is that Castro's Cuba has Latin America's best
infant mortality rate, the details reveal that the relative quality of infant
care has declined. Cuba had Latin America's best infant mortality rate under
Batista as well as under Castro. In general, small-coalition regimes gradually
run their economy into the ground through inefficiencies designed to benefit the
leader and essentials in the short run, at the expense of longer term
productivity. How quickly welfare would erode depends in part on what foreign aid
comes in to offset the economic woes brought on by small-coalition governance. We
should see trends indicating declining quality of health care in Cuba under
Castro compared to Batista, not necessarily because one was more civic minded
than the other, but because time inexorably diminishes the quality of life for
ordinary people in most petty dictatorships.}

{\large Cuba's absolute infant mortality rate has improved markedly since
Batista's overthrow, but Cuba's relative quality of infant care has not kept pace
with the rest of the world. Medical technology has improved health care
substantially since the end of World War II, and especially since the 1960s,
right after Castro's revolution succeeded in 1959. Cuba'simprovements in infant
mortality, though substantial, have lagged behind improvements in infant
mortality in many other countries. In 1957, not long before Batista was
overthrown by Castro's revolution, Cuba's infant mortality rate was 32 per 1,000
live births. This was the thirteenth best in the world at the time. To put this
impressive record in perspective, Cuba was outperforming Austria, Belgium,
France, Israel, Japan, Spain, Portugal, and West Germany. Today, all of these
countries outrank Castro's Cuba in infant mortality rates. Yet, until the demise
of the Soviet Union, Cuba's economic growth rate was one of the highest in Latin
America and its high abortion rate---which terminates difficult, at-risk
pregnancies---is 58.6 per 100, according to the Guttmacher Institute.7 Cuba's
infant mortality story is one of the better ones among countries with long
histories of petty dictatorship. Indeed, even wealth proves a poor way to
inoculate little children from untimely deaths. But having a big coalition is the
best vaccine. Like all medicine, it is not perfect, but it makes a huge
difference.}

{\large The world's 36 governments that depend on the largest groups of
essentials have thirty-one fewer infant deaths per 1,000 births than the
forty-four governments that depend on the smallest groups of essentials.}

{\large Comparing the same eighty countries but now based on per capita income,
the poorest have fifteen more infant deaths per 1,000 births than the richest.
Being rich does facilitate saving babies' lives but not as much as being
democratic!}

\subsection{Clean Drinking Water}

{\large For autocrats, money spent on people---like infants and little
children---who are years away from contributing to the economy is money wasted.}

{\large Resources should instead be focused on those who help the ruler stay in
power now, not those who might be valuable in the distant future. When you see
pictures and images flowing out of populations in crisis, it's apparent that
suffering at the extremes of the life span is hardly uncommon in autocracies.
It's not that these terrible conditions can't be reversed; it's that the autocrat
must choose not to reverse them as a simple matter of cost.}

{\large Funds diverted in such a fashion are taken right out of her own pocket
and the pockets of the coalition.}

{\large Consider the availability of as basic and essential a public good as
clean drinking water. In a world in which easily prevented, waterborne diseases
like cholera, dysentery, and diarrhea kill millions of the young and
old---non-workers---clean water would be a tremendous lifesaver. The problem is
that these are lives that autocrats seem not to value.}

{\large Sure enough, drinking water is cleaner and more widely available in
democratic countries than in small-coalition regimes, independent of the separate
and significant impact of per capita income. Honduras, for instance, is a pretty
poor country. Its per capita income is only \$4,100. Yet, 90 percent of the
people in Honduras have access to clean drinking water.}

{\large Per capita income in Equatorial Guinea is more than \$37,000, nine times
higher than in Honduras.8 And yet only 44 percent of its people enjoy clean
potable water. This is true even though both places have the same burden of a
tropical climate; both were Spanish colonies; and both are predominantly
Christian societies. The big difference: Honduras is considerably more
democratic, with a larger group of essentials, than Equatorial Guinea. Is this
comparison out of the ordinary? Not at all! To be sure, higher income countries
on average do enjoy even higher quality drinking water than poorer countries.
Looking within approximately equalper capita income slices of the world, however,
those regimes that depend on a big coalition on average make quality drinking
water readily accessible to almost their entire population, and those who depend
on a smaller coalition lag behind by 20 percent or more. The availability and
technology of clean water doesn't favor democratic societies; democratic regimes
favor ensuring that drinking water is clean.}

\subsection{Building Infrastructure}

{\large As we've demonstrated, even a nasty dictator provides the people with
basic education and essential health care so that they can work at making the
autocrat rich. There is one more public goods program that is necessary to
translate labor into his or her wealth. Everything the workers make has to get
out to the market so that the leader can sell the product of the workers' labor
for money. That means there is a need for roads to transport what's been made to
markets where people have money.}

{\large Nevertheless, there is still a balance when it comes to infrastructure.}

{\large Since roads run in two directions, one must be careful not to build too
many roads or, especially, roads to the wrong places. Roads are very costly to
build and it is easy to hide their true costs. This makes them a good source of
graft, which in turn makes constructing them attractive. But having a country too
well connected can lead to new regional power centers---political, economic, or
otherwise---that undermine the autocrat.}

{\large And if things ever heat up sufficiently to encourage rebellion, the very
roads that autocrats build can come back to haunt them. Shoddy infrastructure is
often an intentionally designed feature of many countries, not a misfortune
suffered unwillingly.}

{\large Zaire's (today's Democratic Republic of the Congo) Mobutu Sese Seko once
told Rwanda's president Juv\'{e}nal Habyiarama, ``I've been in power in Zaire for
thirty years, and I never built one road.'' Why? As he explained to Habyiarama,
``Now they are driving down your roads to get you.''9 Indeed, when Mobuto came to
power in 1965, Zaire had about 90,000 miles of roads. Thirty-two years later,
when he was finally deposed, only about 6,000 miles remained, just enough to sell
goods and not enough to make it easy to get to Mobutu. So, roads to market: yes.
Roads to get you out of the country: yes. Other roads: no.}

{\large Consider how straight or curvy the roads are from the center of capital
cities to the city's largest airport. Of course, just how straight the road
isdepends on a number of factors. There's topography, how sprawling the capital
city is, the technology when the road was built, and how wealthy the society is.
And then there is the size of the winning coalition.}

{\large Wealth is not randomly distributed. Places that depend on broad-based
support to keep the government in power tend to be wealthy too. That might lead
us to think that airport highways are especially straight in wealthy---read,
large-coalition---societies, since it is rich governments that can most easily
compensate people for tearing down their houses to make efficient roads from the
city to the airport. And yet, that isn't the case.}

{\large Topography, unlike wealth, isn't dictated by politics. A landscape
spattered with wide waterways and high mountains is likely to make the road from
the capital city to its airport pretty curvy regardless of the type of government
that runs the society. To just plow straight ahead in such circumstances means
building tunnels and long bridges. Those are expensive. Tearing through villages
is expensive too if the townspeople need to be properly compensated when the
government invokes its right of eminent domain to knock down houses. And the
people who need to be compensated in such circumstances may well happen to be
both influential and essential. If the townsfolk whose houses are in the way of
an airport road are not influential or essential, then it is cheaper to go
straight ahead than to skirt village after village.}

{\large If the choice of route were just about economics one might think that
straight roads from city to airport are especially prevalent in rich countries.}

{\large But if politics trumps economics, then straight roads will more often be
the province of petty dictatorships rather than representative---and rich---
democracies. That the difference between driving distance and the distance as the
crow flies is related to politics, and especially to how many essentials a leader
needs, is rather interesting and perhaps surprising, but related they are.}

{\large We calculated the ratio of driving distance to the distance as the crow
flies from the major airport serving each national capital for 158 countries.10 A
low ratio means a fairly straight road; higher ratios, more curves. Only two of
the thirty lowest ratios---places where the driving distance is almost equal to
the distance as the crow flies---fall in democracies, taking the average
coalition requirements for governance into account over the past thirty years
(1981--2010). Portugal and Canadahave the straightest roads to their respective
capital-city airports among societies whose leaders rely on lots of essentials to
hold power. Portugal has the world's thirteenth lowest ratio and Canada is
twenty-eighth. Which countries have the ten lowest ratios? Answer: Guinea, Cuba,
Dominica, Colombia, Afghanistan, Pakistan, Yemen, Ecuador, Ethiopia, and
Equatorial Guinea. This certainly is not a who's who of democracy. Only Ecuador
and Colombia's governments, among this motley crew, are making real progress
toward dependence on a large coalition. The average coalition size for these ten
by our estimation method is 42 out of a possible score of 100.11 The world's
average is 62; that is, 50 percent higher!12 The lesson is that when an autocrat
needs a road to the airport (a good route of escape) he can just confiscate
people's property, making the road as straight and as quickly traveled as
possible. As President Obama observed in his State of the Union address on
January 25, 2011, when discussing the similar issue of building railroads, ``If
the central government wants a railroad, they get a railroad---no matter how many
homes are bulldozed.'' He was contrasting what autocrats can do with what he, as
a democratic leader, cannot.}

{\large Democrats find using eminent domain politically costly and so are more
likely to go around a village or house than to knock it down. In the event that a
democrat ignores property rights, it's likely that all the freedoms she must
provide will culminate in people taking to the courts and the streets to redress
any perceived wrong. A smart democrat, of course, tries to avoid such troubles,
using eminent domain only when it benefits many people, especially members of the
democrat's constituency (the influentials). It is incredible to see how easily
leaders can take people's property in the People's Republic of China and how hard
it is to do the same in Hong Kong. When essentials are few, pretty much anything
goes.}

{\large Roads are not the only infrastructure construction that seems to
emphasize their private benefits in autocracies and their public benefits in
democracies. Autocrats and democrats need electric grids. A recent study, for
instance, shows that when governments expand reliance on a large coalition they
shift electricity pricing and availability away from policies that favor industry
and toward policies that help consumers; that is, the masses instead of the
wealthiest in society.13 And then there are theMobutu Sese Sekos of the world,
who have worked out how to use electric power to advance their political
survival.}

{\large Mobutu famously replaced local electricity-generating capacity near
Zaire's copper mines with a hydroelectric station that was more than 1,000 miles
away. This empowered him to cut off electricity at the touch of a button,
guaranteeing that he, and not some local entrepreneur, controlled the flow of
copper wealth. It's worth noting that the power lines bypassed all the people
along the way. That's the right sort of infrastructure project taken up by
someone who wants to use public policy to secure his hold on power.}

{\large Massive construction projects, like the Aswan Dam in Egypt and China's
Three Gorges Dam, are very much like Mobutu's power grid. These sorts of projects
are great for autocrats. Although they dislocate vast numbers of people, they
also generate vast corruption opportunities, making them gems of private rewards
as well as providers of basic public infrastructure.}

{\large It is noteworthy that they also cost vastly more to build than
comparable dams in the United States or other democratic countries, where such
projects serve primarily to advance public---not private---welfare.}

{\large All leaders need to provide some public goods in order that the people
can work to pay taxes. This is just as true in other organizational settings.}

{\large Corporate bosses cannot expect their employees to produce in isolation.}

{\large Communications, training, and team-building skills promote productivity,
although they also facilitate the coordination of protest against the boss.}

{\large For this reason, not all corporate phones connect to everywhere.}

{\large Even the heads of crime families provide public goods that help mobsters
earn, of which perhaps the most important is reputation.}

{\large Mobsters would find it much harder to demand protection money if people
did not believe they were backed by muscle. Mafias also provide muscle and
deterrence to protect their members. Killing a mafioso is not to be undertaken
lightly. Mobs also provide lawyers. Each of these services is a valuable reward.
But more importantly, they keep the mafia earning. As with autocrats, mob bosses
provide those public goods that help mobsters produce the wealth that their
bosses need to stay on top.}

\subsection{Public Goods for the Public Good}

{\large In small-coalition polities, public goods often serve the narrow
interests of the leadership and only indirectly the interests of citizens. The
situation is almost entirely different for those who rely on a big coalition. For
such leaders the desire to stay in office dictates that they must satisfy the
large coalition's desire for access to good education at all levels; to quality
health care at all levels; and, most importantly, to the means to make the wishes
of the coalition easily known by the government at all levels. It is surely no
coincidence that all but one (Singapore) of the twenty-five countries in the
contemporary world with the highest per capita incomes are liberal democracies;
that is, societies that enjoy rule of law, with transparent and accountable
government, a free press, and freedom of assembly. These are places that foster
rather than suppress or obstruct political competition. They foster such
competition not out of civicmindedness but rather out of the necessity of
assembling a large coalition of supporters.}

{\large Some of the richest people in the world live in tiny countries with tiny
populations, like Iceland and Luxembourg. Others live in countries with vast
populations---the United States or Japan---while still others live in expansive
territories with relatively modest populations, like Canada or Australia. Some of
the wealthiest people live in religiously homogeneous societies like Denmark or
Italy, but others reside in religiously heterogeneous nations such as the United
Kingdom or the United States.}

{\large Many of the richest countries are in Europe, but others are in Asia,
North America, or Oceania. Some were imperial powers like Britain and France;
others were themselves colonies like Canada or New Zealand.}

{\large What, then, is it that these countries have in common? It is not their
geographic locale, their culture, religion, history, or size. What they all have
in common is that they are democracies and therefore dependent on a large
coalition, albeit of different shapes and sizes. And being dependenton many
essentials, all of these regimes share in common the provision of the cheap and
yet hugely valuable public good called freedom.}

{\large Although such crucial freedoms as free speech, free assembly, and a free
press are cheap to provide, autocrats avoid them like the plague.}

{\large Democratic leaders, no doubt, wish they could avoid these freedoms since
it is these public goods that make it easy for opponents to organize to overthrow
them. But those who depend on a large coalition can't escape them because they
cannot amass a winning coalition without guaranteeing large numbers of people the
right to say, read, and write what they want, and come together to discuss and
debate at will. And then democrats must listen and deliver what it is their
constituents want or someone else will come to power and do so.}

{\large But when incumbents rely on a small coalition of cronies, then coalition
members are readily satisfied by being made rich through corruption and cronyism.
They do not risk these riches by demanding that incumbents siphon money away from
them and into effective public policies. Under these conditions, leaders can
readily limit the provision of public goods in general and freedom in particular
if they so choose. Hence, democracies escape Hobbes's state of nature and
autocracies generally don't. Indeed, we can see just how dramatic the difference
is in escaping the state of nature by looking at what happens when nature
exercises its freedom to wreak havoc. We have in mind the consequences of natural
disasters like earthquakes, cyclones, tsunamis, and drought. These certainly are
not political events, but their consequences are the product of how rulers best
allocate revenue and how people's freedom to organize shape allocation
decisions.}

\subsection{Earthquakes and Governance}

{\large An earthquake of magnitude 7 on the Richter scale is ten times larger
than one of magnitude 6, just as an 8 is ten times larger than a 7 and 100 times
bigger than a 6. The city of Bam in Iran suffered a terrible earthquake on
December 26, 2003. Its magnitude was between 6.5 and 6.6. Of the city's
approximately 97,000 residents, 26,271 were killed. Chile, with a similar per
capita income to Iran, experienced a magnitude 7.9 earthquake on June 14, 2005.
That is twenty-five times bigger than the Bam earthquake and it struck in a more
populous area. The Chilean quake hit the city of Iquique with a population of
about 238,000. Remarkably, it killed only eleven people. Was this good luck or
good policy at work?}

{\large Chile and Iran both regularly experience substantial seismic activity.
As such, we should expect that their governments are attentive to the risks of
earthquake and the devastation that can befall their people. But everything we
have argued urges us to be cautious about such an optimistic view of governance.}

{\large Just to look at the past fifty years of history, Iran has consistently
been a small-coalition regime. The shah's government may well have depended on a
somewhat smaller group of essentials than Iran's current theocracy, but the two
regimes are in practice not so different. By our way of thinking, therefore, Iran
is not a place expected to foster the kinds of political freedoms that make it
easy for people to express what they want and for the government to make a
serious effort to fulfill those wants.}

{\large Chile's last half century was a bit more complex. The country was a
fairly democratic polity from 1960 until 1973, and was then plunged into a
smallcoalition regime that lasted until the end of the 1980s. By 1989 it was well
on its way back to dependence on a relatively large coalition to sustain the
government. This means that we should expect a substantially more publicgoods
oriented approach to seismic activity in Chile than in Iran at least during the
1960s and since 1990.Chile experienced an extraordinary 9.5 earthquake in 1960.
It killed 1,655 Chileans (as well as sixty-one in far-away Hawaii following the
tsunami that resulted) and left about 2 million people homeless. Chile's fairly
democratic government (at the time) immediately set about developing a new,
rigorous seismic code for all new construction to protect its citizens from such
destruction in the future. Left largely unaltered during the long years of
military dictatorship, the code was revisited in 1993 when the once-again
democratic Chile made upgrades to reflect improvements in technology. It seems
that Chile's seismic code was not only rigorous but also well enforced, resulting
in greatly enhanced public safety against the ominous threat of natural
earthquake disaster.}

{\large Unlike Chile, Iran enjoyed no such period of democratic rule during the
last half century. As a result, there was no impetus for the government to
strengthen its policies for protecting the public from disaster. As reported by
the Iranian studies group at the Massachusetts Institute of Technology (MIT)
following the Bam earthquake, ``Considering the high seismicity of Iran, a
comprehensive hazard reduction program was launched in 1991, but the
effectiveness of the measures have [sic] been limited by lack of adequate funding
and institutional coordination.... The principal causes of vulnerability in the
region include . . . inefficient public policies, and lagging and misguided
investments in infrastructure. [Emphasis added]''14 Translation: the
small-coalition Iranian regimes of the shah and the ayatollahs have siphoned off
funds for their private benefit instead of directing them toward improved public
security against the predictable threat of seismic disasters. They provide no
means for the people to make clear their desires, and they take few actions to
secure their citizens against the predictable danger of death and destruction
from seismic shocks.}

{\large The comparison of Iran and Chile is far from unusual. China, like Chile,
suffered a 7.9 earthquake of its own. It struck in May 2008, bringing down many
shoddily constructed schools and apartment buildings, killing nearly 70,000. Even
accounting for variations in Chile's and China's populations and incomes, it is
impossible to reconcile the difference between China's death toll and Chile's,
except by reflecting on the incentives to enforce proper building standards in
democratic Chile---incentives missing in autocratic China and Iran. And lest it
is thought these are special cases, itis worth noting that democratic Honduras
had a 7.1 earthquake in May 2009, with 6 deaths and Italy a 6.3 in April 2009
with 207 deaths. Even Japan's horrendous death toll following its massive 8.9
magnitude earthquake and tsunami in March 2011 is surely lower than a comparable
event's death toll would have been in a small coalition regime. Japan spent a
fortune on quality construction to withstand earthquakes but almost no one can
afford to protect against a seismic event and tsunami of the magnitude Japan
experienced. Big coalitions save lives because bigcoalition leaders know that if
they don't protect their ordinary citizens they'll be turned out of office in
favor of someone who will.}

{\large Earthquakes and tsunamis are hard to foresee. But their aftermath is
not. When there are lots of essential supporters, rescue is swift and repair is
quick and effective. If it isn't as swift and effective as people expect--- and
in large-coalition systems they expect it to be remarkably swift and
effective---then political heads role. This is what happened, as we will see
later, following Hurricane Katrina in the United States. We will also see that
when there are few essentials, poor relief does not lead to heads rolling.}

{\large Rather, autocrats actually prefer to exaggerate damage to attract more
relief funds. Once aid is secured, it is redirected into the private accounts of
political elites, rather than being steered toward rebuilding. Consider the
relief effort in Sri Lanka following the tsunami of 2004.}

{\large Such differences can be observed within nations too. Edward Luce toured
refugee camps in Tamil Nadu on the east coast of southern India in the wake of
the 2004 tsunami.15 Although 15,000 to 20,000 people were killed and there was
widespread devastation, within a year virtually everyone had been resettled and
the government had provided compensation for the losses of life and property. The
people, although relatively poor, were highly informed about the process. The
reason: elections in Tamil Nadu are highly competitive, as the patronage style of
bloc voting that is still prevalent in northern India has broken down. When Luce
toured the more northerly state of Orissa in 2006 he found people still housed in
tent villages. But these were not victims of the 2004 tsunami: they were still
coping with the ramifications of a cyclone that happened in 1999.}

{\large Each of these examples of natural disasters tells a variation upon the
same story. When governments depend on many essentials, they need toallocate the
government's resources and provide valuable public goods like reliable building
codes, relief efforts to rescue the victims of disaster, and, when possible,
protective barriers like levees and dikes to forestall disaster. To know what the
people need, governments need to make it easy for the public to make clear what
basket of public goodies they desire. That is best done by allowing the least
costly and most precious public good of all: freedom.}

{\large Public goods can be for the public's good. Yet they can also be a means
of exploiting the public. In large-coalition environments, public goods
overwhelmingly enhance public well-being. In small-coalition settings this is not
true.}

{\large Democracies are not lucky. They do not attract civic-minded leaders by
chance. Rather, they attract survival-oriented leaders who understand that, given
their dependence on many essentials, they can only come to and stay in power if
they figure out the right basket of public goods to provide.}

{\large Small-coalition leaders figure out their solution to the exact same
survival problem. It is just that when the coalition on which they rely is small
then the mix of public goods is slimmer and trimmer. It is designed for survival
purposes in both cases.}

{\large We don't need to appeal to civic spirit to explain why people have so
much better a life in a democracy than in an autocracy. Higher levels of
education are accessible to everyone when the coalition is large; education is
basic when the coalition is small. Health care is for those who are productive
when the coalition is small; babies and the elderly are not excluded from health
care when the coalition is large. Good water is for everyone when the coalition
is large; otherwise, it is only for the privileged.}

{\large And most importantly, freedom to say what you want and to dissent when
you don't get it is abundant when the coalition is large, and is scarce in the
extreme when the coalition is small.}

{\large After this exploration of the benefits of living in a large-coalition
system, in the next chapter we will see the dark side of democracy---for
largecoalition regimes are not immune from providing private benefits to a select
set of their citizenry. We will also see that corruption is a boon to
small-coalition leaders and that, in fact, corruption, bribery, and otherprivate
benefits to their cronies help small-coalition leaders stay in power.}

{\large These same benefits could cost large-coalition leaders their jobs. That
is why the world's most corrupt regimes are always led by a small coalition.}
\pagebreak{}


\section{Chapter 6 - If Corruption Empowers, Then Absolute Corruption Empowers
Absolutely}

{\large WE HAVE SEEN HOW LEADERS COME TO POWER, find money, and provide public
goods, sometimes even for the benefit of society. Yet precious few successful
leaders are motivated primarily by the desire to do good works on behalf of their
subjects. Everyone likes to be liked, and there's no reason to think that the
powerful have anything against being beloved and honored by their people. Indeed,
it could well be the case that there are many candidates for high office who
pursue power with the intention of being benevolent leaders. The problem is that
doing what is best for the people can be awfully bad for staying in power.}

{\large The logic of political survival teaches us that leaders, whether they
rule countries, companies, or committees, first and foremost want to get and keep
power. Second, they want to exercise as much control over the expenditure of
revenue as they possibly can. While they can indulge their desires to do good
deeds with any money at their discretion, to come to power, and to survive in
office, leaders must rivet their attention on building and maintaining a
coalition loyal enough that the ruler can beat back any and all rivals. To do
that, leaders must reward their coalition of essential backers before they reward
the people in general and even before they reward themselves.}

{\large We have seen how the coalition's rewards can come in the form of public
goods, especially when the group of essentials is large. However, as the
essential coalition gets smaller, the efficient thing for any ruler to do is to
emphasize more and more the allocation of resources in the form of private
benefits to her cronies. Why? Private goods to a few cost less in total than
public goods for the many, even when the few get really lavish rewards. This is
all the more true when the coalition not only is small but also is drawn from a
very large pool of interchangeable selectoratemembers, each clamoring to become a
member of the winning coalition with its access to myriad private gains.}

{\large Successful leaders must place the urge to do good deeds a distant third
behind their own political survival and their degree of discretionary control.}

{\large Private goods are the benefits that most help rulers keep coalition
loyalty. It is only the private gains that separate the essentials from the
masses.}

{\large For this reason, it's crucial that we next explore the use of private
rewards as the means to survive in power. It remains to be seen what rulers do
with money they do not have to spend on buying their coalition's loyalty; that
is, any money whose use is at the incumbent's discretion. As we investigate these
uses of revenue we will see that Lord Acton's adage, ``Power tends to corrupt,
absolute power corrupts absolutely,'' holds generally true---however it doesn't
quite capture the causality. The causal ties run both ways: power leads to
corruption and corruption leads to power. As the title of this chapter instructs
us, corruption empowers leaders and absolute corruption empowers them
absolutely---or almost so.}

{\large Remember, as we saw with Louis XIV, no leader ever has absolute power.}

{\large That's why leaders need coalition members who support them, and why
coalition members need opportunities for enrichment if they are to remain loyal
to their leader, empowering her to stay on in office, getting and spending
money---on them.}

\subsection{Power and Corruption}

{\large Corrupt politicians are attractive to would-be supporters, and
politicians eager for power find it easiest to attract corrupt people to their
cause.}

{\large Leaders want to stay in power and must take whatever actions are needed
to do so. Successful leaders are not above repression, suppression, oppression,
or even killing their rivals, real and imagined. Anyone unwilling to undertake
the dirty work that so many leaders are called on to do should not pursue
becoming a leader. Certainly anyone reluctant to be a brute will not last long if
everyone knows he is unprepared to engage in the vicious behavior that may be
essential to political survival. If an aspiring leader won't do terrible things,
they can be sure that there are plenty of others who will. And if they don't pay
their backers to do terrible things, they can be pretty confident that those
cronies will be bought off, exchanging terrible deeds for riches and power.}

{\large Genghis Khan (1162--1227) understood this principle. If he came across a
town that did not immediately surrender to him, he killed everyone that lived
there, and then made sure the next town knew he had done so. That way, in
aggregate, he didn't actually have to slaughter that many townspeople. They
worked out that things would be better for them by giving up, turning their
wealth over to him, and accepting that the Mongols would then pass through,
leaving the survivors to fend for themselves.}

{\large Genghis went on to rule much of the known world and to die in his sleep
of old age at sixty-five. True, he doesn't have the greatest reputation in the
West (although he is revered in his homeland of Mongolia), but he most assuredly
was a successful leader.}

{\large It is fair to say that England's Henry V has a better reputation than
Genghis Khan.1 His Saint Crispin's day speech in Shakespeare's play, Henry V, is
received even by the modern reader with passion and admiration. We sometimes
forget that Henry was capable of brutality.}

{\large Much as the English revere him, it may be that he is less warmly
receivedin France where, at the siege of Harfleur, Shakespeare had him announce,
in a properly brutal leader's terms, what he would do if the town's governor did
not surrender: If I begin the battery once again, I will not leave the
half-achieved Harfleur Till in her ashes she lie buried.}

{\large The gates of mercy shall be all shut up, And the flesh'd soldier, rough
and hard of heart, In liberty of bloody hand shall range With conscience wide as
hell, mowing like grass Your fresh-fair virgins and your flowering infants....}

{\large What say you? will you yield, and this avoid, Or, guilty in defence, be
thus destroy'd? 2 Fortunately for Harfleur, on hearing Henry's words, the
governor surrendered.}

{\large The most powerful leaders in history, people like Genghis Khan, Henry V,
or Russia's Catherine the Great, tend to be autocrats beholden to only a small
coalition. Those who are most successful, especially in the modern world, also
enjoy a secure means of extracting vast revenues, such as mineral wealth.
Provided they remain healthy, such leaders are practically unassailable. That is
to say, they are as close to being absolute leaders as one can get.}

{\large What, then, is an autocrat to do once in power? They should tax
excessively---Genghis Khan is said to have levied a tax of 100 percent following
a conquest. Being a nomad, he didn't need those he defeated to produce for the
next year, since by then he and his horde would be elsewhere. They should
enthusiastically suppress the people---Joseph Stalin worked out that killing many
to catch but a few ``enemies of the people'' was worth the expense and loss of
innocent lives. He therefore made clear to his commissars that an exorbitant
error rate in executing potential enemies of the people was perfectly acceptable.
They should hand out lavish rewards to essential supporters---Catherine the Great
made sure that even her ex-lovers remained loyal by giving them control over vast
tracts of land, thousands of serfs, and the income that came withthem. And
finally, they should sock money away for their personal use, giving them a
rainy-day fund to bail themselves out of trouble or assuring a soft landing when
their luck runs out and they are overthrown---Haiti's JeanClaude ``Baby Doc''
Duvalier did just that, living lavishly in exile in France until he lost most of
his fortune to his ex-wife in a nasty divorce.3 How should nearly absolute
leaders behave? In short: Be corrupt.}

{\large As surely as money makes the world go round, so too does it make the
coalition go round. The key to a loyal coalition truly is money. If a leader
wants to oppress, suppress, repress, and even kill his enemies, he needs people
who will do the dirty work for him. Such brutality can be expensive.}

{\large That's why successful rulers pay more than anyone else for just such
purposes and, needless to say, not a penny more than that.}

{\large Leaders, essentials, and influentials of autocratic states can flaunt a
dauntingly extravagant degree of wealth, especially when you consider that their
populations are otherwise destitute, starving, and often dying.}

{\large Nevertheless, their monopoly on power and force keeps the people down,
and it's the money that keeps the select few happy to enforce the regime's will
and to protect the leader's power.}

{\large Lest anyone jump to the conclusion that this is an apt description only
of dictators, private goods in the democrat's domain are indeed worthy of
examination. Needing the help of so many, they don't pay as much as autocrats,
but still, even backers of democrats must have their rewards.}

\subsection{Private Goods in Democracies}

{\large Our version of political logic tells us that private rewards capture a
larger percentage of government spending when there are fewer essentials. That is
surely one reason why we are so much more conscious of gross corruption in
dictatorships than in democracies and rightfully so.}

{\large Transparency International, which rates government corruption every
year, shows that our intuition about dictatorships and autocracies is generally
right. Of the twenty-five most corrupt regimes, according to Transparency
International's 2010 corruption index, not even one is a mature democracy.}

{\large Only a very few---Russia and Venezuela, for instance---might be
described by some as quasi democratic, at least in the sense that they appear to
have multiparty elections. We say ``appear'' because it is also clear in both
cases that the opposition parties are severely restricted in their access to the
media or in their ability even to hold public rallies. So, to be sure, the
highest levels of corruption do belong to illiberal, small-coalition regimes.}

{\large But that does not mean that dependence on a big coalition exempts a
government from corruption. It doesn't even mean that large-coalition regimes
spend absolutely less on corruption than their more autocratic counterparts.}

{\large Because democratic settings foster lower taxes and more spending on
productivity-enhancing public goods than small-coalition regimes, dependence on
lots of essentials tends to correlate with a successful economy. Consequently, it
is likely to promote a bigger revenue pie than small-coalition settings, as we
discussed earlier. Less of the total income pie is taken by big-coalition
governments, but they are taking a smaller share of a bigger pie, so they could
have more revenue at their disposal.}

{\large Even though the private/public goods mix favors more private benefits in
small-coalition regimes, the total amount of private rewards can be greater in a
large-coalition environment.}

{\large Iran and Turkey are two predominantly Muslim countries (one Shia andthe
other Sunni), both situated in the Middle East. Iran has vast oil reserves that
should lighten the people's tax load, or so one might think. Turkey does not have
oil or other substantial natural resource wealth and so needs tax revenue to
sustain the government. Both Iran and Turkey have histories of autocratic rule,
but with Turkey now a maturing (though still transitional) democratic country
while Iran, despite some trappings of democracy, remains authoritarian. In Iran,
the votes of the people and the laws of parliament can all be overturned by the
will of the Supreme Leader. In Turkey, the president has limited legislative veto
power, as in the United States, so basically it takes unlawful action, like a
military coup, to overturn the will of the people.}

{\large Iran's population is 73 million, Turkey's 75 million, meaning that the
two nations are of comparable size. Iran's corruption ranking in 2010 was the
thirty-second worst (that is, it ranked 146 out of 178 countries in honest
business dealings), making it one of the more corrupt regimes in the world.}

{\large Turkey's ranking was fifty-sixth, placing it in the top third of the
world in terms of avoiding corruption. That is, 122 countries were rated as more
corrupt than Turkey. Not stellar, but a good performance for a transitional
democracy. Per capita income in Turkey is about \$13,730; in Iran only \$4,530.4
Thus, despite its vast oil wealth, Iranians, on average, earn only about one
third what Turks earn. Tax rates are higher in Iran than in Turkey so, despite
the oil wealth, Iran extracts more income tax than does Turkey.}

{\large Both countries have progressive income taxes although a small group in
Iran, known as the Bonyads, is exempt from taxation and even exempt from
accusations of corruption. They manage the money of the senior ayatollahs and
some key military leaders. The Bonyads are reputed to control 20 to 25 percent of
Iran's annual income---not bad as private benefits go.}

{\large To properly compare the countries, it's useful to look at how much tax
an Iranian and a Turk must pay. On \$4,530, an Iranian pays \$762 and a Turk pays
only \$680 (based on exchange rates as of December 2, 2010, for the Turkish lira
and the Iranian rial to the US dollar) in income tax.5 At the Turkish per capita
income of \$13,730, a Turk pays \$2,450 and an Iranian \$2,809. So, as expected,
Iran's government takes a bigger part of a smaller gross national product pie. In
fact, the World Bank reports that Turkey's government's revenue came to 22.5
percent of GDP in 2008 (the latest year reported); Iran's in the same year was 32
percent. At Turkey'sGDP, government revenue in 2009 (the latest year reported)
was \$138.8 billion. Iran's government revenue given its 2009 GDP was \$105.9
billion.}

{\large Despite the higher taxes in Iran, Turkey's government's revenue pie is
larger than Iran's.}

{\large Iran's revenue pie, despite its greater tax take, is only about 76
percent of Turkey's. Thus, if Iran spends 25 percent of its revenue on private
goods for its relatively small coalition of essential backers, then Turkey would
only need to allocate 19 percent to private rewards to spend the same amount of
dollars as Iran spends on private benefits. It is very likely that Turkey spends
much less both as a percentage of the pie and in total than does Iran on
corruption and other private goods. But as we saw, it is entirely possible to
engage in an absolute amount of corruption in a large-coalition regime that is
equal to the absolute amount in a small-coalition setting.}

{\large Corruption will be reported as greater in the small coalition
environment because, after all, it is a bigger proportion of the available
revenue or GDP pie.}

{\large It is important to remember that the value of private gains for the
millions of individual supporters in a democracy is small. It is substantial for
the few key individual backers in an autocracy even if the total spent on private
goods is the same in both. For instance, Turkey's winning coalition could easily
be about 20 million voters. Turnout in Turkey's 2007 election was just under 36
million, out of 42.5 million registered voters and 48.4 million total eligible
voters. With 36 million voting, the winning coalition in Turkey's
first-past-the-post elections could have been more than 18 million. Iran's
winning coalition could easily be no more than several thousand. Let's err on the
side of overstating Iran's winning coalition by assuming it is as many as 100,000
people, including religious leaders, local and national political elites,
important civil servants, key military officers, and the government's goon squads
(the Basij led by Ayatollah Khomeini's son), who enforce its antiprotest efforts.
If Iran spends as little as \$5 billion on private rewards (we have no way of
knowing the actual amount), then the average coalition member gets \$50,000, more
than ten times per capita income. If Turkey spends the same amount on private
benefits (and, again we have no way of knowing the actual amount), then the
average Turkish coalition member can expect to receive only \$250, less than 2
percent of per capita income.}

{\large Of course, in either case most coalition members will get much less
thanthe average and a few will get vastly more. But obviously there will not be
many coalition members in Turkey who are willing to beat and even kill their
fellow citizens for \$250. It is equally obvious that in relatively poor Iran,
for \$50,000 a head it should be easy for the regime to get supporters to go out
and oppress their fellow citizens.}

{\large Private goods make up a part of every government's spending just as they
make up a part of every corporation's spending. But it is much tougher to get
people to engage in truly nasty behavior in a large-coalition environment than it
is in a small one even if the totals spent on private gains are equal. History
has not produced large-coalition-dependent leaders as brutal as Genghis Khan.
Equally, we must realize that the nature of private rewards in more democratic
systems are likely to come in the form of distorted public policy rather than
through more overt means such as outright bribery, black marketeering, or extreme
favoritism.}

{\large What, then, are the private rewards provided in democracies? How might
public policy be distorted to create benefits for some and costs for others?}

{\large It is fashionable to talk about politics in terms of concepts like
ideology or left-right continuums. The standard mantras from either side of the
left-right continuum go something like this: Liberals care about the poor and are
dedicated to alleviating their misery. They are often stymied by the rich and
powerful. Those very rich and powerful tend to be conservative.}

{\large Conservatives care about the rich and are dedicated to protecting them
from the taxing and spending inclinations of liberals, whose supporters, not
surprisingly, tend to be relatively poor compared to conservative backers.}

{\large As a simplification of politics that works fine. We do not challenge
this view so much as offer a completely different way to think about it.}

{\large The rules governing how people rule inevitably divorce what policies
politicians really desire from what they say and do. Not that we doubt that
politicians hold sincere views of good and bad public policy---rather those views
are not terribly important and, besides, there are few ways to tell the
difference between declarations based on opportunistic political expediency and
true beliefs.}

{\large From the perspective of this book, so-called liberals and so-called
conservatives appear simply to have carved out separate electoral nichesthat give
them a good chance of winning office. Democrats in the United States like to
raise taxes on the rich, improve welfare for the poor, and seek heavy doses of
benefits for the middle-class swing voters.}

{\large Republicans in the United States like to reduce taxes on the rich,
decrease welfare for the poor, relying on back-to-work programs instead, and,
similarly, look for a heavy dose of benefits for the swing middle voters.}

{\large Many of the taxing and spending policies, pork-barrel programs and the
like, are simply private goods distributed to the relevant party's coalition of
essentials. Both parties pay special attention to the middle-class because there
are an awful lot of middle class voters and they can be tipped either way. They
like to define the rich---those who might be asked to pay higher taxes---as
anyone whose income is higher than their own. They like to think of welfare
payments as rife with fraud and cheating that needs to be ferreted out. And they
are very happy to have government programs that disproportionately benefit
them---no surprise there---such as tax deductions on mortgage interest, expanding
Medicare benefits, subsidies for university tuition for their children, and
increases in social security payments even in the absence of inflation.}

{\large The very poor are not likely to vote but the working poor are, and, of
course, they are likely to vote for people who adopt policies that benefit them.
The less well off love progressive taxes and hate sales taxes. Those who hope for
expanded and more effective programs for jobs training, Medicaid, long-term
unemployment insurance, and low or no taxes at their income level, tend to turn
to candidates most likely to fulfill their wishes.}

{\large These wishes are public policies to be sure, but they are public
policies that benefit primarily the select group whose voting ``bloc'' is
essential to the winning Democratic candidate. They are unlikely to vote
Republican because, let's face it, a winning Republican candidate is not likely
to support the programs we just mentioned, at least not on the same scale as a
Democrat. So these policies are payments for political support, no more and no
less so than any other private reward.}

{\large The rich like subsidies too. Republican candidates trying to build a
coalition around the support of the relatively well-to-do are the candidates most
likely to provide these subsidies. The well off and Republican candidates by and
large favor, for instance, government support for medical research on cancer,
Alzheimer's disease, and other ailments ofthe elderly who happen also to be the
wealthiest age cohort in the United States. What is more, the well-to-do are more
likely to live long enough to suffer from these diseases. They like lower capital
gains taxes since they have enough money that they can invest in the pursuit of
equity gains and they don't like inheritance taxes since they can save enough to
leave a tidy sum to their heirs. The poor rarely consume any of these benefits
but they pay for them to help the rich if they pay taxes at all. But with
Democrats more often controlling legislatures at the federal and state level than
Republicans, it is worth noting that more than 40 percent of Americans--- mostly
at the lower income levels---pay no income taxes at all.6 That, after all, is one
of the private rewards they covet just as in smaller coalition regimes the rich
pay few taxes and covet their private gains. Private benefits, whether in large-
or small-coalition environments, distort economies in exactly the self-serving
ways we should expect. And even in the most democratic of polities these private
benefits are perfectly explainable without appeal to high-falutin principles of
equity, efficiency, or ideology. People support leaders who deliver policies that
specifically benefit them. That's why earmarks---pork in colloquial terms---are
reviled in general and beloved by each constituency when the money goes to them.}

{\large This is true outside of the United States as well. As governments shift
toward or away from democracy, or leaders experience different degrees of
dependence on large or small coalitions in different parts of their domain, they
adjust their private goods giving accordingly. We can see this by comparing two
transitional democracies: Tanzania, which seems on the way to expanding its
coalition, and Russia, whose coalition seems to be shrinking.}

{\large Earlier we talked about how Tanzania's parliament, the Bunge,
superficially looks like it reflects the structure of a large-coalition
democratic government. We saw that, beneath the surface of apparent democracy, it
is a transitional regime that retains many characteristics of a small-coalition
environment. That is, the selectorate expanded more quickly that the winning
coalition, emulating a rigged system. This standard problem of transitional
regimes is accomplished through a variety of means that restrict how large the
coalition can be even as universalsuffrage is introduced. One of the ways in
which the coalition is kept artificially small in Tanzania is by reserving many
parliamentary seats for women who are indirectly elected by the parties in
parliament and by permitting several members of the Bunge to be appointed by the
president. The result is that the true size of the required winning coalition is
much less than a majority of the legislature. And when we zoomed in on
district-level elections in Tanzania, we realized that, just as in multicandidate
elections in Bell, California, in Tanzania's parliamentary districts a winning
coalition only requires one more vote than the second largest of the many parties
competing for office. This translates into needing less than 10 percent of the
vote, and very often much less. To get that crucial percentage, the government
doles out private rewards.}

{\large Tanzania's main crop is maize. The government therefore selectively
provides vouchers for subsidized purchases of maize seed. The vouchers to
different districts are of varying value, providing two opportunities to observe
private rewards at work. Our perspective implies that who gets vouchers and how
much the vouchers are worth should be driven by the size of the winning coalition
in each district. After all, the voucher program could just be a central
government reward to loyal, small-coalition constituencies. Large-coalition
districts, in that case, would be unlikely to receive vouchers or would only
receive vouchers of little value even if they are heavily dependent on maize
production and are impoverished.7 In providing vouchers, the Tanzanian central
government confronts an opportunity to equalize or distort economic and social
conditions. It could make decisions purely on a needs basis (poverty and low
productivity) or it could make decisions to dole out resources on a political
survival basis; that is, rewards for the politically loyal rather than the
economically needy.}

{\large And what do you think they do? Without boring you with the details of
the statistical analysis of the data, here are the essentials when it comes to
maize vouchers in Tanzania.}

{\large As we have sadly come to expect, the impact of coalition size is
substantial, with a doubling of the size of a district's presidential election--
winning coalition being equal to about a 69 percent decline in the prospect of
receiving vouchers. The value of the vouchers is even more dramatically
responsive to coalition size than is the likelihood of receiving them.}

{\large Looking only at the districts that actually received vouchers we found
thatdoubling the number of district-level essentials (remembering that the
districts receiving vouchers are selected on the basis of having a small
coalition to begin with), produces about a one third reduction in the value of
the vouchers they got. Thus we find that even among the small-coalition
districts---those most likely to receive vouchers---the central government
sharply discriminates between those that value private goods the most (the
smallest coalition districts) and those that value such goods least (the
relatively larger small-coalition districts).}

{\large How about handing out vouchers on the basis of need? It turns out that
productivity is linked to the odds of getting vouchers and to their worth---but
it is the higher productivity districts that do better, not the ones needing help
improving their productivity. As for poverty and vouchers---it turns out that
need has no impact on the use of vouchers to help stimulate the agricultural
economy, the very purpose the government gives for the program. Leader
self-interest once again trumps a choice to do what's best for the people,
except, as expected, when the district-level winning coalition is large. In those
districts, just as we have learned to expect, there are more effective public
policies. People living in the large-coalition districts, for instance, have
better access to healthcare, lower infant mortality rates, and more residential
electrification than those living in small-coalition districts.}

{\large Without a doubt, corruption is endemic to small-coalition regimes.}

{\large Governments that transition from autocracy to democracy diminish
corruption in the process. Tanzania, for instance, seems to be slowly improving
in its governance. In 2010, Transparency International ranked it as 116 out of
178 in corruption, considerably better, for instance, than Russia. As we have
come to expect, governments like Russia's, which are making the transition in the
opposite direction, gradually abandoning their shifts toward democracy in favor
of a smaller coalition autocracy, embrace corruption as crucial to their
leadership's political survival.}

{\large As we have noted, Russia is among the world's most corrupt states. As
such, the political logic of private goods can be seen vividly in the workings of
its corruption.}

{\large Low salaries for police forces are a common feature of small coalition
regimes and Russia is no exception. At first blush this might seemsurprising. The
police are crucial to a regime's survival. Police officers are charged with
maintaining civil order---which often boils down to crushing antigovernment
protests and bashing the heads of antigovernment activists. Surely inducing such
behavior requires either great commitment to the regime or good compensation. But
as elsewhere, the logic of corruption takes a more complex turn.}

{\large Though private rewards can be provided directly out of the government's
treasury, the easiest way to compensate the police for their loyalty--- including
their willingness to oppress their fellow citizens---is to give them free rein to
be corrupt. Pay them so little that they can't help but realize it is not only
acceptable but necessary for them to be corrupt. Then they will be doubly
beholden to the regime: first, they will be grateful for the wealth the regime
lets them accumulate; second, they will understand that if they waver in loyalty,
they are at risk of losing their privileges and being prosecuted. Remember
Mikhail Khodorkovsky? He used to be the richest man in Russia. We do not know
whether he was corrupt or not, but we do know that he was not loyal to the Putin
government and duly found himself prosecuted for corruption. Police face the same
threat.}

{\large Consider former police major and whistleblower, Alexei Dymovsky. 8 Mr.}

{\large Dymovsky, by his own admission, was a corrupt policeman in Novorossiysk,
a city of 225,000 people. He noted that on a new recruit's salary of \$413 a
month (12,000 rubles) he could not make ends meet and so had to turn to
corruption. Dymovsky claims he personally only took very small amounts of money.
Whether that is true or not, we cannot know. What we do know is what happened
next.}

{\large In a video he made and sent to Vladimir Putin before it became famous on
YouTube, ``Mr. Dymovsky also described a practice that is considered common in
Russia: When officers end their shifts, they have to turn over a portion of their
bribes to the so-called cashier, a senior member of the department. Typically,
\$25 to \$100 a day. If officers do not pay up, they are disciplined.'' According
to his own account, Mr. Dymovsky eventually grew tired of being corrupt and
feeling compelled to be corrupt. As the NewYork Times reported, he inquired of
Vladimir Putin, ``How can a police officer accept bribes? . . . Do you understand
where our society is heading? . . .}

{\large You talk about reducing corruption,'' he said. ``You say that it should
not be just a crime, that it should be immoral. But it is not like that. I told
my bossthat the police are corrupt. And he told me that it cannot be done away
with.'' Dymovsky became something of a folk hero in Russia. It seems his
whistle-blowing was much appreciated among many ordinary Russians.}

{\large The official response, however, was quite different. He was shunned,
fired, persecuted, prosecuted, and imprisoned. The public uproar that followed
led eventually to his release. No longer a police officer, he established a
business guiding tours of the luxurious homes of some of his police colleagues.
Most notable among these is the home of Chief Chernositov.}

{\large The chief's salary is about \$25,000 a year---yet he owns a beachfront
home on land estimated to be worth \$800,000. The chief offers no account of how
he could afford his home and, it should be noted, he remains in his position as
chief. He certainly has not faced imprisonment for his apparent corruption, but
then, unlike Mikhail Khodorkovsky or Aleksei Dymovsky, Novorossiysk's police
chief has remained loyal to the governing regime.}

{\large As for Dymovsky's whistle-blowing, it did prompt a response from the
Kremlin. Russia's central government passed a law imposing tough penalties on
police officers who criticize their superiors. As the Times notes, the law has
come to be known as ``Dymovsky law.'' Corruption is a private good of choice for
exactly the reasons captured by the Dymovsky Affair. It provides the means to
ensure regime loyalty without having to pay good salaries, and it guarantees the
prosecutorial means to ferret out any beneficiaries who fail to remain loyal.
What could be better from a leader's perspective?}

\subsection{Private Goods in Small Coalition Settings}

{\large Liberia's Sergeant Doe, our by now all-too-familiar case in point of a
``rightthinking'' small-coalition ruler, understood the importance of private
rewards to his cronies. As a US government report observed of his use of US aid
funds, ``The President's primary concern is for political and physical survival.
His priorities are very different from and inconsistent with economic recovery .
. . President Doe has great allegiance to his tribes people and inner circle. His
support of local groups on ill designed projects undercut larger social
objectives.''9 That, in a nutshell, is what private rewards are all
about---physical and political survival; not larger social objectives. What is
most significant about Sergeant Doe's ``misuse'' of government money is that it
kept him in power for a decade. Doe's story is not unique to him, nor is it
unique to Africa; it is not even unique to governments. It applies to all
organizations, especially when they rely on a small group of essentials. Before
reporting on the world's many dictators, let's look at how private rewards work
in a small-coalition regime that most of us think of as benign and even
praiseworthy. We have in mind two sports organizations, the International Olympic
Committee (IOC) and F\'{e}d\'{e}ration Internationale de Football Association
(FIFA, the international governing body of football---or, to people in the United
States, soccer).}

{\large What, after all, could be more important to the IOC than advancing the
quality (and maybe the quantity) of international sports competition, free from
political and personal distortions? The answer: lavish entertainment and money.}

{\large The 2002 Salt Lake City winter games are perhaps remembered almost as
much for scandal and bribery as they are for athletic excellence. The Salt Lake
Organizing Committee (SLOC) spent millions of dollars on entertainment and
bribes, which included cash, lavish entertainment and travel, scholarships and
jobs for relatives of IOC members, real estate deals, and even plastic surgery.
In the fallout, ten IOC members wereremoved or resigned, ten others were
reprimanded, and Tom Welch and Dave Johnson, who headed the SLOC, were prosecuted
for fraud and bribery.}

{\large Yet this was not an isolated incident. Indeed the Salt Lake bid
committee felt they had been unfairly overlooked for the 1998 winter games. The
Japanese city of Nagano, which won those games, spent over \$4.4 million on
entertainment for IOC officials. Improprieties of this sort abound behind
virtually all bids. During its bid for the 1996 games, Melbourne, Australia,
arranged a special concert for the Melbourne Symphony Orchestra to showcase the
piano playing of the daughter of a South Korean IOC official. Clearly, any city
that wants a serious chance at landing the games needs to lay on lavish travel
and entertainment.}

{\large Corruption and private dealing is not limited just to big bribes; money
to be converted into private gains for backers is sought at every level. Indeed,
the 1996 summer games, held in Atlanta, demonstrate that no threat to the IOC's
chance to shift money to its cronies and essential backers is too small to
capture their attention. As the British newspaper, The Independent, reported in
its Business Section (March 26, 1995): Even small entrepreneurs, from T-shirt
vendors to Greek restaurants, need to beware. Under a 1978 US law---the Amateur
Sports Act---the United States Olympic Committee (USOC) has a ``super trademark''
over any Olympic symbols or words....}

{\large The promise of strict action has been critical to attempts by the
Atlanta Committee for the Olympic Games (ACOG) to attract official sponsors, some
of which must pay up to \$40 million for the privilege.}

{\large Those already signed up include Coca-Cola, which is based in Atlanta,
IBM, Kodak, Xerox and the car makers General Motors and BMW. . . .}

{\large Eyebrows have been raised, however, at steps taken to protect the
Olympic trade mark. An Atlanta artist wanted the trade mark ``USAtlanta'' to
market her works. ACOG objected, saying that it evoked the 1996 games.}

{\large ``I think that's stepping over the line a little bit. I find it hard to
believe that anyone is going to misconstrue her logo as being designed to profit
from the games,'' said John Bevilaqua, a sports sponsorshipconsultant in Atlanta,
who none the less sympathizes with the organizers.}

{\large Perhaps the oddest case is that of Theodorus Vatzakas, who opened a
Greek restaurant in Atlanta in 1983---long before the city won the right to stage
the 1996 games---and called it the ``Olympic.'' In 1991, he was advised by ACOG
that he was infringing the 1978 Act and would have to change the name. Eventually
he did, at a cost to himself of \$1,000, calling it ``Olympia Restaurant and
Pizza.'' ``I am very upset about this,'' he complained, ``but I changed the name
because I don't have any money to fight these kind of people.}

{\large Really, I think it's crazy.''10 Even one of the authors of this
book---Bueno de Mesquita--- experienced firsthand how eager Olympic committees
are to control the flow of money and the opportunities for private gains. His
wife, Arlene, together with two friends, founded a company called Cartwheels
(which they eventually sold) to make fun products, like T-shirts, jewelry,
stationery, and music CDs, all on a gymnastics theme, for competitive gymnasts.
As Arlene recalled about Cartwheels's experience with regulations from the IOC
and the USOC leading up to the 1996 Atlanta Olympics, Our company designed
t-shirts and other products for gymnasts. Prior to the Atlanta Olympics we tried
to design some with rings, torch or any other `Olympic' related logo, but were
told that no one would print them and we would wind up with big legal problems.
It didn't matter if we used completely different styles or colors from the
official designs.}

{\large We could not use any form of the word Olympic, nor any allusion to rings
or torch. We even had to stay away from the official colors. In order to fulfill
our clients' demands for Olympic goods, we had to buy only official USOC products
at greatly inflated prices. Some of the quality was awful, making us wonder about
how some of these companies got their sponsorship.}

{\large The answer, according to our way of thinking, is straightforward.}

{\large Cartwheels, like many others, was compelled to pay high prices and buy
from vendors chosen by the IOC or AOC to fund the pot of money that theIOC and
AOC used to enrich itself and pay for the lavish private rewards it doled out to
others. And just as we should expect, quality was as low as prices were high.}

{\large The scandalous corruption that seemed to accompany almost all business
aspects of the Olympics appeared finally to come to a head with Salt Lake City.
The negative publicity surrounding the corruption scandal did inspire the IOC to
promise reforms and to place restrictions on gifts, luxury travel, and perks in
bidding cities. But as the dictates of political survival leads us to expect,
this was unlikely to last since the Olympic organizations are all small-coalition
operations. In fact, an undercover investigation by the BBC's news program
Panorama suggests bribery is still active. In the runup to the announcement of
the location of the 2012 games, secretly taped meetings suggest a price on the
order of around \$100,000--\$200,000 per IOC vote.11 Distressing to sports lovers
to be sure, but this is no surprise to anyone who thinks about political
survival.}

{\large That the IOC is plagued by bribery and corruption allegations is exactly
what we should expect when we explore its institutional structure. The IOC,
created in 1894, runs all aspects of the modern Olympics. The IOC is composed of
only up to 115 members drawn from current athletes (up to fifteen), members of
international sporting federations (IFs) (up to fifteen), senior members of
National Olympic Committees (NOCs) (up to fifteen), and up to seventy
unaffiliated members. IOC members are selected and voted in by existing IOC
members. The IOC is responsible for selecting the senior Olympic executives and
executive committees, regulating IFs and NOCs, and selecting the site of future
games.}

{\large Fifty-eight votes are all that are needed to guarantee someone's
election to become IOC president or host the games. Not surprisingly, IOC
presidents keep their jobs for a long time and maintain lavish expense accounts.
Since 1896, the date of the first modern Olympic games, there have been only
seven presidents. In practice, often even fewer than fiftyeight votes are
required because not all 115 positions on the IOC are filled and representatives
are ineligible to vote on motions involving their home nation. For instance,
London's bid for the 2012 games beat out Paris by fifty-four votes to fifty. At
the level of Panorama's estimates it costs less than \$10 million to win. While
this is a substantial sum of money, it is insignificant when compared to the IOC
revenues (nearly \$5.5 billion for2005--2008, the period covering the Beijing
games) and the estimated 9.3 billion (approximately \$15 billion) Britain will
spend on venues and infrastructure for the 2012 games.12 Building better
stadiums, which benefits the whole Olympic movement---athletes, officials, and
audiences alike---is a much more expensive way to buy support than doling out
\$10 million in private gains to a select few.}

{\large The design of the IOC lies at the heart of the scandals it faces. When
fifty-eight votes guarantee victory, and the IOC president can handpick IOC
members, politics and control will always revolve around corruption and bribery.
As long as the IOC's institutions remain as they are, vote buying and graft will
persist because it is the ``right'' strategy for any IOC president who wants to
survive. Regulating ``gifts'' and travel cannot change the underlying incentives
to compete on the basis of private rewards rather than better management and
facilities for the games.}

{\large When billions of dollars are at stake and winning requires the support
of a mere 58 people, any nation that relies solely on the quality of its sporting
bid will be a loser. Salt Lake learned this lesson bidding for the 1998 winter
games. It was an error they did not repeat for the 2002 games, although they got
caught in the process. Many in Salt Lake may have feigned outrage, but many might
also have been glad. After all, in spite of the subsequent allegations, the games
were not reassigned.}

{\large The IOC is not alone in engendering corruption. FIFA, soccer's
international governing body, is even worse. On December 1, 2010, FIFA announced
that it had chosen Russia and Qatar as the sites for the 2018 and 2022 World Cup
Finals. Russia beat out bids from other European rivals, including England, a
joint bid by Belgium and the Netherlands, and a joint bid from the Iberian
Peninsula. While there were many attractive features to Russia's bid, it is
becoming increasingly difficult to understand Qatar being chosen over Australia,
Japan, South Korea, and the United States.}

{\large Qatar, a tiny state in the Persian Gulf, has the world's third largest
known gas reserves and possibly the highest per capita income in the world.}

{\large However, as a site for a soccer tournament it is problematic. Sharia
forms the basis of its legal code. Alcohol consumption is harshly punished,
homosexuality is banned, and Sepp Blatter, FIFA president, has already been
condemned for making insensitive remarks on this topic. Beyondthese concerns, the
weather remains the most serious impediment to Qatar's sponsorship. It is so hot
and humid that many Qatar's residents even leave for the summer months. To make
it possible for the players to compete, Qatar's bid entailed constructing
specially built, fully airconditioned stadiums. FIFA is now contemplating moving
the tournament from its traditional June and July dates to the winter months,
when the temperature is much cooler. This would severely disrupt domestic
competition in the European football leagues, where many of the world's top
players ply their trade. Needless to say, it is hard to rationalize having this
debate after the vote rather than before if the objective was to do what is best
for soccer/football.}

{\large Since just twenty-four members of FIFA's executive committee determine
the location of the finals, the winner requires the support of only thirteen
members---if that. For the December 2010 vote only twelve votes were required
after two members were suspended for allegedly trying to sell their votes. One of
these members, Amos Adamu, was caught asking for an \$800,000 bribe in a sting
operation by the Sunday Times newspaper. While the money was nominally for
building artificial pitches, the deal required that the \$800,000 be paid
directly to him. Three days prior to the location vote the BBC's Panorama once
again exercised its penchant for unearthing corruption in sports by airing a
documentary entitled FIFA's Dirty Secret, which detailed bribery and corruption
among a number of senior FIFA officials. It is thought this severely harmed
England's bid to host the 2018 finals, since three of the officials accused were
among the twenty-two executive committee voters. Perhaps the fact that the
backers of England's bids, including British prime minister David Cameron,
immediately expressed full confidence in the fidelity of the accused FIFA
officials is a telling sign that bribery is the modus operandi at FIFA. Why call
for an investigation, after all, when it could only imperil England's future
prospects?}

{\large Fortunately, devising reforms that would promote sport and competition
over bribery and corruption is straightforward, and a comparison of bribery at
the two institutions shows why. To buy the Olympics takes approximately four
times as many votes as to buy the World Cup, fifty-eight versus thirteen. And, if
the details of alleged corruption are to be believed, the size of bribes is
substantially smaller, \$100,000--\$200,000 per vote versus\$800,000. This is a
direct illustration of the role of institutions in action, and it makes the
solution clear.}

{\large As the number of supporters needed increases, private goods become less
important. Bribery could easily be made a thing of the past by simply expanding
the IOC. For instance, all Olympians might be made IOC members eligible to vote
for the executive officers and the site of future games. There were nearly 11,000
athletes at the Beijing summer games and over 2,500 at the Vancouver games.
Alternatively, medalists, or to prevent overrepresentation of team sports, one
representative per medal, could become IOC members. Either way, within a few
years the body of the IOC would swell, and officials and bidding cities would
have to compete on the quality of leadership, games, and facilities rather than
on lavish travel trips. (Alastair laments that fixing the English football team
poses a far greater challenge.)}

\subsection{Wall Street: Small Coalitions at Work}

{\large From any boss's perspective, the best way to organize a business is
exactly the same as the best way to organize a government: rely on a small group
of essentials, drawn from a small group of influential selectors, who are drawn
from millions of interchangeable selectors. That, of course, is a perfect
description of most modern, publicly traded corporations. It also happens to be a
pretty good description of organized crime families. A coincidence? Probably
not---and not for the reasons you may be thinking.}

{\large Big corporations do not coerce people to consume their services. In
fact, they provide valuable services that lead people voluntarily to spend money
on them and to make themselves generally better off for having done so. But, like
the mafia, and like monarchies and petty dictatorships, publicly traded
corporations are made up of a small coalition, a small group of influentials, and
masses of interchangeables. That means that for their leaders---the CEOs, CFOs,
and other senior management---to survive in office they must provide lots of
private goods to their coalition of essential supporters.}

{\large The media (itself made up of just such corporations) like to portray
Wall Street businesses as tone-deaf and greedy. We take a broader view: pretty
much all of us are greedy, some for money, some for adulation, some for power,
but all greedy nevertheless. Some few among us have the opportunity to act on our
greed, while most of us are confined to pursuing our greed in minor ways. Wall
Street bankers have the opportunity to satisfy their desire for money and power
in a big way and we should not be surprised that they do so.}

{\large As we all know, the world economy went through a massive tumble in
recent years. Even several years after the near-depression's onset, unemployment
remained high and economic growth meager. And yet--- here being the media's basis
for the accusation of tonedeafness---Wall Street bonuses remained huge even as
the banks lost their proverbialshirts. Wall Street financial houses distributed
\$18.4 billion in bonuses in 2008, even though many of the largest Wall Street
firms begged for and got billions in bailout money from the federal government.
Of course, these bonuses, distributed among the leaders, their coalition, and
their influential backers, are the very private goods that helped keep the
existing managers in their jobs. It is equally worth noting that these bonuses
were more than 40 percent lower than in 2007, the year before the economic
collapse. Private goods are doled out from revenue. If revenue is down, private
goods are likely to go down too, because, after all, leaders want to keep as much
for their discretionary purposes as possible, and when there isn't much money
around it is not as if those getting private goods can easily find a better deal
by defecting to some alternative leadership.}

\subsection{Dealing with Good Deed Doers}

{\large We commented earlier that ``Successful leaders are not above repression,
suppression, oppression, or even killing their rivals, real and imagined.'' The
truth of this statement is demonstrated routinely in the world's smallest
coalition environments. Aleksei Dymovsky's unhappy experience in Russia is
nothing compared to what happens when anticorruption campaigns are mounted in
really small coalition settings.}

{\large Africa provides many of the worst cases. Daniel Kaufman, a senior fellow
at the Brookings Institute, estimates that more than a trillion dollars is spent
annually on bribes worldwide, presumably with most of it going to government
officials. With so much money on the line, it is no wonder that he also reports,
``We are witnessing an era of major backtracking on the anticorruption drive. And
one of the most poignant illustrations is the fate of the few anticorruption
commissions that have had courageous leadership.}

{\large They're either embattled or dead.'' Two examples among many include the
deaths of Ernest Manirumva of Burundi and Bruno Jacquet Ossebi in the Congo. Mr.
Manirumva was investigating corruption at high levels in Burundi when he was
stabbed to death. Although he apparently was not robbed of his personal
possessions, the president of the nonprofit organization he was working with
reported, according to the New York Times, that, ``A bloodstained folder lay
empty on his bed. Documents and a computer flash drive were missing.''
Coincidence, no doubt! Mr. Ossebi's error was to cooperate with Transparency
International in its lawsuit to recover wealth allegedly stolen by Congo's
president. Mr.}

{\large Ossebi died as the result of a suspicious fire in his home. Alexei
Dymovsky, if he knows these facts, must count his good fortune in living in a
country that is transitioning away from democracy rather than in one that never
got close to such a status in his lifetime.}

\subsection{Cautionary Tales: Never Take the Coalition for Granted}

{\large Whistle-blowing is not the only way to get in trouble. Leaders can put
themselves at dire risk if they take their coalition's loyalty for granted. The
rules governing rulers teach us that leaders should never underpay their
coalition whether they do so to reward themselves or the common people.}

{\large Those who want to enrich themselves must do so out of discretionary
funds, not coalition money. Those who want to make the people's lives better
likewise should only do so with money out of their own pockets and not at the
expense of the coalition. Leaders sometimes miscalculate what is needed to keep
the coalition happy. When they make this mistake it costs them their leadership
role and, very often, their life. The stories of crime boss ``Big'' Paul
Castellano and Roman emperor Julius Caesar are cautionary tales for any who would
make the mistake of not giving the coalition its due.}

{\large ``Big'' Paul Castellano, who inherited control of the Gambino crime
family in 1976, made just such a mistake. He shifted the focus of the family
business to racketeering and shaking down the construction industry.}

{\large Indeed it was said that no concrete could be poured on projects worth
over \$2 million in New York City without the mafia's permission. That would have
been fine for his crime family if the wealth from these new activities flowed to
its members, or if he continued to pay sufficient attention to their traditional
revenue sources. Instead, he neglected the traditional businesses, like
extortion, loan sharking, and prostitution that were the source of income for his
coalition of mafiosi. When a moment of opportunity presented itself, triggered by
the death of a key supporter, Aniello ``Neil'' Dellacroce, and the pressures from
the ongoing Mafia Commission Trial prosecuted by Rudy Giuliani, Castellano's
erstwhile backers turned on him. John ``the Dapper Don'' Gotti, Frank DeCicco,
Sammy ``The Bull'' Gravano, and other captains worked together to gundown
Castellano outside of Sparks Steak House on Forty-sixth Street in New York.13
Castellano rewarded himself at the expense of his supporters and it cost him his
life. A few thousand years earlier, Julius Caesar's mistake was to help the
people at the expense of his backers and this too cost him his life.}

{\large Julius Caesar's death at the hands of some of his closest supporters is
often portrayed as the slaying of a despot. But the facts don't support this
interpretation.}

{\large Julius Caesar was a reformer. He undertook important public works, from
redoing the calendar and relieving traffic congestion, to stabilizing food
availability. He also took steps specifically designed to help the poor.}

{\large For instance, he provided land grants to former soldiers and got rid of
the system of tax farming, replacing it with a more orderly and predictable tax
system. Not only that, he relieved the people's debt burden by about 25 percent.}

{\large Not surprisingly, though these policies were popular with the people,
many came at the expense of Rome's prominent citizens. Tax farming was, of
course, lucrative for those lucky few who got to extract money from the people.
High indebtedness was also lucrative for those who were owed money. These groups
found Caesar's reforms hitting them straight in their anachronistic pocketbooks
and, therefore, not at all to their liking. Popular though many of his reforms
might have been with the man on the street, they harmed the welfare of the
powerful influentials and essentials, and it was of course these people who cut
him down.14 Caesar made the mistake of trying to help the people by using a
portion of the coalition's share of rewards. It is fine for leaders to enrich the
people's lives, but it has to come out of the leader's pocket, not the
coalition's. The stories of Caesar and Castellano remind us that too many good
deeds or too much greed are equally punished if the coalition loses out as a
result.}

{\large As we have seen, there is a fine balance between giving enough private
goods to keep the coalition loyal and giving too much or too little. When money
is spent elsewhere that ``rightfully'' belongs to the coalition, there is a
serious risk of a coup d'\'{e}tat. When more money is spent on the coalition than
is their due, then the incumbent wastes funds that would otherwise have been
his.}

\subsection{Discretionary Money}

{\large What is a leader to do with any money that need not go to the coalition
to buy loyalty? There are two answers to this question: sock it away in a secret
account or lavish it on the people. Those who are most successful at stealing for
their own benefit open the door to joining our Haul of Fame.}

{\large Those who are more civic-minded spend discretionary money to help the
people, but only some of them are good at it. The successful join our Hall of
Fame and the unsuccessful, those with bad ideas about civic improvement, become
members of our Hall of Shame.}

{\large According to Hank Gonzalez, a politician in Mexico before
democratization, ``A politician who stays poor is poor at politics.''15 On this
basis, Zaire's Mobutu was a political genius. He allegedly stole billions.}

{\large His biographer, Michela Wrong, observes that, ``No other African
autocrat had proved such a wily survivor. No other president had been presented
with a country of such potential, yet achieved so little. No other leader had
plundered his economy so effectively or lived the high life to such excess.''16
Indeed, the word kleptocrat, meaning rule by theft, was coined to describe
Mobutu's style of governance. But though Mobutu made kleptocracy famous, he
didn't invent it.}

{\large King Solomon is reported to have had 700 wives. One can only wonder for
how many of them the choice was theirs or his alone. And then who can forget the
economic looting of the Caliphate. A serious estimate of the Caliphate's income
for the years 918--919 is 15.5 million dinars, 10.5 million of which was spent on
the Caliphs household.17 To put that in perspective, if President Barack Obama
had that proportion of the US economy available for his household's discretionary
use, he and Michelle would personally control a cool \$5 trillion, give or take a
few hundred billion. There, indeed, is the reason people took such great risks to
become the Caliph.}

{\large Small-coalition leaders have tons of money to use as they see fit.
Eventhough they compensate their coalition of essential backers well, with so few
who need to be bribed, plenty is left over. Some incumbents may choose to use
their discretionary pile of money for civic-minded purposes ---we'll talk about
them when we discuss hall of shame and hall of fame leaders---but an awful lot
just want to sock the money away for a rainy day.}

{\large It is to accommodate just such leaders that secret bank accounts exist.}

{\large The prevalence of master thieves among world leaders is striking. Some
succeed on a relatively small scale like Alberto Fujimori, Peru's president from
1990--2000 (including a so-called self-coup in 1992, in which he suspended his
own Congress and constitution). He probably didn't take more than a few hundred
million. And with Peru's return to democracy, Fujimori, who went into
self-imposed exile, found himself extradited, returned to Peru, put on trial, and
convicted of murder, human rights violations, bribery, and a host of other crimes
for which he was imprisoned.}

{\large He just did what any small-coalition leader does, but he had the
misfortune of being removed following popular discontent with his corruption and
being replaced by a large-coalition regime.}

{\large Others do considerably better considering the meager means of their
society. Serbia's Slobodan Milosevic, for instance, is believed to have
accumulated \$1 billion in a country where per capita income fell by 50 percent
during his rule. He followed key political principles: his coalition was small;
he taxed heavily, allowing him to make a fortune on the backs of the poor Serbs;
and he made sure to keep the people downtrodden.}

{\large Reliable reports indicate that he precipitated food shortages and
massive unemployment for Serbs who opposed him, leaving millions in desperate
circumstances while enriching 10,000 influential supporters.}

{\large Moving up the ladder of success when it comes to treating the national
treasury as one's personal account, we come to Iraq's Saddam Hussein.}

{\large He built billion-dollar palaces for himself while his country's infants
died of easily treated diseases. Other notable national thieves distinguished by
their relative take given the impoverishment of their societies are such figures
as Uganda's Idi Amin, Haiti's Papa Doc Duvalier, and then his son, Baby Doc
Duvalier, and the list goes on. They all typify the rule of successful
autocrats---they know how to build, manage, and finance tight coalitions while
enriching themselves. But they are all---except for Mobutu ---little leaguers
when compared to the champion haul of famers.When it comes to the cr\`{e}me de la
cr\`{e}me of kleptocrats, some of the greats include Indonesia's Suharto
(president from 1967 to 1997), Zaire's Mobutu (president from 1965 to 1997), the
Philippines Ferdinand Marcos (ruled from 1965 to 1986), and perhaps the current
incumbent front-runner, Sudan's Omar al-Bashir. He came to power in 1993 and
still is in office as of this writing, despite indictment by the International
Criminal Court for human rights violations, war crimes, and genocide.}

{\large Mr. Suharto, referred to by The Economist magazine as the king of
kleptocrats, is alleged by Transparency International to have stolen up to \$35
billion from his country.18 His late wife, Madame Tien, was often known as
``Madame Tien percent.'' Of course we cannot know what the true amount captured
by his family was but we do know that he depended on a small coalition, he had
lots of discretionary power, he survived in office for more than thirty years,
and he lived out his life as a free man in Indonesia (he died in 2008).
Apparently he was considered too ill to prosecute.}

{\large Like Suharto, Zaire's Mobutu lasted in power for more than thirty years,
ousted only once he was known to be suffering from terminal cancer.}

{\large Mobutu stole billions and lived the high life, whereas Suharto lived
more modestly considering his alleged means. Mobutu owned villas in the Swiss
Alps, Portugal, the French Riviera, and numerous residences in Brussels.}

{\large In addition he had a presidential palace in just about every major town
in Zaire, including a palace in his home town of Gbadolite. With a population of
about 114,000, one would not have thought the town needed an airport that could
handle the supersonic Concorde, but then one of the 114,000 sometimes residents
was Mobutu. He apparently rented the Concorde from Air France for his personal
use and, needing a proper airfield for it to land and take off, he built one for
himself.}

{\large Ferdinand Marcos, like Suharto, seemingly ran a successful economy.}

{\large The growth rate during many of Marcos's years was quite good, but then
the Philippine population was growing faster than the economy. Whereas Suharto
had been successful at controlling population growth, Marcos did not do so well.
But he certainly did well through his so-called cronycapitalism system in
enriching his coalition and himself. Transparency International estimated that
Marcos looted billions from his country. His wife, Imelda, notorious for her
enormous shoe collection, was brought up on charges related to the family's theft
of Philippine wealth and thegovernment succeeded in recovering \$684 million, a
relatively small portion of the total allegedly taken by Marcos and his family.
Despite their alleged thievery, the Marcos family, remarkably, is making a
political comeback in the Philippines. It seems money really makes the world---of
politics---go round! Omar al-Bashir, Sudan's president, stands accused of having
taken \$9 billion, so far, from his country. This was one of the revelations that
came to light when Wikileaks released US diplomatic cables in late 2010. The
claim, made by Luis Moreno Ocampo, chief prosecutor for the International
Criminal Court, includes the allegation that Bashir's money is held by Lloyd's of
London. They deny it, and, of course, so does Bashir. Indeed, the Guardian
newspaper reports that Khalid al-Mubarak, government spokesperson at the Sudanese
embassy in London, said, ``To claim that the president can control the treasury
and take money to put into his own accounts is ludicrous---it is a laughable
claim by the ICC prosecutor.'' Our way of thinking tells us that it is not only
not ludicrous, it is the way small-coalition petty dictators choose to govern,
and it works for them.}

{\large What would be ludicrous from a political survival perspective is if
Bashir does not ``control the treasury and take money to put into his own
accounts.'' Bashir has been in office for seventeen years so far, and despite his
external legal problems he continues to hold on to power and the country's not
inconsiderable purse.}

{\large Discretion means leaders have choices. So far we have looked at leaders
who use their discretion to enrich themselves but we do not mean to suggest that
people in power must be greedy louts like Marcos, Mobutu, Suharto, and Bashir. It
is entirely possible for autocrats to be civic-minded, well-intentioned people,
eager to do what's best for the people they govern. The trouble with reliance on
such well-intentioned people is that they are unconstrained by the accountability
of a large coalition. It is hard for a leader to know what the people really want
unless they have been chosen through the ballot box, and allow a free media and
freely assembled groups to articulate their wishes. Without the accountability of
free and fair elections, a free press, free speech, and freedom of assembly, even
well-intentioned small-coalition rulers can only do whatever they and their
coalition advisers think is best.}

{\large We close by reflecting on exemplars among well-intentioned leaders
ofwhat we call the hall of shame and the hall of fame---that is, those who wanted
to do well and didn't, and those who wanted to do well and did. The Soviet
Union's Nikita Khrushchev well illustrates a member of the hall of shame.}

{\large Khrushchev visited the United States in 1959 and announced a new
agricultural policy. He asserted that the USSR would overtake the United States
in the production of meat, milk, and butter. He neither knew much about
agriculture nor was he directly accountable to the people who did and who would
be burdened with trying to achieve his goals. There is no reason to believe that
Khrushchev hoped to gain personally from his illconceived agricultural policies.
Indeed, there is no evidence that he socked away public money for his personal
use. Rather, he seems genuinely to have wanted to improve the lot of the Soviet
people.}

{\large Good intentions notwithstanding, his agricultural program and its
implementation were a disaster. Local officials, wishing to please Khrushchev and
sensitive to the potential political consequences of failing to meet his
expectations, committed to fulfilling his demand for increased production. Their
pledges to meet his goals, of course, could not be met with the primitive farming
technology available in the Soviet Union. The upshot of Khrushchev's civic-minded
ideas was that farmers had to slaughter even their breeding cattle to meet the
meat quotas to which they were committed. They even went as far as to buy meat
from state stores, pretending later that they had produced it when they sold it
back to the government. This created both a false sense of improved production
and subsequent increases in prices as the slaughter of the breeding herds reduced
the size of future herds.}

{\large A few short years into his program, food prices skyrocketed, leading to
mass movements against the government. Official Soviet reports indicate that 22
people were killed, 87 wounded, 116 demonstrators were convicted, and 7 were
executed in response to the people taking to the streets.19 Two years later, with
the Soviet economy in shambles, rife with food shortages, and with the nation
humiliated in the Cuban Missile Crisis, Khrushchev was overthrown in a peaceful
coup. A bit more than twenty years later, Mikhail Gorbachev followed in
Khrushchev's footsteps, introducing economic reforms to mobilize the economy. His
programs also failed to have the effect he desired, but in his case they not only
led to hisouster but also to the end of the Soviet Union.}

{\large Mao Zedong and Deng Xiaoping in China mirrored Khrushchev and Gorbachev,
but with an important difference. All of these leaders seem to have been
initially motivated by the sincere desire to improve their economy. All seemed to
have recognized that failing to get their economy moving could pose a threat to
their hold on power. But unlike Mao, Mikhail, and Nikita, Deng belongs squarely
in the hall of fame. Like them, he was not accountable to the people and, like
them, he was not hesitant to put down mass movements against his rule. The
horrors of Tiananmen Square should not be forgotten. But unlike his fellow
dictators, he actually had good ideas about how to improve economic performance.}

{\large Deng and Singapore's Lee Kwan Yew are surely among the contemporary
world's two greatest icons of the authoritarian's hall of fame.}

{\large They did not sock fortunes away in secret bank accounts (to the best of
our knowledge). They did not live the lavish lifestyles of Mobutu Sese Seko or
Saddam Hussein. They used their discretionary power over revenue to institute
successful, market-oriented economic reforms that made Singaporeans among the
world's wealthiest people and lifted millions of Chinese out of abject poverty.
Nothing about their actions contradicts the rules of successful, long-lasting
governance. They were brutal when that served their interest in staying in power,
Deng with murderous violence and Lee Kwan Yew through the power of the courts to
drive his opponents into bankruptcy. Lee's approach was vastly more civilized
than Deng's, but nevertheless it was the arbitrary and tough use of power
dictated by the logic of political survival. And that, in the end, is what
politics is all about.}

{\large Most people think that reducing corruption is a desirable goal. One
common approach is to pass additional legislation and increase sentences for
corruption. Unfortunately such approaches are counterproductive. When a system is
structured around corruption, everyone who matters, leaders and backers alike,
are tarred by that corruption. They would not be where they were if they had not
had their hand in the till at some point. Increasing sentences simply provides
leaders with an additional tool with which to enforce discipline. It is all too
common for reformers and whistle-blowers to be prosecuted for one reason or
another. It is rumored that Yasser Arafat kept a record of all the
corruptactivities of the cabinet members in his government in the Palestinian
Authority. Increasing the punishment for corruption only increases the leverage
people like Arafat and others have over their cronies. Arafat effectively induced
loyalty to him both by allowing and monitoring cronycorruption within his inner
circle. And, while claiming that the Palestinian Authority was bankrupt, he
allegedly personally socked away a vast fortune, between \$4.2 billion and \$6.5
billion, according to Al-Jazeera.}

{\large Legal approaches to eliminating corruption won't ever work, and can
often make the situation worse. The best way to deal with corruption is to change
the underlying incentives. As coalition size increases, corruption becomes a
thing of the past. As we proposed for the IOC and FIFA, increasing the number of
members responsible for choosing the site of the games could end graft. The same
logic prevails in all organizations. If politicians want to end massive bonuses
for bankers then they need to pass legislation that fosters the restructuring of
corporate government, so that chief executive officers and board chairs really
depend on the will of their millions of shareholders (and not on a handful of
government regulators). As long as corporate bosses are beholden to relatively
few people they will provide those few key supporters with fat bonuses. Big
bonuses might not be popular with the public or even with their many
shareholders, but the public and unorganized shareholders can't simply depose
them. Insiders at the bank can. Legislating limits on compensation will simply
force CEOs to resort to convoluted and quasi legal means.}

{\large Such measures cannot improve corporate transparency or make balance
sheets easier to understand.}

{\large Those seeking to regulate corporate compensation and put businesses on
the straight and narrow path of enhancing shareholder welfare would do well to
examine closely the rules by which corporations are ruled. Firstblush fixes, such
as are often proposed by government officials, play well with their political
constituencies but also violate the fundamental logic of governance and so are
likely to undermine good corporate governance.}

{\large Consider the problem of corporate fraud. We have amassed considerable
evidence that securities fraud is more likely to be committed by firms with
financial problems and a large coalition than by firms with comparable financial
problems and a small coalition. After all, executives who depend on a relatively
large coalition are particularly vulnerable to being replacedwhen corporate
performance is poor. Being at greater risk of deposition, larger coalition
executives try to hide poor corporate performance through fraudulent reporting.20
What is more, one of the best early-warning indicators of corporate fraud is that
senior management is paid less---not more---than one would expect given the
firm's reported performance! The same issues hold when examining governments.
Politicians can introduce all sorts of legislation and administrations to seek
out and prosecute corruption. This looks good to the voters. But such measures
are either a fa\c{c}ade behind which it is business as usual, or they are
designed as a weapon to be used against political opponents. Neither a
smokescreen nor a witch hunt will root out sleaze. But make political leaders
accountable to more people and politics becomes a competition for good ideas, not
bribes and corruption. Of course leaders don't want to be more accountable. It
reduces their tenure in office and gives them less discretion. That's why we must
next turn to the difficult problem of how to get leaders to agree to such
actions.}
\pagebreak{}


\section{Chapter 7 - Foreign Aid}

{\large ADEMOCRAT'S LOT IS NOT A HAPPY ONE. SHE MUST continually try to find
better policy solutions to reward her large number of supporters.}

{\large And yet her hands are tied. She has little discretion in her policy
choices.}

{\large Her pet projects must be subjugated to the wishes of her large body of
supporters, and she can steal virtually nothing for herself. She is like a
selfless angel, appearing to place the concerns of her people over her own
interests. That is, until she turns her attention overseas.}

{\large When it comes to foreign policy, a democrat is prone to behave more like
a devil than an angel. In fact, in targeting her policies at foreign governments
she is likely to be little better than the tyrannical leaders who rule those very
foreign regimes.}

{\large In this chapter we explore five questions about foreign aid. Who gives
aid to whom? How much do they give? Why do they give it? What are the political
and economic consequences of aid? And what do the answers to these questions
teach us about nation building?}

{\large For any who were starting to think of democrats as the good guys, this
will serve as a wakeup call. Most of us would like to believe that foreign aid is
about helping impoverished people. The United States Agency for International
Development (USAID), the primary organization for allocating US aid, advertises
itself as ``extending a helping hand to those people overseas struggling to make
a better life, recover from a disaster or striving to live in a free and
democratic country. It is this caring that stands as a hallmark of the United
States around the world.'' Making the world a better place for its inhabitants is
a laudable goal for donors. Yet the people in recipient nations often develop a
hatred for the donor. And recipient governments (and donors too) often have
different views about what the money should be for. As we will see, democrats are
constrained by their big coalition to do the right thing at home. However, these
very domestic constraints can lead them to exploit the peoples of other nations
almost without mercy.}

\subsection{The Political Logic of Aid}

{\large Heart-wrenching images of starving children are a surefire way to
stimulate aid donations. Since the technology to store grain has been known since
the time of the pharaohs, we cannot help but wonder why the children of North
Africa remain vulnerable to famine. A possible explanation lies in the
observations of Ryszard Kapuscinski. Writing about the court of the Ethiopian
emperor Haile Selassie, Kapuscinski describes its response to efforts by aid
agencies to assist millions of Ethiopians affected by drought and famine in 1972:
Suddenly reports came in that those overseas benefactors who had taken upon
themselves the trouble of feeding our ever-insatiable people had rebelled and
were suspending shipments because our Finance Minister, Mr. Yelma Deresa, wanting
to enrich the Imperial treasury, had ordered the benefactors to pay high customs
fees on the aid. ``You want to help?'' the minister asked. ``Please do, but you
must pay.'' And they said, ``What do you mean, pay? We give help! And we're
supposed to pay?'' ``Yes,'' says the minister, ``those are the regulations. Do
you want to help in such a way that our Empire gains nothing by it?'' The antics
of the Ethiopian government should perhaps come as little surprise. Autocrats
need money to pay their coalition. Haile Selassie, although temporarily displaced
by Italy's invasion in the 1930s, held the throne from 1930 until overcome by
decrepitude in 1974. As a long-term, successful autocrat, Selassie knew not to
put the needs of the people above the wants of his essential supporters. To
continue with Kapuscinski's description: First of all, death from hunger had
existed in our Empire for hundreds of years, an everyday, natural thing, and it
never occurred to anyone tomake any noise about it. Drought would come and the
earth would dry up, the cattle would drop dead, the peasants would starve.
Ordinary, in accordance with the laws of nature and the eternal order of things.}

{\large Since this was eternal and normal, none of the dignitaries would dare to
bother His Most Exalted Highness with the news that in such and such a province a
given person had died of hunger.... So how were we to know that there was unusual
hunger up north?}

{\large Selassie fed his supporters first and himself second; the starving
masses had to wait their turn, which might never come. His callous disregard for
the suffering of the people is chilling, at least until you compare it to his
successor. Mengistu Haile Mariam led the Derg military regime that followed
Selassie's reign. He carried out policies that exacerbated drought in the
Northern Provinces of Tigray and Wollo in the mid 1980s.2 With civil war raging
in these provinces and a two-year drought, he engaged in forced collectivization.
Millions were forced into collective farms and hundreds of thousands forced out
of the province entirely. Mass starvation resulted. Estimates of the death toll
are between 300,000 and 1 million people. From the Derg's perspective the famine
seriously weakened the rebels, a good thing as Mengistu saw it. Many of us
remember Live Aid, a series of records and concerts organized by Bob Geldof to
raise disaster relief. Unfortunately, as well intentioned as these efforts were,
much of the aid fell under the influence of the government.3 For instance, trucks
meant for delivering aid were requisitioned to forcibly move people into
collective farms all around the country. Perhaps 100,000 people died in these
relocations.}

{\large There is no shortage of similar instances, where aid is misappropriated
and misdirected by the recipient governments. To take just one prominent example,
the United States gave Pakistan \$6.6 billion in military aid to combat the
Taliban between 2001 and 2008. Only \$500 million is estimated to have ever
reached the army.4 Nevertheless, aid continues to flow into Pakistani coffers.
Given the stated goals of aid agencies, once it becomes clear that money is being
stolen, one would expect them to stop giving. Alas, they do not.}

{\large Indeed, to dispel any pretense that donors are having the wool pulled
over their eyes, it is worthwhile to consider the Kenyan case. In her book,It's
Our Turn to Eat, Michela Wrong describes the exploits of an idealistic
bureaucrat, John Githongo. He was appointed anticorruption czar by the new Kenyan
president Mwai Kibaki.5 Given the notorious corruption of his predecessor, Daniel
Arap Moi, Kibaki ran on an anticorruption ticket.}

{\large International aid agencies began once again to lend to Kenya at
attractive rates. When the IMF gave Kenya a \$252.8 million loan, the Economist
reported that the finance minister was overheard whistling ``Pennies from
Heaven.''6 Githongo quickly discovered that the government thought his agency's
function was more to cover up corruption than to root it out. When he realized
the corruption went all the way to the president, he made secret tape recordings,
then fled to Britain and provided international organizations and banks
documentary evidence of the corruption. He was not alone in his claims. The
British ambassador to Kenya, Edward Clay, in beautifully florid language,
described the corruption as ministers eating ``like gluttons'' and ``vomiting on
the shoes'' of donors.}

{\large Although some years later the IMF and World Bank would eventually stop
lending to Kenya, this was not the immediate reaction. Indeed, the international
financial community shunned Githongo rather than the wrongdoers. His information
was ignored and he became a pariah at development meetings. Banks and bureaucrats
acted like people so desperate to eat at a restaurant that they continued to
ignore the health department's warning that the kitchen was overrun by rats.
Githongo now makes a meager living as a lecturer and consultant. Edward Clay
became persona non grata in Kenya and was discreetly retired by the British
government. Both Githongo and Clay effectively ended their careers by ``doing the
right thing.'' It is hard to believe that aid agencies remain so na\"{\i}ve as to
not understand how misused their funds are. Perhaps the truth lies in another aim
of the USAID---``furthering America's foreign policy interests.'' Perhaps the
United States is more interested in having a reliable ally in its fight against
global terrorism and needs assistance combating Somali pirates in the Indian
Ocean.}

{\large Against this harsh view, that aid is about recipients selling favors
overseas, is the rhetoric of Kenya's first president, Jomo Kenyatta, who at his
Independence Day speech in 1963, said:We shall never agree to friendship through
any form of bribery. And I want all those nations who are present today---whether
from West or from East---to understand our aim. We want to befriend all, and we
want aid from everyone. But we do not want assistance from any person or country
who will say: Kenyatta, if you want aid, you must agree to this or that. I
believe, my brothers, and I tell you now, that it is better to be poor and remain
free, than be technically free but still kept on a string. A horse cannot choose:
reins can be put on him so he can be led around as his owner desires. We will not
be prepared to accept any aid that will tie us like a horse by its reins.7 As
upright as this speech may initially sound, Kenyatta is in fact being
disingenuous. Are aid agencies willingly throwing away money? Or are they getting
something in return? We suspect that the key statement in Kenyatta's speech was
``whether from West or from East.'' In spite of his idealistic words, he was
covertly telegraphing that his government remained open to bids from both sides.}

{\large Political logic suggests that democratic donors are ready to turn a
blind eye to theft and corruption when they need a favor. If you remember,
Sergeant Doe of Liberia received over \$500 million from the United States during
his decade in power. And the United States got a lot in return: ``We [US] were
getting fabulous support from him on international issues. He never wavered [in]
his support for us against Libya and Iran. He was somebody we had to live with.
We didn't feel that he was such a monster that we couldn't deal with him. All our
interests were impeccably protected by Doe.''8 With the end of the cold war, the
United States had much less need for Doe's support. Only then did it find its
moral scruples. In 1989 it published a report, which we quoted earlier but is
nonetheless worth repeating: [Liberia] was managed with far greater priority
given to short-term political survival and deal-making than to any long-term
recovery or nation-building efforts.... The President's primary concern is for
political and physical survival. His priorities are very different from and
inconsistent with economic recovery . . . President Doe has greatallegiance to
his tribes people and inner circle. His support of local groups on ill designed
projects undercut larger social objectives.9 The truth is, foreign aid deals have
a logic of their own. Aid is decidedly not given primarily to alleviate poverty
or misery; it is given to make the constituents in donor states better off. Aid's
failure to eliminate poverty has not been a result of donors giving too little
money to help the world's poor.}

{\large Rather, the right amount of aid is given to achieve its
purpose---improving the welfare of the donor's constituents so that they want to
reelect their incumbent leadership. Likewise, aid is not given to the wrong
people, that is, to governments that steal it rather than to local entrepreneurs
or charities that will use it wisely. Yes, it is true that a lot of aid is given
to corrupt governments but that is by design, not by accident or out of
ignorance. Rather, aid is given to thieving governments exactly because they will
sell out their people for their own political security. Donors will give them
that security in exchange for policies that make donors more secure too by
improving the welfare of their own constituents.}

{\large The fact is, aid does a little bit of good in the world and vastly more
harm. Unless and until it is restructured, aid will continue to be a force for
evil with negative consequences---moreover it will continue to be promoted by
well-meaning citizens who in making themselves feel good are blinded to the harm
they are inflicting on many poor people who deserve a better lot in life.}

{\large Let's be clear, democrats act as if they care about the welfare of their
people because they need their support. They are not helping out of the goodness
of their hearts, and their concern extends only as far as their own people---the
ones from whom they need a lot of supporters. Democrats cannot greatly enrich
their essential backers by handing out cash. There are simply too many people who
need rewarding. Democrats need to deliver the public policies their coalition
wants.}

{\large Autocrats, on the other hand, can richly reward their limited number of
essential backers by disbursing cash. Money, which good governance suggests
should be spent on public goods for the masses, can instead more usefully (from
the autocrat's perspective) be handed out as rewards to supporters. And since
private goods generate such concentrated benefits to the people who matter (and a
good leader never forgets thatwho matters is all that matters), autocrats forsake
the public policy goals of the people. It is not that they necessarily care less
about the people's welfare than do democrats; it is just that promoting the
people's interest jeopardizes their hold on power. Remember the story of Julius
Caesar! Herein lies the basis for making foreign aid deals. Each side has
something to give that the other side holds dear. A democrat wants policies his
people like, and the autocrat wants cash to pay off his coalition.}

{\large Suppose there are two nations, A and B, each with a population of 100
people. The leader in each nation has \$100 with which to buy political support.
Suppose nation A is a democracy and its leader needs to keep fifty people happy
in order to stay in power. In contrast B is an autocracy and its leader needs to
keep five people happy. Suppose the people of both nations care about some policy
initiative taken up by nation B. For instance, to take a common cold war
situation, the policy might be nation B's stance towards the Soviet Union. The
citizens in nation A prefer that B adopt an anti-Soviet stance. Suppose the value
of such a stance to each of the people in nation A is equivalent to \$1. The
citizens of nation B don't want socialism outlawed and they don't want their
government to take an anti-Soviet stance. Indeed, since it is their country's
policy at stake, let's assume that the people of B care about their government's
policy much more than the people in nation A. To keep our example simple, suppose
that if B takes the anti-Soviet policy, then this is equivalent to a \$2 loss in
welfare for each of the 100 people in B.}

{\large In nation A, the leader has \$100 to make fifty people happy. If he
hands out the money to his supporters then each gets \$2. The leader in nation B
has fewer people to satisfy. If he handed out all his money, then each of his
five supporters would get \$20. Now, suppose the leader in B agrees to change to
the anti-Soviet policy in exchange for cash. The essential questions are how much
does B need and how much is A willing to pay to make this deal work?}

{\large The leader of B would only agree to trade policy for aid if it made his
coalition better off. The switch in policy is equivalent to a \$2 loss for each
of his supporters (and each of the inconsequential remaining 95 people in B who
are not influential), because they don't like the policy. So the leader of B
would never agree to the anti-Soviet stance unless the ``aid'' money hegets for
doing so is larger than this loss. Since he has 5 supporters to keep happy, and
each supporter suffers a \$2 loss, he needs at least \$10 in aid to offset the
political cost of turning anti-Soviet. That is, an extra \$2 for each of his 5
essential backers is the minimum required to change B's policy to anti-Soviet.}

{\large The leader in nation A only ``buys'' the anti-Soviet policy if its value
to his supporters is greater than the amount given up by each. Since the fifty
coalition members in A value the anti-Soviet policy at \$1 each, the money they
give up so that their government can buy an anti-Soviet stance from country B
must be less than \$1 each. Otherwise, they prefer the cash to the policy
concession. Since the policy shift is worth \$1 to each supporter in nation A and
there are fifty member of the coalition, this means the leader in A would pay up
to \$50 in ``aid'' to B's government to get B to become anti-Soviet.}

{\large Provided the aid transfer is between \$10 and \$50, the essential
backers in both nations are made better off by trading policy for aid. This
enhances the survival of both leaders. However, it makes each of the remaining
ninety-five people in nation B---those not in the winning coalition---the
equivalent of \$2 worse off. They are not compensated for the anti-Soviet policy
that they don't like.}

{\large This example, while extremely simple, captures the logic of cold war aid
flows. The United States provided Liberia's Sergeant Doe with an average of \$50
million per year in exchange for his anti-Soviet stance. This aid did not provide
for the welfare of his people, and is coincidentally close to the amount of money
Doe and his cronies are alleged to have stolen during his decade in power. From
the perspective of survival-oriented leaders, the rationale for aid becomes
clear. When the cold war ended, the United States no longer valued anti-Soviet
policies and was no longer willing to pay for them. Doe's government didn't have
much else to offer the United States that American voters valued, so he was cut
off. Without aid revenue, Doe could no longer pay his supporters enough for them
to suppress insurgencies, and so he died a gruesome death at the hands of Prince
Johnson.}

{\large For the reader who finds the above example too contrived, it is perhaps
worthwhile to look at a recent failed United States attempt to buy policy. In the
runup to the 2003 invasion of Iraq, the United States sought permissionto base US
troops in the predominantly Muslim nation of Turkey. Such basing rights would
have improved the US army's ability to engage the Iraqi army. Although Turkey is
allied with the United States through NATO, the idea of assisting a predominately
Christian nation to invade a fellow Muslim nation was domestically unpopular in
Turkey. During negotiations in February 2003, the United States offered Turkey
\$6 billion in grants and up to \$20 billion in loan guarantees. Given Turkey's
population of approximately 70 million, these aid totals amounted to about \$370
per capita.10 Turkey is relatively democratic. For a quick, back of the envelope
calculation, let's suppose its leader needs the support of a quarter of the
people. So the value of the United States offer works out to nearly \$1,500 per
essential backer. This is a substantial amount (a bit over 10 percent of today's
Turkish income per capita), but then the policy concession sought was very
politically risky. Indeed, it might be useful for the American reader to think
about how much compensation they would need before agreeing to allow foreign
troops a base in the United States in order to invade Canada.}

{\large It appears that \$1,500 per person was not enough. After much back and
forth, the Turkish government rejected the offer. They were holding out for
significantly more money so we know there was a price at which the policy
concession could have been granted, but it was a high price. The United States
was not willing to pay more and so the deal could not be struck. In the end,
Turkey granted a much less controversial concession for a lot less money. The
United States was allowed to rescue downed pilots using bases in Turkey.}

{\large Buying policy from a democracy is expensive because many people need to
be compensated for their dislike of the policy. Buying policies from autocracies
is quite a bit easier. Suppose Turkey were an autocracy and its leaders were
beholden to only 1 percent of the population. Under such a scenario, the value of
the US offer rejected by Turkey would have approached \$40,000 per essential
backer. Thinking back to the challenge offered to Americans, while few might have
sold out their northern neighbor for \$1,500, \$40,000 might start looking very
attractive to many. It is probably not an accident that the US invasion of Iraq
was launched from the decidedly very small coalition monarchies of Kuwait and
Saudi Arabia.The logic of how coalitions operate gives us a good handle on who
gives how much aid to whom. Getting the people what they want helps democratic
leaders stay in office. It is therefore no surprise that most foreign aid
originates in democracies. The price of buying concessions depends upon the
salience of the issue and the size of the recipient leader's coalition. As
coalition size grows, the recipient leader needs to compensate more and more
people for the adoption of the policy advocated by the donor. That means that the
price of buying a policy concession rises with the size of the prospective
recipient's group of essential backers. This creates an interesting dynamic.}

{\large As a nation becomes more democratic, the amount of aid required to buy
its policy goes up. But because the price is higher, donors are less likely to
buy the policy concession from it because it just gets to be too expensive. Poor
autocracies are most likely to get aid, but they don't get much. Although they
may have great needs, they can be bought cheaply.}

{\large We have confirmed this relationship between coalition size, the chance
to get aid and the amount of aid received (if any) in detailed statistical
studies of aid giving by the United States and other wealthy democratic nations,
namely the members of the Organization for Economic Cooperation and Development
(the OECD).11 Coalition size is not the only factor determining who gets aid or
how much is spent on buying concessions from them. The salience of the issues at
stake---what the policy concessions are worth---is an important determinant of
how much aid gets transferred. Notice that in the formula we just described, need
is not a significant factor. In fact, because an extra dollar is worth more to a
poor country than to a rich one, needier countries are likely to get less aid,
not more than the less needy among those receiving aid at all.}

{\large One extremely salient, and hence expensive aid for policy deal was the
1979 Egyptian-Israeli peace treaty. As part of this agreement, Egypt became the
first Arab nation to officially recognize Israel. Israel and Egypt ended
hostilities that had been nominally ongoing since the 1948 war (and had erupted
into actual warfare in 1956, 1967, and 1973). As part of the 1979 deal, Israel
withdrew from the Sinai Peninsula, which it had captured in the 1967 Six Day War
and both sides agreed to the free passage of shipping through the Suez Canal.
Peace between Israel and Egypt was ofgreat importance to the United States.
Beyond the strong domestic support for Israel, the United States was suffering
the ill effects of oil shocks in the 1970s. The sharp rise in oil prices raised
inflation and harmed the US and other Western economies dependent on oil
imports.}

{\large The United States, being desperate to avoid another oil crisis,
underwrote the deal, thinking, perhaps, that doing so would help stabilize the
situation in the region. As can be seen in Figure 7.1, the United States provided
enormous economic incentives for the Egyptian president, Anwar Sadat, to visit
Israel, attend the Camp David Peace Summit, and sign the treaty.}

{\large The recognition of Israel was an extremely unpopular policy shift in
Egypt. This is why Sadat could extract so much from the United States.}

{\large Unfortunately for Sadat it also led to his assassination in 1981.}

{\large Fundamentalists threw grenades and attacked with automatic rifle fire
during an annual parade. Although it officially recognizes Israel, the Egyptian
government has done almost nothing to encourage the Egyptian people to moderate
their hatred for Israel. In a BBC survey conducted nearly thirty years after the
Camp David agreement was struck, 78 percent of Egyptians indicated that they
perceive Israel as having a negative impact in the world, far above the average
in other countries whose citizens participated in the BBC survey on this
question. 12 Of course, changing the negative attitude toward Israel in Egypt
would just reduce the amount of aid the Egyptian government could extract from
the United States.}

{\large FIGURE 7.1 Total US Assistance to Egypt in Constant 2008 Million US\$
from USAID GreenbookRecent movement toward a more democratic government in Egypt
highlights the dilemma faced by democratic donors. Those who celebrate the
prospects of democracy in Egypt and favor peace with Israel have a problem. As we
have noted, the aid-for-peace-with-Israel deal could be struck exactly because
the autocratic Egyptian leadership and its coalition were compensated for the
anti-Israeli sentiment among its citizenry, a sentiment they helped preserve.
With the people now in charge, it would be natural for Egypt to shift away from
its peace with Israel. To prevent that, greater amounts of foreign aid will be
needed than was true under the Sadat-Mubarak dictatorships. Given the
significance of Israeli-Egyptian peace to American and Israeli voters, it is
likely that the higher price will be paid. That leaves the question, will that
greater aid be used to strengthen the military or improve the lot of ordinary
Egyptians?}

{\large As with Egypt, US assistance to Pakistan is much easier to explain by
looking at aid as a payment for favors rather than a tool for alleviating
poverty. In 2001, the United States gave Pakistan \$5.3 million and Nepal \$30.4
million in aid. Pakistan's aid had been greatly reduced by congressional mandate
following their test of a nuclear weapon in 1998.}

{\large Yet, on September 22, 2001, US president George W. Bush lifted
restrictions on aid. Pakistan received more than \$800 million in 2002.}

{\large Meanwhile, Nepal, not on the frontline of the fight against Al Qaeda and
the Taliban, received about \$37 million, just modestly more than their 2001
receipts. India, also not front and center in the battle against terrorism in
2002, received \$166 million from the United States, up barely from 2001 when
they received about \$163 million. Poverty had not changed in anymeaningful way
in any of these countries between 2001 and 2002, but their importance to American
voters most assuredly had.}

{\large Democrats are often perceived as being in the driver's seat and
dictating terms to autocrats. However, as in other matters, they are often the
ones who are constrained. They need to deliver the policies their backers want.
If they try to cut back on the aid they give or impose strict conditions, then
autocrats simply end the policy concessions.}

{\large Subsequent US relations with Pakistan offer clear evidence of this
pattern of waning and waxing aid. As we saw, aid went up following the terrorist
attacks of September 11, 2001, but then it began to taper as the war against the
Taliban inAfghanistan seemed to have been won by 2003.}

{\large Once Pakistan increasingly became a safe haven for the Taliban and Al
Qaeda, everything changed. Pakistan now found itself in a tough spot. If the
government opposed the Taliban who were infiltrating the Pakistani frontier with
Afghanistan, they were likely to face a domestic insurgency. If they supported
the Taliban they would face severe pressure from the United States. This dilemma
offered an opportunity for Pakistan to make greater demands for US aid if
Pakistan's government was to be induced to resist the Taliban. The demands were
made but the US Congress balked at giving Pakistan more, noting that much
American aid to Pakistan was diverted to uses not intended by the Congress. These
uses include the disappearance of some money and channeling much of the rest by
Pakistan to stave off what the Pakistanis perceive to be the greater threat from
India than from Muslim fundamentalist militants.}

{\large The United States, disgruntled with Pakistan, did not initially agree to
pay the higher price needed to get the Pakistani government to pursue the Taliban
and Al Qaeda militants within Pakistan. What was the upshot? As we have learned
to expect, the Pakistani leadership ignored US pressure and began looking for
ways to work with the Taliban. Aid is basically a pay or don't play program. The
United States wouldn't pay and Pakistan wouldn't play.}

{\large By 2008, the government of Pakistan's leader Asif Ali Zardari, was
paying only lip service to going after the militants. The Bush administration,
lacking more aid to offer, proved unable to change Zardari's mind. In fact, the
second half of 2008 saw only a perfunctory effort by the Zardari government to
fight the militants. There was a brief military offensiveagainst the Taliban,
starting on June 28 and ending in early July, with precisely one militant killed.
After that, although the Taliban aggressively pursued their own territorial
expansion in Pakistan, the Zardari government mostly looked the other way. Rather
than fight the militants, Zardari's regime made a deal with the Taliban in
February 2009, paying them about \$6 million and agreeing to the imposition of
Sharia law in the Swat Valley in exchange for the Taliban agreeing to an
indefinite ceasefire. The ceasefire unraveled by May. By this time, the Zardari
government seemed in trouble and the US government was fearful that the Taliban
might take control of Pakistan altogether. In the face of such dangers, the price
for aid had risen but so too had the desire in the United States to motivate the
Pakistanis to try harder to beat back the Taliban.}

{\large Congress passed the Kerry-Lugar bill at the end of September 2009. It
nearly tripled aid to Pakistan, increasing it to \$1.5 billion. Even then the
Pakistani government balked at taking the greatly increased aid because the bill
included requirements that the Pakistanis be accountable for how the money was
used. Facing resistance from Pakistan, Senator John Kerry clarified that the bill
was not designed to interfere at all with sovereign Pakistani decisions; that is,
he essentially assured the Pakistani leadership that the United States would not
closely monitor use of the funds. Shortly after, the Pakistani government
accepted the aid money and greatly stepped up its pursuit of militants operating
within its borders. By February 2010 they had captured the number two Taliban
leader, but, as we should expect, they have also been careful not to wipe out the
Taliban threat. Doing so would just lead to a termination of US funds.}

{\large The US government, for its part, is frustrated that even with \$1.5
billion in aid, Pakistan is not sufficiently motivated to beat back the Taliban.
As a result, the United States has stepped up drone attacks and the use of the
American military to pursue the Taliban within Pakistani territory, much to the
public---but we doubt private---dismay of the Zardari government. This is all
just the dance of the donors and the takers, the recipients looking for as much
money as possible and the donors looking for a highly salient, costly political
concession: the destruction of the Taliban.}

{\large Perhaps this is distasteful to those who would like to maintain the
fiction that aid is about alleviating poverty. Naturally some aid is given with
purely humanitarian motives, such as that given after a natural disaster. Yet it
ishard to reconcile the large scale of aid flows to Egypt and Pakistan with
idealistic goals. If aid actually helped the poor, then we might expect the
people in recipient nations to be grateful and hold donor nations in esteem.}

{\large Nothing could be further from the truth. In return for its
``benevolence'' to Egypt and Pakistan, the United States is widely reviled by the
people in those two countries; and with good reason.}

{\large In 2002, Pew undertook a study of values and opinion in forty-two
nations. One question asked about people's view of the United States. In Pakistan
69 percent of people reported an extremely unfavorable view of the United States.
In Egypt the figure was 79 percent. In the other forty nations an average of only
11 percent of the people shared this extremely negative view of the United
States. But then Pakistan and Egypt received an average of \$1.6 billion in
economic and military aid from the United States in 2002, while the other forty
nations averaged only \$97 million in aid. The pattern is borne out in detailed
statistical analyses. People in nations receiving lots of US aid seem to hate the
United States. Of course, much may have changed since 2002, and it would be
fascinating to see whether our assessment continues to be borne out in future
surveys.}

{\large Our account of aid may seem to paint the United States as international
bad guy number one. But the United States is far from the only aid donor.}

{\large While the United States is the largest overall donor, as a proportion of
its economic size it gives relatively little, about 0.2 percent of GDP.}

{\large Scandinavian nations give over 1 percent of their economic output in
foreign aid. Provided the policy rewards that a foreign power can provide to a
democrat's supporters are worth more to the supporters than the rewards that
could be directly purchased with the money, democrats support aid. Other nations
and agencies buy favors too, if not perhaps on the same grand scale that the
United States can afford. In fact, careful analysis shows that even the seemingly
generous Scandinavians give aid in exchange for policy concessions rather than
for altruistic reasons. They particularly like to use aid to gain trade
concessions and prosocialist ideologies in recipient regimes.13 Aid agreements
are notorious for being tied to conditions that help the donor. This means that
the agreement often specifies how, and more importantly where, the money is
spent. For instance, Germany might give a recipient money, but only if they use
it to buy German tractors. This mightseem an inefficient way to reward tractor
manufacturers. However, international trade laws often forbid direct subsidies.
Further, tied aid can bring future business, such as spare parts and service.
Canada is notorious for high levels of tied aid, 60--75 percent of all its aid.}

{\large Scandinavia and the UK claim to have the lowest levels of tied aid, but
even there, informal tying is common. For instance, Denmark had allocated \$45
million to repair ferries in Bangladesh. Rather than repair the ferries locally,
Denmark proposed taking the ships to Denmark and repairing them there at four
times the local cost. Amid protest from the Bangladeshi government, Denmark
decided to simply cancel the whole scheme so neither the Bangladeshis nor the
Danes benefited.}

{\large Just as the United States buys security and trade concessions with aid,
and the Europeans trade aid for business concessions, so too does Japan. Whales
need to fear Japanese benevolence. American voters like pork. In contrast,
Japanese voters like blubber, and Japanese leaders have been working hard to
deliver. In 1986, the International Whaling Commission (IWC) instituted a
moratorium on the commercial hunting of whales. While this ban was popular with
the people of most nations, the citizens of Iceland, Norway, and Japan want to
resume hunting. Currently the Japanese hunt a small number of whales through a
loophole that allows hunting for scientific research. These whales, of course,
end up being eaten. The Japanese government buys votes on the IWC with foreign
aid.}

{\large Recently, behind Japan's efforts, the IWC's membership has swelled to
include nations with no history of whaling. Some of these recent members, such as
Laos, Mali, and Mongolia, are landlocked. Japan's efforts have been rewarded with
growing support for the resumption of whale hunting.}

\subsection{The Impact of Aid}

{\large Example after example highlight the simple fact that aid is given in
exchange for policy concessions far more readily and in far larger quantities
than to reduce poverty and suffering. Following World War II, the rich nations
seem genuinely to have thought they could free the world of poverty through their
generosity. But no sooner did aid begin to flow than the politics of survival
intruded on the noble goal of reducing misery. It should not be surprising that
politics prevailed over benevolence. The record is unambiguous: foreign
assistance has proven ineffective at alleviating poverty and promoting economic
growth.}

{\large In the aftermath of World War II, Europe faced many challenges. Even the
victors had suffered enormous human and economic losses. The United States
launched a widespread relief program known as the Marshall Plan. Adjusted for
inflation, the United States pumped over \$182 billion in economic assistance
into Europe between 1946 and 1952.}

{\large Britain was the largest recipient, followed by West Germany, France, and
Italy.14 The United States's goals were to build stalwart states aligned against
communism. To achieve these ends, the United States needed an economically strong
Europe. States that were willing to combat communism and follow US-dictated
economic plans got aid; those not willing to do so, didn't.}

{\large Over the entire postwar period, total US economic assistance was nearly
\$1.3 trillion. Military aid over the same period was about \$650 billion. To
give some perspective, together these economic and military aid packages are
roughly twice the size of the 2009 US stimulus package and Troubled Asset Relief
Program (TARP) funds put together.}

{\large The success of the Marshall Plan proved hard to replicate. Trillions of
dollars have been pumped into developing economies, yet there is precious little
to show for it if we measure performance by assessing improvements in the quality
of life. As we've seen, aid has done virtuallynothing to relieve poverty.}

{\large Among policy makers, this record has prompted a fierce debate about the
efficacy of aid. For critics, it is all too easy to point to many aiddependent
states in Africa that are poorer now than they were at independence. The
development community likes to counter that such a direct comparison is unfair
and argues that, while aid-dependent nations have performed poorly, they would
have done even worse without aid. This defense, while wrong, is a sensible
argument that needs to be taken seriously.}

{\large We cannot simply condemn the aid enterprise just because nations that
received aid performed so poorly. To understand why, consider the following
provocative statement: hospitals kill! There is plenty of evidence to support
this claim. The likelihood of dying is much higher for a person in a hospital
than for a person who is not. Of course, most of us instantly see the error in
the evidence. The people in hospitals are sick. Healthy people are not to be
found staying in hospitals. But this kind of error from looking at statistics
without thinking about where they come from is all too common.}

{\large A colleague of ours, Peter Rosendorff, organized a petition and appealed
to the Santa Monica City Council to put a crosswalk at a dangerous junction near
his former house. The City Engineer said that quite to the contrary, the city was
planning to take out all the crosswalks because their study showed that
pedestrians were more likely to get killed in crosswalks than anywhere else. The
children of Santa Monica should be grateful to Peter. He took the time to explain
that the result was not because crosswalks were inherently more dangerous but
rather, they are where people cross the road.}

{\large Assessing the true impact of hospitals or a particular treatment or drug
is difficult unless we understand who is being treated. The medical community
uses randomized drug trials to test the efficacy of medicines.}

{\large Patients are randomly split into two groups: half get the medicine being
tested and half get a placebo. The effectiveness of the drugs is determined by
comparing the performance of the two groups. If, alternatively, the medicine is
given only to the sickest patients, then even if it is an effective treatment the
group getting the medicine might do worse than the group that did not. Likewise,
if aid agencies target aid at those nations facing themost serious problems, then
aid could appear ineffective even if it was actually working.}

{\large Ideally, to assess the effectiveness of aid, the international community
should undertake controlled experiments, giving aid to some randomly chosen
nations and withholding it from others. But since it is unlikely aid will ever be
allocated in this way, economists need to use complex (and controversial)
statistical procedures to adjust the results according to which nations get
treated. Rather than delve into these convoluted procedures, we offer some simple
evidence based on the United Nations Security Council (UNSC).}

{\large The UNSC is composed of five permanent members (the United States,
Russia, China, Britain, and France) and ten temporary members. The temporary
members are elected for two-year terms on the Security Council and they are
ineligible to be reelected in the two years after their term expires. Election to
the UNSC is highly prestigious and, as it turns out, valuable too. Unfortunately,
its value comes at a cost: bringing hardship for the people in many of the
countries that get elected. On average, nations elected to the UNSC grow more
slowly, become less democratic and experience more restrictions on press freedoms
than eligible nations that are not elected.15 For instance, during a two-year
term on the UNSC, the economy grows an average of 1.2 percent less for nations
elected to the council than for nations not elected. Over a four-year period (the
two years on the UNSC and the following two years of ineligibility), the
difference in growth averages 3.5 percent less for elected UNSC members, that is,
nearly 1 percent per year. The effects are much stronger in autocracies than
democracies.}

{\large The effects of council membership on growth are fascinating and should
cause us to question why the UN is held in such high regard. They also provide an
important piece of evidence about the impact of aid. Nations elected to the UNSC
get more aid. A UNSC seat gives leaders valuable favors to sell in the form of
their vote on the Security Council, and the aid they receive results in worse
performance for their economy. Recently there has been a profusion of studies
that show that nations elected to the UNSC get financial rewards from the
international community. They get more US and UN aid, better terms and more
programs at the IMF, World Bank, and a host of other institutions. 16 Membership
on the UNSC givesnational leaders a say in formulating global policy. Many
leaders, particularly those from autocratic nations, appear to prefer to sell
this influence rather than exercise it on behalf of their people's interests.}

{\large UNSC membership comes as close to a randomized test as we are likely to
get. Although who gets elected is not random, it is unrelated to the need for
aid. Indeed, population size appears to be the only systematic determinant of
UNSC elections. African nations, in particular, appear to have adopted a norm of
rotation. Nations are elected simply because it is their turn. The key point is
that prior to their election, UNSC members behave no differently from other
nations. But once elected they actually underperform. To return to the medical
analogy, nations elected to the UNSC are not sicker than nations not elected.
They get an extra shot of medicine (aid) and it makes them sicker (poorer, less
democratic, and less free press).}

{\large UNSC membership gives leaders the opportunity to sell salient policy
support. As we have seen over and over again, autocrats need to pay off their
coalition. Aid provides the money to do so and that helps leaders survive.
Further, aid encourages autocrats to reduce freedoms for two reasons. First, aid
revenue means leaders are less dependent upon the willingness of people to work,
so the leader does not need to take as many of the risks that arise from freedom,
risks they must take when their revenue and worker productivity depends upon
allowing people to communicate with each other. Second, the policy concessions
are generally unpopular, so leaders need to suppress dissent. UNSC membership
brings prominence and prestige to a nation. For an autocratic leader it also
means more easy money. For the people of autocratic nations the UNSC means fewer
freedoms, less democracy, less wealth, and more misery.}

{\large The historical record shows that aid has largely failed to lift nations
out of poverty. It is perhaps ironic that while aid affords the resources to
alleviate poverty and promote economic growth, it creates the political
incentives to do just the opposite. As Edward Walker, US ambassador to Egypt
(1994-- 1998) succinctly put it, ``Aid offers an easy way out for Egypt to avoid
reform.''}

\subsection{An Assessment of Foreign Aid}

{\large So what are we to think of foreign aid? Is it good for policy, or just
good politics?}

{\large It has certainly had its successes. Foreign aid, in the form of the
Marshall Plan, lifted the predominantly democratic nations of Western Europe out
of economic disaster. But the deck was stacked in the plan's favor. The United
States wanted to promote an economically powerful bloc as a means of combating
Soviet expansion. The plan therefore promoted economic growth. Democrats need
policy success and so were happy to comply with US policy goals in exchange for
substantial aid. Yet as we now know, subsequent aid donations have failed to
replicate the success of the Marshall Plan.}

{\large What aid does well is help dictators cling to power and withhold
freedoms. And yet, the quest to make aid work for the poor is phoenix-like in its
ability to rise and rise again. Or, come to think of it, maybe, like Sisyphus, we
just keep climbing the same hill only to fall down again.}

{\large Every decade or so, donor nations launch new initiatives to ``get aid
working.'' The most recent manifestation of this is the Millennium Development
Goals. Set up by the United Nations Development Program and adopted by world
leaders in 2000, this program sets poverty, health, gender equality, education,
and environmental targets to be reached by 2015. For instance, the poverty
eradication goals call for reducing by half the number of people living on less
than a dollar per day. Commendable as such declarations are, saying you want to
make poor people richer or at least less poor and actually doing so are
completely different things.}

{\large Millennium Development Goals are not the first such declaration to end
poverty. They were preceded by efforts to attain ``self-sustained growth'' first,
in the 1940s and 1950s through infrastructure development; then with the US P-4
program to make scientific and technological breakthroughs readily available to
poor countries; followed in turn by John Kennedy'sdeclaration that the 1960s
would be the ``Decade for Development.'' The goals, set back in the late 1940s,
remain the same and scant evidence suggests that the world is closer to achieving
those goals than it was in the 1950s or 1960s. William Easterly has discussed the
hope and optimism that accompany these roughly once-a-decade initiatives. He
laments that while each new plan says it will be different, they repeat the same
errors of the past. He argues the bureaucracy involved in giving aid ensures
funds are given in ways that impede rather than promote economic activity.}

{\large Poverty persists.18 Still, we don't need to be completely pessimistic
about aid. Our knowledge of how it works has greatly improved. For instance, we
know that aid works much better in the presence of good governance (just as we
know that more often than not it goes to places with bad governance).19
Proponents of development assistance point to the success of NGOs undertaking
directed programs within nations. Some of these programs have produced wonderful
successes. For instance, in 1986 the Carter Center started a plan to combat
Guinea worm disease, a parasite transmitted via dirty drinking water that
affected about 3.5 million people in seventeen nations across Asia and Africa. By
2009, worldwide infections had been reduced to about 3,000, mostly in southern
Sudan.}

{\large Nongovernmental organizations (NGOs) have proven that they can
effectively deliver basic health care and primary education. Yet harking back to
our discussion of public goods provided by small-coalition regimes, we can't help
but notice that these benefits are precisely the kinds of public policy programs
that even the most autocratic leaders want to initiate. NGOs are less successful
at providing advanced education.}

{\large Autocratic leaders in recipient states don't want people to be taught
how to think independently enough that they could organize opposition to the
government.}

{\large The successes of NGOs in promoting basic education, basic health care
and sanitation, and other basic necessities---digging wells, electrifying
villages, making very small business loans (at what we would describe in the
United States as usurious interest rates)---all point to a fundamental failing of
aid programs and to the harm being done unwittingly by many NGOs and their
supporters. It is a simple fact that aid money is fungible. This means recipient
governments have nearly completediscretion about moving funds from one project to
another. With direct government-to-government transfers it is easy to see why
this is so.}

{\large Autocrats want to provide private rewards for their supporters. NGOs
don't typically want to help the rich get richer and so they provide funding for
specific projects or do the work themselves. However, in practice, recipients are
very skilled at converting aid into the kinds of rewards they want rather than
the kind of rewards donors want them to provide.}

{\large The most sensible criterion for assessing aid's effectiveness asks not
how much money is spent or even how many wells are dug, schools built, or
villages electrified, but rather how many people are helped. NGOs count how much
money they spend to evaluate their efficacy, but this is a flawed criterion. It
encourages charities to help the easiest to reach and the more visible cases
while ignoring the difficult and harder to reach people who might well be those
in greatest need. Counting the number of people helped also encourages agencies
to undertake work that the government would have otherwise done on its own.
Remember that NGOs are most successful at providing basic public goods like
primary education and basic health care---services even autocrats want. When aid
funds are used to substitute for government spending, then few, maybe even no
one, has actually been helped unless the government uses the freed-up money for
other projects of benefit to the general population. Of course, they don't.}

{\large They use the money to shore up their political position and the loyalty
of their essential backers.}

{\large Cambodia is a case in point. Half of the Cambodian government's budget
is made up of foreign aid. Rather than supplementing government programs, these
donor funds are largely directed toward the bank accounts of government
officials. Indeed, Cambodia ranks among the world's most corrupt nations. As
USAID reports, ``Donor funds have flowed into education and health, and some of
these are passed on to ordinary citizens. But, there can be little doubt a
significant portion of funds earmarked for schools, teachers and textbooks, and
for clinics, health workers, and medications are diverted.''20 That is, the funds
intended for the people are diverted to rewards for Cambodia's rich. Often when
NGOs provide aid, the amount of assistance is substantially less than the numbers
reflect. Suppose an NGO provides basic education to 100 children in a village at
a cost of \$100 per child peryear, for a total expenditure of \$10,000. It sounds
like 100 people are helped by the NGO, pleasing their donors and bringing in more
money.}

{\large The reality of how many are helped, however, is less clear. The
government might well have paid to educate half of those children (or even all of
them) itself, even if there were no expectation of aid. Nominally the agency
helps 100 children. But in reality they help fifty children at twice the nominal
cost and let the leaders abscond with \$5,000. Is this good? Well, yes, for the
fifty extra children. Is it bad? Well, yes, for all of the people since the NGO
is facilitating the government's opportunity to steal more money and the NGO is
helping to further entrench a bad government in power to plague the people for
many more years to come.}

{\large Even some of the simplest acts of charity have bad consequences that
enhance government control and irresponsibility. To take a personal example,
Alastair took his children on a tour of Kenya in 2009. One of the stops was at a
primary school where they were encouraged to help paint classrooms. It seems like
a nice idea to help out and many people enthusiastically grabbed paintbrushes,
eager to brighten the classroom.}

{\large Alastair objected on principle and went outside and taught some of the
kids how to use a digital camera. Was he being a Grinch or was he encouraging
better economic policy? From the economic perspective, having highly skilled
tourists and their families paint classrooms is at best ineffective and at worst
downright harmful.}

{\large Comparative advantage lies at the heart of economics. Everyone should
specialize in what they are relatively good at and then trade their goods and
services. This way everyone ends up with more than if everyone tried to do a
little of everything by themselves. Consider the comparative advantage of Kenya
relative to Britain, where most of the people on Alastair's trip were from.
Education levels are low in Kenya and there are lots of unemployed manual
laborers. Kenya's comparative advantage is therefore in industries requiring lots
of relatively unskilled labor. Indeed this is where it flourishes: Kenya is a
huge exporter of flowers. It has a great climate for growing and lots of people
to tend to the labor-intensive processes of growing, picking, and packaging
flowers. The flowers are then flown to Western Europe for sale. In exchange,
Europe exports goods that require human and physical capital to
produce---pharmaceuticals, machinery, and computer software. Europe has a
relative abundance ofhuman and physical capital. It trades its capital-intensive
products for Kenya's labor-intensive agricultural products and both nations are
better off.}

{\large So what has this to do with painting school classrooms? Well, painting
classrooms, while fun, deprived a local worker of a much-needed job. If educated
westerners displace locals from manual labor jobs, then where can those workers
possibly compete given the current distribution of skills and capital? How can
they earn enough money to make a living, and perhaps send their children to
school to acquire greater skills that will make them more competitive when they
grow up? Rather than helping out, the wealthy tourists who took up paintbrushes
made some worker worse off. Repeat that exercise thousands of times and in
thousands of different ways and you can see how feel-good charitable acts can
benefit the donor vastly more than it actually benefits the needy.}

{\large On a much larger scale, the means of aiding needy countries can be
dramatically improved by taking stock of comparative advantage. For instance,
agriculture is highly protected from competition in Europe and North America
through price supports and subsidies. Agriculture was deliberately excluded from
the postwar trade settlement established by the General Agreement on Tariffs and
Trade (GATT) and its controversial successor, the World Trade Organization (WTO).
This is because rural areas are disproportionately represented in some countries
and so farmers tend to be the essential backers of leaders in many European
countries. Allowing farmers from developing nations to compete on the basis of
comparative advantage would go much further toward promoting economic growth than
providing poorly targeted and highly bureaucratized aid. Painting schools
provides just one tiny example of how assistance, even when well meaning,
undermines development. Bill Easterly's work shows that rather than this being
the exception, it is the norm.}

\subsection{Aid Shakedowns}

{\large We started this chapter with an account of Haile Selassie's shakedown of
donors. By now it should be clear that this practice is all too common, and
reflects the logic of privately given aid. When private donors provide aid,
governments must either strike deals with them so that the government gets its
cut---that, after all, is the value of aid to a small-coalition regime--- or, in
the absence of such deals, they must shakedown well-intentioned private donors.
Either way, the government must get its piece of the action or it will make it
impossible for donors to deliver assistance. That, for instance, is what the
Myanmar government did following the Nargis cyclone in 2008. They insisted on
having United Nations aid delivered to the government or barred from the country.
Why? Because, as we noted earlier, the military dictatorship wanted to use the
aid to enrich itself by selling food on the black market rather than distributing
it to those most in need. You might think this was the odd behavior of a horrible
regime, atypical of the response of government leaders following natural
disasters.}

{\large Not so! Consider the case of Oxfam relief for Sri Lanka in the wake of
the 2004 Indian Ocean tsunami.}

{\large Following a massive earthquake on December 26, 2004, a tsunami sent huge
waves of water rushing inland, killing over 230,000 people across fourteen
nations. Subsequent assistance totaled over \$14 billion. Yet even while the goal
of aid agencies might have been to relieve suffering, many recipient governments
took it as an opportunity to enrich themselves.}

{\large To distribute aid, Oxfam shipped twenty-five four-wheel-drive trucks to
the region. The Sri Lankan government impounded the trucks and insisted that
Oxfam pay a 300 percent import duty. For over a month (the first critical month
after the tsunami) the trucks sat idle and people went without food and shelter.
Eventually Oxfam paid over \$1 million to have its trucks released.}

{\large Before giving to a charity many people like to assess how much of
theirdonation goes to help people versus how much is spent on overhead.}

{\large Oxfam America, for instance, gets three out of four stars from Charity
Navigator, an organization that rates charities. Oxfam spends 6 percent of its
revenue on administrative expenses and 14 percent on fundraising. The remaining
80 percent is spent on programs, that is, helping people.}

{\large Unfortunately, 80 cents on the dollar is not the effective amount of
help provided. Remember those trucks and the 300 percent import duty---if such
cases are the norm (and they usually are) the actual aid benefit may only equate
to 20 cents on the dollar. If even as careful a charity as Oxfam is being shaken
down, then it makes us wonder what is happening to the rest.}

{\large It is virtually impossible to quantify how much aid gets diverted
towards the recipient government's objectives rather than the donor's intended
goals. However, we suspect this figure is huge. The fundamental problem is that
recipient governments are not appropriately incentivized to fix problems.
Consider the recent case of flooding in Pakistan in 2010. No one can blame the
government for the rains, but they are very much accountable for the subsequent
devastation. Over 20 million people were affected, 4 million made homeless, and
nearly 2,000 died.}

{\large Following severe floods in the 1970s, Pakistan set up a Federal Flood
Commission. On paper this agency has completed about \$900 million worth of dike
construction. Of course the reality is very different. Irrigation and flood
control are a source of graft, not public policy. And when the dikes are built,
they serve the interests of the wealthy; that is, coalition members, not the
people. As the floods swept downstream and threatened huge segments of the
population, President Zardari, who is nicknamed ``Mister 10 percent'' for his
alleged penchant to take that portion as his cut, acted as a good autocrat
should. He ignored the problem, headed off to Europe for a highprofile tour and
left his government to sacrifice the many to save the few. The government
reinforced dikes to protect essential supporters while allowing flooding to
continue in poor areas. Areas with ethnic minorities and large numbers of
opposition supporters were particularly likely to flood.21 Richard Holbrooke, the
late US Special Representative for Pakistan, described the flood as ``an equal
opportunity disaster,'' but this is far from the truth. Beholden to a few,
Pakistani leaders sacrificed the many. Theyreinforced barrages and dikes to
protect the homes and farms of their supporters and ignored the plight of towns
and villages. A local official acknowledges, local government figures in the
Sindh province conspired with prominent landowners to bolster the riverbank
running through their property and others deemed important, at the expense of
other regions, which were left vulnerable to flood waters.... It was not just
incompetence on the part of the authorities to protect the poorest of the poor
from potential floods; it was their deliberate intention that they should suffer
if floods were to take place.22 Obviously, from a good governance stance, this
behavior makes no sense. But in terms of ruling for one's own survival, it is an
ingenious move.}

{\large Supporters were reminded of the consequences of being outside the
coalition of essential backers. That is good for loyalty. And aid agencies rushed
to give money. The UN Secretary General, Ban Ki-Moon described the flooding as
the worst he had ever seen and called for massive foreign assistance. Many
Pakistanis preferred to directly assist those affected, noting ``we don't donate
to the government because we know it's mainly a way for government officials to
make money.''23 The international community was less careful. They gave Pakistan
\$1.7 billion in the first three months. That equates to about \$83 per affected
person. Presumably much of the money was siphoned off. It certainly was not used
for efficient disaster management.}

{\large Pakistan was not the only nation affected by severe floods in 2010.}

{\large Benin also faced historic floods that covered two thirds of the
country.}

{\large Although the absolute numbers were smaller because Benin is a much
smaller country, in proportion to its size the scale of the disaster was very
similar to that in Pakistan. Benin received much less assistance, only about one
twentieth of the aid per affected person. Yet despite this, its response has been
widely praised. But then of course Benin is much more democratic than Pakistan.
With a disastrous earthquake and tsunami having struck Japan in 2011 we are
confident much the same pattern as in Benin will be repeated. Japan, a democratic
country, will receive massive assistance as it should. It will use the money much
more wisely than thenations affected by the 2004 tsunami.}

{\large It is easy to understand why Zardari did so little to minimize the
impact of the flood on the masses, and, as some have suggested, he may have
deliberately made things worse. He had strong financial incentives. As the
magnitude of the disaster increased so did the amount of aid. His survival
depends upon paying off the few rather than protecting the many. Aid incentivizes
autocratic leaders to fail to fix problems. Had Pakistan implemented an effective
flood-management program instead of just saying they had, then the people would
have been much better off, but Zardari would have had no pretext to further
fleece donors.}

{\large Similar incentives plague Pakistani assistance with the war on terror.}

{\large Following the terrorist attacks of 2001, the United States repeatedly
sought their assistance in fighting the Taliban and Al Qaeda and in capturing
international terrorists---foremost of which has been Osama bin Laden, the leader
of Al Qaeda who was believed to be hiding in the tribal regions of Northwestern
Pakistan. Through 2008 the United States has paid Pakistan \$6.5 billion in
economic and military aid for its assistance. If Pakistan had captured bin Laden
and prevented the Taliban from operating in northern Pakistan, then the United
States would have been very grateful. But it would also no longer have needed to
pay Pakistan. As with effective disaster management that limits the number of
disaster victims, capturing bin Laden would have ended aid to Pakistan's leaders,
as his death may do now.}

{\large To understand how aid works, it is essential to take into account the
incentives from the perspective of the leaders who enact policy. Unless aid is
restructured to change these incentives, Pakistan has little reason to end
insurgency and terrorism. Instead both will be allowed to rumble on and
encouraged to expand if the West tries to cut aid. Fortunately, in addition to
identifying problematic incentives, our perspective offers the tools to
restructure aid to create the incentives to fix problems.}

\subsection{Fixing Aid Policy}

{\large The modus operandi of the international community is to give recipient
nations money to fix problems. A common argument is that the locals know much
better how to address their problems than do far-away donors.}

{\large That's probably true, but knowing how to fix local problems and having
the will or interest to do so is quite another matter. This policy of giving
money to recipients in anticipation of their fixing problems should stop. Instead
the United States should escrow money, paying it out only when objectives are
achieved.}

{\large Consider the problem of capturing Al Qaeda's former number two, Ayman
al-Zawahiri. Suppose the United States thinks \$4 billion is a reasonable reward
for his capture. Remember: to date the US government has paid \$6.5 billion
without success. This money could be escrowed, say at a Swiss bank. Upon
Zawahiri's capture, Pakistan could receive a payment of \$2 billion, with, say,
an additional \$1 billion in each of the two subsequent years. The deal could
perhaps be done more cheaply if we dispensed with the fiction that the money is
for the Pakistani people and paid it directly to Pakistani leaders.}

{\large If aid took the form of a reward-in-escrow scheme, then Zardari would
need to hand over Zawahiri to receive money. However, unlike the existing
incentives, he could deliver without fearing that the money will dry up once his
assistance is no longer needed. Zardari might prove unwilling or unable to
capture Zawahiri for \$4 billion. However, if this is the case then the United
States has lost nothing. He would certainly not be more likely to hand him over
if all he has to do is pretend to look for Zawahiri to keep the money flowing.
That, of course, is the way the current system works.}

{\large Undoubtedly there are many operational and procedural problems with
implementing an aid-in-escrow scheme. And these problems would be even more
difficult in terms of designing escrowed aid relief for disaster management. Yet,
it is better to tackle these tricky technical issues within aframework that
incentivizes leaders to solve the donor's problem than to carry on with failed
policies.}

\subsection{Nation Building}
{\large  }
{\large What, then, are the fundamental incentives for one institution to
interfere with the institutions of another? Democracies often claim that they
want to democratize other nations. They frequently justify both aid and military
intervention on this basis, but the evidence that they actually promote democracy
is scant. Those who defend such policies tend to cite Germany and Japan after
World War II, but that was sixty or so years ago, and on close examination it
took many years before these nations developed (or were permitted to develop)
independent foreign policies. The reality is that in most cases democracies don't
want to create democracies.}

{\large In 1939, US president Franklin Delano Roosevelt famously remarked about
Anastasio Somoza Garc\'{\i}a, a brutal Nicaraguan dictator, that, ``He's a son of
a bitch, but at least he's our son of a bitch.'' And herein lies the rub.
Dictators are cheap to buy. They deliver policies that democratic leaders and
their constituents want, and being beholden to relatively few essential backers,
autocrats can be bought cheaply. They can be induced to trade policies the
democrat wants for money the autocrat needs. Buying democrats is much more
expensive. Almost every US president has argued that he wants to foster democracy
in the world. However, the same US presidents have had no problem undermining
democratic, or democratizing, regimes when the people of those nations elect
leaders to implement policies US voters don't like.}

{\large Undermining democracy was the story behind US opposition to the Congo's
first democratically elected prime minister, Patrice Lumumba. He was elected in
June 1960 and he was murdered on January 17, 1961, just half a year later.
Lumumba ran into difficulty with Western democracies because of the policies he
adopted; not because he usurped power. He spoke out vehemently against the years
of Belgian rule over the Congo. In a speech during Congo's independence
celebration less than a week after his election as prime minister, Lumumba
announced, ``Nous ne sommesplus vos singes [We are no longer your monkeys].''24
In an effort to remove Belgian troops and diplomats from the Congo and to defeat
the secessionist movement in Katanga Province led by Moise Tchombe, Lumumba
sought Soviet military assistance. That was a big political error.}

{\large The massive bulk of evidence today points to US and Belgian complicity
in Lumumba's murder. Later the United States would become closely associated with
the Congo's (that is, Zaire's) Mobutu Sese Seko who, unlike Lumumba, was neither
democratic nor pro-Soviet. For a price (totalling billions by the time he fell
out of power thirty-two years after his ascent) Mobutu was willing to back US
policy. Democratically elected Lumumba was not and that meant he had to go.}

{\large Lumumba was not exceptional in his downfall at the hands of democratic
leaders. Hawaii's Queen Liliuokalani was overthrown in 1893. Her sin?}

{\large She wanted Hawaii and Hawaiians (no doubt including herself) to profit
from the exploitation of farming and export opportunities pursued by large
American and European firms operating in Hawaii. As these business interests
organized to depose her, the United States sent marines ostensibly to maintain
peace from a neutral stance, but in fact making it impossible for the Hawaiian
monarch to defend herself. And then we ought not to forget the overthrow of
democratically elected Juan Bosch in the Dominican Republic at the hands of the
American military in 1965. His offense: he liked Fidel Castro. Or Salvador
Allende in Chile, Mohammed Mosaddeq in Iran, or US opposition to the
democratically elected Hamas government in the Palestinian Authority, and the
list goes on. As we write these words, we see this policy of reluctance to
promote democracy at work for the US in the Gulf. The United States has a long
history of supporting useful autocrats. Indeed, US policy in that part of the
world stands as a perfect example of the perils of democratization. The incipient
democracies in the Gulf are unlikely to be positively inclined toward US
interests, in part because of deep policy differences and in part because we've
been funding for decades the oppression under which they were governed.}

{\large In case after case, the story is the same. Democrats prefer compliant
foreign regimes to democratic ones. Democratic interventionists, while
proclaiming to be using military force to pursue democratization, have a profound
tendency to reduce the degree of democracy in their targets,while increasing
policy compliance by easily purchased autocrats.25 Before this chapter you might
have been under the impression that democrats were angels compared to their
autocratic counterparts. This chapter has tarnished that image and there will be
more tarnishing to come. But rather than deplore European and Japanese prime
ministers and US presidents on principle, we need to pause for a moment and
consider what they are doing and why.}

{\large Democrats deliver what the people want. Because they have to stand for
election and reelection, democrats are impatient. They have a short time horizon.
For them, the long run is the next election, not their country's performance over
the next twenty years. However, as long as we the people want cheap gasoline and
an abundance of markets in which to dump agricultural products, and we want that
more than we want to see genuine development in poor countries, then our leaders
are going to carry out our wishes. If they don't, why they'll be replaced with
someone who will.}

{\large That's what democracy is all about---government of, by, and for the
people at home.}

{\large As a classroom experience, Bruce likes to ask his students how many of
them want to help remove poverty in Nigeria or Mali. This idea produces universal
support. And virtually everyone wants the government to provide aid to make it
happen. Yet when push comes to shove, enthusiasm fades.}

{\large For instance, he asks how many students are willing to give up their
mobile phone service and have the funds sent to help Nigeria. Hardly a hand goes
up. And when he asks about reducing their low-interest government loans that help
pay tuition if the money goes to the world's poor, even fewer hands go up even
though he reminds them that they are the world's incredibly rich ``poor'' and
that they profess to want to help the world's truly poor. Not at their own
expense! Aid is a tool for buying influence and policy. Unless we the people
really value development and are willing to make meaningful sacrifices towards
those ends then aid will continue to fail in its stated goals. Democrats are not
thuggish brutes. They just want to keep their jobs, and to do so they need to
deliver the policies their people want. Despite the idealistic expressions of
some, all too many of us prefer cheap oil to real change inWest Africa or the
Middle East. So we really should not complain too much when our leaders try to
deliver what we want. That, after all, is what democracy is about.}
\pagebreak{}


\section{Chapter 8 - The People in Revolt}

{\large ASUCCESSFUL LEADER ALWAYS PUTS THE WANTS OF his essential supporters
before the needs of the people.1 Without the support of his coalition a leader is
nothing and is quickly swept away by a rival. But keeping the coalition content
comes at a price when the leader's control depends only on a few. More often than
not, the coalition's members get paid at the cost of the rest of society. Sure, a
few autocrats become hall of famers who make their citizens better off. Most
don't. And those who don't will spend their time in office running down their
nation's economy for their own and their coalition's benefit. Eventually things
get bad enough that some of the people tire of their burden. Then they too can
threaten the survival of their leader.}

{\large Although not as omnipresent as the threat posed by the risk of coalition
defection, if the people take to the streets en masse then they may succeed in
overwhelming the power of the state. How to prevent and deal with such
revolutionary threats is therefore a crucial lesson for dictators and for
would-be revolutionaries that we must now confront.}

\subsection{To Protest or Not To Protest}

{\large In autocracies the people get a raw deal. Their labor provides tax
revenues that leaders lavish on essential core supporters. Leaders provide them
little beyond the essential minimal health care, primary education, and food to
allow them to work. And if a small-coalition leader is fortunate enough to have
another source of revenue, such as natural resources or a benevolent foreign
donor, then he may even be able to do away with these minimal provisions.
Autocrats certainly don't provide political freedoms. Life for people in most
small-coalition regimes is nasty, solitary, poor, brutish, and short. The people,
seeing the hopeless path they are on, invariably want change. They want a
government that provides for them and under which they can live secure, happy,
and productive lives.}

{\large Why, having suffered long and hard, might they suddenly and often in
multitudes rise up against their government? The answer resides in finding a
crucial moment, a tipping point, at which life in the future under the existing
government is expected to be sufficiently bad that it is worth their while to
risk the undoubted costs of rebellion. They must believe that some few who have
come forward first in rebellion have a decent chance of success and a decent
chance of making the lives of ordinary people better.}

{\large There is a delicate balance here. If a regime excels at convincing
people that stepping out of line means incredible misery and even death, it is
unlikely to experience rebellion. Yes, life under such a government is
horrendous, but the risk of failure in a revolt and the costs of that failure are
way too high for people to rise up. They might be killed or imprisoned, and they
might lose their job or home, even their children. That is why the Hitlers,
Stalins, and Kim Jong Ils of the world manage to avoid revolt. If rule is really
harsh, people are effectively deterred from rising up.}

{\large At first, a few especially bold individuals may rise up in revolt. They
proclaim their intention to make their country a democracy. Everyrevolution and
every mass movement begins with a promise of democratic reform, of a new
government that will lift up the downtrodden and alleviate their suffering. That
is an essential ingredient in getting the masses to take to the streets. Of
course, it doesn't always work.}

{\large The Chinese communists, for instance, declared the formation of a
Chinese Soviet Republic on November 7, 1931. They said of their newly declared
state, It is the state of the suppressed workers, farmers, soldiers, and working
mass. Its flag calls for the downfall of imperialism, the liquidation of
landlords, the overthrow of the warlord government of the Nationalists. We shall
establish a soviet government over the whole of China; we shall struggle for the
interests of thousands of deprived workers, farmers, and soldiers and other
suppressed masses; and to endeavor for peaceful unification of the whole of
China.2 Jomo Kenyatta, the leader of Kenya's independence movement and its first
head of state, likewise declared during a meeting of the Kenya African Union
(KAU) on July 26, 1952: If we unite now, each and every one of us, and each tribe
to another, we will cause the implementation in this country of that which the
European calls democracy. True democracy has no colour distinction.}

{\large It does not choose between black and white. We are here in this
tremendous gathering under the K.A.U. flag to find which road leads us from
darkness into democracy. In order to find it we Africans must first achieve the
right to elect our own representatives. That is surely the first principle of
democracy. We are the only race in Kenya which does not elect its own
representatives in the Legislature and we are going to set about to rectify this
situation.... It has never been known in history that a country prospers without
equality. We despise bribery and corruption, those two words that the European
repeatedly refers to. Bribery and corruption is prevalent in this country, but I
am not surprised. As long as a people are held down, corruption is sure to rise
and the only answer to this is a policy of equality.3Noble words from both Mao
Zedong and Jomo Kenyatta. Neither fulfilled his promises of equality, democracy,
and liberty for the average Chinese or the average Kenyan. Nor did either leader
eliminate corruption and special opportunities for their party faithful. Once
most revolutionaries come to power, their inclination---if they can get away with
it---is to be petty dictators. After all, the democratic institutions that
engender the policies the people want also make it hard for leaders to survive in
office. Leaders won't acquiesce to the people's wants unless the people can
compel them.}

{\large And when can the people compel an old dictator, seemingly set in his
ways, or a recently victorious revolutionary, newly ensconced in power, to look
out for them instead of for himself? The answer to that question is the answer to
when regimes choose the road to democracy rather than to sustained autocracy.}

{\large Before deciding to gamble on the promises of revolutionaries, each
prospective demonstrator must judge the costs and the risks of rebellion to be
tolerable relative to the conditions expected without rebellion and relative to
the gains expected with a successful uprising. Thus it is that middle-of-the-road
dictators, like Cuba's Fulgencio Batista, Tunisia's Ben Ali, Egypt's Hosni
Mubarak, the Soviet Union's Gorbachev (but not Stalin) are more likely to
experience a mass uprising than their worst fellow autocrats. That is not to say
that when the people rise up they are right in thinking life will be better. They
are taking a calculated risk. They surely understand that revolutionary success
holds the prospect of betterment, but not all revolutionary movements end in
democracy and not all result in an outpouring of public goods for the people.}

{\large Many revolutions end up simply replacing one autocracy with another.}

{\large On some occasions the successor regime can actually be worse than its
predecessor. This might well have been the case with Sergeant Doe's deposition of
Liberia's True Whig government or Mao's success against Chiang Kai Shek's
Kuomintang government in China. But the hope of the people when they participate
is that they will improve their lot, either by enlarging the winning coalition
through democratization or at least by becoming part of the new coalition.}

\subsection{Nipping Mass Movements in the Bud}

{\large There are two diametrically opposed ways in which a leader can respond
to the threat of a revolution. He can increase democracy, making the people so
much better off that they no longer want to revolt. He can also increase
dictatorship, making the people even more miserable than they were before while
also depriving them of a credible chance of success in rising up against their
government.}

{\large The extent of expected loyalty from the military is one critical factor
that shapes the direction an incumbent takes in responding to a nascent threat.}

{\large Leaders know that as isolated individuals the people are no threat to
their government. That is precisely why government leaders are reluctant to let
people freely assemble and organize against them. If the people find a way to
take to the streets en masse, the incumbent will certainly need very loyal
supporters willing to undertake the decidedly dirty work of suppressing the
masses if he is to survive.}

{\large We have met many leaders whose backers have deserted them at just such
key times. When insurgents challenged Sergeant Doe in 1990, his soldiers
terrorized and stole from the people of Liberia rather than combat the threat. In
1979, the shah of Iran was deposed when his soldiers joined the supporters of
Ayatollah Khomeini. Similarly, President Ferdinand Marcos in the Philippines lost
power in 1986 because his security forces defected. Russia's Czar Nicholas was
deposed when the people stormed his Winter Palace in St. Petersburg in 1917. The
army, poorly paid and facing deployment to the front in World War I, declined to
stop them. Many other crucial events in modern political history, from the French
Revolution to the collapse of the Soviet Union and its satellite states, also owe
their occurrence to the failure of core supporters to suppress the people at
critical moments. The recent so-called colored revolutions (Georgia's Rose
Revolution in 2003, Ukraine's Orange Revolution in 2004--2005, and the Tulip
Revolution in Kyrgyzstan in 2005), the Jasmine revolution inTunisia, as well as
the uprisings in Egypt are also manifestations of the same phenomenon.}

{\large In each case, coalition support evaporated at the key moment because the
leader could no longer promise his or her supporters an adequate flow of rewards
to justify their undertaking the dirty work required to keep the regime in place.
The Russian czar, France's Louis XVI, and the Soviet Union were all short of
money with which to reward supporters. The Philippines' Marcos and Iran's shah
were both known to be terminally ill.}

{\large New leaders typically reshuffle their coalition, so key backers of the
regime were uncertain whether they would be retained by the successor. Lacking
assurance that they would continue to be rewarded they stood aside and allowed
the people to rebel.}

{\large Revolutionary movements may seem spontaneous but we really need to
understand that they arise when enough citizens believe they have a realistic
chance of success. That is why successful autocrats make rebellion truly
unattractive. They step in quickly to punish harshly those who first take to the
streets. This is what we saw in Iran following the June 2009 presidential
election. The regime quickly stepped in, beating, arresting, and killing
protesters, until the people feared continuing to take to the streets.}

{\large A prudent dictator nips rebellion in the bud. That is why we have
reiterated the claim that only people willing to engage in really nasty behavior
should contemplate becoming dictators. The softhearted will find themselves
ousted in the blink of an eye.}

\subsection{Protest in Democracy and Autocracy}

{\large Dissatisfaction with what a government is doing is an entirely different
matter in democracies than it is in autocracies. In a democracy, protest is
relatively cheap and easy. People have the freedom and, indeed, the right to
assemble. They also have easy means through which to coordinate and organize. We
know from earlier chapters that governments ruled by a large coalition produce
lots of public goods, including a special set of such goods that fall under the
general heading of freedoms. These include a free press, free speech, and freedom
of assembly. These freedom goods make it much easier for large numbers of people
to exchange information about how they feel about their government and to express
objections to any policies they don't like.}

{\large These freedoms also make protest easy. But since people like these
freedoms, granting them can also dissipate their desire to bring down the
government. Protests are common in democracies but revolts intending to overthrow
the institutions of government are not. Democrats provide the policies people
want because otherwise the people will protest, and when people can freely
assemble there is little a leader can do to stop them except give them what they
want. Sometimes, of course, democratic leaders fail to give the people what they
want. Then people are likely to take to the streets to indicate their dislike of
a particular policy. That's what generally happens when a democracy goes to war,
for example. Some people favor the decision and others oppose it. Those who
oppose it frequently make their displeasure known by taking to the streets, and,
if there are enough of them and if they protest for a sufficiently sustained
time, they can provoke a policy change. Lyndon Johnson, for instance, chose not
to seek reelection in the face of deep dissatisfaction with his Vietnam War
policies.}

{\large In democracy, protest is about alerting leaders to the fact that the
people are unhappy, and that, if changes in policy are not made, they'll throw
therascals out. Yet in autocracy, protest has a deeper purpose: to bring down the
very institutions of government and change the way the people are governed.}

{\large Autocrats dislike freedoms because they make it easy for people to learn
of their shared misery and to collaborate with each other to rise up against the
government. Given their druthers, autocrats eliminate freedom of assembly, a free
press, and free speech whenever they can, thereby insulating themselves from the
threat of the people. Unfortunately for autocrats, without the public goods
benefits from these freedoms, people can find it hard to work effectively because
they cannot easily exchange ideas even about how to improve the workplace. And if
the people don't work effectively, then the leader cannot collect tax revenues.}

{\large Autocrats must find the right balance. Without enough freedom the people
are less productive and do little work, but give them too many freedoms and they
pose a threat to the leader. The degree to which autocrats rely on taxation to
fund the government limits the extent to which they can oppress the people.}

{\large Nations awash with natural resource wealth or lavished with foreign aid
rarely democratize. They are the world's most oppressive places. Their leaders
have resources to reward their essential supporters without having to empower the
people. In such societies, though the people really desire change, they cannot
act upon these wants. Without the ability to assemble, coordinating against the
government is difficult. What is more, the people know the leader can afford to
pay the coalition to oppress them. With little chance of success, the people keep
their heads down. Protest is rare and answered with even greater repression.}

{\large But what happens if the money dries up?}

{\large Take a look back at Figure 7.1, where we graphed Egypt's foreign aid
receipts through 2010. US aid to Egypt has been dropping as Egypt's peace with
Israel has aged and matured. The drop in aid has been substantial and that means
Egypt's former president, Hosni Mubarak, found himself in a weaker and weaker
position when it came to buying the loyal support of the military. The global
economic slowdown had compounded the importance of aid for the Egyptian regime.
With money drying up, a chance was created for a rebellion against his
government.}

{\large And, indeed, in early 2011, Mubarak, facing a poor economy anddecreased
aid receipts, also faced a mass rebellion.}

{\large When autocrats lack abundant resources they have a more difficult time
managing the people. First and foremost, leaders must pay their essential backers
or they will be gone. Leaders without adequate revenues from aid, natural
resources, or borrowing must obtain them by encouraging the people to work and by
taxing them. Unfortunately for leaders, many of the public goods that increase
productivity also improve the people's ability to coordinate and, therefore,
protest. Further, because the leader needs the tax revenues the workers provide,
such protests are more likely to be met with concessions than in a resource-rich
nation or one with huge cash reserves.}

{\large The factors that lead to rebellion are relatively uncomplicated. How
much a leader does to enhance the welfare of the people by providing public goods
determines the desire of the people to rebel. The level of freedom determines the
ease with which they can act upon these desires by taking to the streets.}

{\large Yet, though high levels of either factor are in evidence in a host of
countries around the world, protests remain rare. They require a spark.}

\subsection{Shocks Raise Revolts}

{\large Shocks that trigger protest come in many forms. On rare occasions
protests happen spontaneously. But more often it requires an event to shake up
the system and trigger protest. At the collapse of the Soviet Union and other
communist states in Eastern Europe in 1989, contagion played a major role. Once
one state fell, the people in the surrounding states realized that their state
was perhaps no longer invulnerable. Free elections in Communist Poland triggered
protests in East Germany. When it became clear that security forces would not
obey East German leader Erich Honecker's order to break up demonstrations, the
protests grew.}

{\large Successful protest in Germany spawned demonstrations in Czechoslovakia,
and so on. As each state fell, it provided a yet stronger signal to the peoples
of the remaining communist states. The states fell like dominos. And each was
suffering from a poorly performing economy, so that the East European dictators
could no longer assure private advantages to their supporters. Quite the
contrary, they had been reduced to a state in which many of their henchmen
understood it was better to abandon the dictator than go down in a blaze of glory
with their failed regimes. Much the same story repeated itself in the Middle East
in 2011.}

{\large As Tunisia fell, the people of Egypt realized that their leader might
also be vulnerable. So contagious was the belief that rebellion could succeed
that the once rock-steady Middle East quickly became fertile ground for mass
movements. People in Bahrain, Jordan, Yemen, Syria, Libya, and elsewhere tried
their luck.}

{\large A massive natural disaster, an unanticipated succession crisis, or a
global economic downturn that drives the autocrat's local economy to the brink or
beyond the brink of bankruptcy can also provide a rallying cry for protesters.
Other shocks can be ``planned''; that is, events or occasions chosen by an
autocrat who misjudges the risks involved. One common example is a rigged
election.Dictators seem to like to hold elections. Whether they do so to satisfy
international pressure (and gain more foreign aid), to dispel domestic unrest, or
to gain a misleading sense of legitimacy, their preference is to rig the vote
count. Elections are nice, but winning is nicer. Still, sometimes the people
seize the moment of an election to shock the incumbent, voting so overwhelmingly
for someone else that it is hard to cover up the true outcome.}

{\large Liberia's Sergeant Doe was foolish enough to hold an election. In doing
so, he provided the impetus for protest that he was lucky to survive. In 1985,
Thomas Quiwonkpa challenged Samuel Doe after it took weeks for Liberia's
electoral commission to ``count'' the votes. Perhaps Quiwonkpa took the
commission's dalliance as a sign of popular support and equally a sign of the
commission's lack of support for him. As his insurgency approached the capital,
Monrovia, the masses took to the streets against Doe's government. Unfortunately
for them, Doe's essential supporters remained loyal. The costs of protest became
very real. Doe's soldiers killed hundreds in retribution.}

{\large In post-Soviet Eastern Europe, ``legitimizing'' elections helped to
promote citizen uprisings. Rather than sustaining the regimes in power, elections
created the opportunity to replace them. In 2004, the incumbent Ukrainian leader,
Leonid Kuchma, having served two terms, decided, perhaps to the surprise of his
essential backers, to respect the two-term limit and retire. His chosen successor
was Viktor Yanukovych. The runup to the election looked like it came straight out
of a John Le Carre spy novel, with the leading opposition candidate, Viktor
Yushchenko, allegedly poisoned with dioxin, which left him horribly disfigured.}

{\large In the first round of the elections in October each of the leading
candidates received about 39 percent of the vote. This necessitated a runoff
election on November 21, in which the official results differed greatly from exit
polls. Even before the second round presidential runoff was complete, Yushchenko
called for the people to take to the streets. The electoral commission declared
Yanukovych the winner. However, protests mounted and the security forces
withdrew. Eventually the Supreme Court ruled that given the high level of fraud,
another ballot was needed.}

{\large Yushchenko then won the election handsomely.}

{\large Coalition dynamics play a key role in explaining why the security
forcesallowed the people to take to the streets. The president was changing.}

{\large Although the retiring incumbent, Kuchma, backed Yanukovych, he could not
ensure core supporters within the security forces that they would be retained
after the transition. As we saw with Louis XIV and many others, newly empowered
leaders, even when they have been chosen by their predecessor, are wise to shake
up their coalition, bring in their own loyalists, and dump many of their
predecessor's erstwhile backers. The security forces, being uncertain whether
they would keep their long-run privileges, declined to attack the masses, hedging
their bets about who would be more likely to reward them. Without force to
control the masses on the street, Yanukovych's supporters deserted. The people
brought Yushchenko to power, but an essential factor in their willingness to take
to the streets was the apparent lack of support for Yanukovych by the security
forces.}

{\large Sometimes the shocks that spark revolt come as a total surprise.}

{\large Natural disasters, while bringing misery to the people, can also empower
them. One frequent consequence of earthquakes, hurricanes, and droughts is that
vast numbers of people are forced from their homes. If they are permitted to
gather in refugee camps, then they have the opportunity to organize against the
government. You see, refugee camps have the unintended consequence of
facilitating free assembly. Earthquakes, storms, and volcanoes can concentrate
large numbers of desperate people with little to lose. They also can
substantially weaken the state's capacity to control the people.}

{\large On the morning of September 19, 1985, a large 8.1 magnitude earthquake
occurred on the Michoacan fault in the Pacific Ocean about 350 kilometers from
Mexico City. Mexico City is geologically vulnerable as it was built on the soft
foundation of the remains of Lake Texcoco. The clay silts and sands that make up
the lake bed plus the soil's high water content led to liquefaction (wherein the
ground behaves like a liquid) during the earthquake. The city was also built in
the absence of democratic rule, so few building codes had been enforced. As a
result, the distant quake caused enormous devastation throughout the city. The
death toll is highly disputed, but is thought to be between 10,000 and 30,000
people. An additional 250,000 were made homeless. The government did virtually
nothing. Left to rescue themselves, the people formed crews to dig forsurvivors
and organized refugee camps.}

{\large Born of necessity, these camps became the foundation for an important
political force in Mexico City. Instead of separate individuals unhappy with
their government, the earthquake formed a concentrated mass of desperate people.
Forced together into crowded camps, they shared their disillusionment with the
government. Organizing a protest rally was suddenly relatively easy. Ready and
willing participants were on hand and had little to lose. With the government
largely absent, these social groups became important political forces that
rapidly deployed as large antigovernment demonstrations. Unable to oppose these
groups, the government sought to accommodate them. It is widely believed they
played a key role in Mexico's democratization.4 The story of Anastasio Somoza's
deposition in the Nicaraguan Revolution in 1979 is broadly similar. In 1972, a
6.2 magnitude earthquake struck the capital of Managua, killing around 5,000
people and forcing about 250,000 homeless people into camps. Somoza and his
cronies profited from disaster relief but did nothing to resettle the enormous
number of homeless people who had gathered in refugee camps in the capital. These
camps became organizing grounds for the activists who eventually ended Somoza's
reign.}

{\large Not all autocrats make the mistake of ignoring disasters or ignoring the
creation of refugee camps. Consider the case of Myanmar in 2008. Than Shwe is the
military leader of Burma (officially known as Myanmar).}

{\large Although he has been described as an unremarkable man, he understands
the essentials of staying in power.5 On May 2, 2008, a massive cyclone, named
Nargis, swept across the Irrawaddy Delta in southern Burma causing havoc. The
delta's residents, mainly poor fishermen and farmers, received no warning of the
coming storm. The storm destroyed entire towns and villages. The official death
toll is 138,000, though other estimates suggest it might be as high as 500,000.}

{\large No one can blame Than Shwe for the storm or for the low-lying villages'
vulnerability to storm surge. However, Burma's military regime provided no
warning and did nothing to help the survivors, and for that they can be blamed.
Indeed they did worse than nothing: they actively prevented help from being
delivered. Many people in Rangoon, the major city in southern Burma that was
itself heavily damaged by the storm, attempted to helpthose in the delta. They
were not allowed. Small businessmen and traders were reduced to smuggling small
amounts of food into what remained of towns and villages.}

{\large The international community rallied to offer assistance. As tens, or
possibly hundreds, of thousands of people died of hunger and thirst in the
aftermath of the storm, ships full of disaster relief supplies sat off the
coast.}

{\large The military junta refused to allow relief workers in. Visas were almost
impossible to obtain. Information was extremely scarce. The government requested
aid, but asked that it be in the form of bilateral government-togovernment
assistance. Effectively, Than Shwe was saying, ``send cash, but you can't come
in.'' About a week after the disaster, the army started entering the larger towns
and villages of the delta. They were not there to help. They were there to
disperse survivors who had congregated in schools and temples.}

{\large Even though their numbers rarely exceeded a few hundred, survivors were
expelled from their shelters and told to return home. It mattered little that, in
most cases, their entire village had been destroyed and they had no food, water,
clothing, or shelter to return to. Indeed, one report observed, Survivors were
loaded onto boats and ferried back to the destroyed villages they had recently
escaped from. In some areas the clearances happened quickly; as the emergency
phase was now officially over, the authorities wanted people back in their
villages by June 2, when the next school term was scheduled to begin. But
survivors had no idea what they were returning to; was there even anything left
at places they had once called home? And how would they get food and water
there?6 The government did not even attempt to answer these questions.}

{\large In the PBS documentary, Eyes of the Storm, a senior Burmese general is
seen addressing a group of survivors.7 Starving and destitute, they ask for a
handful of rice. The general tells them that he is here now (but still he makes
no offer of assistance) and that they must go back to their village and ``work
hard.'' While the army seized (and sold on the black market) the few relief
supplies allowed in, the people were told they could eat frogs.}

{\large Effectively the government told these survivors to go away and die
quietly:inhumane in the extreme, but good small-coalition politics. Dead people
cannot protest.}

\subsection{Are Disasters Always Disasters for Government Survival?}

{\large Earthquakes and other disasters shake up political systems. However, the
nature of the shakeup is very different under different institutions.}

{\large Democratic leaders are very sensitive to disaster-related casualties.}

{\large Allowing people to die reveals serious policy failure. Democrats need to
deliver good public policy to reward their large number of backers. When they
fail to do so, they are liable to be removed. Disaster-related deaths result in
protest and in the removal of leaders in democracies.}

{\large To illustrate the difference in political responses to poor disaster
relief in a non-democratic and democratic setting, we contrast Cyclone Nargis
with Hurricane Katrina. Katrina struck the US Gulf Coast in August 2005. This was
the most costly natural disaster in US history, with damages estimated at \$81
billion. The death toll was 1,836.}

{\large The government, from President George W. Bush down to New Orleans's
mayor, Ray Nagin, stood accused of mismanagement and lack of leadership. Nagin
delayed the evacuation order for the city until nineteen hours before the storm
struck. As a result many people became trapped.}

{\large Then, once the New Orleans Superdome football stadium was set up as an
emergency center, it became overwhelmed when 30,000 rather than the anticipated
800 people showed up. Federal disaster relief was slow in arriving. Many of the
casualties were the sick and elderly who were overcome by heat and dehydration.}

{\large The tenure of US leaders was seriously jeopardized by the disaster.}

{\large Many observers think Katrina contributed significantly to the Republican
Party's midterm electoral losses in 2006 and their significant losses, including
the presidency, in 2008. Yet, while it is clear that the situation could have
been handled much better, it bears no resemblance to Cyclone Nargis. In contrast,
despite having allowed at least 138,000 people to die, Than Shwe felt
sufficiently well entrenched to allow a farcical election in2010, which the
government-backed parties won easily (at least according to official sources).}

{\large As seen in the cases of Mexico and Nicaragua, disasters can serve as
rallying points in autocracies. Disasters can concentrate opponents of the
regime, making it easier for them to coordinate. Yet the death toll from
disasters has relatively little effect on a dictator's chance of staying in
power. Indeed, if anything, large numbers of people dying in disasters actually
enhance the political survival of autocratic leaders.}

{\large As we know, autocrats don't buy political support with efficient public
policy. Resources spent saving the lives of the people cannot be spent on
cronies. In addition, as we have seen, autocrats are skilled at exploiting the
international community. By letting more people die they may in fact be able to
extract more relief assistance. The implications of these results are
frightening. Small wonder, then, that far more people die in natural disasters in
autocracies than in democracies.}

{\large Letting people die is good governance in autocracy, but it is disastrous
for the tenure of democrats. Although a detailed statistical analysis of the
relationship between disasters,8 deaths and leader tenure is complex, we compared
what happens in a country when 200 or more people die in a magnitude 5+
earthquake, to what happens in the same size earthquake if fewer people die. In
particular, we looked at the effect of such circumstances on the odds of a
country's leader being removed from office within two years following the
earthquake.}

{\large An earthquake alone does not threaten the survival of democrats.}

{\large However, if there are more than 200 people killed by the quake then a
democratic leader is almost certain to be removed from office. Under normal
circumstances, any democrat has a 40 percent chance of being ousted from office
in any two-year period. But for a democrat whose country suffered 200 or more
deaths in an earthquake, those odds rise to 91 percent. We believe this is the
case because democratic leaders are supposed to deliver effective public
policies, and those effective policies include ensuring good building codes are
enforced and excellent rescue and recovery is implemented following a natural
disaster. The death of many in such a disaster is a signal to everyone else that
the leadership has not done an adequate job of protecting the people and so out
go the leaders.Autocrats are less vulnerable to removal than democrats and
earthquake related deaths have little effect on their hold on power. Over a
typical two-year period, 22 percent of autocrats lose power. If their country
suffers a magnitude 5 or greater earthquake in the first year of this twoyear
window, the dictator's risk of being removed goes up to 30 percent.}

{\large However, the autocrat's risk of removal is reduced to 24 percent if the
earthquake killed more than 200 people. Earthquakes pose a threat to autocratic
leaders when people are forced into refugee camps and can organize against the
regime. People dying from an earthquake can't organize and so they do not
endanger a dictator's survival in office. As might be expected, given these facts
and the incentives they suggest, instances of 200 or more people dying in
earthquakes is much more common in autocracies than democracies.9 Not all
disasters are equal in the eyes of autocrats. Dictators are particularly wary of
natural disasters when they occur in politically and economically important
centers. Disaster management in China emphasizes this point. When an earthquake
struck the remote province of Qinghai in 2010, the Chinese government's response
was, at best, halfhearted. In contrast, its handling of disaster relief in the
wake of a 2008 earthquake in Sichuan won the approval of much of the
international community. The differences are stark and driven by politics. The
Sichuan quake occurred in an economically and politically important center where
a massed protest could potentially threaten the government. Qinghai is remote and
of little political importance. Protest there would do little to threaten the
government. The government did much less to assist people who could not threaten
them.}

\subsection{Responding to Revolution or Its Threat}
{\large  }
{\large Whether because of an unforeseen earthquake, a succession crisis, or a
financial meltdown, the threat of rebellion can rise, striking a leader like a
lightning bolt. What then is the right response to such a threat? History teaches
us that some crack down hard on rebels; some succumb to them; and some reform on
their own. The rules governing politics help us understand how different
circumstances lead to different choices among these options.}

{\large Successful rebellions, mass movements, and revolutions are not
commonplace, but neither are they extremely rare. Successful rebellions that turn
into democracy are pretty rare but they do happen. What characterizes revolutions
or revolutionaries who actually do what they promise: create a democracy to try
to better the lives of the people? And what characterizes revolutions that don't
take off or revolutionaries who don't democratize! We start with our old friend,
General Than Shwe of Burma.}

{\large The Than Schwe government makes sure that the people of Burma are kept
poor, isolated, and ignorant. There is no free press. The people are not allowed
to congregate. Few foreigners are allowed in, and those that are, are constantly
watched by the police. All these actions are designed to make it hard for the
people to coordinate and organize against the government. The people are
desperate for change, but the government makes it virtually impossible for them
to achieve it. In a telling 2005 account of how unhappy the people are, a
journalist for the Economist magazine recalls how they were continually asking
him how the United States could be prevailed upon to invade: ``the prospect of a
foreign invasion is a fond hope, not a fear.''10 The people of Burma want to be
the next Iraq! With such demand for change, it is little wonder that Shwe is
terrified of protest and that he focuses his attention on preventing it.}

{\large Than Shwe, like many others, takes the autocrat's preferred path
toeliminating the threat from mass political movements. He suppresses the people.
He doesn't need to buy them off because Burma is blessed, or cursed, depending
upon your point of view, with natural resources. Burma is a huge exporter of
natural gas, hardwood, gems, gold, copper, and iron.11 For instance, it is
thought to earn about \$345 million through the annual export of 1.4--1.6 million
cubic meters of hardwood, much of it extremely valuable teak. We use the term
``thought to'' because it is hard to know the figures for sure. For instance, in
2001, China reported that it imported 514,000 cubic meters of wood from Burma,
but Burma only records exports of 3,240 cubic meters. Presumably the income from
the unaccounted-for hardwood lines the pockets of the generals, rather than
funding the welfare of the people. It certainly does not fund infrastructure.}

{\large Indeed, the timber industry's attempts to process its products before
export have been nearly completely stymied by the absence of infrastructure. Of
course the absence of roads makes it even more difficult for the people to
assemble and threaten the government. This became particularly true after 2005,
when the government moved the capital to a remote mountain location where few
citizens are allowed to visit.}

{\large Burma is also the world's major producer of jade and rubies. Gem
auctions in 2007 are thought to have earned the nation \$370 million. Yet Burma's
biggest export earner is natural gas. Currently the offshore natural gas fields
generate between \$1--1.5 billion. These earnings are likely to increase over the
next few years with the development of additional fields and the opening of a
pipeline to ship gas directly to China. Little of this money makes its way into
the government's economic accounts. The official exchange rate is 6 kyaks to the
dollar. However, the real rate is around two hundred times higher. This means the
regime can deposit all gas export earnings in government accounts at the official
exchange rate and still keep 99.5 percent of the money for themselves.}

{\large Burma is poor. Than Shwe is rich! He is a fortunate leader. Since he
does not rely on the labor of the people he can suppress them ruthlessly.}

{\large This means that despite the miserable conditions they endure, the people
cannot easily rebel. And if they do, Shwe has the resources to buy the army's
loyalty and ensure that he stays on in power.}

{\large In February 2007, various newspapers reported on a minor demonstration
in Burma. Fifteen people (or twenty-five, depending uponreports) congregated to
protest. Their demands were for basic human rights. Within thirty minutes, many
of them, along with a number of journalists covering the protest, were arrested.
The regime perceives any kind of protest as a potential threat to its survival,
and with good reason.}

{\large General Ne Win seized power in a coup in 1962 and implemented a
socialist agenda. Protests and riots erupted in 1988. On August 8, 1988
(8/8/1988---a lucky set of numbers in many Asian cultures), troops fired at
demonstrators killing thousands. Protest over these atrocities forced Ne Win to
resign and agree to elections scheduled for 1990. Aung San Suu Kyi's National
Party for Democracy was the landslide winner, taking 58.7 percent of the popular
vote and capturing 392 out of 492 seats. However, with demonstrations and
protests under control, the military simply ignored the results and carried on
ruling.}

{\large Than Shwe came to power in 1992. His regime stamped out the protest of
February 2007 immediately. However, the junta's fear of protest was justified by
events in August 2007. Following an announcement of fuel price increases, on
August 19 about 500 protestors, led by many of the student protest leaders who
had been active in 1988, took to the streets.}

{\large These protests continued over a number of days. Participation soon
dwindled to double digits as the army engaged in widespread arrests, but in
September these protests reignited when several hundred monks marched. The army
beat the monks. Two monks were chained to a lamppost and beaten. One allegedly
died.}

{\large Monks are revered in Burma. The violence against them generated further
protest. A government delegation was trapped for six hours by protesters. Across
Burma monks took the symbolic act of overturning their alms bowls against the
government, a ritual known as thabeik hmauk.}

{\large Religious services were denied to all members of the military. Across
the country groups of monks began to march. These protests grew daily.}

{\large People began to talk of a Saffron Revolution, saffron being the color of
the monk's robes. This was precisely what Than Shwe feared most.}

{\large On September 25 the government ordered a crackdown. Protesters were
attacked, first with rubber bullets, then with live ammunition and whips. The
army also raided monasteries and carried monks away at night. Many of the
remaining monks were dispersed to their villages to prevent them from
congregating. After three days the protests hadcompletely ended. Although
government forces utterly crushed all opposition, it was a costly operation. The
esteem in which monks are held meant many soldiers were reluctant to harm them.
There were fears that the army might not be willing to attack temples. In the end
they were, but it no doubt cost the regime lots of resources to buy such
loyalty.}

{\large Inhumane as Shwe's actions were, they represented good autocratic
politics. He survived to rule another day. Nor is Shwe alone in placing being a
leader ahead of being a good human being. Life is miserable for the people in
resource-rich autocracies the world over. In these regimes, governments prevent
the people from coordinating. Their lives are isolated, miserable, and
unproductive. But revolution and protest are not hopeless acts, as the next set
of examples make clear.}

\subsection{Power to the People}

{\large A few of history's revolutionaries stand out for their success not only
in overthrowing a nasty regime, but in creating a people-friendly government in
its place. America's George Washington, South Africa's Nelson Mandela, India's
Jawaharlal Nehru, and the Philippines' Corazon Aquino are a few cases in point.
Perhaps even more interestingly, a few leaders threatened with revolution have
also democratized as the path to keep themselves in power. Ghana's Jerry John
``J. J.'' Rawlings is a perfect example. Common threads run through each of these
democratizers--- common threads that are absent from revolutions that replaced
one dictator with another, such as occurred under Mao Zedong in China, Fidel
Castro in Cuba, Porfirio Diaz in Mexico, and Jomo Kenyatta in Kenya.}

{\large Democratic revolutions are most often fought by people who cannot count
on great natural resource wealth to sustain them once they overthrow the
predecessor regime. These ``good'' revolutionaries just are not as lucky as
Libya's Colonel Muammar Qaddafi or Kazakhstan's Nursultan Nazarbayev. Although
contagion prompted an extreme threat to Qaddafi's political survival in 2011, his
oil wealth gave him a substantial fighting chance against the rebels. He had the
money to buy soldiers and keep them loyal, something his resource-poor Tunisian
and Egyptian neighbors did not. They, like good revolutionaries, had to rely on
the productivity of the people to generate the revenues they needed to reward
supporters. To encourage the people to work productively, good revolutionary
leaders needed to increase the people's freedoms. If the people can meet and talk
then they can earn more. As a very simple example, if farmers have access to
telephones, newspapers, and radios, then they can find out about market prices.
This allows them to take the crops to the right markets at the right time. Roads
and transport networks reduce transaction costs. Given the ability to earn more,
farmers work harder and the economy improves. Unfortunately, for a leader, those
same freedoms allowpeople to organize. The same media, telecommunications, and
roads that increase productivity also make it much easier for the same farmers to
hear about antigovernment demonstrations and join them. In much the same way that
Mexico City's 1985 earthquake lowered the barriers for coordination and
organization, increasing the public good of freedom makes protest more likely.}

{\large In the latter half of the 1980s, Mikhail Gorbachev faced a dilemma. The
economy of the Soviet Union was failing. Without additional resources he could
not continue to pay his essential backers. He might have turned to oil ---of
which Russia has plenty---to save the day, but oil prices were depressed in those
years. His best shot at keeping rebellion at bay was to liberalize the Soviet
economy, even though that also meant giving the people more power over their
lives. Gorbachev showed himself willing to take that risk.}

{\large Some might suggest that Gorbachev is a better person than Burma's
General Shwe. Probably he is, although we cannot help but notice that he cracked
down on constitutionally protected secessionist movements in Azerbaijan, Latvia,
Lithuania, and Estonia. The Soviet military response to the efforts of the people
in those republics to gain their freedom is hardly the response of an enlightened
leader. The Soviet ``black beret'' militia killed fourteen and injured 150 people
in Lithuania.12 A week later, 4 more people were killed and twenty injured when
Soviet forces cracked down on Latvia's efforts to attain independence.13 Why did
the enlightened Gorbachev take these harsh actions? He was responding to
political pressure from within his coalition. Topranking Soviet military officers
together with others urged Gorbachev to impose direct Kremlin rule in breakaway
provinces. They wrote in an open letter that was circulated at the Congress of
People's Deputies, ``If constitutional methods prove ineffective against
separatists, criminal speculators and the paramilitary forces that are continuing
to spill the blood of the people, we suggest instituting a state of emergency and
presidential rule in zones of major conflicts.''14 Gorbachev understood the
political risks of ignoring key military and political figures in his coalition
of essentials.}

{\large Gorbachev's failure to quash the secessionist movements was a
significant contributor to the decision by hardliners in his government to launch
a coup that overthrew him. He was restored to power---briefly---when the people,
backed by Boris Yeltsin, occupied Red Square and forced the coup makers to
retreat. But for Gorbachev the damage was done. He returned to power, recognized
the independence of Lithuania, Latvia, and Estonia, only to find himself unable
to sustain his government or even the existence of the Soviet Union. The Soviet
Union was formally dissolved three months later.}

{\large Gorbachev's policy of perestroika, aimed at restructuring the Soviet
political and economic system, can be understood as his effort to increase the
government's revenue to forestall just such problems as the secessionist
movements and their political aftermath. It didn't work out for him or the Soviet
form of government, but that is what it means to take risks. Sometimes they turn
out your way and sometimes they don't.}

{\large Today Russia is backsliding away from democratization. While under Boris
Yeltsin's post-Gorbachev government Russia maintained free and competitive
elections, that is much less true today. Vladimir Putin, former head of the
Soviet secret police (the KGB) and Yeltsin's immediate successor, moved the
political system sharply back from its emerging dependence on a large coalition
and good governance. He made it much more difficult for opposition parties to
compete by severely restricting freedom of assembly. He made it much more
difficult for opposition candidates to get their message across by nationalizing
television and much of the print media. He made it much more difficult for people
to articulate their dissatisfaction by making it a crime to make public arguments
that disparaged the government. In short, he systematically reduced the
availability of freedoms that compel a democratic government to attend to the
wishes of the people. Why could he do this? As we have noted, Russia is awash in
oil wealth. During Putin's time, unlike poor Gorbachev's, oil prices were at
record highs so he could pay key backers to help him quash opposition, and
possibly even have enough extra money from oil to keep the people happy enough
that they don't rebel against their loss of freedom.}

{\large The expansion of freedoms is a sure sign of impending democratization.}

{\large Economic necessity is one factor that produces such a concession.}

{\large Another is coming to power already on the back of a large coalition.
This was George Washington's, Nelson Mandela's, and Jawaharlal Nehru's
circumstance. For different reasons, each started out with a big coalitionand was
pretty much locked into trying to sustain it at least for a while as a necessity
if their government was to survive.}

{\large When Washington became president of the United States, the term ``United
States'' was treated as a plural noun. Back then people identified more strongly
with their state than with the nation. Washington headed an army that depended on
recruits from thirteen distinct colonies, each with their own government and each
paying for their military contingents out of their own pocketbooks. Washington
needed the support of a broad base of colonists and so he was stuck with a large
coalition from the get-go. In that circumstance he had to do what large coalition
leaders do--- disproportionately deliver public goods rather than private
benefits. First among these public goods was the Bill of Rights, guaranteeing the
very freedoms that are central to democratic, large-coalition governance.}

{\large Without these, the colonies could not agree to ratify the constitution
and serve under a single, unified government.}

{\large Nelson Mandela's story is not much different. His political movement,
the African National Congress (ANC), spent decades fighting the whitedominated
apartheid regimes of South Africa. Despite their efforts and the protracted use
of violence, they were unable to grow strong enough to overthrow their oppressors
through force. Nelson Mandela, who served twenty-seven years in prison for his
antigovernment stance and who refused early release from prison on the condition
of eschewing violence, eventually saw another way.}

{\large Possibly due to the effects of sanctions, the South African economy went
into a sharp decline during the 1980s. In 1980, per capita income was \$3,463.
But by 1993, the year in which F. W. de Klerk's apartheid regime passed a new
constitution paving the way for elections for all races, it had fallen to
\$2,903.15 De Klerk, and his long-term predecessor, Pik Botha, were in trouble
because with the economy in decline they did not have sufficient resources to buy
the continued loyalty required to keep the people suppressed. Under those
conditions, more money was needed to sustain the government. That money could
only be gotten from the people and many of them were already rebelling against
the apartheid government. Faced with very tough circumstances, the apartheid
regime had a choice: fight to the bitter end or cut a deal with Mandela.
They---and he---chose the latter course.The large-coalition compromise deal with
Mandela and his ANC meant allowing all South Africans equal rights. In practice,
this meant that the voting majority was turned over to the very people who were
most discriminated against during the years of apartheid. As a result, the
country became more democratic and its people freer. Whether it will last as the
ANC's interests come more and more to dominate the government remains to be seen.
There is the real danger down the road that unless the opposition wins office and
leadership is swapped back and forth between different political parties, South
Africa could go the way of Zimbabwe. Like South Africa, Zimbabwe started out on a
positive path to democracy based on a large-coalition deal between Joshua Nkomo's
ZAPU, Robert Mugabe's ZANU party, and Ian Smith's white-only UDI government. But
once Mugabe became sufficiently entrenched, he, like Putin in Russia, was able to
overturn the progress toward democratization. He plunged Zimbabwe back into the
role of a corrupt, rent-seeking, small-coalition regime that serves the interest
of the few at the expense of the many, black and white.}

{\large The successes of Washington, Mandela, and others were duplicated from a
very different starting place in the case of Ghana. There revolution did not lead
to democracy so much as the anticipation of revolution did.}

{\large Ghana's J. J. Rawlings understood well that liberalizing Ghana's economy
and empowering the people could endanger his hold on power.}

{\large But he also recognized that liberalization did not mean that the people
would inevitably end up revolting or that the coalition will turn on its leader.}

{\large Rawlings became the poster boy for the IMF and World Bank. He
implemented the economic reforms they prescribed, invigorated the economy,
instituted democratic reforms, and after serving two terms as president of Ghana
he stepped down. But that is not how he started out.}

{\large And the people were not as happy with him as this rosy picture would
suggest; at least not if you believe what Adu Boahen, a professor and leading
political opponent, had to say.}

{\large Boahen recounted Rawlings's explanation for the seeming passivity of the
Ghanaian people. As he observed, According to Rawlings, `The people have faced
and continue to face hardship. Naturally, people will grumble. But the fact that
Ghanaianshave been able to put up with shortages, transport difficulties and low
salaries, and other problems without any major protest, is an indication of their
confidence in our integrity, the integrity and good intensions of the PNDC
[Provisional Nations Defense Council] government. Visitors from other countries
have commented that in their countries there would be riots if conditions were
similar to those here. But the people know that they are not suffering to make a
corrupt government rich at all, we are suffering in order to concentrate all our
resources in the building of a just and prosperous society.' To this, Boahen
responded, ``I am afraid that I do not agree with Rawlings' explanation of the
passivity of Ghanaians. We have not protested or staged riots not because we
trust the PNDC but because we fear the PNDC! We are afraid of being detained,
liquidated or dragged before the CVC or NIC or being subjected to all sorts of
molestation. . . .}

{\large They have been [protesting] but in a very subtle and quiet way---hence
the culture of silence.''16 Boahen portrays Ghana in 1989 as permeated by
oppression. Yet by 1989 things were much better than they had been, as evidenced
by the fact that Boahen could make such speeches in the first place.}

{\large Rawlings's seizure of power on January 11, 1982, is often described in
almost biblical terms. Via his initials, ``J.J.,'' he was sometimes referred to
as ``Junior Jesus.'' And this was his second coming. He had been the figurehead
for a military revolt in 1979. Rawlings had movie star looks and exuded charisma.
But charm was not what kept him in power. Oppression and rich rewards for
supporters are the staples of leadership in smallcoalition systems and Rawlings
was no exception. In the first six months of his rule, 180 people were killed and
a thousand more were arrested and tortured. His loyal soldiers were renowned for
their thuggish brutality and Rawlings bought their loyalty through a massive
increase in military spending. Despite a collapse of the economy and a complete
meltdown of government finances, J.J. knew whose support he needed and paid them
first.}

{\large Rawlings had a talent for preventing protest. He stifled any free press
by restricting the supply of paper. His supporters meanwhile infiltrated the
trade unions and effectively made strikes impossible for many years. Heavoided
free assembly at every turn. Events a year into his rule demonstrate his
considerable organizational talents. In January 1983, Nigeria announced the
expulsion of 1.4 million Ghanaians working in Nigeria. In a few weeks 10 percent
of the population, most of them young adults, flooded back into a
poverty-stricken Ghana. The prospects of hundreds of thousands of disgruntled and
unemployed people milling around the capital terrified many in the government,
some of whom advocated closing the border to prevent them from arriving. Instead,
Rawlings welcomed them with open arms, but almost immediately ensured the
returnees were transported back to their home villages. His massive transport
undertaking avoided the camps that overwhelmed Mexico and Nicaragua. And it was a
much more humane approach than Shwe's.}

{\large Rawlings's fundamental problem was that Ghana was broke and the economy
had nearly completely collapsed. Ghana's food production was the second lowest in
Africa, ahead only of Chad. Rigged exchange rates lay at the heart of Ghana's
economic problems and its system of political rewards. The official exchange rate
for Ghana's currency, the cedi, was much higher than the black market rate.
Essential backers were allowed to exchange money at the official exchange rate
and then convert it on the street. Unfortunately this eroded the incentives of
farmers. By the early 1980s, it often cost farmers more for fuel to take goods to
markets than they earned by selling them. Seventy percent of the crops that did
make it to market were carried on people's heads. Smuggling crops across the
border to the Ivory Coast became the norm. The government responded by making
smuggling a capital crime. With little being produced for export, Ghana had
exhausted its capacity to borrow and was going bankrupt.}

{\large Rawlings had a big problem. He had seized power and wanted to pursue a
revolutionary socialist agenda, but he needed money. As Naomi Chazan phrased it,
``the question was no longer where resources were located but if they existed at
all.''17 To start with, Rawlings closed all the universities and had the students
help bring in the harvests. But such measures were not enough. The people were
hungry. Ghana had insufficient funds to pay for food imports and to pay the army.
As a good rule-abiding autocrat, Rawlings knew his priorities: pay the army! Soon
the term Rawlings necklace became a popular euphemism for the protruding
collarbones common among the emaciated people. He approached theSoviet Union, but
they had their own financial problems and, despite his move to the political
left, they declined to support him.}

{\large J.J. was between a rock and a hard place. He needed money and the only
place left to get it was to encourage the people to get back to work. At the
beginning of 1983 he enacted a radical reversal of policy. The cedi was allowed
to devalue. Producer prices paid to farmers were also increased, and subsidies
for gas, electricity, and health care were cut.}

{\large International financial institutions such as the IMF and World Bank were
delighted to have an adherent to their policies, but many of his closest allies
were not. This policy switch was also accompanied by a change in personnel.
Rawlings orchestrated a coup, making it a fait accompli before his targets could
organize and retaliate. Overnight his closest supporters found themselves without
influence. Some, such as J. Amartey Kwei, would be executed (allegedly for his
part in a notorious murder of judges). Others, such as the radical student
activist, Chris Atim, fled into exile.}

{\large It is telling that by 1985, Rawlings was the only remaining member of
the original ruling PNDC council. As a further sign of the direction Rawlings's
administration was taking, that council swelled from six members to ten.}

{\large No leader voluntarily increases the number of people to whom he is
beholden unless he thinks that doing so will help him stay in power.}

{\large As is to be expected, Rawlings was a reluctant democrat. He simply had
few options left. He needed money. To get it, he implemented policies that
empowered the people. Gradually, they could demand more. ``Rawlings was a victim
of his own success.'' He had given the people a voice by liberalizing the economy
and opening the airwaves. There was the perception of increased confidence. With
the economic crisis resolved the people began to feel ``we can do this without
someone telling us what to do.''18 As we have seen, by 1989 Boahen felt
comfortable openly criticizing Rawlings. Even he had to admit reforms had
improved the economy. The ``Rawlings necklace'' had been replaced by the
``Rawlings waistcoat (a fat belly).'' Having to implement policies to keep the
masses happy, Rawlings allowed a gradual expansion of the coalition to accompany
the expansion in public goods. In 1988 and 1989 local elections were allowed.
Rather than provoke mass protest, Rawlings stayed one step ahead. As a loose
affiliation of political interests coalesced into the Movement for Freedomand
Justice and called for multiparty elections, Rawlings defused their thunder by
organizing elections while the opposition was still disorganized.}

{\large In the 1991 presidential election he decisively defeated Adu Boahen, who
ran as the leader of the New Patriotic Party (NPP). Although there were some
discrepancies, international observers declared the results basically fair.}

{\large Elections have been basically fair ever since. Rawlings and his National
Democratic Congress party won again in 1996, beating John Kufuor. In 2000,
Rawlings stepped down and John Kufuor went on to serve the constitutional limit
of two terms. In 2008 the NDC candidate, Atta Mills, became president in a highly
competitive election.}

{\large Rawlings needed money and his only way of getting it was to empower the
people. By allowing the people to assemble and communicate he increased their
productivity. But he also made it easier for them to coordinate and organize
against him. He successfully avoided protest and revolution only by remaining one
step ahead of the people in terms of granting concessions. Yet he could not avoid
protest indefinitely. In 1995 between 50,000 and 100,000 people joined Kume
preko, or ``We have had enough'' marches through downtown Accra, the capital.
Although the government sought to prevent these marches, the courts overruled
them.}

{\large An independent judiciary encourages entrepreneurial zeal, but it also
protects the civil liberties of the people.}

{\large Today Ghana is an economically vibrant democracy. Its transition from
autocracy to democracy took place under the leadership of the larger than life J.
J. Rawlings. Yet it should be remembered that he was a reluctant democrat. Had he
had the resources he would have perpetuated his socialist revolution. Ghana
recently developed an offshore oilfield. Had these funds been available to J.J.,
or had the Soviets had the resources to back him, then it is likely he would
still be in power and Ghana would be a much poorer and more oppressive land.}

{\large Revolutionary moments often arise, as we saw in the cases of Ghana,
South Africa, and the Soviet Union, when an economy is near collapse--- so near,
in fact, that the leadership can no longer buy the military's loyalty.}

{\large Such circumstances are practically inevitable in the life of the vast
majorityof autocracies. Their rent-seeking, corrupt, inefficient economic ways
assure it.}

{\large At such moments the threatened government is more than likely to blame
the international community for their woes. After all, in exchange for policy
concessions, oppressive leaders have been able to borrow on relatively easy terms
from rich foreign governments and the international banks they control. Now these
governments face crushing debt obligations and no money to pay them. Getting more
money becomes difficult exactly because they are in such danger of defaulting on
their debts. And what do many well-intentioned people cry out for them: debt
forgiveness.}

{\large We must repeat what we have indicated earlier. Financial crises, from an
autocratic leader's perspective, are political crises. The leader hasn't cared a
whit about destroying his country's economy by stealing from the public. Now that
money is in such short supply that he can't maintain his coalition's loyalty
there is a moment of opportunity for political change.}

{\large Forgive the debts and the leader will just start borrowing again to pay
his cronies and keep himself in power. Nicolas Van de Walle compares the fates of
regimes in Benin and Zambia with Cameroon and Ivory Coast during crises.19 In the
former cases, international financial institutions withdrew support and the
nations democratized. In the latter cases, France stepped in with financial
support and no reform occurred.}

{\large So the first policy recommendation for outside observers when a dictator
faces national bankruptcy, and the protests likely to follow in its train is
this: don't save the dictator; don't forgive indebtedness unless the dictator
first actually puts his hold on power at real risk by permitting freedom of
assembly, a free press, freedom to create opposition parties, and free,
competitive elections in which the incumbent's party is given no advantages in
campaign funds, rallies, or anything else. Only after such freedoms and real
political competition are in place might any debt forgiveness be considered. Even
the least hint of a fraudulent election and of cutbacks in freedom should be met
by turning off the flow of funds.}

{\large Foreign aid, as we have seen, is a boon to petty dictators and to
democratic donor citizens and leaders. That makes persuading people to cut off
aid to help promote democratization very difficult indeed. But if the opportunity
arises it should be seized. Just as with debt forgiveness or new loans, foreign
aid should be tied to the actuality of political reform and notto its promise.
When leaders put themselves at risk of being thrown out by the people, then they
show themselves worthy of aid. When leaders allow their books to be audited to
detect and publicize corruption, then they are good candidates for aid designed
to improve the well-being of their people. Those who refuse to make politics
competitive and to expose and correct corruption will just steal aid and should
not get it if there is not an overwhelming national security justification for
continuing aid.}

{\large When a succession in leaders takes place, whether through revolution or
through the unexpected death or retirement of the person in power, then there is
a window of opportunity for real democratic change. We have seen that the early
part of an incumbent's time in office is the riskiest in terms of the new leader
being deposed. This is especially true for autocrats.}

{\large Indeed, they have a strong incentive to pretend to be democratic in
their first couple of years exactly because for that first period in office
democrats have a better chance of surviving than autocrats. We have seen just
such reforms coming out of Cuba, for instance, as Raul Castro took over from his
brother Fidel. Raul needed to consolidate his hold on power, reassure his backers
that he could provide for them, and to do that he had to get Cuba's economy to
grow. Solution: introduce some economic competition and a few political acts of
liberalization. Today Cubans can take greater advantage of private businesses
than was true at any time since the revolution. They can have cell phones and
some access to the Internet, expanding their reach for information and their
ability to coordinate with fellow Cubans even when they are not face-to-face. But
will these reforms last once Raul, or any newly ensconced autocrat, consolidates
his control over the flow of money and the loyalty of his key backers? Probably
not, unless the international community exploits its brief window of opportunity.
It can do so by tying economic assistance to a lock-in of political
liberalization.}

{\large All the methods mentioned above are exactly the tools that liberal
governments can adopt to promote lock-in of democratic reforms. But do they have
the will to do it? That, sadly, is unlikely---and for that problem we have not
yet found a cure.}
\pagebreak{}


\section{Chapter 9 - War, Peace, and World Order}

{\large THE BIBLE'S FIRST RECORD OF WAR ARISES WHEN the kings of Shinar,
Ellasar, Elam, and Goiim fight the kings of Sodom, Gomorrah, Admah, Zeboiim, and
Bela, two thousand years after biblical creation. The world has not seen that
long a stretch without war again. Indeed, it is fair to say that our world is
chock full of war---it has little peace, and hardly any order. We think that a
big part of why war is such a scourge is that too many leaders get the wrong
advice about how to solve international problems. Maybe, just maybe, by looking
at war in our political survival terms we will see ways to construct a more
peaceful and orderly world.}

{\large War is often said to transcend everyday politics, to be above the fray
of partisan rancor. But the fact is that war is inherently political. Carl von
Clausewitz, the nineteenth-century Prussian soldier and preeminent military
thinker, expressed it best, ``War is a mere continuation of politics by other
means.'' And as we have seen, political survival is at the heart of all
politics.}

{\large Georges Clemen\c{c}eau, leader of France during the later stages of
World War I famously declared, ``war is too important to be left to the
generals.'' He was right. Relative to parliamentarians, generals do a lousy job
of fighting wars. While completely counterintuitive, military men who lead
juntas, and other forms of autocratic leaders, are much worse at fighting wars
than their civilian counterparts who lead democratic governments.}

{\large That's why it's so important for us to unpack the contrasting advice
different leaders receive about how to and when to fight. It turns out that
autocrats and democrats should receive and follow radically different counsel.
War, being about domestic politics, can be best understood, we believe, by
putting it in the context of interchangeables and essentials and taking it out of
the context of grand ideas about national interest and balances of power.}

\subsection{War Fighting}

{\large 1 Two thousand five hundred years ago, Sun Tzu literally wrote the book
on how to wage war. Although his advice has been influential to leaders down
through the centuries, leading American foreign policy advisers have contradicted
his war-fighting doctrines.}

{\large Ronald Reagan's secretary of defense, Caspar Weinberger, George W.}

{\large Bush's first secretary of state, Colin Powell, and, with slight
modifications, Bill Clinton's second secretary of state, Madeleine Albright, all
prescribe a doctrine of when and how the United States should fight. And it
differs radically from the time-tested advice of Sun Tzu.}

{\large The reason Sun Tzu has served so many leaders so well over twenty-five
centuries is that his is the right advice for kings, chieftains, and autocrats of
every shape to follow. Until recently, and with very few exceptions,
smallcoalition systems have been the dominant form of government. But these are
the wrong policies for a leader beholden to many. Democratic war fighting
emphasizes public welfare, exactly as should be the case when advising a leader
who relies on a large coalition. Sun Tzu's advice is exactly right for a
small-coalition leader. To see this, let's have a look at the ideas expressed by
Sun Tzu and Caspar Weinberger.}

{\large Sun Tzu contended to his king, Ho Lu of Wu, that: The skillful general
does not raise a second levy, neither are his supply wagons loaded more than
twice. Once war is declared, he will not waste precious time in waiting for
reinforcements, nor will he turn his army back for fresh supplies, but crosses
the enemy's frontier without delay. The value of time---that is, being a little
ahead of your opponent---has counted for more than either numerical superiority
or the nicest calculations with regard to commissariat.... Now, in order to kill
the enemy, our men must be roused to anger. For them to perceive the advantage of
defeating the enemy, they must also have theirrewards. Thus, when you capture
spoils from the enemy, they must be used as rewards, so that all your men may
have a keen desire to fight, each on his own account.2 In contrast to Sun Tzu's
perspective, Caspar Weinberger maintained that: First, the United States should
not commit forces to combat overseas unless the particular engagement or occasion
is deemed vital to our national interest or that of our allies....}

{\large Second, if we decide it is necessary to put combat troops into a given
situation, we should do so wholeheartedly, and with the clear intention of
winning. If we are unwilling to commit the forces or resources necessary to
achieve our objectives, we should not commit them at all....}

{\large Third, if we do decide to commit forces to combat overseas, we should
have clearly defined political and military objectives. And we should know
precisely how our forces can accomplish those clearly defined objectives. And we
should have and send the forces needed to do just that....}

{\large Fourth, the relationship between our objectives and the forces we have
committed---their size, composition, and disposition---must be continually
reassessed and adjusted if necessary. Conditions and objectives invariably change
during the course of a conflict. When they do change, then so must our combat
requirements....}

{\large Fifth, before the United States commits combat forces abroad, there must
be some reasonable assurance we will have the support of the American people and
their elected representatives in Congress....}

{\large Finally, the commitment of US forces to combat should be a last resort.3
Sun Tzu's ideas can coarsely be summarized as follows: (1) an advantage in
capabilities is not as important as quick action in war; (2) the resources
mobilized to fight should be sufficient for a short campaign that does not
require reinforcement or significant additional provisions from home; and (3) the
provision of private goods is essential to motivatesoldiers to fight. Sun Tzu
says that if the army initially raised proves insufficient or if new supplies are
required more than once, then the commanders lack sufficient skill to carry the
day. In that case, he advises that it is best to give up the fight rather than
risk exhausting the state's treasure.}

{\large Weinberger's doctrine does not emphasize swift victory, but rather a
willingness to spend however much is needed to achieve victory, a point made even
more emphatically in the Powell Doctrine. Weinberger and Powell argue that the
United States should not get involved in any war in which it is not prepared to
commit enough resources to win. They, and Madeleine Albright too, argue for being
very cautious about risking war.}

{\large Once a decision is made to take that risk, then, as Weinberger (and
Powell) recognize, the United States must be prepared to raise a larger army and
to spend more treasure if necessitated by developments on the ground. War should
only be fought with confidence that victory will follow and that victory serves
the interests of the American people.}

{\large Sun Tzu emphasizes the benefits of spoils to motivate combatants (``when
you capture spoils from the enemy, they must be used as rewards, so that all your
men may have a keen desire to fight, each on his own account''). Weinberger
emphasizes the public good of protecting vital national interests. For Sun Tzu,
the interest soldiers have in the political objectives behind a fight or their
concern for the common good is of no consequence in determining their motivation
to wage war. That is why he emphasizes that soldiers fight, ``each on his own
account.'' Sun Tzu's attentiveness to private rewards and Weinberger's
concentration on the public good of protecting the national interest (however
that may be understood) represent the great divide between small-coalition and
large-coalition regimes. Our view of politics instructs us to anticipate that
leaders who depend on lots of essential backers only fight when they believe
victory is nearly certain. Otherwise, they look for ways to resolve their
international differences peacefully. Leaders who rely only on a few essential
supporters, in contrast, are prepared to fight even when the odds of winning are
not particularly good. Democratic leaders try hard to win if the going gets
tough. Autocrats make a good initial effort and if that proves wanting they quit.
These strategies are clearly in evidence if we consider the Six Day War in 1967.}

\subsection{To Try Hard or Not}

{\large As its name tells us, the Six Day War was a short fight, begun on June
5, 1967, and ending on June 10. On one side were Syria, Egypt (then the United
Arab Republic), and Jordan; on the other was Israel. By the end of the war,
Israel had captured the Sinai from Egypt; Jerusalem, Hebron, and the West Bank
from Jordan; and the Golan Heights from Syria. The air forces of the Arab
combatants were devastated and Egypt accepted an unconditional cease-fire. The
Israelis had easily defeated their opponents.}

{\large From a conventional balance-of-power perspective the outcome must be
seen as extraordinarily surprising. From the political-survival point of view, as
we shall see, it should have been perfectly predictable.}

{\large To understand the war and how our way of thinking explains it, we must
first comprehend some basic facts about the adversaries. The combined armed
forces of the Arab combatants on the eve of war came to 360,000, compared to
Israel's 75,000; that is, the Israeli side represented only 17 percent of the
available soldiers.4 The Arab combatants accounted for 61 percent of the national
military expenditures of the two sides. For starters, comparing these two sets of
values already tells us something very important that reflects a fundamental
difference between large-coalition and small-coalition governments. Although the
Arab side had 83 percent of the soldiers, they spent considerably less per
soldier than did the Israelis.}

{\large Remember that large-coalition leaders must keep a broad swath of the
people happy. In war that turns out to mean that democrats must care about the
people and, of course, soldiers are people. Although conflict involves putting
soldiers at risk, democrats do what they can to mitigate such risk. In
autocracies, foot soldiers are not politically important.}

{\large Autocrats do not waste resources protecting them.}

{\large The difference in expenditures per soldier is greater even than the
numbers alone indicate. The Israeli military, like the military of democracies in
general, spends a lot of its money on buying equipmentthat is heavily armored to
protect soldiers. Better training and equipment enable democracies to leverage
the impact of each soldier so they can achieve the same military output while at
the same time putting few soldiers at risk.5 The Egyptian military's tanks, troop
transports, and other equipment were lightly and cheaply armored. They preferred
to spend money on private rewards with which to ensure the loyalty of the
generals and colonels.}

{\large Gamal Abdul Nasser, Egypt's president at the time, was not elected by
the people; he was sustained in office by a small coterie of generals whose own
welfare depended on the survival of his regime. For that reason, he was not
beholden to the wives and mothers who scream about the avoidable deaths of their
loved ones. Israeli prime ministers are elected by those mothers and wives, and
this is reflected in the superior equipment, armor, and training given to Israeli
soldiers. Give our troops the best, is a democratic refrain. This was why there
was such a stink about US soldiers having insufficient body armor in Iraq and
Afghanistan, and why the United States rushed to fix this deficiency, even if in
some cases the extra armor made some vehicles so heavy that they became close to
inoperable.}

{\large A bit of close reasoning shows us that making an extra effort to win the
war made tons of sense for the Israelis and no sense at all for their opponents.
Let's have a look at why it is that democrats, like Israel's prime minister Levi
Eshkol, try hard to win wars and autocrats, like Egypt's Nasser, don't. Indeed,
we will see that for a small-coalition autocrat like Nasser it could even make
more sense to lose the war but keep on paying off his cronies than to win the war
if doing so came at the cost of asking the cronies to sacrifice their personal
private rewards.}

{\large In a small-coalition regime, the military serves two crucial functions.
It keeps the incumbent safe from domestic rivals and it tries to protect the
incumbent's government from foreign threats. In a large-coalition government, the
military pretty much only has to worry about the latter function. Sure, it might
be called upon to put down some massive domestic unrest from time to time, but
its job is to protect the system of government and not the particular group
running the government. Its job description does not include taking out
legitimate domestic political rivals. Autocrats, of course, don't recognize any
rivals as legitimate. And to do their job in anautocracy, as Sun Tzu eloquently
argued, the soldiers must have their rewards. If they don't they might turn the
guns on the leadership that employed them to keep rivals at bay. With that in
mind we can begin to unravel the seeming surprise of a larger military, backed by
a larger gross domestic product---\$5.3 billion derived from 30 million people in
1967 Egypt, compared to \$4 billion generated by only 2.6 million
Israelis---losing to a puny state.}

{\large Imagine that the Israeli government spent as much as 10 percent of its
revenue on private rewards, probably a high estimate. Imagine that the Egyptian
government spent 30 percent of its revenue on private rewards; that is, more than
the Israelis as befits the comparison of large- and smallcoalition regimes that
we have seen in the earlier chapters. Then how valuable did winning have to be
for Israel's coalition and for Egypt's coalition to justify trying so hard that
it meant spending extra money on the war effort?}

{\large Anticipating the high risk of war, the usually fractious Israelis formed
a unity government in May 1967, reflecting the national commitment to win the
coming war. We know the government allocated \$381 million to the military in
1967. That means, given our assumptions, that \$38 million of that pot of money
might have been available for private rewards to the government's winning
coalition. Of course, even more would have been available across the whole
economy (both in Egypt and in Israel) but we focus just on money committed to the
military in 1967, thereby understating our case. Being a unity government it is
likely that the Israeli winning coalition was very large, but we will err on the
side of conservatism and assume that the government needed just 25 percent of the
population to sustain it. That puts the winning coalition's size at roughly
650,000 people. With these numbers in mind, we see that the potential value of
private rewards taken from the military budget for government supporters in
Israel would have been less than \$60 a head (\$381 million in military
expenditures x 10 percent for private rewards/650,000 coalition members = \$58.62
per coalition member).}

{\large Each member of Israel's coalition could have had a choice: take the
private reward or agree to put that money toward the war effort. Putting it
toward the war effort would certainly have increased the odds of victory, an
attractive public good to offset the small private gain that would besacrificed
by each individual in the coalition. Surely each of the relevant 650,000 Israelis
would have put a greater value on military victory than a paltry \$58.62! Compare
this calculation to that for Egyptians in Nasser's winning coalition. We did a
pilot study a few years ago in which we surveyed country experts about the size
of several governments' winning coalitions from 1955 to 2008. The experts we
interviewed about Egypt placed its winning coalition as being as small as 8
members and as many as 65 in 1967. Wherever one comes down in that range it is
obvious that the coalition was very small. We suspect the experts may have
underestimated its size so we will err, again, on the side of conservatism and
assume it was as many as 1,000 key military officers and essential senior civil
servants. Even with our conservative estimate, each coalition member stood to get
\$150,000 in private rewards (\$500 million in military expenditures x 30 percent
for private rewards/ 1,000 coalition members = \$150,000 per coalition member) if
the funds out of the military budget that were available for that use were turned
over to them instead of being applied to making a concerted increased effort to
win the war. Whereas Israeli coalition members were only asked to sacrifice about
\$60 to help their country win the war, Egypt's coalition members would have had
to personally give up \$150,000 in income to help their country win. It should be
obvious that Nasser would likely have lost the loyal support of lots of his key
backers if he took their \$150,000 a head and spent it on the war instead of on
them. He actually would have increased his chance of being overthrown in a
military coup by making an all-out effort to win the war at the expense of his
cronies. His backers would have had to place a value on winning the war that was
worth their personally giving up \$150,000.}

{\large Victory is nice, but it probably isn't that nice for many people. Levi
Eshkol faced no such problem. His supporters were much more likely to place a
value on victory that was greater than \$58.62.}

{\large Of course, Israel did not just fight Egypt. It took on Syria and Jordan
at the same time. Here again the logic for its victory is the same. As Ryszard
Kapuscinski describes, Israel simply tried harder.}

{\large Why did the Arabs lose the 1967 war? A lot has been said on that
subject. You could hear that Israel won because Jews are brave andArabs are
cowards. Jews are intelligent, and Arabs are primitive. The Jews have better
weapons, and the Arabs worse. All of it untrue! The Arabs are also intelligent
and brave and they have good weapons.}

{\large The difference lay elsewhere---in the approach to war, in varying
theories of war. In Israel, everybody takes part in war, but in the Arab
countries---only the army. When war breaks out, everyone in Israel goes to the
front and civilian life dies out. While in Syria, many people did not find out
about the 1967 war until it was over. And yet Syria lost its most important
strategic area, the Golan Heights, in that war. Syria was losing the Golan
Heights and at the same time, that same day, that same hour, in Damascus---twenty
kilometres from the Golan Heights---the cafes were full of people, and others
were walking around, worrying about whether they would find a free table. Syria
lost fewer than 100 soldiers in the 1967 war. A year earlier, 200 people had died
in Damascus during a palace coup. Twice as many people die because of a political
quarrel as because of a war in which the country loses its most important
territory and the enemy approaches within shooting distance of the capital.6
Kapuscinki's numbers are wrong, since about 2,500 Syrians were killed in the war,
but his point is not. Autocrats don't squander precious resources on the
battlefield. And elite well-equipped units are more for crushing domestic
opposition than they are fighting a determined foreign foe. Syrian president,
Hafez al-Assad, did just that. In February 1982, he deployed around 12,000
soldiers to besiege the city of Hama in response to an uprising of a conservative
religious group, the Muslim Brotherhood.}

{\large After three weeks of shelling, the city was destroyed and tens of
thousands of civilians were massacred.}

{\large When they need to, democracies try hard. However, often they don't need
to. Indeed they are notorious for being bullies and picking on weaker states, and
negotiating whenever they are confronted by a worthy adversary. Thus the United
States readily fights small adversaries like Grenada, Panama, and the Dominican
Republic, and many democracies expanded their influence in the world by
colonizing the weak. But when it came to the Soviet Union, the United States and
its democratic, NATO allies negotiated whether the dispute was over Cuba, issues
in Europe, orelsewhere in the world. Indeed, the cold war stayed cold precisely
because the United States, a large-coalition regime, even with enormous effort,
could not be confident of victory. When extra effort does make victory likely, as
in the Iraqi surge, democrats try hard.}

{\large Unfortunately, sometimes negotiations fail, as was the case when Britain
and France sought to appease Adolf Hitler before World War II. They agreed to
Germany occupying Austria and the German-speaking part of Czechoslovakia. Even
when he invaded Poland, some in Britain hesitated to declare war. No concession,
however, was sufficient to satisfy Hitler's appetite for Lebensraum. This left
Britain and France with a very serious fight on their hands, and one in which
Britain tried enormously hard. In contrast, Germany did not switch its economy
onto a full war footing until the later stages of the war when it was clear to
Hitler and his cronies that their government's survival---and their personal
survival---was at risk.}

{\large In other cases the fight turns out to be significantly more difficult
than initially thought. US involvement in Vietnam, Iraq, and Afghanistan would be
just such cases. When confronted by these difficult fights democracies increase
their effort. In Vietnam, the United States continually reassessed the resources
needed to win before negotiating a settlement with North Vietnam, only to see
that agreement collapse a year after American withdrawal. In both Iraq and
Afghanistan, the United States has needed troop surges to advance its objectives.
That is, the United States follows Weinberger's counsel and not Sun Tzu's
time-tested advice. Autocratic leaders are wary of expending resources on the war
effort, even if victory demands it. They know their fate depends more upon the
loyalty of their coalition than success on the battlefield. They don't generally
make that extra effort.}

{\large World War I provides a great case study in these principles. Its origins
are complex and contentious, so we limit ourselves to describing the chain of
events. The war started as a dispute between Austria and Serbia after Serbian
nationalists murdered the heir to the Austrian throne, Archduke Franz Ferdinand,
in June 1914. When Austria threatened war, Serbia's ally, Russia, became
involved. This activated Germany's alliance with Austria. Given that war with
Russia also meant war with her ally, France, the Germans launched a rapid
invasion of France in the hope of quickly defeating it, as they had in 1871. The
German invasion of France wentthrough Belgium, and, since the British had pledged
to protect Belgium's neutrality, this brought Britain in on the side of the
allies.}

{\large A tangled web! Although many nations were involved, the war was
basically a struggle between the central powers of Austria and Germany and the
allied powers of France, Russia, and Britain. After a dynamic beginning---the war
was famously supposed to be over by Christmas---the conflict stagnated and
devolved into trench warfare, particularly on the Western Front. Russia dropped
out of the war in late 1917, after the Bolshevik Revolution. Doing so cost it
enormous amounts of its Western territory, but the political genius of Lenin knew
it was better to preserve resources to pay supporters than it was to carry on
fighting. In late 1917, the United States entered the war on the allied side.7
Allied victory was sealed with an armistice signed on the eleventh hour of the
eleventh day of the eleventh month of 1918.}

{\large Figure 9.1 plots the military expenditures of the primary combatants. 8
On a per capita basis, Russia spent less than the others. It was both massive and
poor. Of these nations only Britain and France were democratic. After the war
started in 1914, all combatants ramped up their military spending. However, after
1915, the autocratic nations didn't increase their effort much and their
expenditures plateaued as the war dragged on. German spending does increase again
in 1917 as it becomes clear that defeat will mean the replacement of the German
government. In contrast to the meager efforts by autocracies like Austria and
Russia, the democracies continue to increase expenditure until victory was
achieved.}

{\large FIGURE 9.1 Military Expenditures in World War ISun Tzu's advice to his
king predicts the behavior of autocrats in World War I: they didn't make an
extraordinary effort to win. The effort by the democratic powers in that same war
equally foreshadowed what Caspar Weinberger and so many other American advisers
have said to their president: if at first you don't succeed, try, try again.}

{\large When it comes to fighting wars, institutions matter at least as much as
the balance of power. The willingness of democracies to try harder goes a long
way to explaining why seemingly weaker democracies often overcome seemingly
stronger autocracies. The United States was once a weak nation. And yet, in the
Mexican-American War (1846--1848) it defeated the much larger, better-trained,
and highly favored Mexican army.}

{\large The miniscule Republic of Venice survived for over a thousand years
until it was finally defeated by Napoleon in 1797. Despite its small size and
limited resources it fought above its weight class throughout the Middle Ages. It
played a crucial role in the Fourth Crusade that led to the sacking of
Constantinople, in which Venice captured the lion's share of the Byzantine
Empire's wealth. The smaller, but more democratic government of Bismarck's
Prussia defeated the larger---widely favored---Austrian monarchy in the Seven
Weeks War in 1866. Prussia then went on to defeat Louis Napoleon's monarchical
France in the 1870--1871 FrancoPrussian War. And as we have seen, tiny Israel has
repeatedly beaten its larger neighbors. History is full of democratic Davids
beating autocratic Goliaths.}

\subsection{Fighting for Survival}

{\large Autocrats and democrats, at one level, fight over the exact same thing:
staying in power. At another level, they are motivated to fight over different
things. Democrats more often than autocrats fight when all other means of gaining
policy concessions from foreign foes fail. In contrast, autocrats are more likely
to fight casually, in the pursuit of land, slaves, and treasure.}

{\large This has important implications. As Sun Tzu suggested, autocrats are
likely to grab what they can and return home. On the other hand, democrats fight
where they have policy concerns, be these close to home, or, as can be the case,
in far-flung lands. Further, once they have won, democrats are likely to hang
around to enforce the policy settlement. Frequently this can mean deposing
vanquished rivals and imposing puppet regimes that will do their policy bidding.9
Thinking back to our discussion of foreign aid, we can see that war for democrats
is just another way of achieving the goals for which foreign aid would otherwise
be used. Foreign aid buys policy concessions; war imposes them. Either way, this
also means that democrats, eager as they are to deliver desired policies to the
folks back home, would much prefer to impose a compliant dictator (surely with
some bogus trappings of democracy like elections that ensure the outcome desired
by the democrat) than take their chances on the policies adopted by a democrat
who must answer to her own domestic constituents.}

{\large The idea that democrats and autocrats fight for their own political
survival may seem awfully cynical at best and downright offensive at worst.}

{\large Nevertheless we believe the evidence also shows this is the way the
world of politics, large and small, actually works. A look at the First Gulf War
will validate all of our suspicions.}

{\large Before 1990, relations between Iraq and Kuwait had long been fractious.}

{\large Iraq claimed that Kuwait, with its efficient modern oil export industry,
had been pumping oil from under Iraq's territory. On numerous occasions it
haddemanded compensation and threatened to invade. After misreading confused
signals from US president George H. W. Bush (on previous occasions the United
States had deployed a naval fleet to the region in response to Iraqi threats but
had also told Iraq's government that what it did in Kuwait was of no concern to
the United States), Saddam Hussein's forces invaded and occupied Kuwait in August
1990. His goal was to exploit its oil wealth for the benefit of himself and his
cronies---fairly typical for an autocrat at war. However, despite initially
confused signals, the United States did not look the other way: President Bush
organized an international coalition, and in January 1991 launched Operation
Desert Storm to displace Iraqi forces.}

{\large The goals and conduct of each side in the First Gulf War differed
greatly.}

{\large In contrast to Hussein's motives, President Bush did not attempt to grab
oil wealth to enrich cronies. Rather, the goal was to promote stability in the
Middle East and restore the reliable, undisrupted flow of oil. Protestors against
the war would chant ``no blood for oil.'' It would be na\"{\i}ve to argue that
energy policy was not a major, if not the major, determinant of US policy in the
Middle East, but it was not an exchange of soldiers' lives for oil wealth. The
objective was to protect the flow of oil, which is the energy running the
machines of the world's economy. Economic stability, not private gain, was the
goal of the coalition. To be sure, soldiers from the United States and other
coalition members died, although in very small numbers. Of the 956,600 coalition
troops in Iraq, the total number of casualties was 358, of which nearly half were
killed in noncombat accidents. In contrast, Iraq experienced tens of thousands of
casualties.}

{\large These coalition deaths brought concessions from Saddam Hussein, not
booty.}

{\large The conduct of the First Gulf War also fits the patterns predicted by a
political survival outlook. The United States first tried negotiations to get
Iraqi forces to leave. When these failed, the United States assembled an
overwhelmingly powerful coalition of highly trained and superbly equipped troops.
Saddam Hussein had elite troops, such as the Republican Guard, that perhaps came
close to matching the training and capability of coalition forces. But his elite
Republican Guard did not confront the coalition forces; Saddam had them pulled
back to safety so they could protect him rather than protect Iraq. Instead the
brunt of the coalition attack was borne by rawrecruits and poorly equipped units.
As the casualty figures show, many of these units suffered horribly.}

{\large Facing the possibility that coalition forces would invade Baghdad to
depose him, on February 28 Saddam Hussein agreed to terms of surrender. The
United States retained forces in the Gulf to ensure Saddam complied with the
terms to which he had agreed. Yet no-fly exclusion zones, diplomatic isolation,
and economic sanctions did not stop Saddam from repeatedly reneging on the
agreement he accepted. He also survived domestically. After his military defeat,
several groups, including Shiites in the South and Kurds in the North, rebelled.
Unfortunately for them, Saddam had preserved his best troops and retained enough
resources to buy their continued loyalty. The suppression of these uprisings
killed tens of thousands and led to the displacement of hundreds of thousands of
others.}

{\large Saddam would subsequently remain in power until the United States
deposed him in the Second Gulf War in 2003.}

{\large Saddam was not alone in placing survival and enrichment over fighting
well. Dictators would like to win wars if they can secure control over extra
riches that way, but keeping their job takes priority over pursuing those riches.
Mengistu Haile Mariam, who came to power in 1974 when he overthrew Ethiopian
Emperor Haile Selassie, embraced communism and was handsomely rewarded by the
Soviet Union. Over a fourteen-year period the USSR gave his regime about \$9
billion, much of it as military aid with which to fight Eritrean rebel forces
seeking independence.}

{\large Despite all this money, his war against the Eritreans did not go well.
It seems Mengistu was more interested in the Soviet money as a means to enrich
himself and ensure his political survival than in the successful conduct of the
war. He certainly had little concern for his soldiers' welfare, as we shall see
later. Michela Wrong, for instance, reports that the Soviets eventually worked
out that Mengistu's devotion to the fight was not all it was cracked up to be:
```He kept telling us that if we helped him he could achieve this military
victory,' remembers Adamishin, with bitterness. `I remember how he told me with
tears in his eyes: ``We may have to sell our last shirt, but we will pay you
back. We Ethiopians are a proud people, we settle our debts.'' Looking back, I
almost feel I hate him. Because I believed that what mattered to him was what was
best for the country. While really all that mattered to him was his own
survival.'''10Unfortunately for Mengistu Haile Miriam, the collapse of the Soviet
Union meant the end of his gravy train. In 1989 the Soviets departed. Mengistu
needed a new source of money. In an effort to salvage his situation, he decided
to try to get blood money from the United States and Israel by offering to trade
Ethiopian Jews (Falashas) for money and military aid. The Falashas dated back in
North Africa for thousands of years and are counted among those who fled from the
Babylonian captivity in 586 BCE.}

{\large To resettle these people, the United States allegedly paid \$20 million
and Israel agreed to pay \$58 million (but eventually only paid \$35 million).
With the money transferred, the rescued Falashas were then settled in Israel.11
This blood money was not enough to buy the loyalty of his supporters. It was a
far cry from the annual amounts doled out by the Soviets. As his military
collapsed to the much weaker Eritrean forces, Mengistu fled to Zimbabwe, where he
lives in luxury with around fifty former colleagues and family members.}

\subsection{Who Survives War}

{\large Democrats are much more sensitive to war outcomes than autocrats.12
Indeed, even victory in war does not guarantee a democrat's political survival.
For instance, within eighteen months of defeating Saddam Hussein, and the over 80
percent approval ratings that went with it, President George Herbert Walker Bush
was defeated at the polls by Bill Clinton in 1992. Similarly, British voters
threw Winston Churchill out of office despite his inspired leadership during
World War II. Still, while it is no guarantee of political survival, military
victory clearly helps. British prime minister Margaret Thatcher turned her career
around with the defeat of Argentina in the Falklands war in 1982. Her economic
reforms and confrontations with trade unions had led to recession and high
unemployment. Prior to the war she was deeply unpopular. At the end of 1981 her
approval rating stood at 25 percent. After the war this jumped to over 50
percent, and a year later she won a decisive electoral victory that would have
looked virtually impossible eighteen months earlier.}

{\large Military success helps democrats retain power while defeat makes removal
a near certainty for democrats. A failure to achieve victory in Vietnam ended US
president Johnson's career. French premier Joseph Laniel suffered a similar fate.
His government collapsed following the French defeat by Vietnamese forces in 1954
at the Battle of Dien Bien Phu. British prime minister Anthony Eden was forced to
resign after his disastrous invasion of Egypt's Suez Canal Zone in 1956.}

{\large Autocrats are much less sensitive to defeat. Despite defeat in the First
Gulf War and a costly and inconclusive result in the Iran-Iraq war (1980-- 1988),
Saddam Hussein outlasted four US presidents (Carter, Reagan, Bush, and Clinton).
Only defeat in the Second Gulf War cost him his job, and that war was fought
primarily to remove him. Unless they are defeated by a democracy seeking policy
concessions, autocrats can generally survive military defeat provided that they
preserve their resources.Autocrats even survive if their loss involves huge
causalities. In contrast, even in victory democrats are liable to be deposed if
they get lots of soldiers killed in the process. That presumably is why democrats
do much more to protect soldiers than autocrats do.}

{\large Hermann Goring, Hitler's number two in the Nazi German regime, knew
that, while it is the people who do the fighting, it is leaders who start wars.}

{\large Naturally the common people don't want war. . . . But, after all, it is
the leaders of the country who determine the policy, and it is always a simple
matter to drag the people along, whether it is a democracy, or a fascist
dictatorship, or a parliament or a communist dictatorship....}

{\large All you have to do is tell them they are being attacked, and denounce
the pacifists for lack of patriotism and exposing the country to danger.}

{\large It works the same way in any country.13 Goring is right. Leaders of
every flavor can deploy troops and the people in democracies are liable to rally
around the flag. But democrats don't recklessly put soldiers in harm's way. And
when they do, they do much more to protect them. The value of a soldier's life
differs drastically between small- and large-coalition systems. To illustrate
this sad truth we compare two conflicts fought a few years apart in the Horn of
Africa.}

{\large The US military operates on the principle of no soldier left behind. For
an accurate and gory drama of this principle, we recommend Ridley Scott's 2001
film, Black Hawk Down, which portrays an account of the battle of Mogadishu,
October 3--4, 1993. US troops entered Somalia as part of a United
Nations--sponsored humanitarian mission. In 1993 Somalia was a collapsed state.
Between 1969 and 1991 it had been ruled by Siad Barre, someone who understood
that policy should always be subordinate to survival. As the real-life Barre
bluntly stated, ``I believe neither in Islam, nor socialism nor tribalism, nor
Somali nationalism, nor pan-Africanism. The ideology to which I am committed is
the ideology of political survival.''14 And this focus allowed him to
successfully survive in office for twenty-two years before being caught up and
deposed in the myriad of civil wars that plague the Horn of Africa. Following his
deposition, the Somali state collapsed, with control divided between tribal
warlords whose militia terrorized the people. Mohamed Farrah Aideed, who led the
Habar Gidirclan, controlled one of the strongest factions. Aideed was strongly
opposed to the United States's presence in Somalia because he believed the United
States was backing his adversaries. After several failed attempts to capture or
kill Aideed, the United States received intelligence that several of his senior
colleagues were meeting at a house. The US plan was to helicopter elite troops
into the building, capture the senior Habar Gidir members, and get out via a
military convoy.}

{\large Unfortunately the mission went sour. Two Black Hawk helicopters went
down and two others were damaged. Thousands of Somalis took to the streets and
erected barricades so that the convoy became trapped. Both the helicopter crews
and many in the convoy became trapped overnight and subject to small-arms fire,
and it was not until the next day that they could be rescued. Although the
operation was a debacle, the US commitment to its soldiers was unwavering. As is
to be expected when soldiers' lives are highly valued, the United States sent
forces in to retrieve the downed helicopter crews. We might take this for granted
but it is not the behavior of autocrats---the Ethiopian-Eritrean conflict in the
Horn of Africa provides a case in point.}

{\large The Battle of Afabet (March 17--20, 1988) was an important turning point
in the decades-long battle for Eritrean independence from Ethiopia. As we have
seen, Ethiopia had an enormous military of about 500,000 men that was lavishly
equipped by Soviet military aid. In contrast, virtually all the Eritrean's
equipment had been captured from the Ethiopians.}

{\large In a switch from its usual guerrilla tactics, the Eritrean rebel force
(the Eritrean People's Liberation Front, or EPLF) decided to challenge the
Ethiopian army in a head-on battle. The Ethiopians resisted solidly for sixteen
hours. On multiple occasions the EPLF commander, Mesfin, was told to withdraw but
he carried on pressing his attack. The Ethiopian commanders decided to withdraw
to the garrison town of Afabet and assembled a convoy of seventy vehicles.
Unfortunately for them, the withdrawal went through the Ad Shirum Pass that forms
a natural bottleneck. When an advancing EPLF tank hit a truck in the front of the
column, the Ethiopian forces were stuck.}

{\large The Ethiopian command was concerned that their heavy weapons not fall
into enemy hands. Fortunately for them they had a sizable air force. Yet, rather
than attempt to relieve their trapped countrymen and fellow soldiers,they
embarked on a two-hour aerial bombardment that destroyed everything. The
Ethiopian motto was: Leave no working tank behind. As an Ethiopian general put
it, ``when you lose an area you better destroy your equipment---it's a principle
of war. If you cannot separate your men from their equipment then you bomb them
both together.''15 It's likely that few readers have ever heard of this battle,
in which Ethiopian causalities were perhaps as high as 18,000 men. In contrast,
many Americans are familiar with the disastrous policy failure in which, for the
loss of thirteen lives, the US army killed possibly as many as 1,000 Somali
militants.}

\subsection{The Peace Between Democracies}

{\large Democracies hardly ever (some might even say never) fight wars with each
other. This is not to say they are peace loving. They are not shy about fighting
other states. But the reasoning behind the tacit peace between democracies
provides some clues to how the world could become more peaceful and why achieving
that end is so difficult.}

{\large Democratic leaders need to deliver policy success or they will be turned
out of office. For this reason they only fight wars when they expect to win.}

{\large Of course they may turn out to be wrong, in which case, as we have
argued, they then double down to turn the fight in their direction. That is just
what happened in Vietnam, where the United States committed massive numbers of
troops and huge amounts of money to no avail. Only after many long, costly years
of trying did the United States settle for a negotiated peace that ultimately
turned all of Vietnam over to the North Vietnamese regime.}

{\large If we are correct, we should hardly ever witness two large-coalition
regimes fighting against each other. According to our reasoning, democrats will
only fight when they believe they are almost certain that they will win. But how
can two adversaries each sustain such certainty?}

{\large Autocrats, as we saw, don't need to think they have a great chance of
winning. They are prepared to take bigger risks because they have good reason to
think that the personal consequences of defeat are not as bad for them as the
personal consequences of not paying off their few essential supporters. Now,
following the logic of political survival closely, we must recognize that just
because two democrats are not likely to fight with each other, we cannot say that
one will not use force against another. Largecoalition systems certainly may be
prepared to engage in disputes with each other and one might even use force
against the other. How does this work?}

{\large As long as a large coalition leader believes that his dispute is
unlikely toescalate to war, he can move partially up the escalation ladder,
pressing his foe into backing down or else backing down himself, and negotiating
if he concludes that the other side is prepared to fight and that his own
prospects of victory are too small to justify fighting. Now imagine the two
disputants are both democracies dependent on a large coalition. The logic of
large-coalition politics tells us that a large-coalition state will attack
another large-coalition state only if the target is sufficiently weak that the
target is expected to prefer to negotiate rather than fight back. Since the
democratic target will also try hard if it chooses to fight back, the initiating
democracy must either have or be capable of having a great military advantage or
it must be confident that its rival's resources are insufficient for the target
to believe it can be nearly certain of victory. Thus, the attacking democracy
must be sure that its target democracy is unsure of victory; this is of paramount
importance in a head-to-head military dispute between two democracies.16 Here we
have an explanation for the history of US attacks against very weak democratic
rivals such as Lyndon Johnson's 1965 attack and overthrow of Juan Bosch's
democratically elected regime in the Dominican Republic, France's invasion of
Weimar Germany in 1923, and the list goes on.}

{\large Democracies don't fight with each other, true. Rather, big democracies
pick on little opponents whether they are democratic or not, with the expectation
that they won't fight back or won't put up much of a fight.}

{\large Indeed, that could very well be viewed as a straightforward explanation
of the history of democracies engaged in imperial and colonial expansion against
weak adversaries with little hope of defending themselves.}

{\large This democratic propensity to pick on weak foes is nothing new.}

{\large Looking at all wars for nearly the past two centuries, we know that
about 93 percent of wars started by democratic states are won by them. In
contrast, only about 60 percent of wars started by nondemocracies are won by
them.}

\subsection{Defending the Peace and Nation Building}

{\large In his 1994 State of the Union address, US president Bill Clinton
declared ``democracies don't attack each other,'' and therefore ``the best
strategy to insure our security and to build a durable peace is to support the
advance of democracy elsewhere.'' This is a common theme for US presidents.}

{\large Unfortunately, actions have not matched the rhetoric. More unfortunately
still, the problem lies not in a failure on the presidential level, but with
``we, the people.'' In democracies, leaders who fail to deliver the policies
their constituents want get deposed. Democrats might say they care about the
rights of people overseas to determine their own future, and they might actually
care too, but if they want to keep their jobs they will deliver the policies that
their people want. Earlier we examined how democrats use foreign aid to buy
policy. If that fails, or gets too expensive, then force is always an option.}

{\large Military victory allows the victors to impose policy.}

{\large We should dismiss any pretense that such policies are paternal and
imposed with the foreigners' long-term best interests in mind. They are not.}

{\large They are done for the benefit of the democrat's supporters and sometimes
these policies can be very unpleasant. For instance, the opium wars (1839--1842
and 1856--1860) got their name because the British wanted to finance their
purchases of Chinese exports by selling the Chinese opium grown in India. China
was reluctant to become a nation of addicts. The British used force to open up
China to the drugs market. Hong Kong started out as a base from which the British
could enforce this trade openness. It is telling that, while the settlements that
ended the wars are officially known as the Treaties of Nanking and Tianjin, the
Chinese often refer to them as the Unequal Treaties.}

{\large One of the problems with seeking a policy solution is that after the
democrat's army leaves, the vanquished nation can renege. Enforcing the
settlement can be very expensive, as was the case after the Gulf War. Acommon
solution, and the one eventually used against Saddam Hussein, is leader
replacement. Democrats remove foreign leaders who are troublesome to them and
replace them with puppets. The leaders that rise to the top after an invasion are
more often than not handpicked by the victor.}

{\large A difficult leader whom democrats don't trust to honor an agreement will
often find himself replaced. The Congo's Patrice Lumumba, democratically elected,
didn't have policies that pleased the Belgian or American governments and before
you knew it, Lumumba was dead, replaced by horrible successors who also happened
to be prepared to toe the line favored by the United States and Belgium. France
has been no different, stepping into its ex-colony of Chad to make sure that a
French-friendly government is in charge rather a Libyan-friendly or Arab-friendly
regime.}

{\large Democratic leaders profess a desire for democratization. Yet the reality
is that it is rarely in their interest. As the coalition size grows in a foreign
nation, its leader becomes more and more compelled to enact policies that his
people want and not the policies desired by the puppeteer's people. If a
democratic leader wants a foreign leader to follow his prescribed policies then
he needs to insulate his puppet from domestic pressures. This means reducing
coalition size in vanquished states. This makes it cheaper and easier to sustain
puppets and buy policy. US foreign policy is awash with examples where the United
States overtly or covertly undermines the development of democracy because it
promoted the policies counter to US interests. Queen Liliuokalani of Hawaii in
1893, Salvador Allende of Chile in 1973, Mohammad Mosaddegh of Iran in 1953, and
Jacobo Arbenz of Guatemala in 1954 all suffered such fates.}

{\large Democracy overseas is a nice thing to believe in, in the abstract. In
practice it's probably not what we, the people want. Let's return to reconsider
Egypt and Israel and the case for democratization. Western democracies used to
complain, albeit not too emphatically, about electoral malpractice in Egypt under
Mubarak. With Mubarak gone, they now worry that true democracy in Egypt might be
contrary to the interests of friends of Israel. Buying peace with Israel under
Mubarak was costly but moves toward democracy in Egypt will make continued peace
costlier at least until and if Egypt becomes a full-fledged, mature democracy
whose leaders will then only fight if they are virtually sure of victory. We can
hopethat in the long run a democratized Egypt and democratic Israel might develop
mutual trust, understanding, and tolerance. However, there is also a chance that
Israel would not survive long enough to reach this long run.}

{\large While it is true that democracies generally don't fight each other, we
have also noted that they do have lopsided conflicts, and those conflicts often
end with the weaker side capitulating. If a democratic Egypt mobilizes and arms
itself, tiny Israel would have little hope of resisting unless the United States
or NATO were prepared to make a large effort to defend it. Anyone who thinks a
democratic Egypt attacking Israel is too fanciful a scenario might ask democratic
Native American tribes from the American plains about their dealings with the
expanding United States in the 1800s. Democratization sounds good in principle
only.}

{\large Of course, many may think that we are just too cynical. Advocates of
democratization are fond of pointing out the success stories. Yet all of these
cases---Germany, Japan, South Korea, and Taiwan---also happen to involve
countries whose population's values largely coincide with American values in
resisting for decades large communist neighbors.}

{\large The big problem with democratizing overseas continues to lie with we,
the people. In most cases we seem to prefer that foreign nations do what we want,
not what they want. However, if our interests align then successful
democratization is more likely. This is particularly so if there is a rival power
that wishes to influence policy. The postwar success stories fit this category
well. Generally, the people of West Germany and Japan preferred what the United
States wanted to the vision expounded by the Soviet Union. Creating powerful
states that wanted to resist communism and would try hard was in the US interest.
As occupying powers, the United States, Britain, and France might have set
Germany on a course to democracy but they did so only because it was advantageous
for them.}

{\large This confluence of interests is rare, and so is externally imposed
democratization.}

{\large Sun Tzu exerted a lasting influence on the study of war precisely
because his recommendations are the right recommendations for leaders, like
monarchs and autocrats, who rule based on a small coalition. The Weinberger
Doctrine---like its more recent replacement, the PowellDoctrine---exerts
influence over American security policy precisely because it recommends the most
appropriate actions for leaders who are beholden to a large coalition.}

{\large We have seen that larger coalition systems are extremely selective in
their decisions about waging war and smaller coalition systems are not.}

{\large Democracies only fight when negotiation proves unfruitful and the
democrat's military advantage is overwhelming, or when, without fighting, the
democrat's chances of political survival are slim to none. Furthermore, when war
becomes necessary, large-coalition regimes make an extra effort to win if the
fight proves difficult. Small-coalition leaders do not if doing so uses up so
much treasure that would be better spent on private rewards that keep their
cronies loyal. And finally, when a war is over, larger coalition leaders make
more effort to enforce the peace and the policy gains they sought through
occupation or the imposition of a puppet regime.}

{\large Small-coalition leaders mostly take the valuable private goods for which
they fought and go home, or take over the territory they conquered so as to enjoy
the economic fruits of their victory for a long time.}

{\large Clausewitz had war right. War, it seems, truly is just domestic politics
as usual. For all the philosophical talk of ``a just war,'' and all the
strategizing about balances of power and national interests, in the end, war,
like all politics, is about staying in power and controlling as many resources as
possible. It is precisely this predictability and normality of war that makes it,
like all the pathologies of politics we have discussed, susceptible to being
understood and fixed.}
\pagebreak{}


\section{Chapter 10 - What Is To Be Done?}

{\large A man always has two reasons for doing anything: a good reason and the
real reason.}

{\large ---J. P. MORGAN IN LATE 1901, VLADIMIR ILYICH LENIN WROTE A
revolutionary essay called ``What is to be Done?'' His question was directed at
justifying the creation of the communist party as the vanguard of the people. We
are more interested in his literal question than in his reason for asking it and,
equally, we are intrigued by his unintended answer three years later (really in a
different context, but nevertheless, apt) in the title to another essay, ``One
Step Forward, Two Steps Back.'' Too often, the real world of politics and
business responds to problems by taking one step forward and two steps back,
resulting in no progress on the problem at hand. Backsliding is, and should be,
the way leaders deal with problems. It is the existing rules that have allowed
them to seize and control resources to date. A headlong plunge into new ways of
conducting politics might only heighten a leader's risk of being overthrown.}

{\large After the past nine chapters of our cynical---but we fear, accurate---
portrayal of politics, it is time at last to more seriously confront Lenin's
first question: What is to be done? We hope that, informed by the lessons of
leadership, we can offer a much better and more democratic answer than he
provided.It is an understatement to say that making the world better is a
difficult task. If it were not, then it would already have been improved. The
misery in which so many live would already have been overcome. The enrichment of
CEOs while their stockholders lose their shirts would be a thing of the past.}

{\large However, the inherent problem with change is that improving life for one
group generally means making at least one other person worse off, and that other
person is likely to be a leader if change really will solve the people's
problems. If the individual harmed by change is the ruler or the CEO---the same
person who has to initiate the changes in the first place--- then we can be
confident that change is never going to happen.}

{\large From the beginning we said we would focus on what is rather than what
ought to be. Now we need to talk a bit about what ought to be. In doing so, we
want to lay down the ground rules. First among these is that we should never let
the quest for perfection block the way to lesser improvement.}

{\large Utopian dreams of a perfect world are just that: utopian. Pursuing the
perfect world for everyone is a waste of time and an excuse for not doing the
hard work of making the world better for many.}

{\large It is impossible to make the world great for everyone. Everyone doesn't
want the same thing. Think about what is good for interchangeables, influentials,
and essentials, the three dimensions of political life: hardly ever is it true
that what is good for leaders and their essential backers is good for everyone
else. If they all had the same wants there wouldn't be misery in the world. So,
even as we are trying to change the world for the better, we are tied to the
dictates of political reality. A fix is not a fix unless it can actually be done!
What can be done must satisfy the needs of everyone required to implement change.
Wishful thinking is not a fix and a perfect solution is not our goal and should
not be any well-intentioned person's goal. Even minor improvements in governance
can result in significant improvements in the welfare of potentially millions of
people or shareholders.}

\subsection{Rules to Fix By}
{\large     }
{\large Whether we are looking at the welfare of shareholders in publicly traded
corporations, the quality of life for citizens in a democracy or the conditions
under which billions live in oppressive and impoverished third-world countries,
there are certain common principles behind bettering the world.}

{\large These commonalities need to be laid bare before we tackle the specifics
of fixing particular problems in particular places.}

{\large If we have learned anything in the preceding pages it is to be
suspicious of people's motives. Appeals to ideological principles and rights are
generally a cover. J. P. Morgan had it right: There is always some principled way
to defend any position, especially one's own interests. In one overseas nation,
our government supports protest and advocates the will of the people to determine
their own future. That is, for instance, a popular refrain for leaders in the
United States when it comes to places like Hugo Chavez's Venezuela or Kim Jong
Il's North Korea. Elsewhere we call for stability. That's the principle invoked
when people try to bring down a government that is our friend and ally, such as
the governments of Bahrain or Saudi Arabia. Both freedom and stability are
principled positions (the good reason) selectively asserted depending upon how we
like the incumbent (the real reason). In devising fixes to the world's ills, the
essential first step is to understand what the protagonists want and how
different policies and changes will affect their welfare. A reformer who takes
what people say at face value will quickly find their reforms at a dead end.}

{\large Everyone has an interest in change, but interchangeables, influentials,
essentials, and leaders don't often agree on what changes they want.}

{\large Leaders, given their druthers, would always like the set of
interchangeables to be very large, and the groups of influentials and essentials
to be very small. That's why the world of business has so many massive
corporations with millions of shareholders, a few influential largeowners, and a
handful of essentials on the board of directors who agree to pay CEOs handsomely
regardless of how the company fares. That's why so much of humanity for so much
of human history has been governed by petty despots who steal from the poor to
enrich the rich.}

{\large The masses---whether members of the selectorate or the wholly
disenfranchised---agree that their group, the interchangeables, should be large
but they want all other groups to be big as well. Their best chance at having a
better life comes from the coalition and the influential group growing in size,
such that they have a realistic chance of becoming one of its members and of
benefiting from the profusion of public goods such governance provides, even if
they remain excluded from the coalition. As we have seen, it is this very hope of
improving the people's lot that revolutionaries use as their rallying cry to get
them to take to the streets.}

{\large But even in a large-coalition system, these masses are unlikely to get
what they want all the time. Their hope is to get what they want more of the
time.}

{\large The group whose desires are most interesting from the perspective of
lasting betterment is the set of essentials. More often than not, they are the
people who can make things happen. You see, they don't like the idea that they
might be purged to make the coalition smaller. But at the same time, ending up in
a smaller coalition can provide them with fabulous wealth.}

{\large Remember Saddam Hussein's videotaped takeover: at the outset everyone in
the audience was terrified. At the end, those still sitting in the auditorium
were thrilled. They knew they had survived to collect their rewards for another
day. What political insiders want when it comes to institutional change is
complex, but to understand the reforms they can be expected to support and those
they will oppose we need to understand their wants.}

{\large Coalition members like small selectorates. Their welfare is enhanced if
there are relatively few replacements for them. The incumbent cannot use the
implicit threat of replacing them with a cheaper backer as a way to keep more
benefits for himself rather than paying his essentials their due.}

{\large This creates tension between a leader and his coalition. The leader
would like to establish a Leninist style, corrupt, rigged electoral system that
guarantees him an eager supply of replacement supporters. The coalition prefers
monarchical, theocratic, or junta style institutional arrangements that restrict
those who can be brought into the coalition to a select group ofaristocrats,
clerics, or military elites.}

{\large Leaders and their essentials share a preference for dependence upon a
small coalition, at least so long as the coalition is very small. However, as the
coalition continues to expand, a wedge is eventually driven between what a king
wants and what his court needs. When that wedge gets big enough we have an
explanation for the emergence of a mature democracy that is so stable it will
almost certainly remain democratic and not backslide into autocratic rule. The
switch in the coalition's desires for institutional change results from tradeoffs
between declines in private goods as the coalition expands, and the increase in
public goods and societal productivity that accompanies such enlargement.}

{\large Given the complexity of the trade-off between declining private rewards
and increased societal rewards, it is useful to look at a simple graphical
illustration, which, although based on specific numbers, reinforces the
relationships highlighted throughout this book. Imagine a country of 100 people
that initially has a government with two people in the winning coalition. With so
few essentials and so many interchangeables, taxes will be high, people won't
work very hard, productivity will be low, and therefore the country's total
income will be small. Let's suppose the country's income is \$100,000 and that
half of it goes to the coalition and the other half is left to the people to
feed, clothe, shelter themselves and to pay for everything else they can
purchase. Ignoring the leader's take, we assume the two coalition members get to
split the \$50,000 of government revenue, earning \$25,000 a piece from the
government plus their own untaxed income. We'll assume they earn neither more nor
less than anyone else based on whatever work they do outside the coalition.}

{\large Now we illustrate the consequences of enlarging the coalition. Figure
10.1 shows how the rewards directed towards those in the coalition (that is,
private and public benefits) compare to the public rewards received by everyone
as more people enter the coalition. Suppose that for each additional essential
member of the winning coalition taxes decrease by half of 1 percent (so with
three members the tax rate drops from 50 percent to 49.5 percent), and national
income improves by 1 percent for each extra coalition member. Suppose also that
spending on public goods increases by 2 percent for each added coalition member.
As coalition size grows, tax rates drop, productivity increases, and the
proportion of governmentrevenue spent on public goods increases at the expense of
private rewards. That is exactly the general pattern of change we explained in
the previous chapters.}

{\large FIGURE 10.1 The Welfare of Essentials and of Ordinary Citizens What we
see in Figure 10.1 is that as the coalition initially expands, the welfare of its
essential members declines. These supporters are made worse off because their
share of private goods is greatly diluted as additional supporters are brought
onboard. However, as the coalition gets ever larger, the extent of the dilution
declines. As a rough approximation, each of the two original coalition members
must give up a third of their lucrative private rewards to compensate bringing in
a third coalition member. They are in part compensated for this loss by the
greater availability of public goods and a more productive society, but they take
huge personal losses in exchange for their societal gains. The trade-off works
out differently in an initially larger coalition. Again, as a rough
approximation, consider the costs and benefits for a coalition of six members. To
bring in a seventh coalition member each of the six existing coalition members
forsake about a seventh of their private benefits in exchange for the societal
gain. As the losses in private rewards from an expanded coalition decline, the
coalition's members, far from continuing to oppose expansion, support additional
members being brought into the coalition. From this point onwards, which occurs
at a coalition size of around seven members in our admittedly simple example, the
essentialsprefer to continue expanding the coalition. This puts them at odds with
their leader who remains committed to the first rule of staying in power: keep
the coalition small.}

{\large Figure 10.1 illustrates numerous features of the logic of institutional
changes from Saddam Hussein's purge to the stability of mature democracies. In
very small coalition settings, leaders can generate support from some existing
members of the coalition to purge other members. In Figure 10.1 this is exactly
the incentive when the coalition is initially sized between one and six. Of
course, no coalition member wants a purge unless he is going to survive it. It is
for this very reason that Saddam Hussein's videotaped purge initially filled the
Ba'ath party members with such fear and why those who were retained were so happy
to be kept on.}

{\large Their survival in the essential group after the purge meant they would
get even more private rewards.}

{\large If coalition size starts out pretty large, beyond six in the
illustrative example in Figure 10.1, then orchestrating a purge or a coup gets
harder and harder. Leaders, whether incumbents or potential coup makers, find it
increasingly difficult to get supporters to go along with reducing their
coalition. While, for example, it is possible for a leader with an initial
coalition of ten to find supporters who could be better off after a purge, the
coalition would have to shrink all the way down to three before those still in it
would be better off after the purge. And to benefit from the dirty work entailed
in such a contraction, the coalition's members that help perpetrate the purge
would have to be absolutely certain that their names were not also on the list of
those to be eliminated.}

{\large As the coalition gets even larger it becomes nearly impossible for a
leader to induce coalition members to perpetrate a purge or for a rival to
organize a coup. Figure 10.1 illustrates this stability of mature democracies.
Once the winning coalition size expands to at least twentyseven, in our example,
the leader could not make his supporters better off even if he could convince
them to contract the coalition all the way back down to just two members.}

{\large The essential facts of political life are that people do what is best
for them. Thus, except under extreme duress, leaders don't expand the coalition;
the masses press for democratization; and essential supporters vary in what they
want. This latter group can be made better off bycontractions in the number of
coalition members---that is, with coups and purges---provided they are the ones
retained. Democratization can also make them better off. It is therefore this
group that offers the greatest prospect for constructive, as well as destructive
change. With them lies the possibility of both ``one step forward'' and ``two
steps back.'' The prospect of being dropped from the coalition encourages its
members to take the single step forward rather than risk becoming a casualty of
the two steps back. Times and circumstances that heighten the risk of coalition
turnover engender an appreciation of democracy among political insiders.}

{\large Members of a small coalition live in luxurious, but constant, fear: make
the coalition smaller, as their leader wants, and they may be out; make the
coalition bigger and their special privileges diminish. But decreased privileges
are much better than the danger of being out altogether. So, there are two times
when the coalition is most receptive to the urge to improve life for the many,
whether those are the people or shareholders: when a leader has just come to
power, or when a leader is so old or decrepit that he won't last much longer. In
these circumstances coalition members cannot count on being retained. At the
beginning and the end of an incumbent's reign the danger of being purged is
greatest and so, at these times, coalition members should be most receptive to
reform.}

{\large Effective reform means expanding the coalition and that means that
everyone, including the current essentials, has a good chance of being needed by
tomorrow's new leader.}

{\large Not only is there a good time to look for the opportunity for reform.
There also are good circumstances when reforms that can improve the people's
welfare are welcomed. Coalitions whose leaders face serious economic strains
understand that their days of luxury and splendor are numbered.}

{\large That is one of the reasons companies sometimes commit fraud: CEOs,
senior management, and board members believe they will be ousted because of the
firm's failure and so they cover up how poorly the business is doing while they
try to fix it and save themselves. Little white lies work well the first year,
but if things do not turn around, then each year they need to lie a little bit
more until their reports are outright fiction and legally fraudulent.}

{\large As we have learned, when a country's economy is in trouble the big
problem from a ruler's perspective is that she doesn't have enough moneyto buy
continued loyalty. When the privileges enjoyed by essentials are shrinking they
are likely to be tuned in to the possibility of change. They know the leader will
want to purge people to use what little money is around more effectively. They,
not wanting to be purged, will be amenable to expanding their group, trading
their privilege for their future security and well-being. Coalition members are
not the only ones willing to contemplate changing the rules when circumstances
warrant. If the economic crisis is severe enough (and foreign aid donors stay
away), then even leaders must ponder whether they might be better off
liberalizing. Democratization jeopardizes their long-term future, but if they
don't pay their supporters today whether they can win an election tomorrow is not
a salient consideration.}

{\large Blind fools don't often get to rule countries or companies. Pretty much
any leader worth his salt can see the dangers he faces when economic
circumstances leave him bereft of funds to buy loyalty. Under such circumstances
even leaders can believe that reform is their best shot at political survival.
They might look for a fix even before their coalition does.}

{\large Consider the experience of Chiang Kai Shek, who certainly was no fool.}

{\large We might well ask why he encouraged much more successful economic
policies on Taiwan than on the mainland of China. In the latter, even with
extensive poverty, because there were so many people, there was plenty to enrich
himself and his coalition. But when Chiang Kai Shek and his backers retreated to
Taiwan, they took over an island with relatively few people and barely any
resources. Only economic success could provide the way to reward his coalition.
In the process of achieving that success, he also gradually expanded the
coalition, perhaps in response to pressure from his essential cronies or perhaps
under pressure from the United States, until one day he woke up to a democracy.}

{\large When the time or circumstances are ripe for change, coalition members
must recognize that if they do not pressure for an expansion of public goods and
public welfare, then others will. Provided that the chances of success are good
enough and the expected gains from success outstrip the costs involved in
gambling on a revolt, an intransigent coalition and leadership will find itself
besieged by an uprising. In this circumstance, such as was seen in Tunisia,
Egypt, Yemen, and elsewhere in the Middle East and North Africa, and as we saw in
the proxy fight at HP over CarlyFiorina's decision to merge with Compaq, people
are willing to take big risks to improve their lot. They do so to call for
exactly the same change as is widely favored by smart coalition members when and
if any change becomes necessary.}

{\large A wise coalition, therefore, works together with the masses to foster an
expanded coalition. The people cooperate because it will mean more public goods
for them and the coalition cooperates because it will mean reducing the risk of
their ending up out on their ear. Egypt's military leaders, essential members of
the Mubarak government, understood this choice very well in the early months of
2011. They ensured their continued place as important players in Egypt's future
by cooperating with the mass movement and supporting an expanded coalition,
rather than hunkering down and risking losing everything.}

{\large What are the lessons here for change? First, coalition members should
beware of their susceptibility to purges. Remember that it ticks up when there is
a new boss, a dying boss, or a bankrupt boss. At such times, the essential group
should begin to press for its own expansion to create the incentives to develop
public-spirited policies, democracy, and benefits for all. Purges can still
succeed if they can be mounted surreptitiously, so wise coalition members who are
not absolutely close to the seat of power would do well to insist on a free
press, free speech, and free assembly to protect themselves from unanticipated
upheaval. And should they be unlucky enough to be replaced, at least they will
have cushioned themselves for a soft landing. Outsiders would be wise to take
cues from the same lessons: the time for outside intervention to facilitate
democratic change or improved corporate responsibility is when a leader has just
come to power or when a leader is near the end of his life.}

{\large Knowing what people want, and the conditions under which they will
oppose reform and the circumstances under which the swing coalition members will
support reform, we can now turn to concrete ideas about fixing, at least
partially, the worlds of business and governance.}

\subsection{Lessons from Green Bay}

{\large The Green Bay Packers, a football team based in the cold climes of
Wisconsin, are remarkable for the loyalty their fans show them. In fact, win or
lose, Packer fans are nearly always satisfied. Virtually every one of their home
games since 1960 has been sold out. Attendance averages 98.9 percent despite
often appalling weather. The Packers have one of the longest waiting lists for
season tickets among professional football teams.1 Despite being a small market
team (Green Bay is a city of only about 100,000), they attract a larger, more
loyal fan base than teams from many much larger cities. Their success with their
fans, if not their success on the field, stems from their institutional
structure.}

{\large The Packers are the only nonprofit, community-owned franchise in
American major league professional sports. Their 112,120 shareholders are mainly
local fans. The ownership rules preclude a small clique taking control of the
team. No one is allowed to own more than 200,000 shares in the Packers and there
are about 4.75 million shares outstanding. Thus, a tiny band of owners cannot
easily overturn the many and run the team for their personal gain at the expense
of the larger, small-owners fan base.}

{\large The Packers have a forty-three-member board of directors.}

{\large We can see the relative representativeness of the Packers' essential
coalition by comparing the size of their board of directors to Carly Fiorina's
board at Hewlett Packard. Remember that the Hewlett Packard Board varied between
ten and fourteen members. HP has about 2.2 billion shares outstanding. Roughly
speaking, each Packard board member nominally represents the interests of about
185 million shares. Each Packers board member represents about 110,000 shares.
The Packers have a vastly larger winning coalition in absolute terms (forty-three
to about twelve). They also have a vastly larger coalition relative to the size
of their nominal selectorate---about 1,700 times larger. Can it be any wonder
that the Packer owners are extremely happy with their company/team and
thatsentiments are more mixed when it comes to HP?}

{\large The lesson to be extracted from the Green Bay Packers is that if firms
can be made to rely on a bigger coalition they are likely to do a better job of
serving the interests of their owners. But how can corporate governance be turned
on its head to make this happen?}

{\large Consider what the main difficulties are for shareholders. They suffer
from two big problems: First, in big corporations there tend to be millions of
little shareholders, a handful of big, institutional shareholders, and a bunch of
insider owners. The millions of little shareholders might as well not exist.}

{\large They are not organized and the cost to any of them to organize the mass
of owners just isn't worth it. Second, the flow of information about the firm's
performance comes from pretty much only two sources: the firm itself and the
financial media. Few owners read annual reports or SEC filings and the financial
media don't spend much time reporting on any one firm unless it is in huge
trouble. By then it is usually too late for the shareholders to save the day.}

{\large We live in the age of networking. Much of the world, including owners of
shares, Twitter and chat with ``friends'' on Facebook; they are LinkedIn; they
can easily communicate with one another, even if they don't always do so.}

{\large Surely it would be relatively simple to design firm-specific Facebooks
or other networking sites.}

{\large Companies maintain lively web sites to put their view across but
entrepreneur-owners have not stepped forward to do the same to help organize the
mass of little owners and to provide a way for them to share views. Sure, there
are bloggers writing about anything and everything, but there don't seem to be
shareholder-controlled sites to exchange thoughts and ideas about a company that
participants own in common. If something like this existed, the size of the
influential, informed voters in any corporation would go way up. Then, for the
first time, boards would really be elected by their owners and then the board
would need, like any leadership group, to be responsive to their large coalition
of constituents.}

{\large A simple change that exploits the Internet to be a conduit for
increasing coalition size can turn the AIGs, Bank of Americas, General Motors,
and AT\&Ts of the world into big-coalition regimes that serve their millions of
small owners instead of a handful of senior managers.}

{\large Ah, you are thinking, senior management can thwart such efforts.
Theywill, as they already do, hold shareholder meetings in places most owners
can't afford to go, or the meetings will be so brief that it will be impossible
for dissidents to express their views (the preferred shareholder meeting strategy
in Japan) and, after all, proxies pour in, turning millions of votes over to a
handful of board members. None of that, of course, will stop shareholder control
once the millions of little owners have a cheap and easy way to exchange views.
Then they will set the rules---by majority vote ---for who casts proxies. They
can set some of their own up to represent competing ``parties'' and they can make
the annual shareholders' meeting a purely decorative event. All such skeptics
should remember that social networking web sites have already successfully
mobilized revolutions and brought down governments. Changing corporate governance
is far easier.}

{\large Corporations don't have armies that can go out and bash in the heads of
dissidents. Pursue a course of connecting and informing shareholders, and we will
see whether shareholders who limit CEO salaries do better or worse; whether firms
that alter behavior to meet the social expectations of their shareholders do
better or worse; and whether shareholders care more about employees or about
themselves. Whatever the millions of little owners decide to do, they will be
responsible for their own fate.}

{\large Management will serve them just as democratic leaders are more
constrained than autocrats to do what their citizens want.}

{\large We also ought to comment a bit on how not to improve corporate
governance. In the wake of Enron's collapse and other big frauds, Congress
decided to regulate corporate governance, ostensibly to make it better. By now
every reader knows that the interest of government leaders is not in making
shareholders or even the man or woman on the street better off. Their interest is
in making themselves better off. The regulations they imposed on corporate
governance may have played well with voters, many of whom had little stake in
many of the companies that were harmed by the regulations, but they have not made
corporate governance better.}

{\large The Sarbanes-Oxley Bill, passed in 2002, was supposed to tamp down
management's greed and make companies responsive to their shareholders' interest
in equity growth. Study after study, however, shows us that this is not what
happened. In a brilliant summary of the statistical assessments of each of the
governance planks in Sarbanes-Oxley, for instance, Yale law professor Roberto
Romano shows that Sarbanes-Oxleydid not do what it was ``supposed'' to do and
often made things worse.}

{\large Even a seemingly obvious reform---requiring an independent audit
committee---turns out not to have been beneficial. Costly, yes! But it did not
improve corporate governance or performance. Romano goes on to document the
failings of Congress and regulators to get it right.2 The wishes of a large
coalition of shareholders with a big stake in finding the right answers to any
given corporation's problems is likely to make businesses work better. A
coalition of government regulators bent on improving their own electoral
prospects is not.}

\subsection{Fixing Democracies}

{\large For the citizens of democracies, life is good. But good does not
preclude better. At the very beginning we mentioned that we would be lazy and not
constantly make subtle distinctions between the size of one democracy's coalition
and another's. Rather, we have repeatedly leaned on the rhetorical distinction
between democracy and autocracy. It is a useful convention, but such a broad
brush risks blurring important distinctions.}

{\large Our approach really depends on the subtle organizational differences in
the size of the three political dimensions on which we focus. For convenience,
these distinctions are often dropped, but even small differences matter. It is
time, then, to confront those small differences head on and see how good can be
made better.}

{\large At the time of its independence, the United States was composed of
thirteen states. They all had broadly the same first-past-the-post electoral
rules and yet their record of performance was remarkably different. It is easy to
be sloppy and think that they all had the same political system--- governed by
the United States Constitution---so that their differences must have come from
somewhere else. In reality, however, their political systems were not the same.
The constitution is silent on many issues that are central to governance. The
constitution tells us nothing, for instance, about how to add up votes. As we
saw, just by changing this simple rule, Harvey Milk could change American
politics by getting elected to San Francisco's Board of Supervisors in 1977, even
though he could not do so in 1975. Seemingly small differences in enfranchisement
rules and districting decisions led to big disparities in the economic (and
social) development of the States of the United States.}

{\large On average, the Northern states developed more rapidly than the Southern
states. It is tempting to ascribe this to the traditional historical narratives
and attribute the general difference to climate or slavery.}

{\large However, a careful examination of the subtle differences between
thestates suggests that variations in their political institutions were the main
culprit behind how differently they developed. Jeffrey Jensen, a former student
of ours and now a faculty member at NYU, Abu Dhabi, did a very careful study of
the differences in the size of the interchangeables, influentials, and essential
groups across the original states.3 He understood that many thought the
differences in development depended on slavery and climate and so he corrected
for these possibilities. Jeff took the size of the slave population carefully
into account just as he also took carefully into account how many frost-free days
there were per year in each of the original thirteen states. He investigated the
distinctions in electoral rules within the early American states that created
different levels of dependency on a large or small coalition drawn from a large
or small set of interchangeables. His discoveries may not only rewrite how we
understand America's early development, but they can also help us understand how
to make our own modern democracy do better.}

{\large Who could vote differed greatly from place to place in the early United
States. Obviously, slavery played an important---but not decisive---role.}

{\large For purposes of allocating seats in the House of Representatives (and
many state legislatures as well), slaves counted as three fifths of a person,
but, of course, they had no right to vote. They were not alone. Women were not
given the franchise until the twentieth century, and in the postcolonial period
there were sizable genderbalance differences between the states.}

{\large Some states also imposed substantial property or educational
qualifications for voting, while others did not. Electoral districts were
typically based upon county lines. Many of these inadequately reflected the
population distribution, so in some legislative districts it took vastly more
votes to win a seat than in others. The modern principle of one-person, one-vote
was not yet the acknowledged law of the land.}

{\large The upshot of these differences was that state political leaders were
accountable to greatly different numbers of voters---that is interchangeables and
essentials. Through painstaking research, Jeff Jensen estimated the proportion of
the states' populations that constituted the minimal winning coalition across
states and across the years. It turns out that the size of the group of
essentials varied enormously from a low of 8.8 percent of adult white males (and
0.9 percent of the total population) in South Carolina to a high of 23.9 percent
of adult white males (and 4.9percent of the total population) in Pennsylvania.}

{\large As the rules to rule by lead us to expect, states in which leaders
required support from a larger proportion of the population developed faster.
Such states build more extensive canal, rail, and road networks. They also
achieved higher educational attainment and were more attractive places for other
Americans to migrate into. People left small-coalition states and flocked to
big-coalition states where public services were better and all manner of public
goods were more extensively provided. Foreign immigrants also flocked to the
larger coalition states, even after correcting for proximity to large ports. Per
capita incomes were much higher and varied almost directly with coalition size
even after correcting for preindependence differences. States with bigger
coalitions simply did better.}

{\large The lesson here is clear. While all the states had the same nominal
rules, redistricting and enfranchisement criteria matter in creating differences
in the competitiveness of political systems and the development of the states. If
properly attended to, districting and enfranchisement decisions could make the
United States an even better place than it currently is.}

{\large Let's start with the decennial redistricting of Congress. The Supreme
Court insists on the principle of one-person, one-vote and that is an excellent
guideline. But it is a principle so easily distorted as to make congressional
elections almost a farce except under extreme conditions.}

{\large This is true for the simple reason that it is politicians in state
legislatures who get to draw up congressional district boundaries. Shockingly
enough, they design the boundaries to make it easier for their party to win.}

{\large Gerrymandering is especially pernicious because it translates into two
conflicting consequences. The average American is greatly dissatisfied with the
job that Congress does and the average American is happy with his or her member
of Congress. The latter is true because districts are constructed by politicians
to give their preferred party a majority and so, by definition, the majority in
any district is likely to be content. But this is a great perversion of
governance. A small coalition of state legislators pick their voters instead of
millions of voters picking their representatives.}

{\large When politicians pick who votes for them it comes as no surprise that
politicians are easily reelected and barely held accountable.Fixing
gerrymandering is something that can be done only once a decade in the United
States. It can be done more frequently in many parliamentary democracies that
equally suffer from this perversion of representative politics. Whether the
opportunity is ongoing or infrequent, fixing gerrymandering is easy, but to be
feasible the voters must take up the cause and fight for it.}

{\large Many scholars of American politics have worked out lots of better ways
to allocate congressional districts than the way it is done now. All the methods
come down to variations on a common theme: district boundaries should not be
manipulated to squeeze some voters in here and others out there. Boundaries
should reflect some basic principles of geometry and the natural constraints of
the terrain, like major rivers or mountains. As a simple principle,
gerrymandering could be greatly diminished by turning redistricting over to some
computer programmers and mathematical political scientists, who could design
rules that are not district specific but that instead apply common principles of
fair representation across all districts.}

{\large A voter initiative in California has taken a step in this direction. It
calls for the appointment of a nonpartisan commission to handle redistricting. We
will see how well that does at being nonpartisan. A computer program drawn up in
ignorance of any specific district's distribution of political preferences would
be much more likely to achieve fairness and impartiality while fulfilling the
spirit as well as the letter of the Supreme Court's insistence on one-person,
one-vote.}

{\large Along with wiping out coalition-reducing gerrymanders, the time may well
have come to amend the US Constitution to get rid of the electoral college. Here
we have an institution whose founders' original intent is pretty clear. They
wanted to ensure that the slave states would join the United States and that
meant erecting constitutional provisions that would protect slavery.4 The
electoral college was one of those institutions. Here is a great example where
original intent most assuredly should not guide modern-day politics. Slavery has
been outlawed for about 150 years and yet the electoral college persists, and the
primary reason, even if rarely spoken out loud, for its survival is that it
allows politicians to construct a coalition of essential supporters that is
substantially smaller than would be the case under direct election.Today the
electoral college is justified by its defenders on the principle that it protects
the interests of the small states since they are overrepresented in terms of
electoral college votes. Indeed, that is exactly what it does. But what happens
to the idea of one-person, one-vote?}

{\large Apparently a vote in Wyoming or Montana should, by this argument, count
more toward choosing the president (and vice president) than a vote in California
or New York. That's a convenient argument---if you're from Wyoming or Montana.
The rules of the electoral college make it possible in a two-candidate race for
one candidate to win a majority of the popular vote and the other candidate to be
elected president of the United States.}

{\large Indeed, judiciously placed votes in a multicandidate race, such as in
the cases of John Quincy Adams and Abraham Lincoln, can allow someone to become
president with a surprisingly small amount of support in the general electorate.
Win the right combination of states---rather than the most voters---and you can
be president. This is just another mechanism to keep the winning coalition
smaller than it could be and, thereby, to empower politicians more and the people
less. Just this mechanism helped distort American politics right up to---and
contributed to precipitating---the Civil War and it helps today to favor
candidates popular in the right places rather than across all the country.}

{\large Immigration policy is a hot topic of debate in the United States and
across much of democratic Europe. The reason for the debate is pretty much the
same whether it is taking place in Phoenix or Paris, Shropshire or San Francisco.
Immigration policies come in three flavors. In one, immigrants have an easy time
becoming citizens in their new homeland. In another, immigrants are welcomed as
guest workers but cannot gain citizenship. And in the third, immigrants just
aren't welcomed. It turns out that which immigration flavor is chosen has big
effects on the size of the groups that dictate whether a government, in governing
for itself, also governs for the people.}

{\large Immigrants without citizenship opportunities increase the size of a
country's disenfranchised group. As such, they are, barring open rebellion (which
is rare among poor immigrants), an impotent source of demand for public goods.
They are not in the interchangeable selectorate and they cannot become
influentials or essentials. Guest worker immigration policies put immigrants in
exactly this boat. The monarchies of the MiddleEast love this sort of immigration
because it doesn't interfere with the control of the few over the many, and if
any immigrant misbehaves he can just be deported.}

{\large We see a similar pattern of constraints that keep immigrants from having
a shot at being in a winning coalition in some democracies as well.}

{\large Gaining citizenship rights is extremely difficult in Japan, for
instance.}

{\large Although over the centuries there have been many waves of immigration to
Japan (from, for instance, North Korea), the limits on access to citizenship
ensures that immigrants do not compel an expansion of the winning coalition.}

{\large In places like Great Britain, immigrants from commonwealth countries
like India or Pakistan can easily enter the country and gain citizenship. This
means that they are quickly made part of the selectorate. Because the size of the
winning coalition in democracies is tied, at least indirectly, to how many people
can vote, this also means that immigration expands the coalition. Naturally many
politicians will be unhappy about this as it diminishes their control over
discretionary money. Current citizens may also be unhappy especially if they back
the party in power. Expanding the coalition reduces the value of their private
rewards. But, for farsighted members of a current winning coalition in a
democracy and for the many citizens who voted for the losing party, increased
immigration means increased pressure on the government to produce more public
goods.}

{\large That's good for just about everyone and especially for those not in the
coalition of essentials.}

{\large Expanding immigrant access and rights, then, can boost the required size
of the winning coalition and, in the process, improve the quality of public
policy. But with so many interests aligned against immigration because of its
short-term costs, it is hard to change immigration rules. Or is it?}

{\large A simple fix that lifts everyone's longer term welfare is to grandfather
in immigrants. Amnesty for illegal immigrants---a dirty word in American
political circles---is a mechanism to choose selectively those who demonstrate
over a fixed period their ability to help produce revenue by working, paying
taxes, and raising children who contribute to the national economy, national
political life, and national social fabric. Give us your poor and let's see if
they can make a better life. Give us your tired and let'ssee if they can be
energized by participating in making a more publicgoods oriented government work
better. Give us your huddled masses longing to be free and let's see if their
children don't grow up to be the foundation of a stronger, more peaceful, and
more prosperous society than they first came to. For generation after generation,
the waves of immigrants to the United States have made our winning coalitions
bigger and better. They have turned from poor, tired, huddled masses into modern
America's success story. This was no happenstance of time or place. This is the
straightforward consequence of easy citizenship and, with it, an expanded winning
coalition that makes for better governance.}

\subsection{Removing Misery}

{\large Beneficial change in the third world is among the most difficult
challenges to overcome. Rampant poverty, frequent exposure to the resource curse,
and long-entrenched autocratic regimes all stand in the way. But change can and
does happen, as the stories of South Africa, Tunisia, Taiwan, and Mexico
demonstrate. When change does happen, it can come from two sources: internal
political upheaval or external threat, and between these two, external threat is
far less likely to succeed in making many better off at the expense of the few.
American presidents and European prime ministers have long advocated a democratic
world and they might even claim some qualified success. The world is much more
democratic today than it was fifty years ago, but it is not likely that our cries
for freedom in the world---rarely backed up by effective efforts---turned many
dictators into freedom lovers. As recent events in the Middle East demonstrate,
effective change comes mostly from local circumstances. After nearly a decade,
the US government has spent over \$1.1 trillion dollars on combat and nation
building in Iraq and Afghanistan.5 The resulting governments are still largely
isolated from the need to improve the welfare of the people. The citizens of
several Middle Eastern nations achieved more in the matter of weeks with
virtually no expense. And if these changes are solidified by winning the backing
of essentials and influentials, then they stand a much better chance of producing
meaningful democratization.}

{\large In the winter of 2011, waves of protests swept across North Africa and
the Middle East. Educated, unemployed people in places like Tunisia and Egypt set
the wheels in motion. With 25 percent to 30 percent of people under twenty-five
unemployed in these countries, the downside for rebels was small. The upside was
great and success was quickly achieved with relatively little violence. At the
same time that these countries were engulfed in regime-changing mass upheaval,
uprisings in Libya and Bahrain also occurred but with very different
consequences.Unemployment among the young educated classes was no better in these
and other oil-rich lands of the Middle East. But unemployment is not that big a
deal for their leaders because, after all, they get their money from oil, not
labor. In terms of comparative deprivation it may seem odd to some that the
Tunisians were the first to rebel, since, at least in relative terms, they were
well looked after by their government. They had a relatively free press and the
ability to assemble. Revolution, however, neither comes to those most deprived of
freedom nor to those who are already free. It is most likely in the great in
between.}

{\large Oppression is a tourist turnoff, so Tunisia's former president, Ben Ali,
whose economy relied significantly on tourism, was compelled to allow more
freedom than he probably would have liked. All those tourists were unwitting
agents for change. They placed Tunisia at risk of revolt to overthrow the
government, because to get their dollars the government had to grant more
freedoms to the people. These freedoms translated into education and access to
information and the means to communicate through the Internet. That, in turn,
meant the possibility of organizing and coordinating fellow dissenters, creating
free assembly online that could and was translated into mass assembly in the
streets. Egypt, another economy with a large tourist sector, was perhaps in the
same boat. Hosni Mubarak ran an oppressive, often violent regime against his
fellow Egyptians, but he never ruled with the iron fist of a world-class
oppressor like natural gas--rich Than Shwe or the vicious Cambodian murderer Pol
Pot. Mubarak couldn't afford to. With US aid waning, Mubarak needed tourist
revenue more and more and so he showed a modicum of restraint.}

{\large No such restraint was seen in Libya, where oil dollars provided Qaddafi
with ample funds to buy mercenary soldiers who did not hesitate to slaughter
rebels seeking to overthrow the colonel's regime.}

{\large In looking for places that may be good targets for democratization, it
is probably a good idea to look to places that rely on tourists for a big chunk
of their economy, like Kenya, Fiji, and an independent Palestine, which hopes to
be a big tourist destination. Reliance on tourism is, of course, only one reason
that an autocrat might allow just enough freedom that opponents might see how to
organize and revolt. Any profound economic strain will do just fine in turning
thought to liberalization provided the strain is so deep that there isn't enough
money around to buy political loyalty.If some mass organizers see how to mount an
uprising, the problem still remains of when to strike. The right moment almost
always depends on their country having a new leader, a sick leader, or a bankrupt
leader.}

{\large Tunisia's Ben Ali, for instance, was rumored to be seriously ill,
possibly suffering from prostate cancer, and Egypt's Mubarak, in his eighties,
may also have been ailing. Those who want to protest when the time is right and
those overseas who want to see democracy blossom can work on laying the
groundwork ahead of time. This may be much easier to achieve than we sometimes
believe.}

{\large Cell phone technology and access to the Internet can transform the lives
of people, even poor people, in developing nations. Even simple information, like
market prices for crops, can make an enormous difference in the income of a
farmer, and of course, the more they can earn the harder they will work. Such
mobile technologies are also giving poor people access to services, such as
banking and insurance, which many of us in developed nations take for granted.
Mobile phone accounts will be increasingly used to transfer money so that with a
simple text message a farmer can pay for fertilizer or receive payment for his
crops.6 The political empowerment of the people by such technology goes way
beyond economic benefits. The adoption of such technologies will make it
impossible for leaders to turn off an important means through which the citizens
can coordinate, without also turning off the commerce and economic activities
that the leadership needs to provide the tax resources with which to sustain
themselves in power. When economic circumstances dictate that a despot's flow of
cash depends on allowing the people to converse, the dictator is truly between a
rock and a hard place. Turn off the technology for long and there will not be
enough money to buy coalition loyalty. Leave the technology on and the people can
coordinate to overthrow their autocrat. Given such circumstances, a smart
dictator will look ahead and work out that he is better off liberalizing now than
risk being exiled, jailed, or killed later. It is not happenstance that a SIM
(subscriber identity module) card for mobile phones costs over \$1,000 in Burma.
It is also not a coincidence that J. J. Rawlings liberalized when Ghana faced
economic collapse.}

{\large Rawlings, recall, went from abusive dictator to competitive democrat. He
did so because the ruinous economic policies associated with decades
ofdictatorship had driven Ghana's economy so far into the ground that he could no
longer ensure even enough food for the people to have the strength and the will
to work to produce revenue for him. Liberalizing the economy to encourage people
back to work was the only way he could continue to pay his coalition. It helped
him cling to power, but the freedom that went along with liberalization empowered
the people.}

{\large Democratization does not require a leader to be benevolent; such leaders
are hard to come by, and often misguided. Rawlings was a ``reluctant democrat,''
but he became a democrat nevertheless. Economic need is a far more reliable path
to empowering the people. Foreign aid all too often eases the financial
stranglehold on leaders. Rawlings first went cap in hand looking for an
international backer. Only when that failed did he embark on market reforms.
There are important lessons to be extracted from his experience, especially when
combined with the power of contemporary technology.}

{\large Rawlings had the right anticipatory response to economic disaster. Not
all leaders can be counted on even to be reluctant democrats. When they are
unprepared to liberalize, even in the face of economic disaster, there is still
plenty that foreign aid donors could do to swing the tide in favor of personal
and economic freedom, and even to persuade petty dictators that it is in their
interest to liberalize. Using foreign aid to set up nationwide wireless access to
the Internet and to provide the poor with mobile phones could be a
win-win-win-win among the four constituencies affected by aid.}

{\large Leaders will gain because commerce will improve, generating more revenue
for their discretionary use. Some donor constituents will benefit because they
will sell the necessary technology to their government to be given in aid. That
will make them happier with their incumbent, improving the democratic donor's
chances for reelection. And unlike most aid, citizens in the recipient countries
will also benefit. First, they will have a better chance to make a good living.
Second, they will be in a better position to freely assemble over the Internet
and press their government for greater freedom and reliance on a larger
coalition. And, as we said, smart leaders, benefiting as they will from the flow
of money, will accept the technology and will, in time, be likely to liberalize
so that they can stay on in power.}

{\large Those who reject the technology will also be helping the cause
offreedom. By saying no to technology that helps the people help themselves, they
will make clear that they are intransigent autocrats. Then donors will know
better than to waste their resources on them and that will free up more aid
dollars to help those people, places, and leaders who are willing to take the
political risks to gain the economic benefits. If the policy concessions to be
bought with aid are economic, then this is just the sort of aid that can satisfy
the donor's interests, the recipient's interests, the wishes of the donor's
coalition, and the poor people that we all give lip service to wanting to help.
For those who want to buy security concessions, aid will, alas, probably continue
as it has in the past. But then those buying security concessions might also
think about how large a business advantage they are giving to competitor cell
phone producers at the expense of their homegrown industry.}

{\large Finally, even when aid is given for security reasons, it can be utilized
more effectively. Lots of aid is bad for poor people as we know. Even when it is
just about buying policy concessions, however, it could be made to work better,
at least from the donor's perspective. Instead of giving aid on the promise by
recipients that they will change their policies, aid money could be put in an
independently controlled escrow account. Aid deals then would need to define
precise performance criteria. If those criteria are met, then the funds are
released. If the criteria are not met or performance does not come up to agreed
standards, the money reverts to the donor. In such a world donors would pay to
get what they want and would not throw good money after bad if the expectation of
payment is not adequate to change the recipient's behavior. Remember, that was
one of the fixes we proposed for dealing with the great aid sinkhole that
Pakistan has become.}

{\large Perhaps the toughest cases for improvement are resource-rich monarchies
and autocracies. The people in such places are kept down, the leaders become
fabulously wealthy, and they have the means to brutalize opponents. But even in
these places there are means to achieve change without the horrendous
consequences experienced by those who revolted against oil-rich Qaddafi. Both the
international community and domestic would-be rebels could provide the right
foundation for the peaceful shift from dictatorship to democracy. Let us begin
with the harder sell---the international community.South Africa's Nelson Mandela
taught the world an important lesson when he came to power. Alas, it is a lesson
only poorly learned. Following the collapse of the apartheid government, he
organized truth and reconciliation commissions. These were designed to provide
people who had oppressed the apartheid regime's opponents to come forward,
confess their crimes, and be granted amnesty. The United Nations certainly could
build a body of international law that motivates dictators facing rebellion to
turn power over to the people peacefully. The UN could prescribe a process for
transition from dictatorship to democracy. At the same time it could stipulate
that any dictator facing the pressure to grant freedom to the people would have a
brief, fixed period of time, say a week, to leave the country in exchange for a
blanket perpetual grant of amnesty against prosecution anywhere for crimes
committed as his nation's leader.}

{\large There is clear precedence for such a policy. It is common practice to
give criminals immunity if they agree to testify. Some victims are bound to
resent that the perpetrator of heinous acts goes unpunished. Unfortunately, the
alternative is to leave the dictator with few options but to gamble on holding
onto power through further murderous acts. Certainly there is little justice in
letting former dictators off the hook. But the goal should be to preserve and
improve the lives of the many who suffer at the hands of desperate leaders, who
might be prepared to step aside in exchange for immunity.}

{\large The incentives to encourage leaders to step aside could be further
strengthened if, in exchange for agreeing to step down quickly, they would be
granted the right to retain some significant amount of ill-gotten gains, and safe
havens for exile where the soon-to-be ex-leadership and their families can live
out their lives in peace. Offering such deals might prove self-fulfilling. Once
essential supporters believe their leader might take such a deal, they themselves
start looking for his replacement, so even if the leader had wanted to stay and
fight he might no longer have the support to do so. The urge for retribution is
better put aside to give dictators a reason to give up rather than fight. Muammar
Qadaffi had none of these opportunities and so faced a stark choice: live the
life of the hunted or fight to the death. He chose the latter, to the detriment
of the Libyan people and anyone who values humanity.}

{\large Additional choices can be provided. Britain's transition from monarchyto
constitutional monarchy provides a valuable lesson. Leaders want to survive in
office and maximize their control over money. But what if their choice is to
trade the power of office in exchange for the right to the money? The English
monarchy once had both power and money but it faced severe pressure that could
have ended, as in so many other places, with the erstwhile royal family having
neither power nor money. That is what happened to the Russian and French royal
families, and for that matter the Stewart branch of the English royal family, in
the wake of revolution.}

{\large Imagine, instead, that they had the option of keeping the crown but
turning power over to a properly elected government of the people, as William and
Mary and the subsequent Hanoverian dynasty did in England. As compensation for
doing so, they could have been granted the right to keep the family's wealth and
even the assurance of further income from the state for a long, specified period
of time (say 100 years). The transition to being fabulously wealthy figureheads
of constitutional monarchies is an option the Saudi Arabian royal family, the
Jordanian royal family, and the royal families of the Emirates might well
contemplate as a better option than trying to crush rebellion. Revolutionaries
might fail today or tomorrow, but leaders have only to lose once and by then it
will be too late for them to negotiate their way to a soft landing.}

\subsection{Free and Fair Elections: False Hope}

{\large Just as there are actions that can promote beneficial change, there are
also actions that hamper progress. One of the most popular unhelpful solutions is
an election. Leaders at risk often decide to hold fraudulent elections to create
the impression of openness and fairness. Needless to say, bogus elections don't
move a country toward better policies or more freedom for the people. Rather,
fake elections empower the ruler by increasing the ranks of the interchangeables
without adding in any meaningful way to the size of the influential and essential
groups.}

{\large True, meaningful elections might be the final goal, but elections for
their own sake should never be the objective. When the international community
pushes for elections without being careful about how meaningful they are, all
that is accomplished is to further entrench a nasty regime. International
inspectors, for instance, like to certify whether people could freely go to the
polling place and whether their votes were properly counted, as if that means
there was a free and fair election. There's no reason to impede the opportunity
to vote or to cheat when counting votes if, for instance, a regime first bans
parties that might be real rivals, or if a government sets up campaign
constraints that make it easy for the government's party to tell its story and
makes it impossible for the opposition to do the same.}

{\large Russian incumbents don't need to cheat in counting votes to get the
outcome they want. They don't need to block people from getting into the polling
place. They deprive the opposition from having access to a free press and from
holding rallies so, sure, observers will easily conclude that elections were free
and fair in the narrow sense, and just as easily we can all recognize that they
were neither really free nor fair.}

{\large Ultimately, elections need to follow expanded freedom and not be thought
of as presaging it! Sometimes the problems of the world seem beyond our capacity
to solve.Yet there is no mystery about how to eradicate much of the world's
poverty and oppression. People who live with freedom are rarely impoverished and
oppressed. Give people the right to say what they want; to write what they want;
and to gather to share ideas about what they want, and you are bound to be
looking at people whose persons and property are secure and whose lives are
content. You are looking at people free to become rich and free to lose their
shirts in trying. You are looking at people who are not only materially well off
but spiritually and physically, too. Sure, places like Singapore and parts of
China prove that it is possible to have a good material life with limited
freedom---yet the vast majority of the evidence suggests that these are
exceptions and not the rule. Economic success can postpone the democratic moment
but it ultimately cannot replace it.}

{\large A country's relative share of freedom is ultimately decided by its
leaders.}

{\large Behind the world of misery and oppression lie governments run by small
cliques of essentials who are loyal to leaders who can make them rich.}

{\large Behind the world of freedom and prosperity lie governments that depend
on the backing of a substantial coalition of ordinary people drawn from a large
pool of influentials, who are in turn drawn from a large pool of
interchangeables. It is not difficult to draw a line from the poverty and
oppression of the world to the corrupt juntas and brutal dictators who skim from
their country's revenues to stay in power. Politics, and political institutions,
define the bounds of the people's lives.}

{\large By now it should be clear that there is a natural order governing
politics, and it comes with an ironclad set of rules. They cannot be altered. But
that does not mean that we cannot find better paths to work within the laws of
politics.}

{\large We have suggested some ways to work within the rules to produce better
outcomes. At the end of the day, the solutions we have suggested will not be
applied perfectly. There are good reasons for that. Entrenched ways of thinking
make altering our approach to problems difficult. Many will conclude that it is
cruel and insensitive to cut way back on foreign aid. They will tell us that all
the money spent on aid is worth it if just one child is helped. They will forget
to ask how many children are condemned to die of neglect because, in the process
of helping a few, aid props up leaders who look after the people only after they
have looked after themselves and their essential backers, if at all. But before
we shift blame onto our ``flawed''democratic leaders for their failures to make
the world a better place, we need to remember why it is that they enact the
policies that they do. The sworn duty of democratic leaders is to do precisely
what we, the people, want.}

{\large American presidents, virtually since the nation's founding, have
routinely endorsed the idea, if not the reality, of spreading democracy.
President Woodrow Wilson, in calling on the Congress to declare war against
Germany on April 2, 1917, reflected his deeply held view that, ``The world must
be made safe for democracy.... We have no selfish ends to serve.}

{\large We desire no conquest, no dominion.'' His sentiment was echoed nearly
ninety years later when George W. Bush, in his second inaugural address,
proclaimed, ``The survival of liberty in our land increasingly depends on the
success of liberty in other lands. The best hope for peace in our world is the
expansion of freedom in all the world . . . So it is the policy of the United
States to seek and support the growth of democratic movements and institutions in
every nation and culture, with the ultimate goal of ending tyranny in our
world.'' Yet Wilson set his noble sentiments aside when it came to standing up
for self-determination in the colonies controlled by America's allies. In the
same spirit, President Bush, during the same speech in which he called for
democracy ``in all the world,'' also noted: ``My most solemn duty is to protect
this nation and its people against further attacks and emerging threats.'' The
president's ``solemn duty'' highlights the problem. There is an inherent tension
between promoting democratic reform abroad and protecting the welfare of the
people here at home. Free, democratic societies typically live in peace with each
other and promote prosperity at home as well as between nations, making
representative government attractive to people throughout the world. Yet
democratic reform, as the experiences of the United States with Khomeini's Iran
and Hamas-led Palestine make clear, does not always also enhance the security or
welfare of Americans (or citizens elsewhere in the world) against foreign threats
and may even jeopardize that security.}

{\large Our individual concerns about protecting ourselves from unfriendly
democracies elsewhere typically trump our longer term belief in the benefits of
democracy. Democratic leaders listen to their voters because that is how they and
their political party get to keep their jobs. Democraticleaders were elected,
after all, to advance the current interests at least of those who chose them. The
long run is always on someone else's watch.}

{\large Democracy overseas is a great thing for us if, and only if, the people
of a democratizing nation happen to want policies that we like. When a foreign
people are aligned against our best interest, our best chance of getting what we
want is to keep them under the yoke of an oppressor who is willing to do what we,
the people, want.}

{\large Yes we want people to be free and prosperous, but we don't want them to
be free and prosperous enough to threaten our way of life, our interests, and our
well-being---and that is as it should be. That too is a rule to rule by for
democratic leaders. They must do what their coalition wants; they are not
beholden to the coalition in any other country, just to those who help keep them
in power. If we pretend otherwise we will just be engaging in the sort of
utopianism that serves as an excuse for not tackling the problems that we can.}

{\large We began with Cassius imploring Brutus to act against Julius Caesar's
despotism: ``The fault, dear Brutus, is not in our stars, but in ourselves.'' We
humbly add that the reason the fault is in ourselves is because we, the people,
care so much for ourselves and so little for the world's underlings.}

{\large But we have also seen that there is hope for the future. Every
government and every organization that relies on a small coalition eventually
erodes its own productivity and entrepreneurial spirit so much that it faces the
risk of collapsing under the weight of its own corruption and inefficiency. When
those crucial moments of opportunity arise, when the weight of bad governance
catches up with despots, then a few simple changes can make all the difference.}

{\large We have learned that just about all of political life revolves around
the size of the selectorate, the influentials, and the winning coalition. Expand
them all, and the interchangeables no more quickly than the coalition, and
everything changes for the better for the vast majority of people. They are
liberated to work harder on their own behalf, to become better educated,
healthier, wealthier, happier, and free. Their taxes are reduced and their
opportunities in life expand dramatically. We can get to these moments of change
faster through some of the fixes proposed here but sooner or later every society
will cross the divide between small-coalition, largeselectorate misery to a large
coalition that is a large proportion of theselectorate---and peace and plenty
will ensue. With a little bit of hard work and good luck this can happen
everywhere sooner, and if it does we all will prosper from it.Acknowledgments The
Dictator's Handbook is the culmination of nearly two decades of research into the
motivation and constraints of leaders. We owe a huge debt of gratitude to
friends, colleagues, coauthors, and critics who have helped sharpen our
understanding of what makes the world tick and given us insight into how it can
be made to tick more smoothly.}

{\large In academic circles, our work has become known as selectorate theory.}

{\large Together with two other founders of this way of thinking, Randolph
Siverson and James Morrow, we published a comprehensive exposition of the theory,
The Logic of Political Survival, in 2003 with MIT Press. That massive 500-plus
page tome was full of mathematical models and complex statistical tests. Although
we readily admit it is not an easy read, it is the most comprehensive statement
of the theory. However, it was not the origin; nor was it the finale.}

{\large The genesis of selectorate theory was Bruce Bueno de Mesquita and Randy
Siverson's foray into examining what happens to leaders after they fight wars.
Surprisingly, no one had systematically looked at how winning or losing wars
affects leader survival. Given their background in international relations, Bruce
and Randy continued to pursue warrelated topics, and brought in James Morrow and
Alastair Smith---and the collaborative team of BdM2S2 was born. In 1999 the four
of us published ``An Institutional Explanation of the Democratic Peace'' in the
American Political Science Review. This paper offered a solution to what at the
time was the dominant question in international relations: why don't democratic
nations fight each other? Many of the existing theories relied on asserting
different normative motivations for democrats and autocrats. Unfortunately, all
too often democrats act contrary to these alleged higher values. In contrast,
selectorate theory assumed leaders had the same objective, to stay in power, and
what differentiated democrats from autocrats was that the former's dependence on
a large coalition of supporters means democrats direct state resources to winning
wars. Autocrats enhance their survival by hoarding resources to pay off cronies,
even if this means losing the war. What started off as a desire to know why
democracies don't fighteach other ended up telling us how nations fight and what
they fight over.}

{\large As science is supposed to do, the answer to one problem provides answers
to other problems and ends up posing a new set of questions.}

{\large In 2002, BdM2S2 published a mathematical representation of the
selectorate theory, ``Political Institutions, Policy Choice and the Survival of
Leaders,'' in the British Journal of Political Science. We further refined this
model and then tested its predictions. This material became the basis for The
Logic of Political Survival. Since its publication we have continued to advance
the theory. In articles in the Journal of Conflict Resolution in 2007 and
International Organization in 2009, we examined how nations trade aid for policy
concessions. Recent extensions of the mathematical model incorporate
revolutionary movements and have been published in the Journal of Politics in
2008, Comparative Political Studies in 2009, and the American Journal of
Political Science in 2010.}

{\large Selectorate theory offers a powerful, yet simple to use, model of
politics.}

{\large It forms the basis for the models in Punishing the Prince, for instance.
That book, by Fiona McGillivray and Alastair Smith, examines how leaders sanction
leaders in other states. By targeting punishments at leaders rather than the
nations they represent, a leader leverages the effectiveness of their state's
policies in three ways. First, such mechanisms provide an explicit means through
which to restore relations between states. Second, they encourage the citizens in
targeted nations to remove their leaders in order to restore cooperation. Third,
since leaders fear removal, the threat of such targeted punishments encourage
leaders to abide by international norms in the first place. By focusing on the
interactions of leaders instead of thinking of international cooperation as only
between nations, Fiona enriched our understanding of interstate relations. As was
characteristic of her scholarship, she asked questions that no one else had
thought to ask and provided elegant answers that pushed scholarship in new
directions.}

{\large For instance, she examined how the dynamics of trade flows between
nations depend upon the turnover of their national leaders. She found that the
replacement of autocrats systematically altered trade flows in predictable ways.}

{\large Punishing the Prince was published in 2008, just a few days before Fiona
died. She is missed every day by everyone who knew her, but most especially by
Alastair and their three children, Angus, Duncan, and Molly.She was both our
greatest supporter and harshest critic. Fiona endured a long and terrible
illness, but her humor and spirit never failed even in her darkest hour. She died
waiting on a transplant list. Please sign your donor card. The doctors and nurses
at Columbia-Presbyterian hospital and elsewhere, and especially Erika
Berman-Rosenzweig and Nazzareno Gali\`{e} gave us extra time with her; they have
our profound thanks. Although she ruled Alastair's life with a rod of iron, Fiona
was the embodiment of a benevolent dictator.}

{\large Developing selectorate theory and writing this book have been a huge
undertaking that we could never have done without the assistance of others.
Randolph Siverson and James Morrow have been our collaborators from the start and
many of the ideas presented here are as much theirs as ours. Financial support is
also vital for any research and early developments of selectorate theory
benefited from generous grants from the National Science Foundation. We also wish
to thank Roger Hertog for his support through the Alexander Hamilton Center for
Political Economy at New York University.}

{\large Hans Hoogeveen, formerly Chief Economist for the World Bank in Tanzania,
commissioned a study applying the selectorate framework to help explain why the
World Bank's efforts in Tanzania had not been as successful as they had hoped.
The opportunity to do that study helped sharpen our own understanding of
selectorate theory and proved essential to advancing our views on the formation
of blocs of interests, whether ethnic, linguistic, geographic, or occupational.
The work undertaken at Hans's request was a great stimulus for us and we are most
appreciative of his support and the opportunity he gave us. Our current employer,
New York University, is a superb organization that has never hesitated in
supporting our research and teaching. We are also grateful to the Hoover
Institution, Yale University, and Washington University in St. Louis for their
support. The generosity of such organizations has allowed us to benefit from
superb research assistance. Alexandra Bear and especially Michal Harari greatly
assisted us in preparing materials for this book.}

{\large Colleagues, students, and friends always improve any endeavor,
especially when they are critics as well as supporters; and this book is no
exception. We are truly fortunate to be connected to such a wonderful network of
scholars and friends from whom we learn everyday.Conversations with Neal Beck,
Ethan Bueno de Mesquita, George Downs, William Easterly, Sandy Gordon, Mik Laver,
Jim Morrow, Lisa Howie, Jeff Jensen, Yanni Kotsonis, Alex Quiroz-Flores, Shinasi
Rama, Peter Rosendorff, Harry Roundell, Shanker Satyanath, John Scaife, Randy
Siverson, Alan Stam, Federico Varesse, James Vreeland, Leonard Wantchekon, and
many others helped shape this book.}

{\large Much of our previous work has been aimed at an academic audience.}

{\large Writing a ``readable'' book is a very different enterprise. Fortunately
Eric Lupfer, our agent, took us under his wing. He worked tirelessly with us on
structure, style, and presentation, and he fixed us up with a phenomenal press.
PublicAffairs has been superbly supportive throughout the process.}

{\large Their entire team has helped us and supported us every step of the way.}

{\large We thank Brandon Proia who made the book more readable, clearer, and
more tightly argued than it would otherwise have been; and, in alphabetical
order, Lindsay Jones, Lisa Kaufman, Jamie Leifer, Clive Priddle, Melissa Raymond,
Anais Scott, Susan Weinberg, and Michelle Welsh-Horst, each of whom contributed
mightily to improving our book. Alas, we cannot hold them responsible for its
continued failing, for which Alastair and Bruce acknowledge that the other is
responsible.}

{\large Of all the organizations we study, the ones we care most about are
family. These are the people that brighten our world: Ethan and Rebecca and
Abraham and Hannah; Erin and Jason and Nathan and Clara; Gwen and Adam and
Isadore; Angus, Duncan and Molly. And most of all, we thank Arlene and Fiona, to
whom we dedicate this book and ourselves.}

{\large Our fondest hope is for the well-being and success of those who imperil
their lives to keep dictators in check.}


\end{document}
