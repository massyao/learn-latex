% header.tex
% ----------
% Contains the preamble of the document and all customizations.
%
% -----------------------------------------------------------------------------
%
% Copyright (C) 2014  Claudio Vellage (claudio.vellage@latex-tutorial.com)
%
% This program is free software: you can redistribute it and/or modify
% it under the terms of the GNU General Public License as published by
% the Free Software Foundation, either version 3 of the License, or
% (at your option) any later version.
%
% This program is distributed in the hope that it will be useful,
% but WITHOUT ANY WARRANTY; without even the implied warranty of
% MERCHANTABILITY or FITNESS FOR A PARTICULAR PURPOSE.  See the
% GNU General Public License for more details.
%
% You should have received a copy of the GNU General Public License
% along with this program.  If not, see <http://www.gnu.org/licenses/>.
%
% -----------------------------------------------------------------------------
%

% The scrartcl documentclass provides easier customization compared to the
% default article class. For more information read the KOMA-Script manual.
%
\documentclass[a4paper,12pt,notitlepage,numbers=noenddot]{scrartcl}

%
% Libraries
% ---------
% A short explanation is given behind this, so you can decide whether you need
% a certain library or not. (More libraries in pgf/tikz section)
%
\usepackage[english]{babel} % Hyphenation for us english
\usepackage[]{amsmath} % Environments to typeset math
\usepackage[utf8]{inputenc} % Support for Unicode
\usepackage[]{geometry} % Allows to change the layout of a page
\usepackage{graphicx} % Optional arguments for \includegraphics
\usepackage{latexsym} % Defines a few more characters
\usepackage{sectsty} % Allows to change the layout of section headings
\usepackage{float} % Float commands (e.g. to force position of figure [H])
\usepackage{setspace} % Allows to change the linespacing
\usepackage{tabularx} % Extended tabular environment
\usepackage{cite} % Citations
\usepackage{hyperref} % Support for hyperlinks
\usepackage{booktabs} % Includes \toprule, \midrule etc. to make nice tables
\usepackage{enumitem} % Make bullet lists and more (e.g. \begin{itemize})
\usepackage{listings} % Beautiful listings
\usepackage{trfsigns} % Symbols for transformations (e.g. Laplace transform)
\usepackage{caption} % Captions for figures
\usepackage{subcaption} % Support for subfigures (pictures side-by-side etc.)
\usepackage{multirow} % Support for multirow headings in tables
\usepackage{nicefrac} % Alternative to \frac{}{} useful for inline fractions
\usepackage{lmodern} % Use enhanced version fo Computer Modern
\usepackage{pdfpages} % Include pdf files
\usepackage{pgfplotstable} % Automatic generation of tables (can read .csv)

%
% Layout customizations
% ---------------------
\geometry{a4paper,left=25mm,right=30mm, top=2cm, bottom=3cm} % Page layout 
\onehalfspacing % Linespacing
\allsectionsfont{\sffamily} % Set font for sections
\setitemize{noitemsep,parsep=0pt,partopsep=0pt} % Linespacing for "itemize"
\floatstyle{plain} % No boxes around figures
\newfloat{code}{H}{myc} % New float setting for code
\renewcommand\lstlistingname{Listing} % Changes listings name if require
\setlength{\parindent}{0mm} % Don't indent paragraphs

% -----------------------------------------------------------------------------

%
% Tikz/pgfplots customizations
% ----------------------------
\usepackage{tikz} % Create graphics in LaTeX
\usepackage[european]{circuitikz} % Create electric circuits in latex
\usetikzlibrary{calc,arrows,shapes} % A few useful shapes for tikz

% This is a hack to get volt- and ammeters without an arrow. (Replacement for
% the default circuitikz meters)
\tikzset{component/.style={
  draw,
  thick,
  circle,
  fill=white,
  minimum size=0.75cm,
  inner sep=0pt}}

\usepackage{pgf} % Create graphics in latex
\usepackage{pgfplots} % Create plots in latex (can read .csv files)
\usepackage{pgfplotstable} % Automatic generation of tables (can read .csv)
\pgfplotsset{compat=newest} % Better customization of plots
\usepgfplotslibrary{units} % Set units for axes in plots

% -----------------------------------------------------------------------------

\usepackage{siunitx} % Supreme typesetting of units
\sisetup{
  exponent-to-prefix  = true,
  round-mode          = places, 
  round-precision     = 2,
%  scientific-notation = engineering, % Use multiples of 3 as exponent
  locale=DE, % Typeset numbers and units the German way
  output-complex-root=\ensuremath{\mathrm{j}} % Change complex i to j
}

% -----------------------------------------------------------------------------

%
% Additional Includes
% -------------------
% If multiple users work on a single document, it happens that some users need
% additional libraries. To avoid conflicts, people should declare their
% packages in a seperate file, so changes are not overwritten if two users
% change the header at the same time. (This _will_ happen in Dropbox)
%
%\input{include/user1.tex}
%\input{include/user2.tex}
