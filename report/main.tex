%&main
%
% main.tex
% --------
%
% Quick-Start
% -----------
% 1. To create the document, run e.g. "pdflatex main.tex", all other files will
%    be included automatically using the "\input" command.
%
% 2. You can also use the Makefile to generate the document. This requires
%    additional programs to be installed and most likely only works on Linux.
%    See README for further instructions.
%
% 3. The preamble, source files and resources should be in the according
%    folders (include,src,res) the subfolders are self-explanatory.
%
% 4. It is imperative that two people do not work on the same file at once when
%    using Dropbox, otherwise you _will_ experience data loss.
%
% -----------------------------------------------------------------------------
%
% This preset is still under active development and I'd be glad to receive
% questions, feedback and suggestions.
%
% -----------------------------------------------------------------------------
%
% Copyright (C) 2014  Claudio Vellage (claudio.vellage@latex-tutorial.com)
%
% This program is free software: you can redistribute it and/or modify
% it under the terms of the GNU General Public License as published by
% the Free Software Foundation, either version 3 of the License, or
% (at your option) any later version.
%
% This program is distributed in the hope that it will be useful,
% but WITHOUT ANY WARRANTY; without even the implied warranty of
% MERCHANTABILITY or FITNESS FOR A PARTICULAR PURPOSE.  See the
% GNU General Public License for more details.
%
% You should have received a copy of the GNU General Public License
% along with this program.  If not, see <http://www.gnu.org/licenses/>.
%
% -----------------------------------------------------------------------------
%
% Include: Header (Preamble)
% --------------------------
% The header file contains the preamble of the LaTeX document. All
% customizations go into this file to keep the the document clean and easy to
% maintain.
%
% If you're not using subversion, you maybe want to check out the section:
% "custom headers" in this file. It is less prone to accidental overwrites.
%
% header.tex
% ----------
% Contains the preamble of the document and all customizations.
%
% -----------------------------------------------------------------------------
%
% Copyright (C) 2014  Claudio Vellage (claudio.vellage@latex-tutorial.com)
%
% This program is free software: you can redistribute it and/or modify
% it under the terms of the GNU General Public License as published by
% the Free Software Foundation, either version 3 of the License, or
% (at your option) any later version.
%
% This program is distributed in the hope that it will be useful,
% but WITHOUT ANY WARRANTY; without even the implied warranty of
% MERCHANTABILITY or FITNESS FOR A PARTICULAR PURPOSE.  See the
% GNU General Public License for more details.
%
% You should have received a copy of the GNU General Public License
% along with this program.  If not, see <http://www.gnu.org/licenses/>.
%
% -----------------------------------------------------------------------------
%

% The scrartcl documentclass provides easier customization compared to the
% default article class. For more information read the KOMA-Script manual.
%
\documentclass[a4paper,12pt,notitlepage,numbers=noenddot]{scrartcl}

%
% Libraries
% ---------
% A short explanation is given behind this, so you can decide whether you need
% a certain library or not. (More libraries in pgf/tikz section)
%
\usepackage[english]{babel} % Hyphenation for us english
\usepackage[]{amsmath} % Environments to typeset math
\usepackage[utf8]{inputenc} % Support for Unicode
\usepackage[]{geometry} % Allows to change the layout of a page
\usepackage{graphicx} % Optional arguments for \includegraphics
\usepackage{latexsym} % Defines a few more characters
\usepackage{sectsty} % Allows to change the layout of section headings
\usepackage{float} % Float commands (e.g. to force position of figure [H])
\usepackage{setspace} % Allows to change the linespacing
\usepackage{tabularx} % Extended tabular environment
\usepackage{cite} % Citations
\usepackage{hyperref} % Support for hyperlinks
\usepackage{booktabs} % Includes \toprule, \midrule etc. to make nice tables
\usepackage{enumitem} % Make bullet lists and more (e.g. \begin{itemize})
\usepackage{listings} % Beautiful listings
\usepackage{trfsigns} % Symbols for transformations (e.g. Laplace transform)
\usepackage{caption} % Captions for figures
\usepackage{subcaption} % Support for subfigures (pictures side-by-side etc.)
\usepackage{multirow} % Support for multirow headings in tables
\usepackage{nicefrac} % Alternative to \frac{}{} useful for inline fractions
\usepackage{lmodern} % Use enhanced version fo Computer Modern
\usepackage{pdfpages} % Include pdf files
\usepackage{pgfplotstable} % Automatic generation of tables (can read .csv)

%
% Layout customizations
% ---------------------
\geometry{a4paper,left=25mm,right=30mm, top=2cm, bottom=3cm} % Page layout 
\onehalfspacing % Linespacing
\allsectionsfont{\sffamily} % Set font for sections
\setitemize{noitemsep,parsep=0pt,partopsep=0pt} % Linespacing for "itemize"
\floatstyle{plain} % No boxes around figures
\newfloat{code}{H}{myc} % New float setting for code
\renewcommand\lstlistingname{Listing} % Changes listings name if require
\setlength{\parindent}{0mm} % Don't indent paragraphs

% -----------------------------------------------------------------------------

%
% Tikz/pgfplots customizations
% ----------------------------
\usepackage{tikz} % Create graphics in LaTeX
\usepackage[european]{circuitikz} % Create electric circuits in latex
\usetikzlibrary{calc,arrows,shapes} % A few useful shapes for tikz

% This is a hack to get volt- and ammeters without an arrow. (Replacement for
% the default circuitikz meters)
\tikzset{component/.style={
  draw,
  thick,
  circle,
  fill=white,
  minimum size=0.75cm,
  inner sep=0pt}}

\usepackage{pgf} % Create graphics in latex
\usepackage{pgfplots} % Create plots in latex (can read .csv files)
\usepackage{pgfplotstable} % Automatic generation of tables (can read .csv)
\pgfplotsset{compat=newest} % Better customization of plots
\usepgfplotslibrary{units} % Set units for axes in plots

% -----------------------------------------------------------------------------

\usepackage{siunitx} % Supreme typesetting of units
\sisetup{
  exponent-to-prefix  = true,
  round-mode          = places, 
  round-precision     = 2,
%  scientific-notation = engineering, % Use multiples of 3 as exponent
  locale=DE, % Typeset numbers and units the German way
  output-complex-root=\ensuremath{\mathrm{j}} % Change complex i to j
}

% -----------------------------------------------------------------------------

%
% Additional Includes
% -------------------
% If multiple users work on a single document, it happens that some users need
% additional libraries. To avoid conflicts, people should declare their
% packages in a seperate file, so changes are not overwritten if two users
% change the header at the same time. (This _will_ happen in Dropbox)
%
%\input{include/user1.tex}
%\input{include/user2.tex}


% These macros are used in the statement to set several dates and names related
% to the assignment.
%
% The due date of the report can be set like this: 
%
%   \newcommand{\city}{Berlin}
%
\newcommand{\city}{} % city
\newcommand{\dueData}{} % due date
\newcommand{\university}{} % university
\newcommand{\department}{} % department
\newcommand{\mySubject}{} % subject
\newcommand{\reportNo}{} % number of experiment
\newcommand{\reportName}{} % name of experiment
\newcommand{\reportDate}{} % date of experiment
\newcommand{\participants}{} % participants
\newcommand{\supervisor}{} % supervisor

% 
% Titlepage settings
% ------------------
% These settings are used by the \maketitle command and are set by macros by
% default.
%
\titlehead{
  \university{}\\
  \fakulty{}
} 
\title{\reportName{}} % Name of report
\subject{\mySubject{}} % Subject
\date{\reportDate{}} % Date (Set by Macro)
\author{\participants{}} % Authors (Set by Macro)
\publishers{\supervisor{}} % Publishers or supervisor (Set by Macro)

% -----------------------------------------------------------------------------
%
% BEGIN OF DOCUMENT
% -----------------
% This is the section containing the actual text of the document. The default
% structure of this section looks like this:
%
% main.tex
% -titlepage.tex (Title)
% -intro.tex (Introduction)
% -methods.tex (Methods)
% -results.tex (Results)
% -discussion.tex (Discussion)
% -Bibliography (Bibliography)
% -Appendix (Appendix)
% --listings.tex (Listings)
% --tables.tex (Tables)
% --figures.tex (Figures)
% statement.tex (Statement)
%
\begin{document}
\pagenumbering{gobble}
% Titlepage
% ---------
% This usually contains the line \maketitle, a custom titlepage or imports a
% .pdf file.
%
% titlepage.tex
% -------------
% Creates the title using \maketitle or include a .pdf file as titlepage.
%
% NOTE: The \maketitle command is configured in the header.tex file. This file 
% is only meant to provide an easy way of switching between using a .pdf or
% \maketitle for the titlepage.
%
% -----------------------------------------------------------------------------
%
% Copyright (C) 2014  Claudio Vellage (claudio.vellage@latex-tutorial.com)
%
% This program is free software: you can redistribute it and/or modify
% it under the terms of the GNU General Public License as published by
% the Free Software Foundation, either version 3 of the License, or
% (at your option) any later version.
%
% This program is distributed in the hope that it will be useful,
% but WITHOUT ANY WARRANTY; without even the implied warranty of
% MERCHANTABILITY or FITNESS FOR A PARTICULAR PURPOSE.  See the
% GNU General Public License for more details.
%
% You should have received a copy of the GNU General Public License
% along with this program.  If not, see <http://www.gnu.org/licenses/>.
%
% -----------------------------------------------------------------------------
%
\begin{titlepage}

%\includegraphics[width=\linewidth]{res/title.pdf}
\maketitle

\end{titlepage}


%
% Table of Contents
% -----------------
% I prefer roman page numbering for the table of contents. Comment this out if
% you prefer arabic numbering. The pagecounter is reset to 1 for the rest of
% the document.
%
\tableofcontents
\pagenumbering{roman}
\newpage
\pagenumbering{arabic}
\setcounter{page}{1}

%
% Abstract
% --------
% Uncomment and put your abstract in this section if you need one.
%
%\begin{abstract}
%  
%\end{abstract}
%\newpage

%
% Introduction
% ------------
% This section includes the intention of this document. In case of lab reports
% it usually contained a brief explanation of the experiment.
%
\section{Introduction}
% intro.tex
% ---------
% Introduction of your report. 
%
% -----------------------------------------------------------------------------
%
% Copyright (C) 2014  Claudio Vellage (claudio.vellage@latex-tutorial.com)
%
% This program is free software: you can redistribute it and/or modify
% it under the terms of the GNU General Public License as published by
% the Free Software Foundation, either version 3 of the License, or
% (at your option) any later version.
%
% This program is distributed in the hope that it will be useful,
% but WITHOUT ANY WARRANTY; without even the implied warranty of
% MERCHANTABILITY or FITNESS FOR A PARTICULAR PURPOSE.  See the
% GNU General Public License for more details.
%
% You should have received a copy of the GNU General Public License
% along with this program.  If not, see <http://www.gnu.org/licenses/>.
%
% -----------------------------------------------------------------------------
%


%
% Methods
% -------
% This part usually contains the theoretical background necessary to solve the
% assignments in the lab reports. For research papers, this would be called 
% related works.
%
\section{Methods}
% methods.tex
% --------------
% Includes explanations of theoretical background of this experiment
%
% -----------------------------------------------------------------------------
%
% Copyright (C) 2014  Claudio Vellage (claudio.vellage@latex-tutorial.com)
%
% This program is free software: you can redistribute it and/or modify
% it under the terms of the GNU General Public License as published by
% the Free Software Foundation, either version 3 of the License, or
% (at your option) any later version.
%
% This program is distributed in the hope that it will be useful,
% but WITHOUT ANY WARRANTY; without even the implied warranty of
% MERCHANTABILITY or FITNESS FOR A PARTICULAR PURPOSE.  See the
% GNU General Public License for more details.
%
% You should have received a copy of the GNU General Public License
% along with this program.  If not, see <http://www.gnu.org/licenses/>.
%
% -----------------------------------------------------------------------------
%


%
% Results
% -------
% All results, measurements, solutions to assignments and whatnot go here. This
% is the main part of the document.
%
\section{Results}
% results.tex
% -----------
% Includes all assignments. Every assignment should go into a different file.
%
% Example:
%
%   \subsection{Task A}
%   \input{src/tasks/taska.tex}
%
%   \subsection{Task B}
%   \input{src/tasks/taskb.tex}
%
% The folder tasks does _not_ exist by default.
%
% -----------------------------------------------------------------------------
%
% Copyright (C) 2013  Claudio Vellage (vellage@ieee.org)
%
% This program is free software: you can redistribute it and/or modify
% it under the terms of the GNU General Public License as published by
% the Free Software Foundation, either version 3 of the License, or
% (at your option) any later version.
%
% This program is distributed in the hope that it will be useful,
% but WITHOUT ANY WARRANTY; without even the implied warranty of
% MERCHANTABILITY or FITNESS FOR A PARTICULAR PURPOSE.  See the
% GNU General Public License for more details.
%
% You should have received a copy of the GNU General Public License
% along with this program.  If not, see <http://www.gnu.org/licenses/>.
%
% -----------------------------------------------------------------------------
%


%
% Discussion
% ----------
% Contains a discussion of the results, errors and applications to real world
% situations.
%
\section{Discussion}
% discussion.tex
% --------------
% Includes discussion of results
%
% -----------------------------------------------------------------------------
%
% Copyright (C) 2014  Claudio Vellage (claudio.vellage@latex-tutorial.com)
%
% This program is free software: you can redistribute it and/or modify
% it under the terms of the GNU General Public License as published by
% the Free Software Foundation, either version 3 of the License, or
% (at your option) any later version.
%
% This program is distributed in the hope that it will be useful,
% but WITHOUT ANY WARRANTY; without even the implied warranty of
% MERCHANTABILITY or FITNESS FOR A PARTICULAR PURPOSE.  See the
% GNU General Public License for more details.
%
% You should have received a copy of the GNU General Public License
% along with this program.  If not, see <http://www.gnu.org/licenses/>.
%
% -----------------------------------------------------------------------------
%

\newpage

%
% Bibliography
% ------------
% All cited books and other materials are stored in main.bib and imported here.
% Refer to the BiBTeX documentation to get an idea how this works.
% 
\bibliography{main}
\bibliographystyle{ieeetr}
\newpage

%
% Appendix
% --------
% Listings, tables, figures and statement go here. Especially when too large.
%
\begin{appendix}
  %
  % Listings
  % --------
  %
  \section{Listings} 
  % listings.tex
% ------------
% Contains listings
%
% -----------------------------------------------------------------------------
%
% Copyright (C) 2013  Claudio Vellage (vellage@ieee.org)
%
% This program is free software: you can redistribute it and/or modify
% it under the terms of the GNU General Public License as published by
% the Free Software Foundation, either version 3 of the License, or
% (at your option) any later version.
%
% This program is distributed in the hope that it will be useful,
% but WITHOUT ANY WARRANTY; without even the implied warranty of
% MERCHANTABILITY or FITNESS FOR A PARTICULAR PURPOSE.  See the
% GNU General Public License for more details.
%
% You should have received a copy of the GNU General Public License
% along with this program.  If not, see <http://www.gnu.org/licenses/>.
%
% -----------------------------------------------------------------------------
%

  
  %
  % Tables
  % ------
  %
  \section{Tables}
  % tabellen.tex
% ------------
% Contains tables
%
% -----------------------------------------------------------------------------
%
% Copyright (C) 2013  Claudio Vellage (vellage@ieee.org)
%
% This program is free software: you can redistribute it and/or modify
% it under the terms of the GNU General Public License as published by
% the Free Software Foundation, either version 3 of the License, or
% (at your option) any later version.
%
% This program is distributed in the hope that it will be useful,
% but WITHOUT ANY WARRANTY; without even the implied warranty of
% MERCHANTABILITY or FITNESS FOR A PARTICULAR PURPOSE.  See the
% GNU General Public License for more details.
%
% You should have received a copy of the GNU General Public License
% along with this program.  If not, see <http://www.gnu.org/licenses/>.
%
% -----------------------------------------------------------------------------
%


  %
  % Figures
  % -------
  %
  \section{Figures}
  % abbildungen.tex
% ---------------
% Contains figures
%
% -----------------------------------------------------------------------------
%
% Copyright (C) 2013  Claudio Vellage (vellage@ieee.org)
%
% This program is free software: you can redistribute it and/or modify
% it under the terms of the GNU General Public License as published by
% the Free Software Foundation, either version 3 of the License, or
% (at your option) any later version.
%
% This program is distributed in the hope that it will be useful,
% but WITHOUT ANY WARRANTY; without even the implied warranty of
% MERCHANTABILITY or FITNESS FOR A PARTICULAR PURPOSE.  See the
% GNU General Public License for more details.
%
% You should have received a copy of the GNU General Public License
% along with this program.  If not, see <http://www.gnu.org/licenses/>.
%
% -----------------------------------------------------------------------------
%


  %
  % Statement
  % ---------
  % We state that we created this document ourselves and only using the listed
  % sources.
  %
  %\section{Statement}
  %% erklaerung.tex
% --------------
% Contains the statement. The names of participants etc. should be set in
% the header file (include/header.tex).
%
% -----------------------------------------------------------------------------
%
% Copyright (C) 2013  Claudio Vellage (vellage@ieee.org)
%
% This program is free software: you can redistribute it and/or modify
% it under the terms of the GNU General Public License as published by
% the Free Software Foundation, either version 3 of the License, or
% (at your option) any later version.
%
% This program is distributed in the hope that it will be useful,
% but WITHOUT ANY WARRANTY; without even the implied warranty of
% MERCHANTABILITY or FITNESS FOR A PARTICULAR PURPOSE.  See the
% GNU General Public License for more details.
%
% You should have received a copy of the GNU General Public License
% along with this program.  If not, see <http://www.gnu.org/licenses/>.
%
% -----------------------------------------------------------------------------
%
\vspace*{3cm}


\normalsize Hiermit erkl\"aren die Verfasser, dass die vorstehende Arbeit nur mit Hilfe der eigenen Erkenntnisse und unter Verwendung der angegebenen Quellen erstellt wurde.
\vspace*{4cm}\\

\abgabeOrt{}, \abgabeDatum{}
\vspace*{2cm}

\dotfill\\
\emph{\versuchsTeilnehmer{}}


\end{appendix}
\end{document}
