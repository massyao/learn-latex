% --------------------------------
% -------  chapter 6  ------------
% --------------------------------

\maketitle
\newpage
\section{Footnotes}
  \textbf{
    Learn how to create footnotes in LaTeX and how to refer to them, using the builtin commands footnote, label and ref.
  }
  % \begin{enumerate} % list_type 有 enumerate、 itemize 和 description
  %   \item Creating a .bib file
  %   \item Using BibTeX
  %   \item Autogenerate footnotes with BibLaTeX
  %   \item BibTeX Format
  %   \item BibTeX Styles
  % \end{enumerate} 
  One of the benifits of using LaTeX is how easy footnotes can be added and referred to. So how to use footnotes? LaTeX offers the \textbackslash footnote command and referencing works using the \textbackslash label and \textbackslash ref commands.

  \paragraph{}
  The following code shows some example text and how to add a footnote with a label:
  \paragraph{}
  I'm referring to footnote \ref{myfootnote}.
  %  lstlisting verbatim minted
  \begin{lstlisting}[language={[LaTeX]TeX},breaklines=true,frame=single]
    ...
    I'm referring to footnote \ref{myfootnote}.
    ...
  \end{lstlisting}

  \paragraph{}
  The following code shows some example text and how to add a footnote with a label:
  \paragraph{}
  This is some example text\footnote{\label{myfootnote}Hello footnote}.
  %  lstlisting verbatim minted
  \begin{lstlisting}[language={[LaTeX]TeX},breaklines=true,frame=single]
  ...
  This is some example text\footnote{\label{myfootnote}Hello footnote}.
  ...
  \end{lstlisting}


  \paragraph{Summary}
    % list
    \begin{itemize} % list_type 有 enumerate、 itemize 和 description
      \item Create footnotes with the \textbackslash footnote command and label them with \textbackslash label
      \item Make sure that the label is contained within the braces of the footnote command
      \item Use the \textbackslash ref command to refer to footnotes
    \end{itemize} 

  \maketitle
  \newpage

