% --------------------------------
% -------  chapter 7  ------------
% --------------------------------

\maketitle
\newpage
\section{Tables}
  \textbf{
    Learn to create tables in LaTeX including all features such as multi row, multi column, multi page and landscape tables. All in one place.
  }
  \begin{enumerate} % list_type 有 enumerate、 itemize 和 description
    \item Your first table / table template
    \item Align numbers at decimal point
    \item Adding rows and columns
    \item Cells spanning multiple rows and multiple columns
      \begin{enumerate}
        \item Using multirow
        \item Using multicolumn
        \item Combining multirow and multicolumn
      \end{enumerate}
    \item Prettier tables with booktabs
    \item Tables spanning multiple pages
    \item Landscape / sideways tables
    \item Tables from Excel (.csv) to LaTeX
  \end{enumerate} 
  In this tutorial we're going to learn how to use the table and tabular environments to create tables in LaTeX. At first we're going to create a simple table like this:
  
  \subsection{Your first table}
  \paragraph{}
    Tables in LaTeX can be created through a combination 
    of the table environment and the tabular environment. 
    The table environment part contains the caption and 
    defines the float for our table, i.e. where in our 
    document the table should be positioned and whether 
    we want it to be displayed centered. The
    \textbackslash caption and \textbackslash label commands 
    can be used in the same way as for pictures. 
    The actual content of the table is contained 
    within the tabular environment.
  
  \paragraph{}
  Now let's take a look at some actual code for a basic table, which you can easily copy-and-paste into your document and modify it to your needs.

  \begin{lstlisting}[language={[LaTeX]TeX},breaklines=true,frame=single]
    \documentclass{article}

    \begin{document}
    
    \begin{table}[h!]
      \begin{center}
        \caption{Your first table.}
        \label{tab:table1}
        \begin{tabular}{l|c|r} % <-- Alignments: 1st column left, 2nd middle and 3rd right, with vertical lines in between
          \textbf{Value 1} & \textbf{Value 2} & \textbf{Value 3}\\
          $\alpha$ & $\beta$ & $\gamma$ \\
          \hline
          1 & 1110.1 & a\\
          2 & 10.1 & b\\
          3 & 23.113231 & c\\
        \end{tabular}
      \end{center}
    \end{table}
    
    \end{document}
  \end{lstlisting}

  \paragraph{}
  The above code will print out the table which I've already shown you in the introduction and it looks like this:

  \begin{table}[h!]
    \begin{center}
      \caption{First table.}
      \label{tab:table1}
      \begin{tabular}{l|c|r} % <-- Alignments: 1st column left, 2nd middle and 3rd right, with vertical lines in between
        \textbf{Value 1} & \textbf{Value 2} & \textbf{Value 3}\\
        $\alpha$ & $\beta$ & $\gamma$ \\
        \hline
        1 & 1110.1 & a\\
        2 & 10.1 & b\\
        3 & 23.113231 & c\\
      \end{tabular}
    \end{center}
  \end{table}
  \paragraph{}
  While this table already works, it's not very satisfying and readable that the numbers in the center column are not aligned at the decimal point. Fortunately, we don't have to add spacing somehow manually, but we can use the siunitx package for this purpose.
  
  \subsection{Align numbers at decimal point}
  \paragraph{}
    The first thing we have to do is to include the siunitx
    package in our preamble and use the command 
    \textbackslash sisetup to tell the package
    how many digital places it should display:

   
  \begin{lstlisting}[language={[LaTeX]TeX},breaklines=true,frame=single]
    %...

    \usepackage{siunitx} % Required for alignment
    
    \sisetup{
      round-mode          = places, % Rounds numbers
      round-precision     = 2, % to 2 places
    }
    
    \begin{document} 
    
    %...
  \end{lstlisting}


  \paragraph{}
  Afterwards we can use a new alignment setting in our tables, so besides left (l), center (c) and right (r), there's now an additional setting S, which will align the numbers automatically. In our previous table, there was an alignment problem with the middle column, so I've now changed the alignment setting of the middle column from (c) to (S):
  \begin{lstlisting}[language={[LaTeX]TeX},breaklines=true,frame=single]
    %...

    \begin{table}[h!]
      \begin{center}
        \caption{Table with aligned units.}
        \label{tab:table1}
        \begin{tabular}{l|S|r} % <-- Changed to S here.
          \textbf{Value 1} & \textbf{Value 2} & \textbf{Value 3}\\
          $\alpha$ & $\beta$ & $\gamma$ \\
          \hline
          1 & 1110.1 & a\\
          2 & 10.1 & b\\
          3 & 23.113231 & c\\
        \end{tabular}
      \end{center}
    \end{table}
    
    %...
  \end{lstlisting}

  % \usepackage{siunitx} % Required for alignment
  \sisetup{
    round-mode          = places, % Rounds numbers
    round-precision     = 2, % to 2 places
  }

  \paragraph{}
  We can now observe, that LaTeX will now properly align the numbers at their decimal points and round the numbers to two decimal places:
  \begin{table}[h!]
    \begin{center}
      \caption{Table with aligned units.}
      \label{tab:table2}
      \begin{tabular}{l|S|r} % \begin{tabular}{l|S|r} <-- Changed to S here.
        \textbf{Value 1} & \textbf{Value 2} & \textbf{Value 3}\\
        $\alpha$ & $\beta$ & $\gamma$ \\
        \hline
        1 & 1110.1 & a\\
        2 & 10.1 & b\\
        3 & 23.113231 & c\\
      \end{tabular}
    \end{center}
  \end{table}

  \subsection{Adding rows and columns}
  \paragraph{}
    Now that we've setup our table properly, we can focus 
    on adding more rows and columns. As I've mentioned before,
    LaTeX uses column separators (\&) and row separators 
    (\textbackslash \textbackslash ) to layout the cells 
    of our table. For the 5x3 table shown above we can 
    count five times (\textbackslash \textbackslash ) 
    behind each row and two times (\&) per row, 
    separating the content of three columns.
  \paragraph{}
  If we now want to add an additional column, it's as simple as copy and pasting the previous column and changing the contents. I will be reusing the table from above for this example and add an additional column:
  
  
   
  \begin{lstlisting}[language={[LaTeX]TeX},breaklines=true,frame=single]
    %...

    \begin{table}[h!]
      \begin{center}
        \caption{More rows.}
        \label{tab:table1}
        \begin{tabular}{l|S|r}
          \textbf{Value 1} & \textbf{Value 2} & \textbf{Value 3}\\
          $\alpha$ & $\beta$ & $\gamma$ \\
          \hline
          1 & 1110.1 & a\\
          2 & 10.1 & b\\
          3 & 23.113231 & c\\
          4 & 25.113231 & d\\ % <-- added row here
        \end{tabular}
      \end{center}
    \end{table}
    
    %...
  \end{lstlisting}
  \paragraph{}
  This will generate the following output:
  \begin{table}[h!]
    \begin{center}
      \caption{More rows.}
      \label{tab:table1}
      \begin{tabular}{l|S|r}
        \textbf{Value 1} & \textbf{Value 2} & \textbf{Value 3}\\
        $\alpha$ & $\beta$ & $\gamma$ \\
        \hline
        1 & 1110.1 & a\\
        2 & 10.1 & b\\
        3 & 23.113231 & c\\
        4 & 25.113231 & d\\ % <-- added row here
      \end{tabular}
    \end{center}
  \end{table}
  \paragraph{}
  Adding an additional column is also possible, but you have to be careful, because you have to add a column separator (\&) to every column:
  \begin{lstlisting}[language={[LaTeX]TeX},breaklines=true,frame=single]
    %...

    \begin{table}[h!]
      \begin{center}
        \caption{More columns.}
        \label{tab:table1}
        \begin{tabular}{l|S|r|l}
          \textbf{Value 1} & \textbf{Value 2} & \textbf{Value 3} & \textbf{Value 4}\\ % <-- added & and content for each column
          $\alpha$ & $\beta$ & $\gamma$ & $\delta$ \\ % <--
          \hline
          1 & 1110.1 & a & e\\ % <--
          2 & 10.1 & b & f\\ % <--
          3 & 23.113231 & c & g\\ % <--
        \end{tabular}
      \end{center}
    \end{table}
    
    %...
  \end{lstlisting}
  \paragraph{}
  We will now see an additional column in our output:
  \begin{table}[h!]
    \begin{center}
      \caption{More columns.}
      \label{tab:table1}
      \begin{tabular}{l|S|r|l}
        \textbf{Value 1} & \textbf{Value 2} & \textbf{Value 3} & \textbf{Value 4}\\ % <-- added & and content for each column
        $\alpha$ & $\beta$ & $\gamma$ & $\delta$ \\ % <--
        \hline
        1 & 1110.1 & a & e\\ % <--
        2 & 10.1 & b & f\\ % <--
        3 & 23.113231 & c & g\\ % <--
      \end{tabular}
    \end{center}
  \end{table}

  \subsection{Cells spanning multiple rows or multiple columns}
    \begin{enumerate}[label=\alph*)]
      \item Using multirow
      \item Using multicolumn
      \item Combining multirow and multicolumn
    \end{enumerate}
  \paragraph{}
  Sometimes it's necessary to make a row span several cells. For this purpose we can use the multirow package, so the first thing we're going to do is adding the required package to our preamble:

  \begin{lstlisting}[language={[LaTeX]TeX},breaklines=true,frame=single]
    %...

    \usepackage{multirow} % Required for multirows
    
    \begin{document} 
    
    %...
  \end{lstlisting}
  \paragraph{}
  We can now use multirow and multicolumn environments, which allow us to conveniently span multiple rows or columns.
  


  \subsubsection{Using multirow}
  \paragraph{}
  In order for a cell to span multiple rows, we have to use the multirow command. This command accepts three parameters:
   
  \begin{lstlisting}[language={[LaTeX]TeX},breaklines=true,frame=single]
    \multirow{NUMBER_OF_ROWS}{WIDTH}{CONTENT}
  \end{lstlisting}

  \paragraph{}
  % textsuperscript
  I usually use an asterisk (\text{*}) as a parameter for the width,
  since this basically means, that the width should be determined
  automatically.

  \paragraph{}
  Because we're combining two rows in our example, it's necessary to omit the content of the same row in the following line. Let's look at how the actual LaTeX code would look like:

  \begin{lstlisting}[language={[LaTeX]TeX},breaklines=true,frame=single]
    %...

    \begin{table}[h!]
      \begin{center}
        \caption{Multirow table.}
        \label{tab:table1}
        \begin{tabular}{l|S|r}
          \textbf{Value 1} & \textbf{Value 2} & \textbf{Value 3}\\
          $\alpha$ & $\beta$ & $\gamma$ \\
          \hline
          \multirow{2}{*}{12} & 1110.1 & a\\ % <-- Combining 2 rows with arbitrary with (*) and content 12
          & 10.1 & b\\ % <-- Content of first column omitted.
          \hline
          3 & 23.113231 & c\\
          4 & 25.113231 & d\\
        \end{tabular}
      \end{center}
    \end{table}
    
    %...
  \end{lstlisting}
  \paragraph{}
  The modified table looks like this:
  \begin{table}[h!]
    \begin{center}
      \caption{Multirow table.}
      \label{tab:table1}
      \begin{tabular}{l|S|r}
        \textbf{Value 1} & \textbf{Value 2} & \textbf{Value 3}\\
        $\alpha$ & $\beta$ & $\gamma$ \\
        \hline
        \multirow{2}{*}{12} & 1110.1 & a\\ % <-- Combining 2 rows with arbitrary with (*) and content 12
        & 10.1 & b\\ % <-- Content of first column omitted.
        \hline
        3 & 23.113231 & c\\
        4 & 25.113231 & d\\
      \end{tabular}
    \end{center}
  \end{table}
  \paragraph{}
  You can now see, that the cell containing 12 spans two rows.


  \subsubsection{Using multicolumn}
  \paragraph{}
  If we want a cell to span multiple columns, we have to use the multicolumn command. The usage differs a bit from multirow command, since we also have to specifiy the alignment for our column. The command also requires three parameters:
  \begin{lstlisting}[language={[LaTeX]TeX},breaklines=true,frame=single]
    \multicolumn{NUMBER_OF_COLUMNS}{ALIGNMENT}{CONTENT}
  \end{lstlisting}
  \paragraph{}
  In our example, we will again combine two neighboring cells, note that in the row where we're using multicolumn to span two columns, there's only one column separator (\&) (instead of two for all other rows):
  \begin{lstlisting}[language={[LaTeX]TeX},breaklines=true,frame=single]
    %...

    \begin{table}[h!]
      \begin{center}
        \caption{Multicolumn table.}
        \label{tab:table1}
        \begin{tabular}{l|S|r}
          \textbf{Value 1} & \textbf{Value 2} & \textbf{Value 3}\\
          $\alpha$ & $\beta$ & $\gamma$ \\
          \hline
          \multicolumn{2}{c|}{12} & a\\ % <-- Combining two cells with alignment c| and content 12.
          \hline
          2 & 10.1 & b\\
          3 & 23.113231 & c\\
          4 & 25.113231 & d\\
        \end{tabular}
      \end{center}
    \end{table}
    
    %...
  \end{lstlisting}
  \paragraph{}
  This will result in the following content:
  \begin{table}[h!]
    \begin{center}
      \caption{Multicolumn table.}
      \label{tab:table1}
      \begin{tabular}{l|S|r}
        \textbf{Value 1} & \textbf{Value 2} & \textbf{Value 3}\\
        $\alpha$ & $\beta$ & $\gamma$ \\
        \hline
        \multicolumn{2}{c|}{12} & a\\ % <-- Combining two cells with alignment c| and content 12.
        \hline
        2 & 10.1 & b\\
        3 & 23.113231 & c\\
        4 & 25.113231 & d\\
      \end{tabular}
    \end{center}
  \end{table}


  \subsubsection{Combining multirow and multicolumn}
  \paragraph{}
  Of course it's also possible to combine the two features, to make a cell spanning multiple rows and columns. To do this, we simply use the multicolumn command and instead of specifying content, we add a multirow command as the content. We then have to add another multicolumn statement for as many rows as we're combining.
  \paragraph{}
  Because this is a little hard to explain, it will be much clearer when looking at the code. In this example, we're going to combine two columns and two rows, so we're getting a cell spanning a total of four cells:
  \begin{lstlisting}[language={[LaTeX]TeX},breaklines=true,frame=single]
    %...

    \begin{table}[h!]
      \begin{center}
        \caption{Multirow and -column table.}
        \label{tab:table1}
        \begin{tabular}{l|S|r}
          \textbf{Value 1} & \textbf{Value 2} & \textbf{Value 3}\\
          $\alpha$ & $\beta$ & $\gamma$ \\
          \hline
          \multicolumn{2}{c|}{\multirow{2}{*}{1234}} & a\\ % <-- Multicolumn spanning 2 columns, content multirow spanning two rows
          \multicolumn{2}{c|}{} & b\\ % <-- Multicolumn spanning 2 columns with empty content as placeholder
          \hline
          3 & 23.113231 & c\\
          4 & 25.113231 & d\\
        \end{tabular}
      \end{center}
    \end{table}
    
    %...
  \end{lstlisting}
  \paragraph{}
  Our document will now contain a table, with a huge cell:
  \begin{table}[h!]
    \begin{center}
      \caption{Multirow and -column table.}
      \label{tab:table1}
      \begin{tabular}{l|S|r}
        \textbf{Value 1} & \textbf{Value 2} & \textbf{Value 3}\\
        $\alpha$ & $\beta$ & $\gamma$ \\
        \hline
        \multicolumn{2}{c|}{\multirow{2}{*}{1234}} & a\\ % <-- Multicolumn spanning 2 columns, content multirow spanning two rows
        \multicolumn{2}{c|}{} & b\\ % <-- Multicolumn spanning 2 columns with empty content as placeholder
        \hline
        3 & 23.113231 & c\\
        4 & 25.113231 & d\\
      \end{tabular}
    \end{center}
  \end{table}



  \subsection{Prettier tables with booktabs}
  \paragraph{}
  Of course beauty is always in the eye of the beholder, but I personally think, that the default hlines used by the table environment are not very pretty. For my tables, i always use the booktabs package, which provides much prettier horizontal separators and the usage is not harder compared to simply using hlines.
  \paragraph{}
  Again, we have to add the according booktabs package to our preamble:
  \begin{lstlisting}[language={[LaTeX]TeX},breaklines=true,frame=single]
    %...

    \usepackage{booktabs} % For prettier tables
    
    \begin{document} 
    
    %...
  \end{lstlisting}
  \paragraph{}
  We can now replace the hlines in our example table with toprule, midrule and bottomrule provided by the booktabs package:
  \begin{lstlisting}[language={[LaTeX]TeX},breaklines=true,frame=single]
    %...

    \begin{table}[h!]
      \begin{center}
        \caption{Table using booktabs.}
        \label{tab:table1}
        \begin{tabular}{l|S|r}
          \toprule % <-- Toprule here
          \textbf{Value 1} & \textbf{Value 2} & \textbf{Value 3}\\
          $\alpha$ & $\beta$ & $\gamma$ \\
          \midrule % <-- Midrule here
          1 & 1110.1 & a\\
          2 & 10.1 & b\\
          3 & 23.113231 & c\\
          \bottomrule % <-- Bottomrule here
        \end{tabular}
      \end{center}
    \end{table}
    
    %...
  \end{lstlisting}
  \paragraph{}
  You can decide for yourself, if you prefer the hlines or the following output:
  \begin{table}[h!]
    \begin{center}
      \caption{Table using booktabs.}
      \label{tab:table1}
      \begin{tabular}{l|S|r}
        \toprule % <-- Toprule here
        \textbf{Value 1} & \textbf{Value 2} & \textbf{Value 3}\\
        $\alpha$ & $\beta$ & $\gamma$ \\
        \midrule % <-- Midrule here
        1 & 1110.1 & a\\
        2 & 10.1 & b\\
        3 & 23.113231 & c\\
        \bottomrule % <-- Bottomrule here
      \end{tabular}
    \end{center}
  \end{table}


  \subsection{Multipage tables}
  \paragraph{}
  If you have a lot of rows in your table, you will notice that by default, the table will be cropped at the bottom of the page, which is certainly not what you want. The package longtable provides a convenient way, to make tables span multiple pages. Of course we have to add the package to our preamble before we can start using it:
  \begin{lstlisting}[language={[LaTeX]TeX},breaklines=true,frame=single]
    %...

    \usepackage{longtable} % To display tables on several pages
    
    \begin{document} 
    
    %...
  \end{lstlisting}
  \paragraph{}
  It's actually not harder, but easier to use than the previous code for tables. I will first show you what the code looks like and than explain the differences between longtable and tabular, in case they're not obvious.
  \begin{lstlisting}[language={[LaTeX]TeX},breaklines=true,frame=single]
    %...

    \begin{longtable}[c]{l|S|r} % <-- Replaces \begin{table}, alignment must be specified here (no more tabular)
      \caption{Multipage table.}
      \label{tab:table1}\\
      \toprule
      \textbf{Value 1} & \textbf{Value 2} & \textbf{Value 3}\\
      $\alpha$ & $\beta$ & $\gamma$ \\
      \midrule
      \endfirsthead % <-- This denotes the end of the header, which will be shown on the first page only
      \toprule
      \textbf{Value 1} & \textbf{Value 2} & \textbf{Value 3}\\
      $\alpha$ & $\beta$ & $\gamma$ \\
      \midrule
      \endhead % <-- Everything between \endfirsthead and \endhead will be shown as a header on every page
      1 & 1110.1 & a\\
      2 & 10.1 & b\\
      % ...
      % ... Many rows in between
      % ...
      3 & 23.113231 & c\\
      \bottomrule
    \end{longtable}
    
    %...
  \end{lstlisting}
  \paragraph{}
  In the previous examples, we've always used the table 
  and tabular environments. The longtable environment 
  replaces both of them or rather combines both of them 
  into a single environment. We now use 
  \begin{verbatim} \begin{longtable}[POSITION_ON_PAGE]{ALIGNMENT}  \end{verbatim} 
  as an environment for our tables. Using this environment, 
  we create a table, that is automatically split between 
  pages, if it has too many rows.
  \paragraph{}
  The content on the table on the first page looks like this:
  \begin{longtable}[c]{l|S|r} % <-- Replaces \begin{table}, alignment must be specified here (no more tabular)
    \caption{Multipage table.}
    \label{tab:table1}\\
    \toprule
    \textbf{Value 1} & \textbf{Value 2} & \textbf{Value 3}\\
    $\alpha$ & $\beta$ & $\gamma$ \\
    \midrule
    \endfirsthead % <-- This denotes the end of the header, which will be shown on the first page only
    \toprule
    \textbf{Value 1} & \textbf{Value 2} & \textbf{Value 3}\\
    $\alpha$ & $\beta$ & $\gamma$ \\
    \midrule
    \endhead % <-- Everything between \endfirsthead and \endhead will be shown as a header on every page
    1 & 1110.1 & a\\
    2 & 10.1 & b\\
    % ...
    % ... Many rows in between
    % ...
    3 & 23.113231 & c\\
    \bottomrule
  \end{longtable}
  \paragraph{}
  Imagine that this table has many rows (...) being a placeholder for more rows and that the table continues the following page like this:
  \paragraph{}
  Note that there's again a header on this page, but without the caption. This is because I've added the commands \textbackslash endfirsthead and \textbackslash endhead to my table, where everything written before \textbackslash endfirsthead declares the header for the first page the table occurs on and everything between \textbackslash endfirsthead and \textbackslash endhead denotes the header, which should be repeated on every following page. If you don't want a header on the following pages, you can simply remove everything in between \textbackslash endfirsthead and \textbackslash endhead including the \textbackslash endhead command.

  \subsection{Landscape tables}
  Now that we have a solution for too many rows, we could also be facing the same problem if we had too many columns. If we add too many columns, we might be getting a table that's too wide for the page. In this situation, it's often best to simply rotate the table and print it in sideways. While there are many different ways to rotate the table, the only that I've found to be satisfying was using the rotating package.
  \paragraph{}
  First we include the required package in our preamble:
  \begin{lstlisting}[language={[LaTeX]TeX},breaklines=true,frame=single]
    %...

    \usepackage{rotating} % To display tables in landscape
    
    \begin{document} 
    
    %...
  \end{lstlisting}
  \paragraph{}
  This package provides the sidewaystable environment, which is very easy to use. Just replace the table environment with the sidewaystable environment like this:
  \begin{lstlisting}[language={[LaTeX]TeX},breaklines=true,frame=single]
    %...

    \begin{sidewaystable}[h!] % <--
      \begin{center}
      \caption{Landscape table.}
      \label{tab:table1}
      \begin{tabular}{l|S|r}
        \toprule
        \textbf{Value 1} & \textbf{Value 2} & \textbf{Value 3}\\
        $\alpha$ & $\beta$ & $\gamma$ \\
        \midrule
        1 & 1110.1 & a\\
        2 & 10.1 & b\\
        3 & 23.113231 & c\\
        \bottomrule
      \end{tabular}
      \end{center}
    \end{sidewaystable}
    
    %...
  \end{lstlisting}
  \paragraph{}
  This will automatically rotate the table for us, so it can be read when flipping the page sideways:
  \begin{sidewaystable}[h!] % <--
    \begin{center}
    \caption{Landscape table.}
    \label{tab:table1}
    \begin{tabular}{l|S|r}
      \toprule
      \textbf{Value 1} & \textbf{Value 2} & \textbf{Value 3}\\
      $\alpha$ & $\beta$ & $\gamma$ \\
      \midrule
      1 & 1110.1 & a\\
      2 & 10.1 & b\\
      3 & 23.113231 & c\\
      \bottomrule
    \end{tabular}
    \end{center}
  \end{sidewaystable}



  \subsection{Tables from Excel (.csv) to LaTeX}
  There are two disadvantages of writing tables by hand as described in this tutorial. While it works for small tables similar to the one in our example, it can take a long time to enter a large amount of data by hand. Most of the time the data will be collected in form of a spreadsheet and we don't want to enter the data twice. Furthermore once put into LaTeX tables, the data can not be plotted anymore and is not in a useful form in general. For this reason, the next lesson shows you, how to generate tables from .csv files (which can be exported from Excel and other tools) automatically.


  \paragraph{Summary}
    % list
    \begin{itemize} % list_type 有 enumerate、 itemize 和 description
      \item LaTeX offers the table and tabular environment for table creation
      \item The table environment acts like a wrapper for the tabular similar to the figure environment
      \item Alignment and vertical separators are passed as an argument to the tabular environment 
            \begin{verbatim}(e.g. \begin{tabular}{l|c||r}) \end{verbatim}
      \item It's possible to align the content left (l), centered (c) and right (r), where the number of alignment operators has to match the desired number of columns
      \item The columns can be seperated by adding | in between the alignment operators
      \item Rows can be seperated using the \textbackslash hline command and columns using the ampersand \& symbol
      \item The newline \textbackslash \textbackslash \enspace operator indicates the end of a row       
      \item It's possible to refer to tables using \textbackslash ref and \textbackslash label
      \item Align numbers at the decimal point using the siunitx package
      \item Combine multiple rows and columns with the multirow package
      \item Prettify your tables using the booktabs package
      \item Make your tables span multiple pages with the longtable package
      \item Display your tables in landscape using the rotating package
    \end{itemize} 