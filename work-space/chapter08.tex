% --------------------------------
% -------  chapter 8  ------------
% --------------------------------

\maketitle
\newpage
\section{Tables from .csv}
  \textbf{
    Use the package pgfplotstable to read tables from .csv files automatically and add them to your document.
  }
  \begin{enumerate} % list_type 有 enumerate、 itemize 和 description
    \item Syntax and usage of pgfplotstable
    \item Adding new columns
    \item Layout of a .csv file
  \end{enumerate} 
  In this tutorial we're going to learn how to use the table and tabular environments to create tables in LaTeX. At first we're going to create a simple table like this:
  
  \subsection{Syntax and usage of pgfplotstable}
  \paragraph{}
  When writing papers, it is sometimes necessary to present a large amount of data in tables. Writing such tables in LaTeX by hand is a very time-consuming and error-prone task. To avoid this, we can simply import the data directly from .csv (comma-separated value) files. Programs such as Excel, OpenOffice Calc or even emacs org-mode can export data sheets as .csv files. LaTeX can not work with them directly, but we can use the following code, to generate tables from .csv files:
  
  \begin{lstlisting}[language={[LaTeX]TeX},breaklines=true,frame=single]
    \documentclass{article}

    \usepackage{booktabs} % For \toprule, \midrule and \bottomrule
    \usepackage{siunitx} % Formats the units and values
    \usepackage{pgfplotstable} % Generates table from .csv
    
    % Setup siunitx:
    \sisetup{
      round-mode          = places, % Rounds numbers
      round-precision     = 2, % to 2 places
    }
    
    \begin{document}
    
    \begin{table}[h!]
      \begin{center}
        \caption{Autogenerated table from .csv file.}
        \label{table1}
        \pgfplotstabletypeset[
          multicolumn names, % allows to have multicolumn names
          col sep=comma, % the seperator in our .csv file
          display columns/0/.style={
        column name=$Value 1$, % name of first column
        column type={S},string type},  % use siunitx for formatting
          display columns/1/.style={
        column name=$Value 2$,
        column type={S},string type},
          every head row/.style={
        before row={\toprule}, % have a rule at top
        after row={
          \si{\ampere} & \si{\volt}\\ % the units seperated by &
          \midrule} % rule under units
          },
        every last row/.style={after row=\bottomrule}, % rule at bottom
        ]{table.csv} % filename/path to file
      \end{center}
    \end{table}
    
    \end{document}
  \end{lstlisting}

  \paragraph{}
  This will generate the following output:
  % Setup siunitx:
  \sisetup{
    round-mode          = places, % Rounds numbers
    round-precision     = 2, % to 2 places
  }

  \begin{table}[h!]
    \begin{center}
      \caption{Autogenerated table from .csv file.}
      \label{table1}
      \pgfplotstabletypeset[
        multicolumn names, % allows to have multicolumn names
        col sep=comma, % the seperator in our .csv file
        display columns/0/.style={
          column name=test1, % name of first column
          column type={S},string type},  % use siunitx for formatting
        display columns/1/.style={
          column name=$Value 2$,
          column type={S},string type},
        every head row/.style={
          before row={\toprule}, % have a rule at top
          after row={
            \si{\ampere} & \si{\volt}\\ % the units seperated by &
            \midrule} % rule under units
            },
        every last row/.style={after row=\bottomrule}, % rule at bottom
      ]{resources/table.csv} % filename/path to file
    \end{center}
  \end{table}
  \paragraph{}
  This is a fairly long snippet just to include a table, but don't worry, we can just reuse it and don't have to code it every time again. We only have to change this part:
  \begin{lstlisting}[language={[LaTeX]TeX},breaklines=true,frame=single]
    ...
    display columns/0/.style={
       column name=$Value 1$,
       column type={S},string type},
         display columns/1/.style={
       column name=$Value 2$,
       column type={S},string type},
         every head row/.style={
       before row={\toprule},
       after row={
         \si{\ampere} & \si{\volt}\\
         \midrule}
         },
       every last row/.style={after row=\bottomrule},
       ]{table.csv} % filename/path to file
   ...
  \end{lstlisting}

  \subsection{Adding new columns}
  \paragraph{}
  The part that controls our column names and formatting is:
  \begin{lstlisting}[language={[LaTeX]TeX},breaklines=true,frame=single]
    ...
    display columns/1/.style={
      column name=$Value 2$,
      column type={S},string type},
    ...
  \end{lstlisting}
  \paragraph{}
  The display columns/NUM/ lets us choose the name for a certain column. Note that numbering starts at zero. In order to add a new column, we simply copy the whole block and change the number to:
  \begin{lstlisting}[language={[LaTeX]TeX},breaklines=true,frame=single]
    ...
      after row={
        \si{\ampere} & \si{\volt}\\
        \midrule}
        },
    ...
  \end{lstlisting}
  \paragraph{}
  We just put a new ampersand \& and use the siunitx package
  to choose a new unit like so:
  \begin{lstlisting}[language={[LaTeX]TeX},breaklines=true,frame=single]
    ...
      after row={
        \si{\ampere} & \si{\volt} & \si{\tesla}\\
        \midrule}
        },
    ...
  \end{lstlisting}
  \paragraph{}
  In order for this to work, we must make sure that we export the .csv file in the correct format. It must have comma as the column seperator and newline as the row seperator. We can also change the respective seperators in our code above, but i recomment to use this layout to avoid problems. The raw .csv datafile for this example looks like:

  \subsection{The layout of a .csv file} 
  \begin{lstlisting}[language={[LaTeX]TeX},breaklines=true,frame=single]
    column 1,column 2
    1,2
    11.432,2342.23123123
  \end{lstlisting}
  \paragraph{}
  As you can see, the numbers contained are actually longer than what shows up in the table. This is because we have:
  \begin{lstlisting}[language={[LaTeX]TeX},breaklines=true,frame=single]
    \sisetup{
      round-mode          = places, % Rounds numbers
      round-precision     = 2, % to 2 places
    }
  \end{lstlisting}
  \paragraph{}
  in the beginning of our document. Changing these values, will change all units formatted by siunitx automatically, which includes our table. It's pretty straightforward to add a table, this snippet works for all tables, unless they are too long to fit on one page. For this purpose, I will introduce a new command in a different tutorial.



  \subsection{Cells spanning multiple rows or multiple columns}
    \begin{enumerate}[label=\alph*)]
      \item LaTeX can generate tables from .csv files automatically
      \item Copy and paste the above snippet and packages to get it to work hassle free
      \item To add new columns, simply duplicate the display column line and change the number and name
      \item Add new units using the \textbackslash siunitx command and the ampersand seperator as shown above
      \item Have a .csv file seperated with comma as column seperator and newline as row seperator
      \item Does only work for tables smaller than one page
    \end{enumerate}
  \paragraph{}
  Sometimes it's necessary to make a row span several cells. For this purpose we can use the multirow package, so the first thing we're going to do is adding the required package to our preamble:

  \begin{lstlisting}[language={[LaTeX]TeX},breaklines=true,frame=single]
    %...

    \usepackage{multirow} % Required for multirows
    
    \begin{document} 
    
    %...
  \end{lstlisting}
  \paragraph{}
  We can now use multirow and multicolumn environments, which allow us to conveniently span multiple rows or columns.
  


  \subsubsection{Using multirow}
  \paragraph{}
  In order for a cell to span multiple rows, we have to use the multirow command. This command accepts three parameters:
   
  \begin{lstlisting}[language={[LaTeX]TeX},breaklines=true,frame=single]
    \multirow{NUMBER_OF_ROWS}{WIDTH}{CONTENT}
  \end{lstlisting}

  \paragraph{}
  % textsuperscript
  I usually use an asterisk (\text{*}) as a parameter for the width,
  since this basically means, that the width should be determined
  automatically.

  \paragraph{}
  Because we're combining two rows in our example, it's necessary to omit the content of the same row in the following line. Let's look at how the actual LaTeX code would look like:

  \begin{lstlisting}[language={[LaTeX]TeX},breaklines=true,frame=single]
    %...

    \begin{table}[h!]
      \begin{center}
        \caption{Multirow table.}
        \label{tab:table1}
        \begin{tabular}{l|S|r}
          \textbf{Value 1} & \textbf{Value 2} & \textbf{Value 3}\\
          $\alpha$ & $\beta$ & $\gamma$ \\
          \hline
          \multirow{2}{*}{12} & 1110.1 & a\\ % <-- Combining 2 rows with arbitrary with (*) and content 12
          & 10.1 & b\\ % <-- Content of first column omitted.
          \hline
          3 & 23.113231 & c\\
          4 & 25.113231 & d\\
        \end{tabular}
      \end{center}
    \end{table}
    
    %...
  \end{lstlisting}
  \paragraph{}
  The modified table looks like this:
  \begin{table}[h!]
    \begin{center}
      \caption{Multirow table.}
      \label{tab:table1}
      \begin{tabular}{l|S|r}
        \textbf{Value 1} & \textbf{Value 2} & \textbf{Value 3}\\
        $\alpha$ & $\beta$ & $\gamma$ \\
        \hline
        \multirow{2}{*}{12} & 1110.1 & a\\ % <-- Combining 2 rows with arbitrary with (*) and content 12
        & 10.1 & b\\ % <-- Content of first column omitted.
        \hline
        3 & 23.113231 & c\\
        4 & 25.113231 & d\\
      \end{tabular}
    \end{center}
  \end{table}
  \paragraph{}
  You can now see, that the cell containing 12 spans two rows.


  \subsubsection{Using multicolumn}
  \paragraph{}
  If we want a cell to span multiple columns, we have to use the multicolumn command. The usage differs a bit from multirow command, since we also have to specifiy the alignment for our column. The command also requires three parameters:
  \begin{lstlisting}[language={[LaTeX]TeX},breaklines=true,frame=single]
    \multicolumn{NUMBER_OF_COLUMNS}{ALIGNMENT}{CONTENT}
  \end{lstlisting}
  \paragraph{}
  In our example, we will again combine two neighboring cells, note that in the row where we're using multicolumn to span two columns, there's only one column separator (\&) (instead of two for all other rows):
  \begin{lstlisting}[language={[LaTeX]TeX},breaklines=true,frame=single]
    %...

    \begin{table}[h!]
      \begin{center}
        \caption{Multicolumn table.}
        \label{tab:table1}
        \begin{tabular}{l|S|r}
          \textbf{Value 1} & \textbf{Value 2} & \textbf{Value 3}\\
          $\alpha$ & $\beta$ & $\gamma$ \\
          \hline
          \multicolumn{2}{c|}{12} & a\\ % <-- Combining two cells with alignment c| and content 12.
          \hline
          2 & 10.1 & b\\
          3 & 23.113231 & c\\
          4 & 25.113231 & d\\
        \end{tabular}
      \end{center}
    \end{table}
    
    %...
  \end{lstlisting}
  \paragraph{}
  This will result in the following content:
  \begin{table}[h!]
    \begin{center}
      \caption{Multicolumn table.}
      \label{tab:table1}
      \begin{tabular}{l|S|r}
        \textbf{Value 1} & \textbf{Value 2} & \textbf{Value 3}\\
        $\alpha$ & $\beta$ & $\gamma$ \\
        \hline
        \multicolumn{2}{c|}{12} & a\\ % <-- Combining two cells with alignment c| and content 12.
        \hline
        2 & 10.1 & b\\
        3 & 23.113231 & c\\
        4 & 25.113231 & d\\
      \end{tabular}
    \end{center}
  \end{table}


  \subsubsection{Combining multirow and multicolumn}
  \paragraph{}
  Of course it's also possible to combine the two features, to make a cell spanning multiple rows and columns. To do this, we simply use the multicolumn command and instead of specifying content, we add a multirow command as the content. We then have to add another multicolumn statement for as many rows as we're combining.
  \paragraph{}
  Because this is a little hard to explain, it will be much clearer when looking at the code. In this example, we're going to combine two columns and two rows, so we're getting a cell spanning a total of four cells:
  \begin{lstlisting}[language={[LaTeX]TeX},breaklines=true,frame=single]
    %...

    \begin{table}[h!]
      \begin{center}
        \caption{Multirow and -column table.}
        \label{tab:table1}
        \begin{tabular}{l|S|r}
          \textbf{Value 1} & \textbf{Value 2} & \textbf{Value 3}\\
          $\alpha$ & $\beta$ & $\gamma$ \\
          \hline
          \multicolumn{2}{c|}{\multirow{2}{*}{1234}} & a\\ % <-- Multicolumn spanning 2 columns, content multirow spanning two rows
          \multicolumn{2}{c|}{} & b\\ % <-- Multicolumn spanning 2 columns with empty content as placeholder
          \hline
          3 & 23.113231 & c\\
          4 & 25.113231 & d\\
        \end{tabular}
      \end{center}
    \end{table}
    
    %...
  \end{lstlisting}
  \paragraph{}
  Our document will now contain a table, with a huge cell:
  \begin{table}[h!]
    \begin{center}
      \caption{Multirow and -column table.}
      \label{tab:table1}
      \begin{tabular}{l|S|r}
        \textbf{Value 1} & \textbf{Value 2} & \textbf{Value 3}\\
        $\alpha$ & $\beta$ & $\gamma$ \\
        \hline
        \multicolumn{2}{c|}{\multirow{2}{*}{1234}} & a\\ % <-- Multicolumn spanning 2 columns, content multirow spanning two rows
        \multicolumn{2}{c|}{} & b\\ % <-- Multicolumn spanning 2 columns with empty content as placeholder
        \hline
        3 & 23.113231 & c\\
        4 & 25.113231 & d\\
      \end{tabular}
    \end{center}
  \end{table}



  \subsection{Prettier tables with booktabs}
  \paragraph{}
  Of course beauty is always in the eye of the beholder, but I personally think, that the default hlines used by the table environment are not very pretty. For my tables, i always use the booktabs package, which provides much prettier horizontal separators and the usage is not harder compared to simply using hlines.
  \paragraph{}
  Again, we have to add the according booktabs package to our preamble:
  \begin{lstlisting}[language={[LaTeX]TeX},breaklines=true,frame=single]
    %...

    \usepackage{booktabs} % For prettier tables
    
    \begin{document} 
    
    %...
  \end{lstlisting}
  \paragraph{}
  We can now replace the hlines in our example table with toprule, midrule and bottomrule provided by the booktabs package:
  \begin{lstlisting}[language={[LaTeX]TeX},breaklines=true,frame=single]
    %...

    \begin{table}[h!]
      \begin{center}
        \caption{Table using booktabs.}
        \label{tab:table1}
        \begin{tabular}{l|S|r}
          \toprule % <-- Toprule here
          \textbf{Value 1} & \textbf{Value 2} & \textbf{Value 3}\\
          $\alpha$ & $\beta$ & $\gamma$ \\
          \midrule % <-- Midrule here
          1 & 1110.1 & a\\
          2 & 10.1 & b\\
          3 & 23.113231 & c\\
          \bottomrule % <-- Bottomrule here
        \end{tabular}
      \end{center}
    \end{table}
    
    %...
  \end{lstlisting}
  \paragraph{}
  You can decide for yourself, if you prefer the hlines or the following output:
  \begin{table}[h!]
    \begin{center}
      \caption{Table using booktabs.}
      \label{tab:table1}
      \begin{tabular}{l|S|r}
        \toprule % <-- Toprule here
        \textbf{Value 1} & \textbf{Value 2} & \textbf{Value 3}\\
        $\alpha$ & $\beta$ & $\gamma$ \\
        \midrule % <-- Midrule here
        1 & 1110.1 & a\\
        2 & 10.1 & b\\
        3 & 23.113231 & c\\
        \bottomrule % <-- Bottomrule here
      \end{tabular}
    \end{center}
  \end{table}


  \subsection{Multipage tables}
  \paragraph{}
  If you have a lot of rows in your table, you will notice that by default, the table will be cropped at the bottom of the page, which is certainly not what you want. The package longtable provides a convenient way, to make tables span multiple pages. Of course we have to add the package to our preamble before we can start using it:
  \begin{lstlisting}[language={[LaTeX]TeX},breaklines=true,frame=single]
    %...

    \usepackage{longtable} % To display tables on several pages
    
    \begin{document} 
    
    %...
  \end{lstlisting}
  \paragraph{}
  It's actually not harder, but easier to use than the previous code for tables. I will first show you what the code looks like and than explain the differences between longtable and tabular, in case they're not obvious.
  \begin{lstlisting}[language={[LaTeX]TeX},breaklines=true,frame=single]
    %...

    \begin{longtable}[c]{l|S|r} % <-- Replaces \begin{table}, alignment must be specified here (no more tabular)
      \caption{Multipage table.}
      \label{tab:table1}\\
      \toprule
      \textbf{Value 1} & \textbf{Value 2} & \textbf{Value 3}\\
      $\alpha$ & $\beta$ & $\gamma$ \\
      \midrule
      \endfirsthead % <-- This denotes the end of the header, which will be shown on the first page only
      \toprule
      \textbf{Value 1} & \textbf{Value 2} & \textbf{Value 3}\\
      $\alpha$ & $\beta$ & $\gamma$ \\
      \midrule
      \endhead % <-- Everything between \endfirsthead and \endhead will be shown as a header on every page
      1 & 1110.1 & a\\
      2 & 10.1 & b\\
      % ...
      % ... Many rows in between
      % ...
      3 & 23.113231 & c\\
      \bottomrule
    \end{longtable}
    
    %...
  \end{lstlisting}
  \paragraph{}
  In the previous examples, we've always used the table 
  and tabular environments. The longtable environment 
  replaces both of them or rather combines both of them 
  into a single environment. We now use 
  \begin{verbatim} \begin{longtable}[POSITION_ON_PAGE]{ALIGNMENT}  \end{verbatim} 
  as an environment for our tables. Using this environment, 
  we create a table, that is automatically split between 
  pages, if it has too many rows.
  \paragraph{}
  The content on the table on the first page looks like this:
  \begin{longtable}[c]{l|S|r} % <-- Replaces \begin{table}, alignment must be specified here (no more tabular)
    \caption{Multipage table.}
    \label{tab:table1}\\
    \toprule
    \textbf{Value 1} & \textbf{Value 2} & \textbf{Value 3}\\
    $\alpha$ & $\beta$ & $\gamma$ \\
    \midrule
    \endfirsthead % <-- This denotes the end of the header, which will be shown on the first page only
    \toprule
    \textbf{Value 1} & \textbf{Value 2} & \textbf{Value 3}\\
    $\alpha$ & $\beta$ & $\gamma$ \\
    \midrule
    \endhead % <-- Everything between \endfirsthead and \endhead will be shown as a header on every page
    1 & 1110.1 & a\\
    2 & 10.1 & b\\
    % ...
    % ... Many rows in between
    % ...
    3 & 23.113231 & c\\
    \bottomrule
  \end{longtable}
  \paragraph{}
  Imagine that this table has many rows (...) being a placeholder for more rows and that the table continues the following page like this:
  \paragraph{}
  Note that there's again a header on this page, but without the caption. This is because I've added the commands \textbackslash endfirsthead and \textbackslash endhead to my table, where everything written before \textbackslash endfirsthead declares the header for the first page the table occurs on and everything between \textbackslash endfirsthead and \textbackslash endhead denotes the header, which should be repeated on every following page. If you don't want a header on the following pages, you can simply remove everything in between \textbackslash endfirsthead and \textbackslash endhead including the \textbackslash endhead command.

  \subsection{Landscape tables}
  Now that we have a solution for too many rows, we could also be facing the same problem if we had too many columns. If we add too many columns, we might be getting a table that's too wide for the page. In this situation, it's often best to simply rotate the table and print it in sideways. While there are many different ways to rotate the table, the only that I've found to be satisfying was using the rotating package.
  \paragraph{}
  First we include the required package in our preamble:
  \begin{lstlisting}[language={[LaTeX]TeX},breaklines=true,frame=single]
    %...

    \usepackage{rotating} % To display tables in landscape
    
    \begin{document} 
    
    %...
  \end{lstlisting}
  \paragraph{}
  This package provides the sidewaystable environment, which is very easy to use. Just replace the table environment with the sidewaystable environment like this:
  \begin{lstlisting}[language={[LaTeX]TeX},breaklines=true,frame=single]
    %...

    \begin{sidewaystable}[h!] % <--
      \begin{center}
      \caption{Landscape table.}
      \label{tab:table1}
      \begin{tabular}{l|S|r}
        \toprule
        \textbf{Value 1} & \textbf{Value 2} & \textbf{Value 3}\\
        $\alpha$ & $\beta$ & $\gamma$ \\
        \midrule
        1 & 1110.1 & a\\
        2 & 10.1 & b\\
        3 & 23.113231 & c\\
        \bottomrule
      \end{tabular}
      \end{center}
    \end{sidewaystable}
    
    %...
  \end{lstlisting}
  \paragraph{}
  This will automatically rotate the table for us, so it can be read when flipping the page sideways:
  \begin{sidewaystable}[h!] % <--
    \begin{center}
    \caption{Landscape table.}
    \label{tab:table1}
    \begin{tabular}{l|S|r}
      \toprule
      \textbf{Value 1} & \textbf{Value 2} & \textbf{Value 3}\\
      $\alpha$ & $\beta$ & $\gamma$ \\
      \midrule
      1 & 1110.1 & a\\
      2 & 10.1 & b\\
      3 & 23.113231 & c\\
      \bottomrule
    \end{tabular}
    \end{center}
  \end{sidewaystable}



  \subsection{Tables from Excel (.csv) to LaTeX}
  There are two disadvantages of writing tables by hand as described in this tutorial. While it works for small tables similar to the one in our example, it can take a long time to enter a large amount of data by hand. Most of the time the data will be collected in form of a spreadsheet and we don't want to enter the data twice. Furthermore once put into LaTeX tables, the data can not be plotted anymore and is not in a useful form in general. For this reason, the next lesson shows you, how to generate tables from .csv files (which can be exported from Excel and other tools) automatically.


  \paragraph{Summary}
    % list
    \begin{itemize} % list_type 有 enumerate、 itemize 和 description
      \item LaTeX offers the table and tabular environment for table creation
      \item The table environment acts like a wrapper for the tabular similar to the figure environment
      \item Alignment and vertical separators are passed as an argument to the tabular environment 
            \begin{verbatim}(e.g. \begin{tabular}{l|c||r}) \end{verbatim}
      \item It's possible to align the content left (l), centered (c) and right (r), where the number of alignment operators has to match the desired number of columns
      \item The columns can be seperated by adding | in between the alignment operators
      \item Rows can be seperated using the \textbackslash hline command and columns using the ampersand \& symbol
      \item The newline \textbackslash \textbackslash \enspace operator indicates the end of a row       
      \item It's possible to refer to tables using \textbackslash ref and \textbackslash label
      \item Align numbers at the decimal point using the siunitx package
      \item Combine multiple rows and columns with the multirow package
      \item Prettify your tables using the booktabs package
      \item Make your tables span multiple pages with the longtable package
      \item Display your tables in landscape using the rotating package
    \end{itemize} 
