% --------------------------------
% -------  chapter 11  ------------
% --------------------------------

\maketitle
\newpage
\section{Highlight source code}
  \textbf{
    The listings package offers source code highlighting for various languages. Learn by example how to use it in your LaTeX documents. 
  }
  % \begin{enumerate} % list_type 有 enumerate、 itemize 和 description
  %   \item Basic example
  %   \item Syntax of tikz
  % \end{enumerate} 

  \subsection{Example}

  \begin{lstlisting}[language={[LaTeX]TeX},breaklines=true,frame=single]
    \documentclass{article}

    \usepackage{listings}
    \usepackage{color}
    
    \renewcommand\lstlistingname{Quelltext} % Change language of section name
    
    \lstset{ % General setup for the package
      language=Perl,
      basicstyle=\small\sffamily,
      numbers=left,
       numberstyle=\tiny,
      frame=tb,
      tabsize=4,
      columns=fixed,
      showstringspaces=false,
      showtabs=false,
      keepspaces,
      commentstyle=\color{red},
      keywordstyle=\color{blue}
    }
    
    \begin{document}
    \begin{lstlisting}
    #!/usr/bin/perl
    print S(@ARGV);sub S{$r=(@_[0]%4==0&&@_[0]%100!=0)||@_[0]%400=0;}
  \end{lstlisting}
  \paragraph{}
  The listings package is a powerful way to get nice source code highlighting in LaTeX. It's fairly easy to use and there's good documentation available on how to use it.
  
  \renewcommand\lstlistingname{Quelltext} % Change language of section name
  \lstset{ % General setup for the package
    language=Perl,
    basicstyle=\small\sffamily,
    numbers=left,
     numberstyle=\tiny,
    frame=tb,
    tabsize=4,
    columns=fixed,
    showstringspaces=false,
    showtabs=false,
    keepspaces,
    commentstyle=\color{red},
    keywordstyle=\color{blue}
  }
  \begin{lstlisting}
    #!/usr/bin/perl
    print S(@ARGV);sub S{$r=(@_[0]%4==0&&@_[0]%100!=0)||@_[0]%400=0;}
  \end{lstlisting}

  \paragraph{}
  I first use the include the color and listings 
  package and then set up the language of the 
  package headings to german using 
  \begin{verbatim}
    \renewcommand\lstlistingname{Quelltext}  
  \end{verbatim}
  . This is not necessary if you're planning to use it in English.
  \paragraph{}
  Afterwards I set up the general layout for the package with the \textbackslash lstset command. The options I set there should be self-explanatory. Note that it's required to manually set the colors for keywords and comments, otherwise the output would be only black on white. The desired output must then be embedded within a listings environment.
  \paragraph{}
  Assuming we have a Perl script saved in a file script.pl, we could also simply use the following syntax to get the same result:
  \begin{lstlisting}
    \lstinputlisting{script.pl}
  \end{lstlisting}
  \paragraph{}
  This will keep your LaTeX source clean and you can still use all features of the package.
  
  \paragraph{Summary}
    % list
    \begin{itemize} % list_type 有 enumerate、 itemize 和 description
      \item After some initial setup, all source code can be embedded in a lstlistings environment
      \item A list of all languages and more documentation is available in the manual of the listings package
      \item Use the \textbackslash lstlinputlisting\{FILENAME\} command to read the content of source files directly into a lstlistings environment.
    \end{itemize} 